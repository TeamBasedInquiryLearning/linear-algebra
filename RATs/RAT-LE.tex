\documentclass{article}
\usepackage{enumerate,amssymb,tikz, tikz-cd,amsmath,amsthm,multicol,hyperref,environ,etoolbox,graphicx,catchfile}

\usepackage[left=1in,right=1in,top=1in,bottom=1.1in,includeheadfoot]{geometry}
\usepackage{fancyhdr}

%%%Page layout setup%%%
\newcommand{\header}{Linear Algebra}
\newcommand{\setHeader}[1]{\renewcommand{\header}{#1}}

\newcommand{\moduleLetter}{X}

\pagestyle{fancy}
\fancyhf{}
\chead{\header}
\rfoot{Page \thepage}
\renewcommand{\headrulewidth}{1pt}
\renewcommand{\footrulewidth}{1pt}


\NewEnviron{module}[2]{
  \renewcommand{\moduleLetter}{#1}
  \setHeader{Module \moduleLetter{}: #2}
  \BODY
}
\NewEnviron{readinessAssuranceTest}{
  \newpage
  \subsection*{Readiness Assurance Test}
  Choose the most appropriate response for each question.
  \begin{enumerate}[1)]
    \BODY
  \end{enumerate}
}


\NewEnviron{readinessAssuranceTestChoices}{
  \begin{enumerate}[(a)]
    \BODY
  \end{enumerate}
}

\newcommand{\systemWithOneSolutionA}[1][0.25]{
\begin{tikzpicture}[scale=#1]
\draw[thin,gray,<->] (-5,0) -- (5,0);
\draw[thin,gray,<->] (0,-5) -- (0,5);
\draw[thick,blue] (-5,4.5) -- (5,-0.5);
\draw[thick,red] (-5,-3.67) -- (5,3);
\end{tikzpicture}
}

\newcommand{\systemWithOneSolutionB}[1][0.25]{
\begin{tikzpicture}[scale=#1]
\draw[thin,gray,<->] (-5,0) -- (5,0);
\draw[thin,gray,<->] (0,-5) -- (0,5);
\draw[thick,blue] (-5,-3) -- (5,-0.5);
\draw[thick,red] (-5,-4) -- (5,2);
\end{tikzpicture}
}

\newcommand{\systemWithInfinitelyManySolutions}[1][0.25]{
\begin{tikzpicture}[scale=#1]
\draw[thin,gray,<->] (-5,0) -- (5,0);
\draw[thin,gray,<->] (0,-5) -- (0,5);
\draw[thick,purple] (-3,5) -- (5,-3);
\end{tikzpicture}
}

\newcommand{\systemWithNoSolutions}[1][0.25]{
\begin{tikzpicture}[scale=#1]
\draw[thin,gray,<->] (-5,0) -- (5,0);
\draw[thin,gray,<->] (0,-5) -- (0,5);
\draw[thick,blue] (3,-5) -- (1,5);
\draw[thick,red] (0,-5) -- (-2,5);
\end{tikzpicture}
}

\begin{document}
% IF-AT Form D012
%A A D C A D B B A A



\begin{module}{LE}{Solving Systems of Linear Equations}
\begin{readinessAssuranceTest}

%1 A
\item Which of the following describe the set of all points on the line \(2x+3y=0\)?
\begin{multicols}{4}
\begin{readinessAssuranceTestChoices}
\item \(\displaystyle \left\{ (x,y) \,\middle|\, 2x+3y=0 \right\}\) %Correct
\item \(\displaystyle \left\{ (x,y) \right\}\)
\item \(\displaystyle \left\{ (2x,3y) \right\}\)
\item \(\displaystyle \left\{ (2x,3y) \,\middle|\, 2x+3y=0 \right\}\)
\end{readinessAssuranceTestChoices}
\end{multicols}


%2  A
\item How many solutions are there for the system of linear equations
      represented by the following graph?
    \begin{center}
      \systemWithOneSolutionB[0.23]
    \end{center}

\begin{multicols}{4}
\begin{readinessAssuranceTestChoices}
\item One % correct
\item Two
\item Infinitely-many
\item Zero
\end{readinessAssuranceTestChoices}
\end{multicols}

%3 D
\item Which of the following points is an element of the set 
		\(\displaystyle \left\{ (x,y) \,\middle|\, 3x+4y=12 \right\}\) ?
\begin{multicols}{4}
\begin{readinessAssuranceTestChoices}
\item \( (1,1) \)
\item \( (3,4) \)
\item \( (4,-3) \)
\item \( (8,-3) \) %Correct
\end{readinessAssuranceTestChoices}
\end{multicols}

%4  C
\item How many solutions are there for the system of linear equations
      represented by the following graph? (This graph represents two completely
      overlapping lines.)
    \begin{center}
      \systemWithInfinitelyManySolutions
    \end{center}

\begin{multicols}{4}
\begin{readinessAssuranceTestChoices}
\item One
\item Two
\item Infinitely-many % correct
\item Zero
\end{readinessAssuranceTestChoices}
\end{multicols}


%5  A
\item How many solutions are there for the system of linear equations
      represented by the following graph?
    \begin{center}
      \systemWithOneSolutionA
    \end{center}

\begin{multicols}{4}
\begin{readinessAssuranceTestChoices}
\item One % correct
\item Two
\item Infinitely-many
\item Zero
\end{readinessAssuranceTestChoices}
\end{multicols}

\newpage

%6  D
\item How many solutions are there for the system of linear equations
      represented by the following graph? (This graph represents two
      non-overlapping parallel lines.)
    \begin{center}
      \systemWithNoSolutions
    \end{center}

\begin{multicols}{4}
\begin{readinessAssuranceTestChoices}
\item Infinitely-many
\item Two
\item One
\item Zero % correct
\end{readinessAssuranceTestChoices}
\end{multicols}


%7  B
\item Solve the following system of linear equations.
      \begin{align*}
      y   &=   2x+5 \\
      y  &=  -x+2
      \end{align*}

\begin{multicols}{4}
\begin{readinessAssuranceTestChoices}
\item \((x,y)=(4,-2)\)
\item \((x,y)=(-1,3)\) % correct
\item There are no solutions.
\item There are infinitely-many solutions.
\end{readinessAssuranceTestChoices}
\end{multicols}


%8  B
\item Solve the following system of linear equations.
      \begin{align*}
      y   &=  3x+5 \\
      y  &=  3x+2
      \end{align*}

\begin{multicols}{4}
\begin{readinessAssuranceTestChoices}
\item There are infinitely-many solutions.
\item There are no solutions. % correct
\item
\((x,y)=(3,4)\)
\item
\((x,y)=(-5,1)\)
\end{readinessAssuranceTestChoices}
\end{multicols}


%9  A
\item Solve the following system of linear equations.
      \begin{align*}
      x+2y   &=   4 \\
      2x-3y  &=  1
      \end{align*}

\begin{multicols}{4}
\begin{readinessAssuranceTestChoices}


\item
\((x,y)=(2,1)\) % correct
\item
\((x,y)=(-1,4)\)
\item There are no solutions.
\item There are infinitely-many solutions.
\end{readinessAssuranceTestChoices}
\end{multicols}


%10 A
\item Solve the following system of linear equations.
      \begin{align*}
      4x-8y   &= 12 \\
      -6x+12y  &=  -18
      \end{align*}

\begin{multicols}{4}
\begin{readinessAssuranceTestChoices}
\item There are infinitely-many solutions. % correct
\item There are no solutions.

\item
\((x,y)=(3,3)\)
\item
\((x,y)=(-2,1)\)

\end{readinessAssuranceTestChoices}
\end{multicols}

\end{readinessAssuranceTest}
\end{module}


\end{document}
