\documentclass{article}[12pt]

\usepackage{tbil-la}

\usepackage[left=1in,right=1in,top=1in,bottom=1in]{geometry}

\begin{document}



\begin{readinessAssuranceOutcomes}
\item Solve a system of linear equations (including finding a basis of the solution space if it is homogeneous) by interpreting as an augmented matrix and row reducing \standardList{E1, E2, E3, E4}.
\item State the definition of linear independence, and determine if a set of vectors is linearly dependent or independent \standardList{V5}.
\item State the definition of a spanning set, and determine if a set of vectors spans a vector space or subspace \standardList{V6, V7}.
\item State the definition of a basis, and determine if a set of vectors is a basis \standardList{V8, V9}.
\end{readinessAssuranceOutcomes}

\begin{readinessAssuranceResources}
\item TODO
\end{readinessAssuranceResources}




\begin{readinessAssuranceTest}

\item Which of the following is a solution to the system of linear equations
      \begin{align*}
      x+3y-z    &=   2 \\
      2x+8y+3z  &=  -1 \\
      -x-y+9z   &= -10
      \end{align*}

\begin{multicols}{4}
\begin{readinessAssuranceTestChoices}
\item $\begin{bmatrix} 1 \\ 1 \\ 0 \end{bmatrix}$
\item $\begin{bmatrix} 0 \\ 1 \\ -1 \end{bmatrix}$
\item $\begin{bmatrix} 1 \\ 0 \\ -1 \end{bmatrix}$
\item $\begin{bmatrix} 1 \\ -1 \\ 1 \end{bmatrix}$
\end{readinessAssuranceTestChoices}
\end{multicols}


\item Find a basis for the solution set of the following homogeneous system of
      linear equations
      \begin{align*}
      x+2y+-z-w    &= 0 \\
      -2x-4y+3z+5w &= 0
      \end{align*}

\begin{multicols}{4}
\begin{readinessAssuranceTestChoices}
\item $\left\{ \begin{bmatrix} 2 \\ 1 \\ 0 \\ 0 \end{bmatrix}, \begin{bmatrix} 2 \\ 0 \\ 3 \\ 1 \end{bmatrix} \right\}$
\item $\left\{ \begin{bmatrix} 2 \\ 2 \\ 0 \\ 0 \end{bmatrix}, \begin{bmatrix} 0 \\ 0 \\ 3 \\ 0 \end{bmatrix} \right\}$
\item $\left\{ \begin{bmatrix} 2 \\ 1 \\ 3 \\ 1 \end{bmatrix} \right\}$
\item None of these are a basis.
\end{readinessAssuranceTestChoices}
\end{multicols}


\item Determine which property applies to the set of vectors $$\left\{ \begin{bmatrix}  1 \\ 0 \\ 0 \end{bmatrix}, \begin{bmatrix} 0 \\ 1 \\ 0 \end{bmatrix} \right\} \subset \IR^3.$$

\begin{readinessAssuranceTestChoices}
\item It does not span and is linearly dependent
\item It does not span and is linearly independent
\item It spans but it is linearly dependent
\item It is a basis of $\IR^3$.
\end{readinessAssuranceTestChoices}


\item Determine which property applies to the set of vectors $$\left\{ \begin{bmatrix}  1 \\ 0 \\ 0 \end{bmatrix}, \begin{bmatrix} 2 \\ 1 \\ 0 \end{bmatrix} , \begin{bmatrix} 1 \\ 1 \\ 3 \end{bmatrix} \right\}\subset \IR^3.$$

\begin{readinessAssuranceTestChoices}
\item It does not span and is linearly dependent
\item It does not span and is linearly independent
\item It spans but it is linearly dependent
\item It is a basis of $\IR^3$.
\end{readinessAssuranceTestChoices}


\item Determine which property applies to the set of vectors $$\left\{ \begin{bmatrix}  1 \\ 0 \\ 0 \end{bmatrix}, \begin{bmatrix} -2 \\ 0 \\ -2 \end{bmatrix} , \begin{bmatrix} 1 \\ 1 \\ 0 \end{bmatrix} , \begin{bmatrix} 3 \\ 3 \\ -3 \end{bmatrix}\right\}\subset \IR^3.$$

\begin{readinessAssuranceTestChoices}
\item It does not span and is linearly dependent
\item It does not span and is linearly independent
\item It spans but it is linearly dependent
\item It is a basis of $\IR^3$.
\end{readinessAssuranceTestChoices}


\item Determine which property applies to the set of vectors $$\left\{ \begin{bmatrix}  2 \\ 2 \\ -1 \end{bmatrix}, \begin{bmatrix} -3 \\ 1 \\ -2 \end{bmatrix} , \begin{bmatrix} 1 \\ 5 \\ -4 \end{bmatrix}\right\}\subset \IR^3.$$

\begin{readinessAssuranceTestChoices}
\item It does not span and is linearly dependent
\item It does not span and is linearly independent
\item It spans but it is linearly dependent
\item It is a basis of $\IR^3$.
\end{readinessAssuranceTestChoices}


\item Find a basis for the subspace of $\IR^4$ spanned by the vectors ...


\item Suppose you know that every vector in $\IR^5$ can be written as a linear combination of the vectors $\{\vec{v}_1, \ldots, \vec{v}_n\}$.  What can you conclude about $n$?

\begin{readinessAssuranceTestChoices}
\item $n \leq 5$
\item $n=5$
\item $n \geq 5$
\item $n$ could be any positive integer
\end{readinessAssuranceTestChoices}

\item Suppose you know that every vector in $\IR^5$ can be written uniquely as a linear combination of the vectors $\{\vec{v}_1, \ldots, \vec{v}_n\}$.  What can you conclude about $n$?

\begin{readinessAssuranceTestChoices}
\item $n \leq 5$
\item $n=5$
\item $n \geq 5$
\item $n$ could be any positive integer
\end{readinessAssuranceTestChoices}

\item Suppose you know that every vector in $\IR^5$ can be written uniquely as a linear combination of the vectors $\{\vec{v}_1, \ldots, \vec{v}_n\}$.  What can you conclude about the set $\{\vec{v}_1, \ldots, \vec{v}_n\}$?

\begin{readinessAssuranceTestChoices}
\item It does not span and is linearly dependent
\item It does not span and is linearly independent
\item It spans but it is linearly dependent
\item It is a basis of $\IR^3$.
\end{readinessAssuranceTestChoices}

\end{readinessAssuranceTest}



\begin{applicationActivities}{Day 1}

\begin{definition}
A \term{linear transformation} is a map between vector spaces that preserves the vector space operations.  More precisely, if $V$ and $W$ are vector spaces, a map $T:V\rightarrow W$ is called a linear transformation if
\begin{enumerate}
\item $T(\vec{v}+\vec{w}) = T(\vec{v})+T(\vec{w})$ for any $\vec{v},\vec{w} \in V$
\item $T(c\vec{v}) = cT(\vec{v})$ for any $c \in \IR$, $\vec{v} \in V$.
\end{enumerate}
In other words, a map is linear if one can do vector space operations before applying the map or after, and obtain the same answer.

$V$ is called the \term{domain} of $T$ and $W$ is called the \term{co-domain} of $T$.
\end{definition}

\begin{activity}
Determine if each of the following maps are linear transformations
\begin{enumerate}[(a)]
\item $T_1 : \IR^2 \rightarrow \IR$ given by $T_1\left(\begin{bmatrix} a \\ b \end{bmatrix} \right) = \sqrt{a^2+b^2}$
\item $T_2 : \IR^3 \rightarrow \IR^2$ given by $T_2\left(\begin{bmatrix} x \\ y \\ z \end{bmatrix} \right) = \begin{bmatrix} x-z \\ y \end{bmatrix}$
\item $T_3: \P_d \rightarrow \P_{d-1}$ given by $T_3(f(x)) = f^\prime(x)$.
\item $T_4: C(\IR) \rightarrow C(\IR)$ given by $T_4(f(x)) = f(-x)$
\item $T_5: \P \rightarrow \P$ given by $T_5(f(x)) = f(x)+x^2$
\end{enumerate}
\end{activity}

\begin{activity}
Suppose $T: \IR^3 \rightarrow \IR^2$ is a linear transformation, and you know $T\left(\begin{bmatrix} 1 \\ 0 \\ 0 \end{bmatrix} \right) = \begin{bmatrix} 2 \\ 1 \end{bmatrix} $ and $T\left(\begin{bmatrix} 0 \\ 0 \\ 1 \end{bmatrix} \right) = \begin{bmatrix} -3 \\ 2 \end{bmatrix} $.  Compute each of the following:
\begin{enumerate}[(a)]
\item $T\left(\begin{bmatrix} 3 \\ 0 \\ 0 \end{bmatrix}\right)$
\item $T\left(\begin{bmatrix} 0 \\ 0 \\ -2 \end{bmatrix}\right)$
\item $T\left(\begin{bmatrix} 1 \\ 0 \\ 1 \end{bmatrix}\right)$
\item $T\left(\begin{bmatrix} -2 \\ 0 \\ 5 \end{bmatrix}\right)$
\end{enumerate}
\end{activity}

\begin{activity}
Suppose $T: \IR^4 \rightarrow \IR^3$ is a linear transformation.  What is the smallest number of vectors needed to determine $T$?  In other words, what is the smallest number $n$ such that there are $\vec{v}_1,\ldots,\vec{v}_n \in \IR^4$ and given  $T(\vec{v}_1), \ldots, T(\vec{v}_n)$ you can determine $T(\vec{w})$ for \textit{any} $\vec{w} \in \IR^2$?
\end{activity}

\begin{observation}
Fix an ordered basis for $V$.  Since every vector can be written \textit{uniquely} as a linear combination of basis vectors, a linear transformation $T:V \rightarrow W$ corresponds exactly to a choice of where each basis vector goes.  For convenience, we can thus encode a linear transformation as a matrix, with one column for the image of each basis vector (in order).
\end{observation}

\begin{activity}
Let $T: \IR^3 \rightarrow \IR^2$ be a linear transformation with
\begin{align*}
T\left(\begin{bmatrix} 1 \\ 0 \\ 0 \end{bmatrix} \right) &= \begin{bmatrix} 3 \\ 2\end{bmatrix} &
T\left(\begin{bmatrix} 0 \\ 1 \\ 0 \end{bmatrix} \right) &= \begin{bmatrix} -1 \\ 4\end{bmatrix} &
T\left(\begin{bmatrix} 0 \\ 0 \\ 1 \end{bmatrix} \right) &= \begin{bmatrix} 5 \\ 0\end{bmatrix}
\end{align*}
Write the matrix corresponding to this linear transformation with respect to the standard ordered basis.
\end{activity}

\begin{activity}
  Let $T: \IR^3 \rightarrow \IR^2$ be a linear transformation with
\begin{align*}
T\left(\begin{bmatrix} 1 \\ 0 \\ 0 \end{bmatrix} \right) &= \begin{bmatrix} 3 \\ 2\end{bmatrix} &
T\left(\begin{bmatrix} 0 \\ 1 \\ 0 \end{bmatrix} \right) &= \begin{bmatrix} -1 \\ 4\end{bmatrix} &
T\left(\begin{bmatrix} 0 \\ 0 \\ 1 \end{bmatrix} \right) &= \begin{bmatrix} 5 \\ 0\end{bmatrix}
\end{align*}
Write the matrix corresponding to this linear transformation with respect to the ordered basis
\[\left\{ \begin{bmatrix} 2 & 1 & 1 \end{bmatrix} , \begin{bmatrix} -1 & -1 & 3 \end{bmatrix} , \begin{bmatrix} 0 & 1 & 2 \end{bmatrix} \right\}\]
\end{activity}

\begin{activity}
Let $D: \P_3 \rightarrow \P_2$ be the derivative map (recall this is a linear transformation).  Write the matrix corresponding to $D$ with respect to the ordered basis $\{1,x,x^2,x^3\}$.
\end{activity}

\end{applicationActivities}



\begin{applicationActivities}{Day 2}

\begin{definition}
Let $T: V \rightarrow W$ be a linear transformation.
\begin{itemize}
\item $T$ is called \term{injective} or \term{one-to-one} if $T$ does not map two distinct values to the same place.  More precisely, $T$ is injective if $T(\vec{v}) \neq T(\vec{w})$ whenever $\vec{v} \neq \vec{w}$.
\item $T$ is called \term{surjective} or \term{onto} if every element of $W$ is mapped to by an element of $V$.  More precisely, for every $\vec{w} \in W$, there is some $v \in V$ with $T(\vec{v})=\vec{w}$.
\end{itemize}
\end{definition}

\begin{activity}
Let $T: \IR^3 \rightarrow \IR^2$ be given by the matrix $\begin{bmatrix} 1 & 0 \\ 0 & 1 \\ 0 & 0 \end{bmatrix}$.  Determine if $T$ is injective, surjective, both, or neither.
\end{activity}

\begin{activity}
Let $T: \IR^2 \rightarrow \IR^3$ be given by the matrix $\begin{bmatrix} 1 & 0 &0  \\ 0 & 1 & 0 \end{bmatrix}$.  Determine if $T$ is injective, surjective, both, or neither.
\end{activity}

\begin{definition}
We also have two important sets called the \term{kernel} of $T$ and the \term{image} of $T$.
\begin{align*}
\ker T &= \left\{ \vec{v} \in V\ \big|\ T(\vec{v})=0\right\} \\
\Im T &= \left\{ \vec{w} \in W\ \big|\ \text{there is some }v\in V \text{ with } T(\vec{v})=\vec{w}\right\}
\end{align*}
\end{definition}

\begin{activity}
Let $T: \IR^3 \rightarrow \IR^2$ be given by the matrix $\begin{bmatrix} 1 & 0 \\ 0 & 1 \\ 0 & 0 \end{bmatrix}$ (for the standard basis).  Find the kernel and image of $T$.
\end{activity}

\begin{activity}
Let $T: \IR^2 \rightarrow \IR^3$ be given by the matrix $\begin{bmatrix} 1 & 0 &0  \\ 0 & 1 & 0 \end{bmatrix}$ (for the standard basis).  Find the kernel and image of $T$.
\end{activity}

\begin{activity}
Describe surjective linear transformations in terms of the image.
\end{activity}

\begin{activity}
Describe injective linear transformations in terms of the kernel.
\end{activity}

\begin{activity}
Let $T: \IR^3 \rightarrow \IR^2$ be the linear transformation given by the matrix $A=\begin{bmatrix} 3 & 4 & -1 \\ 1 & 2 & 1 \end{bmatrix}$ (for the standard basis).
\begin{enumerate}[1)]
\item Write a system of equations whose solution set is the kernel.
\item Compute $\RREF(A)$ and solve the system of equations.
\item Compute the kernel of $T$
\item Find a basis for the kernel of $T$
\end{enumerate}
\end{activity}

\begin{activity}
Let $S: \IR^3 \rightarrow \IR^2$ be the linear transformation given by the matrix $B=\begin{bmatrix} 3 & 4 & 1 \\ 1 & 2 & 4 \\ 5 & 8 & 9 \end{bmatrix}$ (for the standard basis).
\begin{enumerate}[1)]
\item Write a system of equations whose solution set is the kernel.
\item Compute $\RREF(A)$ and solve the system of equations.
\item Compute the kernel of $T$
\item Find a basis for the kernel of $T$
\end{enumerate}
\end{activity}

\begin{activity}
Let $T: \IR^3 \rightarrow \IR^3$ be the linear transformation given by the matrix $A=\begin{bmatrix} 3 & 4 & -1 \\ 1 & 2 & 1 \end{bmatrix}$ (for the standard basis).
\begin{enumerate}[1)]
\item Find a set of vectors that span the image of $T$
\item Find a basis for the image of $T$.
\end{enumerate}
\end{activity}

\begin{activity}
Let $S: \IR^3 \rightarrow \IR^3$ be the linear transformation given by the matrix $B=\begin{bmatrix} 3 & 4 & 1 \\ 1 & 2 & 4 \\ 5 & 8 & 9  \end{bmatrix}$ (for the standard basis).
\begin{enumerate}[1)]
\item Find a set of vectors that span the image of $T$
\item Find a basis for the image of $T$.
\end{enumerate}
\end{activity}

\end{applicationActivities}



\begin{applicationActivities}{Day 3}

\begin{activity}
Let $T: \IR^n \rightarrow \IR^m$ be a linear map with matrix $A \in M_{m,n}$ (for the standard basis).  Consider the following statements about $T$
\begin{enumerate}[(a)]
\item $T$ is injective
\item $T$ is not injective
\item $T$ is surjective
\item $T$ is not surjective
\item The system of linear equations given by the augmented matrix $\begin{bmatrix}[c|c]A & \vec{b} \end{bmatrix}$ has a solution for all $\vec{b} \in \IR^m$
\item The system of linear equations given by the augmented matrix $\begin{bmatrix}[c|c]A & \vec{b} \end{bmatrix}$ has a unique solution for all $\vec{b} \in \IR^m$
\item The system of linear equations given by the augmented matrix $\begin{bmatrix}[c|c] A & \vec{0} \end{bmatrix}$ has a non-trivial solution.
\item The columns of $A$ span $\IR^m$
\item The columns of $A$ are linearly independent
\item The columns of $A$ are a basis of $\IR^m$
\item Every column of $\RREF(A)$ is a pivot column
\item $\RREF(A)$ has a non-pivot column
\item $\RREF(A)$ has $n$ pivot columns
\end{enumerate}
Sort these statements into groups of equivalent statements.
\end{activity}

\begin{activity}
Gallery walk--switch boards with a different team.  If they have two things grouped together that you know are not equivalent, write a reason or counter-example on a sticky note.
\end{activity}

\begin{activity}
Update your team's groupings based on feedback.
\end{activity}

\begin{activity}
Repeat?
\end{activity}

\begin{activity}
Can you add any statements to any groups?
\end{activity}

\end{applicationActivities}





\end{document}
