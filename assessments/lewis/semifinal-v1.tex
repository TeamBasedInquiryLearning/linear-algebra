%% 
%% This is file, `semifinal-v1.tex',
%% generated with the extract package.
%% 
%% Generated on :  2017/11/21,16:43
%% From source  :  semifinal-v1_solutions.tex
%% Using options:  active,generate=semifinal-v1,extract-env={problem}
%% 
\documentclass{sbgLAsemi}
\usepackage{amsmath,amssymb,amsthm,enumerate}
\coursetitle{Math 237}
\courselabel{Linear Algebra}
\calculatorpolicy{You may use a calculator, but you must show all relevant work to receive credit for a standard.}

\newcommand{\IR}{\mathbb{R}}

\makeatletter
\renewcommand*\env@matrix[1][*\c@MaxMatrixCols c]{%
  \hskip -\arraycolsep
  \let\@ifnextchar\new@ifnextchar
  \array{#1}}
\makeatother


\title{Semifinal}
\version{1}

\begin{document}

\begin{problem}{E1}
Write a system of linear equations corresponding to the following
augmented matrix.
\[
\begin{bmatrix}[ccc|c]
2 & -1 & 0 & 1  \\
-1 & 4 & 1 & -7  \\
1 & 2 & -1 & 0
\end{bmatrix}
\]
\end{problem}

\begin{problem}{E2}
Find the reduced row echelon form of the matrix below.
$$\begin{bmatrix}[cccc|c] 2 & 1 & -1 & 0 & 5 \\ 3 & -1 & 0 & -2 & 0 \\ -1 & 0 & 5 & 0 & -1 \end{bmatrix}$$
\end{problem}

\begin{problem}{E3}
Solve the system of linear equations.
\begin{align*}
2x+y-z+w &=5 \\
3x-y-2w &= 0 \\
-x+5z+3w&=-1
\end{align*}
\end{problem}

\begin{problem}{E4}
Find a basis for the solution set to the homogeneous system of equations
given by
\begin{align*}
3x+2y+z &= 0 \\
x+y+z &= 0
\end{align*}
\end{problem}

\begin{problem}{V1}
Let $V$ be the set of all pairs of real numbers with the operations, for any $(x_1,y_1), (x_2,y_2) \in V$, $c \in \IR$,
\begin{align*}
(x_1,y_1) \oplus (x_2,y_2) &= (x_1+x_2,y_1+y_2) \\
c \odot (x_1,y_1) &= (c^2x_1, c^3y_1)
\end{align*}
\begin{enumerate}[(a)]
\item Show that scalar multiplication \textbf{distributes scalars} over
      vector addition:
      \(c\odot((x_1,y_1) \oplus (x_2,y_2))=
      c\odot(x_1,y_1) \oplus c\odot(x_2,y_2)\).
\item Determine if $V$ is a vector space or not.  Justify your answer.
\end{enumerate}
\end{problem}

\begin{problem}{V2}
  Determine if
  \(\begin{bmatrix} 3 \\ -2 \\ 4 \end{bmatrix}\)
  belongs to the span of the set
  \(\left\{
    \begin{bmatrix} 1 \\ 2 \\ -3 \end{bmatrix},
    \begin{bmatrix} 2 \\ 4 \\ -6 \end{bmatrix},
    \begin{bmatrix} 0 \\ 0 \\ 0 \end{bmatrix}
    \right\}
  \).
\end{problem}

\begin{problem}{V3}
Determine if the vectors $\begin{bmatrix} 1 \\ 1 \\ 2 \\1 \end{bmatrix}$, $\begin{bmatrix} 3 \\ 3 \\ 6 \\ 3 \end{bmatrix}$, $\begin{bmatrix}3 \\ -1 \\ 3 \\ -2\end{bmatrix}$, and $\begin{bmatrix} 7 \\ -1 \\ 8 \\ -3 \end{bmatrix}$  span $\IR^4$.
\end{problem}

\begin{problem}{V4}
Determine if $\left\{ \begin{bmatrix} x \\ y \\ 0 \\ z \end{bmatrix}\  \bigg|\ x,y,z \in \IR\right\}$  a subspace of $\IR^4$.
\end{problem}

\begin{problem}{S1}
Determine if the set of vectors $\left\{ \begin{bmatrix} -3 \\ -8 \\ 0 \end{bmatrix}, \begin{bmatrix} 1 \\ 2 \\ 2 \end{bmatrix}, \begin{bmatrix} 0 \\ -1 \\ 3 \end{bmatrix} \right\}$ is  linearly dependent or linearly independent
\end{problem}

\begin{problem}{S2}
Determine if the set $\left\{ x^2+x-1, 3x^2-x+1, 2x^2-2 \right\}$ is a basis of $\P^2$.
\end{problem}

\begin{problem}{S3}
Let \(
  W={\rm span}\left\{
    \begin{bmatrix} 2 & 0 \\ -2 & 0 \end{bmatrix},
    \begin{bmatrix} 3 & 1 \\ 3 & 6 \end{bmatrix},
    \begin{bmatrix} 0 & 0 \\ 1 & 1 \end{bmatrix},
    \begin{bmatrix} 1 & 2 \\ 0 & 1 \end{bmatrix}
  \right\}
\). Find a basis for this vector space.
\end{problem}

\begin{problem}{S4}
Let $W={\rm span}\left( \left\{ \begin{bmatrix} 1 \\ -1 \\ 3 \\ -3 \end{bmatrix},\begin{bmatrix} 2 \\ 0 \\ 1 \\ 1 \end{bmatrix}, \begin{bmatrix} 3 \\ -1 \\ 4 \\ -2 \end{bmatrix},  \begin{bmatrix} 1 \\ 1 \\ 1 \\ -7 \end{bmatrix} \right\}\right)$.  Compute the dimension of $W$.
\end{problem}

\begin{problem}{A1}
Let $T: \IR^3 \rightarrow \IR$ be the linear transformation given by $$T\left(\begin{bmatrix} x_1 \\ x_2 \\ x_3  \end{bmatrix} \right) = \begin{bmatrix} x_3+3x_1 \end{bmatrix}.$$ Write the matrix for $T$ with respect to the standard bases of $\IR^3$ and $\IR$.
\end{problem}

\begin{problem}{A2}
 Determine if $T: \IR^2 \rightarrow \IR^2$ given by $T\left(\begin{bmatrix} x \\ y \end{bmatrix}\right) = \begin{bmatrix} e^{x} \\ e^y \end{bmatrix}$ is a linear transformation.
\end{problem}

\begin{problem}{A3}
Determine if each of the following linear transformations is injective (one-to-one) and/or surjective (onto).
\begin{enumerate}[(a)]
\item
  \(S: \IR^2 \rightarrow \IR^3\) where
  \(S(\vec e_1)=\begin{bmatrix}
    2 \\
    1 \\
    0
  \end{bmatrix}\) and
  \(S(\vec e_2)=\begin{bmatrix}
    1 \\
    2 \\
    1
  \end{bmatrix}\).
\item
  \(T: \IR^3 \rightarrow \IR^2\) where
  \(T(\vec e_1)=\begin{bmatrix}
    2 \\
    2
  \end{bmatrix}\),
  \(T(\vec e_2)=\begin{bmatrix}
   1  \\
   0
  \end{bmatrix}\), and
  \(T(\vec e_3)=\begin{bmatrix}
    1 \\
    1
  \end{bmatrix}\).
\end{enumerate}
\end{problem}

\begin{problem}{A4}
Let $T: \IR^3\rightarrow \IR^3$ be the linear transformation given by $$T\left(\begin{bmatrix} x \\ y \\ z \\  \end{bmatrix} \right) = \begin{bmatrix} -3x+y \\ -8x+2y-z \\ 2y+3z \end{bmatrix}$$
Compute a basis for the kernel and a basis for the image of $T$.
\end{problem}

\begin{problem}{M1}
Let
\begin{align*}
A &= \begin{bmatrix} 2 & 3 \\ 0 & 1 \end{bmatrix} & B&= \begin{bmatrix} 3 & 1 & 0 \end{bmatrix} & C&= \begin{bmatrix} 3 & -1 & 4 \\ 1 & 0 & 2 \end{bmatrix}
\end{align*}

Exactly one of the six products $AB$, $AC$, $BA$, $BC$, $CA$, $CB$ can be computed.  Determine which one, and compute it.
\end{problem}

\begin{problem}{M2} Determine if the matrix $\begin{bmatrix} 1 & 3 & -1 \\ 2 & 7 & 0 \\ -1 & -1 & 5 \end{bmatrix}$ is invertible.
\end{problem}

\begin{problem}{M3}
  Find the inverse of the matrix
  \(\begin{bmatrix}
    2 & -1 & -3  \\
    -14 & 9 & 24  \\
    3 & -2 & -5
  \end{bmatrix}\).
\end{problem}

\begin{problem}{G1}
Compute the determinant of the matrix $\begin{bmatrix} 2 & 3 & 0 & 1 \\ -1 & 3 & 1 & 4 \\ 0 & 2 & 0 & 3 \\ 1 & -1 & 3 & 5 \end{bmatrix}$.
\end{problem}

\begin{problem}{G2}
Compute the eigenvalues, along with their algebraic multiplicities, of the matrix $ \begin{bmatrix} 8 & -3 & 2 \\ 23 & -9 & 5 \\ -7 & 2 & -3 \end{bmatrix}$.
\end{problem}

\begin{problem}{G3}
Find the eigenspace associated to the eigenvalue $-1$ in the matrix $A=\begin{bmatrix}9 & -3 & 2 \\ 19 & -6 & 5 \\ -11 & 4 & -2 \end{bmatrix}$
\end{problem}

\begin{problem}{G4}
Compute the geometric multiplicity of the eigenvalue $3$ in the matrix $A=\begin{bmatrix}1 & -2 & -1 & 0 \\ -4 & -1 & -2 & 0 \\ 14 & 12 & 11 & 2 \\ -14 & -10 & -9 & -1\end{bmatrix}$.
\end{problem}

\end{document}
