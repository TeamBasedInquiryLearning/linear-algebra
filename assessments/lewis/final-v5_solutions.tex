\documentclass{sbgLAexam}

\begin{extract*}
\usepackage{amsmath,amssymb,amsthm,enumerate}
\coursetitle{Math 237}
\courselabel{Linear Algebra}
\calculatorpolicy{You may use a calculator, but you must show all relevant work to receive credit for a standard.}

\newcommand{\IR}{\mathbb{R}}

\makeatletter
\renewcommand*\env@matrix[1][*\c@MaxMatrixCols c]{%
  \hskip -\arraycolsep
  \let\@ifnextchar\new@ifnextchar
  \array{#1}}
\makeatother


\title{Final Exam}
\standard{E1,E2,E3,E4,V1,V2,V3,V4,S1,S2,S3,S4,A1,A2,A3,A4,M1,M2,M3,G1,G2,G3,G4}
\version{5}
\end{extract*}

\begin{document}

\begin{problem}{E1}
Write an augmented matrix corresponding to the following system of linear equations.
\begin{align*}
x_1+3x_2-4x_3 +x_4 &= 5 \\
3x_1+9x_2+x_3-7x_4 &= 0 \\
x_1-x_3 +x_4 &= 1
\end{align*}
\end{problem}
\begin{solution}
\[
\begin{bmatrix}[cccc|c]
1 & 3 & -4 & 1 & 5 \\
3 & 9 & 1 & -7 & 0 \\
1 & 0 & -1 & 1 &  1
\end{bmatrix}
\]
\end{solution}

\begin{problem}{E2}
Put the following matrix in reduced row echelon form.
$$\begin{bmatrix}[ccc|c] -3 & 1 & 0 & 2 \\ -8 & 2 & -1 & 6 \\ 0 & 2 & 3 & -2 \end{bmatrix}$$
\end{problem}
\begin{solution}
$$\begin{bmatrix}[ccc|c]
-3 & 1 & 0 & 2 \\
 -8 & 2 & -1 & 6 \\
 0 & 2 & 3 & -2
\end{bmatrix} \sim
\begin{bmatrix}[ccc|c]
1 & -\frac{1}{3} & 0 & -\frac{2}{3} \\
 -8 & 2 & -1 & 6 \\
 0 & 2 & 3 & -2
\end{bmatrix} \sim
\begin{bmatrix}[ccc|c]
1 & -\frac{1}{3} & 0 & -\frac{2}{3} \\
 0 & -\frac{2}{3} & -1 & \frac{2}{3} \\
 0 & 2 & 3 & -2
\end{bmatrix} $$
$$\sim
\begin{bmatrix}[ccc|c]
1 & -\frac{1}{3} & 0 & -\frac{2}{3} \\
 0 & 1 & \frac{3}{2} & -1 \\
 0 & 2 & 3 & -2
\end{bmatrix} \sim
\begin{bmatrix}[ccc|c]
1 & 0 & \frac{1}{2} & -1 \\
 0 & 1 & \frac{3}{2} & -1 \\
 0 & 0 & 0 & 0
\end{bmatrix}$$
\end{solution}

\begin{extract}\newpage\end{extract}
\begin{problem}{E3}
Find the solution set for the following system of linear equations.
\begin{align*}
2x_1+3x_2-5x_3+14x_4 &= 8 \\
x_1+x_2-x_3+5x_4&= 3
\end{align*}
\end{problem}
\begin{solution}
Let \(A =
  \begin{bmatrix}[cccc|c]
    2 & 3 & -5 & 14 & 8 \\
    1 & 1 & -1 & 5 & 3
  \end{bmatrix}
\), so \(\RREF A =
  \begin{bmatrix}[cccc|c]
    1 & 0 & 2 & 1 & 1 \\
    0 & 1 & -3 & 4 & 2 \\
  \end{bmatrix}
\). It follows that the solution set is given by \(
  \begin{bmatrix}
    1 - 2a - b \\
    2 + 3a - 4b \\
    a \\
    b
  \end{bmatrix}
\) for all real numbers \(a,b\).
\end{solution}

\begin{problem}{E4}
Find a basis for the solution set to the homogeneous system of equations
given by
\begin{align*}
3x+2y+z &= 0 \\
x+y+z &= 0
\end{align*}
\end{problem}
\begin{solution}
Let \(A =
  \begin{bmatrix}[ccc|c]
    3 & 2 & 1 & 0 \\
    1 & 1 & 1 & 0
  \end{bmatrix}
\), so \(\RREF A =
  \begin{bmatrix}[ccc|c]
    1 & 0 & -1 & 0 \\
    0 & 1 & 2 & 0
  \end{bmatrix}
\).
It follows that the basis for the solution set is given by \(\left\{
  \begin{bmatrix}
    1 \\
    -2 \\
    1
  \end{bmatrix}
\right\}\).
\end{solution}

\begin{extract}\newpage\end{extract}
\begin{problem}{V1}
Let $V$ be the set of all pairs of real numbers with the operations, for any $(x_1,y_1), (x_2,y_2) \in V$, $c \in \IR$,
\begin{align*}
(x_1,y_1) \oplus (x_2,y_2) &= (x_1+x_2,y_1+y_2) \\
c \odot (x_1,y_1) &= (c^2x_1, c^3y_1)
\end{align*}
\begin{enumerate}[(a)]
\item Show that scalar multiplication \textbf{distributes scalars} over
      vector addition:
      \(c\odot((x_1,y_1) \oplus (x_2,y_2))=
      c\odot(x_1,y_1) \oplus c\odot(x_2,y_2)\).
\item Determine if $V$ is a vector space or not.  Justify your answer.
\end{enumerate}
\end{problem}
\begin{solution}
Let $(x_1,y_1), (x_2,y_2) \in V$ and let $c \in \IR$.
\begin{align*}
c \odot \left( (x_1,y_1) \oplus (x_2,y_2) \right) &= c \odot (x_1+x_2,y_1+y_2) \\
&= (c^2(x_1+x_2), c^3(y_1+y_2) ) \\
&= (c^2x_1,c^3y_1) \oplus (c^2x_2,c^3y_2) \\
&= c \odot (x_1,y_1) \oplus c \odot (x_2,y_2)
\end{align*}
However, $V$ is not a vector space, as the other distributive law fails:
$$(c+d) \odot (x_1,y_1) = ( (c+d)^2 x_1, (c+d)^3 y_1) \neq ((c^2+d^2)x_1, (c^3+d^3)y_1) = c \odot (x_1,y_1) \oplus d \odot (x_1,y_1).$$
\end{solution}


\begin{problem}{V2}
  Determine if
  \(\begin{bmatrix} 3 \\ -2 \\ 4 \end{bmatrix}\)
  belongs to the span of the set
  \(\left\{
    \begin{bmatrix} 1 \\ 2 \\ -3 \end{bmatrix},
    \begin{bmatrix} 2 \\ 4 \\ -6 \end{bmatrix},
    \begin{bmatrix} 0 \\ 0 \\ 0 \end{bmatrix}
    \right\}
  \).
\end{problem}
\begin{solution}
  Since
  \[
    \RREF\left(
      \begin{bmatrix}[ccc|c]
        1 & 2 & 0 & 3 \\
        2 & 4 & 0 & -2 \\
        -3 & -6 & 0 & 4
      \end{bmatrix}
    \right) =
    \begin{bmatrix}[ccc|c]
      1 & 2 & 0 & 0 \\
      0 & 0 & 0 & 1 \\
      0 & 0 & 0 & 0
    \end{bmatrix}
  \]
  contains the contradiction \(0=1\),
  \(\begin{bmatrix} 3 \\ -2 \\ 4 \end{bmatrix}\) is
  not a linear combination of the three vectors.
\end{solution}
\begin{extract}\newpage\end{extract}
\begin{problem}{V3}
Determine if the vectors $\begin{bmatrix} 1 \\ 0 \\ 2 \\1 \end{bmatrix}$, $\begin{bmatrix} 3 \\ 1 \\ 0 \\ -3 \end{bmatrix}$,$\begin{bmatrix} 0 \\ 3 \\ 0 \\ -2 \end{bmatrix}$, and $\begin{bmatrix}-1 \\ 1 \\ -1 \\ -1 \end{bmatrix}$ span $\IR^4$.
\end{problem}
\begin{solution}
$$\RREF\left(\begin{bmatrix}1 & 3 & 0 & -1 \\ 0 & 1 & 3 & 1 \\ 2 & 0 & 0 & -1 \\ 1 & -3 & -2 & -1 \end{bmatrix} \right) =\begin{bmatrix} 1 & 0 & 0 & 0 \\ 0 & 1 & 0 & 0 \\ 0 & 0 & 1 & 0 \\ 0 & 0 & 0 & 1 \end{bmatrix}$$
Since every row contains a pivot, the vectors span $\IR^4$.
\end{solution}

\begin{problem}{V4}
Determine if the set of all lattice points, i.e. $\{(x,y)\ \big|\ \text{$x$ and $y$ are integers} \}$ is a subspace of $\IR^2$.
\end{problem}
\begin{solution}
This set is closed under addition, but not under scalar multiplication so it is not a subspace.
\end{solution}

\begin{extract}\newpage\end{extract}
\begin{problem}{S1}
Determine if the set of vectors $\left\{\begin{bmatrix} 3 \\ -1 \\ 0 \\ 4 \end{bmatrix}, \begin{bmatrix} 1  \\ 2 \\ -2 \\ 1 \end{bmatrix}, \begin{bmatrix} 3 \\ -8 \\ 6 \\ 5 \end{bmatrix} \right\}$  is linearly dependent or linearly independent.
\end{problem}
\begin{solution}
$$\RREF\left(\begin{bmatrix} 3 & 1 & 3 \\ -1 & 2 & -8 \\ 0 & -2 & 6 \\ 4 & 1 & 5 \end{bmatrix} \right) = \begin{bmatrix} 1 & 0 & 2 \\ 0 & 1 & -3 \\ 0 & 0 & 0 \\ 0 & 0 & 0 \end{bmatrix}$$
Since the reduced row echelon form has a nonpivot column, the vectors are linearly dependent.
\end{solution}

\begin{problem}{S2}
  Determine if the set \(\left\{
    \begin{bmatrix} 3 \\ -1 \\ 2 \end{bmatrix},
    \begin{bmatrix} 2 \\ 0 \\ 2 \end{bmatrix},
    \begin{bmatrix} 1 \\ 4 \\ -1 \end{bmatrix}
  \right\}\) is a basis of $\IR^3$.
\end{problem}
\begin{solution}
  \[\RREF\left(
    \begin{bmatrix}
      3 & 2 & 1 \\
      -1 & 0 & 4 \\
      2 & 2 & -1
    \end{bmatrix} \right)= \begin{bmatrix}
      1 & 0 &0 \\
      0 & 1 & 0 \\
      0 & 0 & 1
    \end{bmatrix}
  \]
Since the resulting matrix is the identity matrix, it is a basis.
\end{solution}


\begin{extract}\newpage\end{extract}
\begin{problem}{S3}
Let \(
  W={\rm span}\left\{
    \begin{bmatrix} 2 & 0 \\ -2 & 0 \end{bmatrix},
    \begin{bmatrix} 3 & 1 \\ 3 & 6 \end{bmatrix},
    \begin{bmatrix} 0 & 0 \\ 1 & 1 \end{bmatrix},
    \begin{bmatrix} 1 & 2 \\ 0 & 1 \end{bmatrix}
  \right\}
\). Find a basis for this vector space.
\end{problem}
\begin{solution}
\[
  \RREF\left(\begin{bmatrix}
    2 & 3 & 0 & 1 \\
    0 & 1 & 0 & 2 \\
    -2 & 3 & 1 & 0 \\
    0 & 6 & 1 & 1
  \end{bmatrix} \right) =
  \begin{bmatrix}
    1 & 0 & 0 & -\frac{5}{2} \\
    0 & 1 & 0 & 2 \\
    0 & 0 & 1 & -11\\
    0 & 0 & 0 & 0
  \end{bmatrix}
\]
Thus \(\left\{
  \begin{bmatrix} 2 & 0 \\ -2 & 0 \end{bmatrix},
  \begin{bmatrix} 3 & 1 \\ 3 & 6 \end{bmatrix},
  \begin{bmatrix} 0 & 0 \\ 1 & 1 \end{bmatrix}
\right\}\) is a basis of $W$.
\end{solution}


\begin{problem}{S4}
  Let \(
    W={\rm span}\left\{ 2x^2-x+3, 2x^2+2, -x^2+4x+1 \right\}\).
  Find the dimension of \(W\).
\end{problem}
\begin{solution}
  \[\RREF\left(
    \begin{bmatrix}
      2 & 2 & -1 \\
      -1 & 0 & 4 \\
      3 & 2 & 1
    \end{bmatrix} \right)= \begin{bmatrix}
      1 & 0 &0 \\
      0 & 1 & 0 \\
      0 & 0 & 1
    \end{bmatrix}
  \]
  Since it has three pivot columns, its dimension is \(3\).
\end{solution}
\begin{extract}\newpage\end{extract}
\begin{problem}{A1}
Let $T: \IR^3 \rightarrow \IR$ be the linear transformation given by $$T\left(\begin{bmatrix} x_1 \\ x_2 \\ x_3  \end{bmatrix} \right) = \begin{bmatrix} x_2+3x_3 \end{bmatrix}.$$ Write the matrix for $T$ with respect to the standard bases of $\IR^3$ and $\IR$.
\end{problem}
\begin{solution}
$$\begin{bmatrix} 0 & 1 & 3 \end{bmatrix}$$
\end{solution}


\begin{problem}{A2}
Determine if the map $T: \P^3 \rightarrow \P^4$ given by $T(f(x))=xf(x)-f(x)$ is a linear transformation or not.
\end{problem}
\begin{extract}\newpage\end{extract}
\begin{problem}{A3}
Determine if the following linear maps are injective (one-to-one) and/or surjective (onto).
\begin{enumerate}[(a)]
\item $S: \IR^2 \rightarrow \IR^3$ given by $S\left(\begin{bmatrix} x \\ y  \end{bmatrix} \right) = \begin{bmatrix} 3x+2y \\ x-y \\ x+4y \end{bmatrix} $
\item $T: \IR^3 \rightarrow \IR^3$ given by $T\left(\begin{bmatrix} x \\ y \\ z  \end{bmatrix} \right) = \begin{bmatrix} x+y+z \\ 2y+3z \\ x-y-2z \end{bmatrix} $
\end{enumerate}
\end{problem}

\begin{solution}
\begin{enumerate}[(a)]
\item $$\RREF\left( \begin{bmatrix} 1 &  1 & 1 \\ 0  & 2 & 3 \\ 1  & -1 & -2 \end{bmatrix} \right) = \begin{bmatrix} 1 &  0 & -\frac{1}{2} \\ 0  & 1 & \frac{3}{2} \\ 0& 0 & 0  \end{bmatrix}$$
Since there is a nonpivot column, $T$ is not injective.  Since there is a zero row, $T$ is not surjective.
\item $$\RREF \left( \begin{bmatrix} 3 & 2 \\ 1 & -1 \\ 1 & 4 \end{bmatrix} \right) = \begin{bmatrix} 1 & 0 \\ 0 & 1 \\ 0 & 0 \end{bmatrix}$$
Since all columns are pivot columns, $S$ is injective.  Since there is a zero row, $S$ is not surjective.
\end{enumerate}
\end{solution}



\begin{problem}{A4}
Let $T: \IR^3 \rightarrow \IR^3$ be the linear map given by \(
  T\left(\begin{bmatrix} x \\ y \\ z \end{bmatrix} \right) =
  \begin{bmatrix}
    8x-3y-z \\
    y+3z \\
    -7x+3y+2z
  \end{bmatrix}
\). Compute a basis for the kernel and a basis for the image of $T$.
\end{problem}
\begin{solution}
\[
  \RREF \left( \begin{bmatrix}
    8 & -3 & -1 \\
    0 & 1 & 3 \\
    -7 & 3 & 2
  \end{bmatrix} \right) = \begin{bmatrix}
    1 & 0 & 1 \\
    0 & 1 & 3 \\
    0 & 0 & 0
  \end{bmatrix}
\]

Thus \(\left\{
  \begin{bmatrix} 8 \\ 0 \\ -7 \end{bmatrix},
  \begin{bmatrix} -3 \\ 1 \\ 3 \end{bmatrix}
\right\} \) is a basis for the image, and \(\left\{
  \begin{bmatrix} -1 \\ -3 \\ 1 \end{bmatrix}
\right\} \) is a basis for the kernel.
\end{solution}


\begin{extract}\newpage\end{extract}
\begin{problem}{M1}
Let 
\begin{align*}
A &= \begin{bmatrix} 1 & 3 & -1  \\ 0 & 0 & 7  \end{bmatrix} & B &= \begin{bmatrix} 0 & 1 & 7 & 7 \\ -1 & -2 & 0 & 4 \\ 0 & 0 & 1 & 5 \end{bmatrix} & C&=\begin{bmatrix} 3  \\  1 \end{bmatrix}
\end{align*}
Exactly one of the six products $AB$, $AC$, $BA$, $BC$, $CA$, $CB$ can be computed.  Determine which one, and compute it.
\end{problem}
\begin{solution}
$AB$ is the only ones that can be computed, and 
$$AB = \begin{bmatrix} -3 & -5 & 6 & 14 \\ 0 & 0 & 7 & 35 \end{bmatrix}$$
\end{solution}


\begin{problem}{M2}
Determine if the matrix $\begin{bmatrix} 3 & -1 & 0 & 4 \\ 2 & 1 & 1 & 1 \\ 0 & 1 & 1 & -1 \\ 1 & -2 & 0 & 3 \end{bmatrix}$ is invertible.
\end{problem}
\begin{solution}
$$\RREF \begin{bmatrix} 3 & -1 & 0 & 4 \\ 2 & 1 & 1 & 1 \\ 0 & 1 & 1 & -1 \\ 1 & -2 & 0 & 3 \end{bmatrix} = \begin{bmatrix} 1 & 0 & 0 & 1 \\ 0 & 1 & 0 & -1 \\ 0 & 0 & 1 & 0 \\ 0 & 0 & 0 & 0 \end{bmatrix}$$
This matrix is not row equivalent to the identity matrix, so it is not invertible.
\end{solution}


\begin{extract}\newpage\end{extract}
\begin{problem}{M3}
Find the inverse of the matrix $\begin{bmatrix} 3 & -1 & 0  \\ 2 & 1 & 1  \\ 0 & 1 & 1   \end{bmatrix}$.
\end{problem}
\begin{solution}
$$\begin{bmatrix} 3 & -1 & 0  \\ 2 & 1 & 1  \\ 0 & 1 & 1   \end{bmatrix}^{-1} = \begin{bmatrix} 0 & \frac{1}{2} & -\frac{1}{2} \\ -1 & \frac{3}{2} & -\frac{3}{2} \\ 1 & -\frac{3}{2} & \frac{5}{2}\end{bmatrix}$$
\end{solution}



\begin{problem}{G1}
Compute the determinant of the matrix
\[
  \begin{bmatrix}
    1 & 3 & 2 & 4 \\
    5 & 0 & -4 & 0 \\
    -2 & 3 & -1 & 1 \\
    0 & 1 & 0 & 1
  \end{bmatrix}
.\]
\end{problem}
\begin{solution}
\(15\).
\end{solution}

\begin{extract}\newpage\end{extract}
\begin{problem}{G2} 
Compute the eigenvalues, along with their algebraic multiplicities, of the matrix $ \begin{bmatrix} 9 & -3 & 2 \\ 19 & -6 & 5 \\ -5 & 2 & 0 \end{bmatrix}$.
\end{problem}
\begin{solution}
1 with algebraic multiplicity 3
\end{solution}

\begin{problem}{G3}
Find the eigenspace associated to the eigenvalue $1$ in the matrix $A=\begin{bmatrix} 8 & -3 & -1 \\ 21 & -8 & -3 \\ -7 & 3  & 2 \end{bmatrix}$
\end{problem}
\begin{solution}
The eigenspace is spanned by $\begin{bmatrix} \frac{3}{7} \\ 1 \\ 0 \end{bmatrix}$ and $\begin{bmatrix}\frac{1}{7} \\ 0 \\ 1 \end{bmatrix}$.
\end{solution}

\begin{extract}\newpage\end{extract}
\begin{problem}{G4}
Compute the geometric multiplicity of the eigenvalue $-1$ in the matrix $A=\begin{bmatrix}9 & -3 & 2 \\ 19 & -6 & 5 \\ -11 & 4 & -2 \end{bmatrix}$
\end{problem}
\begin{solution}
The eigenspace is spanned by $\begin{bmatrix} -\frac{5}{7} \\ - \frac{12}{7} \\ 1 \end{bmatrix}$, so the geometric multiplicity is $1$.
\end{solution}

\end{document}