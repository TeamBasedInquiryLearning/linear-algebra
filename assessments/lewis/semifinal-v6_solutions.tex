\documentclass{sbgLAsemi}

\begin{extract*}
\usepackage{amsmath,amssymb,amsthm,enumerate}
\coursetitle{Math 237}
\courselabel{Linear Algebra}
\calculatorpolicy{You may use a calculator, but you must show all relevant work to receive credit for a standard.}

\newcommand{\IR}{\mathbb{R}}

\makeatletter
\renewcommand*\env@matrix[1][*\c@MaxMatrixCols c]{%
  \hskip -\arraycolsep
  \let\@ifnextchar\new@ifnextchar
  \array{#1}}
\makeatother


\title{Semifinal}
\version{6}
\end{extract*}

\begin{document}

\begin{problem}{E1}
Write a system of linear equations corresponding to the following
augmented matrix.
\[
\begin{bmatrix}[ccc|c]
-4 & -1 & 3 & 2  \\
1 & 2 & -1 & 0 \\
-1 & 4 & 1 & 4
\end{bmatrix}
\]
\end{problem}
\begin{solution}
\begin{align*}
-4x_1-x_2+3x_3&=2 \\
x_1+2x_2-x_3 &= 0 \\
-x_1+4x_2+x_3&=4
\end{align*}
\end{solution}

\begin{problem}{E2}
Put the following matrix in reduced row echelon form.
$$\begin{bmatrix}[ccc|c] -3 & 1 & 0 & 2 \\ -8 & 2 & -1 & 6 \\ 0 & 2 & 3 & -2 \end{bmatrix}$$
\end{problem}
\begin{solution}
$$\begin{bmatrix}[ccc|c]
-3 & 1 & 0 & 2 \\
 -8 & 2 & -1 & 6 \\
 0 & 2 & 3 & -2
\end{bmatrix} \sim
\begin{bmatrix}[ccc|c]
1 & -\frac{1}{3} & 0 & -\frac{2}{3} \\
 -8 & 2 & -1 & 6 \\
 0 & 2 & 3 & -2
\end{bmatrix} \sim
\begin{bmatrix}[ccc|c]
1 & -\frac{1}{3} & 0 & -\frac{2}{3} \\
 0 & -\frac{2}{3} & -1 & \frac{2}{3} \\
 0 & 2 & 3 & -2
\end{bmatrix} $$
$$\sim
\begin{bmatrix}[ccc|c]
1 & -\frac{1}{3} & 0 & -\frac{2}{3} \\
 0 & 1 & \frac{3}{2} & -1 \\
 0 & 2 & 3 & -2
\end{bmatrix} \sim
\begin{bmatrix}[ccc|c]
1 & 0 & \frac{1}{2} & -1 \\
 0 & 1 & \frac{3}{2} & -1 \\
 0 & 0 & 0 & 0
\end{bmatrix}$$
\end{solution}

\begin{problem}{E3}
Solve the system of linear equations.
\begin{align*}
2x+y-z+w &=5 \\
3x-y-2w &= 0 \\
-x+5z+3w&=-1
\end{align*}
\end{problem}
\begin{solution}
$$\RREF\left( \begin{bmatrix}[cccc|c] 2 & 1 & -1 & 0 & 5 \\ 3 & -1 & 0 & -2 & 0 \\ -1 & 0 & 5 & 0 & -1 \end{bmatrix} \right) = \begin{bmatrix}[cccc|c] 1 & 0 & 0 & -\frac{1}{12} & 1 \\ 0 & 1 & 0 & \frac{7}{4} & 3 \\ 0 & 0 & 1 & \frac{7}{12} & 0 \end{bmatrix}$$
So the solutions are $$\left\{ \begin{bmatrix} 1+a \\ 3-21a \\ -7a \\ 12a \end{bmatrix}\ \big|\ a \in \IR\right\}$$
\end{solution}

\begin{problem}{E4}
Find a basis for the solution set of the system of equations
\begin{align*}
x+3y+3z+7w &= 0 \\
 x+3y-z-w &= 0 \\
  2x+6y+3z+8w &= 0 \\
   x+3y-2z-3w &= 0
\end{align*}
\end{problem}
\begin{solution}
$$\RREF\left(\begin{bmatrix} 1 & 3 & 3 & 7 \\ 1 & 3 & -1 & -1 \\ 2 & 6 & 3 & 8 \\ 1 & 3 & -2 & -3 \end{bmatrix}\right) = \begin{bmatrix} 1 & 3 & 0 & 1 \\ 0 & 0 & 1 & 2 \\ 0 & 0 & 0 & 0 \\ 0 & 0 & 0 & 0 \end{bmatrix}$$
Then the solution set is
$$\left\{ \begin{bmatrix} -3a-b \\ a \\ -2b \\ b \end{bmatrix}\ \big|\ a,b \in \IR\right\}$$
So a basis for the solution set is $$ \left\{ \begin{bmatrix} 3 \\ -1 \\ 0 \\ 0 \end{bmatrix} , \begin{bmatrix} 1\\0 \\ 2 \\ -1 \end{bmatrix} \right\} $$
\end{solution}


\begin{problem}{V1}
Let $V$ be the  set of all real numbers together with the operations $\oplus$ and $\odot$ defined by, for any $x,y \in V$ and $c \in \IR$,
\begin{align*}
x\oplus y  &= x+y-3 \\
c \odot x &= cx-3(c-1)
\end{align*}
\begin{enumerate}[(a)]
\item Show that \textbf{scalar multiplication} is
      \textbf{associative}: \(a\odot(b\odot x)=(ab)\odot x\).
\item Determine if $V$ is a vector space or not.  Justify your answer
\end{enumerate}
\end{problem}

\begin{solution}
Let $x,y \in V$, $c,d \in \IR$.
To show associativity:
\begin{align*}
c\odot \left( d \odot x\right) &= c\odot \left( dx-3(d-1) \right) \\
&= c\left(dx-3(d-1)\right)-3(c-1) \\
&= cdx-3(cd-1) \\
&= (cd) \odot x
\end{align*}

We verify the remaining 7 properties to see that $V$ is a vector space.
\begin{enumerate}[1)]
\item Real addition is associative, so $\oplus$ is associative.
\item $x\oplus 3 = x+3-3=x$, so $3$ is the additive identity.
\item $x \oplus (6-x) = x+(6-x)-3=3$, so $6-x$ is the additive inverse of $x$.
\item Real addition is commutative, so $\oplus$ is commutative.
\item Associativity shown above
\item $1 \odot x = x-3(1-1)=x$
\item \begin{align*} c \odot (x \oplus y) &=
c \odot (x+y-3) \\
&= c(x+y-3)-3(c-1) \\
&= cx-3(c-1) + cy-3(c-1) -3 \\
&= (c\odot x ) \oplus (c\odot y)
\end{align*}
\item \begin{align*} (c+d) \odot x &= (c+d)x-3(c+d-1) \\
&= cx-3(c-1)+dx-3(c-1)-3 \\
&= (c\odot x ) \oplus (d \odot x)
\end{align*}
\end{enumerate}

Therefore $V$ is a vector space.
\end{solution}


\begin{problem}{V2}
Determine if $\begin{bmatrix}0 \\ -1 \\ 2 \\ 6 \end{bmatrix}$ can be written as a linear combination of the vectors $\begin{bmatrix} 3 \\ -1 \\ -1 \\ 0 \end{bmatrix}$ and $\begin{bmatrix} -1 \\ 0 \\ 1 \\ 2 \end{bmatrix}$.
\end{problem}
\begin{solution}
$$\RREF\left(\left[\begin{array}{cc|c} 3 & -1 & 0 \\ -1 & 0 & -1 \\ -1 & 1 & 2 \\ 0 & 2 & 6\end{array} \right] \right)= \left[ \begin{array}{cc|c} 1 & 0 & 1 \\ 0 & 1 & 3 \\ 0 & 0 & 0 \\ 0 & 0 & 0 \end{array} \right]$$

Since this system has a solution, $\begin{bmatrix}0 \\ -1 \\ 2 \\ 6 \end{bmatrix}$ can be written as a linear combination of the vectors $\begin{bmatrix} 3 \\ -1 \\ -1 \\ 0 \end{bmatrix}$ and $\begin{bmatrix} -1 \\ 0 \\ 1 \\ 2 \end{bmatrix}$, namely
$$\begin{bmatrix}0 \\ -1 \\ 2 \\ 6 \end{bmatrix}=\begin{bmatrix} 3 \\ -1 \\ -1 \\ 0 \end{bmatrix}+3\begin{bmatrix} -1 \\ 0 \\ 1 \\ 2 \end{bmatrix}.$$
\end{solution}

\begin{problem}{V3}
Determine if the vectors $\begin{bmatrix} 1 \\ 1 \\ 2 \\1 \end{bmatrix}$, $\begin{bmatrix} 3 \\ 3 \\ 6 \\ 3 \end{bmatrix}$, $\begin{bmatrix}3 \\ -1 \\ 3 \\ -2\end{bmatrix}$, and $\begin{bmatrix} 7 \\ -1 \\ 8 \\ -3 \end{bmatrix}$  span $\IR^4$.
\end{problem}
\begin{solution}
$$\RREF\left(\begin{bmatrix} 1 & 3 & 3 & 7 \\ 1 & 3 & -1 & -1 \\ 2 & 6 & 3 & 8 \\ 1 & 3 & -2 & -3 \end{bmatrix}\right) = \begin{bmatrix} 1 & 3 & 0 & 1 \\ 0 & 0 & 1 & 2 \\ 0 & 0 & 0 & 0 \\ 0 & 0 & 0 & 0  \end{bmatrix}$$
Since there are zero rows, they do not span.  Alternatively, by inspection $\begin{bmatrix} 3 \\ 3 \\ 6 \\ 3 \end{bmatrix}=3\begin{bmatrix} 1 \\ 1 \\ 2 \\1 \end{bmatrix}$, so the set is linearly dependent, so it spans a subspace of dimension at most 3, therefore it does not span $\IR^4$.
\end{solution}

\begin{problem}{V4} Let $W$ be the set of all polynomials of even degree.  Determine if $W$ is a subspace of the vector space of all polynomials.
\end{problem}
\begin{solution}
$W$ is closed under scalar multiplication, but not under addition.  For example, $x-x^2$ and $x^2$ are both in $W$, but $(x-x^2)+(x^2)=x \notin W$.
\end{solution}


\begin{problem}{S1}
Determine if the set of vectors $\left\{ \begin{bmatrix} -3 \\ 8 \\ 0 \end{bmatrix}, \begin{bmatrix} 1 \\ 2 \\ 2 \end{bmatrix}, \begin{bmatrix} 0 \\ -1 \\ 3 \end{bmatrix} \right\}$ is  linearly dependent or linearly independent
\end{problem}
\begin{solution}
$$\RREF\left( \begin{bmatrix}-3 & 1 & 0 \\ 8 & 2 & -1 \\ 0 & 2 & 3 \end{bmatrix}\right) = \begin{bmatrix} 1 & 0 & 0 \\ 0 & 1 & 0 \\ 0 & 0 & 1 \end{bmatrix}$$
Every column is a pivot column, therefore the set is linearly independent.
\end{solution}

\begin{problem}{S2}
Determine if the set $\left\{\begin{bmatrix} 1 \\ 1 \\ -1 \end{bmatrix}, \begin{bmatrix} 3 \\ -1 \\ 1 \end{bmatrix},\begin{bmatrix} 2 \\ 0 \\ -2 \end{bmatrix}\right\}$ is a basis of $\IR^3$.
\end{problem}
\begin{solution}
$$\RREF\left(\begin{bmatrix} 1 & 3 & 2 \\ 1 & -1 & 0 \\ -1 & 1 & -2 \end{bmatrix} \right)= \begin{bmatrix} 1 & 0 &0 \\ 0 & 1 & 0 \\ 0 & 0 & 1\end{bmatrix}$$
Since the resulting matrix is the identity matrix, it is a basis.
\end{solution}


\begin{problem}{S3}
Let $W={\rm span}\left( \left\{ \begin{bmatrix} 1 \\ -1 \\ 3 \\ -3 \end{bmatrix},\begin{bmatrix} 2 \\ 0 \\ 1 \\ 1 \end{bmatrix}, \begin{bmatrix} 3 \\ -1 \\ 4 \\ -2 \end{bmatrix},  \begin{bmatrix} 1 \\ 1 \\ 1 \\ -7 \end{bmatrix} \right\}\right)$.  Find a basis of $W$.
\end{problem}
\begin{solution}
$$ \RREF \left( \begin{bmatrix} 1 & 2 & 3 & 1 \\ -1 & 0 & -1 & 1 \\ 3 & 1 & 4 & 1 \\ -3 & 1 & -2 & -7 \end{bmatrix} \right) =  \begin{bmatrix} 1 & 0 & 1 & 0 \\ 0 & 1 & 1 & 0 \\ 0 & 0 & 0 & 1 \\ 0 & 0 & 0 & 0\end{bmatrix}$$
Then  $\left\{ \begin{bmatrix} 1 \\ -1 \\ 3 \\ -3 \end{bmatrix},\begin{bmatrix} 2 \\ 0 \\ 1 \\ 1 \end{bmatrix},   \begin{bmatrix} 1 \\ 1 \\ 1 \\ -7 \end{bmatrix} \right\}$ is a basis for $W$.
\end{solution}


\begin{problem}{S4}
Let $W={\rm span}\left( \left\{ \begin{bmatrix} 1 \\ -1 \\ 3 \\ -3 \end{bmatrix},\begin{bmatrix} 2 \\ 0 \\ 1 \\ 1 \end{bmatrix}, \begin{bmatrix} 3 \\ -1 \\ 4 \\ -2 \end{bmatrix},  \begin{bmatrix} 1 \\ 1 \\ 1 \\ -7 \end{bmatrix} \right\}\right)$.  Compute the dimension of $W$.
\end{problem}
\begin{solution}
$$ \RREF \left( \begin{bmatrix} 1 & 2 & 3 & 1 \\ -1 & 0 & -1 & 1 \\ 3 & 1 & 4 & 1 \\ -3 & 1 & -2 & -7 \end{bmatrix} \right) =  \begin{bmatrix} 1 & 0 & 1 & 0 \\ 0 & 1 & 1 & 0 \\ 0 & 0 & 0 & 1 \\ 0 & 0 & 0 & 0\end{bmatrix}$$
This has 3 pivot columns so $\dim(W)=3$.
\end{solution}


\begin{problem}{A1}
Let $T: \IR^3\rightarrow \IR^4$ be the linear transformation given by $$T\left(\begin{bmatrix} x \\ y \\ z \\  \end{bmatrix} \right) = \begin{bmatrix} -3x+y \\ -8x+2y-z \\ 2y+3z \\ 0 \end{bmatrix}.$$  Write the matrix for $T$ with respect to the standard bases of $\IR^3$ and $\IR^4$.
\end{problem}
\begin{solution}
$$\begin{bmatrix} 3 & 1 & 0 \\ -8 & 2 & -1 \\ 0 & 2 & 3 \\ 0 & 0 & 0 \end{bmatrix}$$
\end{solution}

\begin{problem}{A2}
Determine if $D: \IR^{2\times 2} \rightarrow \IR$ given by $D\left(\begin{bmatrix} a & b \\ c & d \end{bmatrix} \right) = a-3c$ is a linear transformation or not.
\end{problem}

\begin{problem}{A3}
Determine if each of the following linear transformations is injective (one-to-one) and/or surjective (onto).
\begin{enumerate}[(a)]
\item $S: \IR^2 \rightarrow \IR^2$ given by the standard matrix $\begin{bmatrix} 0 & 1 \\ -1 & 0 \end{bmatrix}$.
\item $T: \IR^4 \rightarrow \IR^3$ given by the standard matrix $\begin{bmatrix} 2 & 3 & -1 & -2 \\ 0 & 1 & 3 & 1 \\ 2 & 1 & -7 & -4 \end{bmatrix}$
\end{enumerate}
\end{problem}
\begin{solution}
\begin{enumerate}[(a)]
\item $ \RREF\begin{bmatrix} 0 & 1 \\ -1 & 0 \end{bmatrix}=\begin{bmatrix}1 & 0 \\ 0 & 1 \end{bmatrix}$.  Since each column is a pivot column, $S$ is injective.  Since there is no zero row, $S$ is surjective.
\item Since $\dim \IR^4 > \dim \IR^3$, $T$ is not injective.
$$\RREF\left(\begin{bmatrix} 2 & 3 & -1 & -2 \\ 0 & 1 & 3 & 1 \\ 2 & 1 & -7 & -4 \end{bmatrix}\right) = \begin{bmatrix} 1 & 0 & -5 & -\frac{5}{2} \\ 0 & 1 & 3 & 1 \\ 0 & 0 & 0 & 0\end{bmatrix}$$
Since there are only two pivot columns, $T$ is not surjective.
\end{enumerate}
\end{solution}


\begin{problem}{A4}
Let $T: \IR^{2\times 3} \rightarrow \IR^3$ be the linear map given by \(
  T\left(\begin{bmatrix} a & b & c \\ x & y & z \end{bmatrix} \right) =
  \begin{bmatrix}
    a+x \\ b+y \\ c+z
  \end{bmatrix}
\). Compute a basis for the kernel and a basis for the image of $T$.
\end{problem}
\begin{solution}
Rewrite as \(
  T'\left(\begin{bmatrix} a \\ b \\ c \\ x \\ y \\ z \end{bmatrix} \right) =
  \begin{bmatrix}
    a+x \\ b+y \\ c+z
  \end{bmatrix}
\).
\[
  \RREF \left( \begin{bmatrix}
    1 & 0 & 0 & 1 & 0 & 0\\
    0 & 1 & 0 & 0 & 1 & 0\\
    0 & 0 & 1 & 0 & 0 & 1
  \end{bmatrix} \right) = \begin{bmatrix}
    1 & 0 & 0 & 1 & 0 & 0\\
    0 & 1 & 0 & 0 & 1 & 0\\
    0 & 0 & 1 & 0 & 0 & 1
  \end{bmatrix}
\]

Thus \(\left\{
  \begin{bmatrix} 1 \\ 0 \\ 0 \end{bmatrix},
  \begin{bmatrix} 0 \\ 1 \\ 0 \end{bmatrix},
  \begin{bmatrix} 0 \\ 0 \\ 1 \end{bmatrix}
\right\} \) is a basis for the image, and \(\left\{
  \begin{bmatrix} -1 & 0 & 0 \\ 1 & 0 & 0 \end{bmatrix},
  \begin{bmatrix} 0 & -1 & 0 \\ 0 & 1 & 0 \end{bmatrix},
  \begin{bmatrix} 0 & 0 & -1 \\ 0 & 0 & 1 \end{bmatrix}
\right\} \) is a basis for the kernel.
\end{solution}


\begin{problem}{M1}
Let 
\begin{align*}
A &= \begin{bmatrix} 2 & 3 \\ 0 & 1 \end{bmatrix} & B&= \begin{bmatrix} 3 & 1 & 0 \end{bmatrix} & C&= \begin{bmatrix} 3 & -1 & 4 \\ 1 & 0 & 2 \end{bmatrix}
\end{align*}

Exactly one of the six products $AB$, $AC$, $BA$, $BC$, $CA$, $CB$ can be computed.  Determine which one, and compute it.
\end{problem}
\begin{solution}
$AC$ is the only one that can be computed, and 
$$AC = \begin{bmatrix} 9 & -2 & 14 \\ 1 & 0 & 2 \end{bmatrix}$$
\end{solution}


\begin{problem}{M2}
Determine if the matrix $\begin{bmatrix} 3 & -1 & 0  \\ 2 & 1 & 1  \\ 0 & 1 & 1   \end{bmatrix}$ is invertible.
\end{problem}
\begin{solution}
It is row equivalent to the identity matrix, so it is invertible.
\end{solution}


\begin{problem}{M3}
  Find the inverse of the matrix
  \(\begin{bmatrix}
    2 & -1 & -3  \\
    -14 & 9 & 24  \\
    3 & -2 & -5
  \end{bmatrix}\).
\end{problem}
\begin{solution}
  \(\begin{bmatrix}[ccc|ccc]
    2 & -1 & -3 & 1 & 0 & 0 \\
    -14 & 9 & 24 & 0 & 1 & 0 \\
    3 & -2 & -5 & 0 & 0 & 1
  \end{bmatrix}\sim\begin{bmatrix}[ccc|ccc]
    1 & 0 & 0 & 3 & 1 & 3  \\
    0 & 1 & 0 & 2 & -1 & -6  \\
    0 & 0 & 1 & 1 & 1 & 4
  \end{bmatrix}\). Thus the inverse is
  \(\begin{bmatrix}
    3 & 1 & 3  \\
    2 & -1 & -6  \\
    1 & 1 & 4
  \end{bmatrix}\).
\end{solution}
\begin{problem}{G1}
Compute the determinant of the matrix $\begin{bmatrix} 2 & 3 & 0 & 1 \\ -1 & 3 & 1 & 4 \\ 0 & 2 & 0 & 3 \\ 1 & -1 & 3 & 5 \end{bmatrix}$.
\end{problem}
\begin{solution}
$-60$.
\end{solution}

\begin{problem}{G2} 
Compute the eigenvalues, along with their algebraic multiplicities, of the matrix $ \begin{bmatrix} 9 & -3 & 2 \\ 19 & -6 & 5 \\ -5 & 2 & 0 \end{bmatrix}$.
\end{problem}
\begin{solution}
1 with algebraic multiplicity 3
\end{solution}

\begin{problem}{G3}
Compute the eigenspace of the eigenvalue $-1$ in the matrix $\begin{bmatrix} 4 & -2 & -1 \\ 15 & -7 & -3 \\ -5 & 2 & 0 \end{bmatrix}$. 
\end{problem}
\begin{solution}
$$\RREF\left(A+I\right) = \begin{bmatrix} 1 & - \frac{2}{5} & -\frac{1}{5} \\ 0 & 0 & 0 \\ 0 & 0 & 0 \end{bmatrix}$$
So the eigenspace is spanned by $\begin{bmatrix} 2 \\5 \\  0 \end{bmatrix}$ and $\begin{bmatrix} 1 \\ 0 \\ 5 \end{bmatrix}$.
\end{solution}


\begin{problem}{G4}
Compute the geometric multiplicity of the eigenvalue $2$ in the matrix $\begin{bmatrix} -1 & 1 & 0 \\ -9 & 5 & 0 \\ 15 & -5 & 2 \end{bmatrix}$.
\end{problem}

\begin{solution}
The eigenspace is the solution space of the system $(B-2I)X=0$.
$$\RREF(B-2I)=\RREF\left(\begin{bmatrix} -3 & 1 & 0 \\ -9 & 3 & 0 \\ 15 & - 5 & 0 \end{bmatrix} \right) = \begin{bmatrix} 1 & -\frac{1}{3} & 0 \\ 0 & 0 & 0 \\ 0 & 0 & 0 \end{bmatrix}$$
Thus the geometric multiplicity is 2.
\end{solution}
\end{document}