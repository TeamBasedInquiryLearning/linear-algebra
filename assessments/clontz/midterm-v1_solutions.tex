\documentclass{sbgLAexam}

\begin{extract*}
\usepackage{amsmath,amssymb,amsthm,enumerate}
\coursetitle{Math 237}
\courselabel{Linear Algebra}
\calculatorpolicy{You may use a calculator, but you must show all relevant work to receive credit for a standard.}

\newcommand{\IR}{\mathbb{R}}

\makeatletter
\renewcommand*\env@matrix[1][*\c@MaxMatrixCols c]{%
  \hskip -\arraycolsep
  \let\@ifnextchar\new@ifnextchar
  \array{#1}}
\makeatother


\title{Midterm Exam}
\standard{E1,E2,E3,E4,V1,V2,V3,V4,S1,S2,S3,S4}
\version{1}
\end{extract*}

\begin{document}

\begin{problem}{E1}
Write a system of linear equations corresponding to the following
augmented matrix.
\[
\begin{bmatrix}[ccc|c]
-4 & -1 & 3 & 2  \\
1 & 2 & -1 & 0 \\
-1 & 4 & 1 & 4
\end{bmatrix}
\]
\end{problem}
\begin{solution}
\begin{align*}
-4x_1-x_2+3x_3&=2 \\
x_1+2x_2-x_3 &= 0 \\
-x_1+4x_2+x_3&=4
\end{align*}
\end{solution}

\begin{problem}{E2}
Put the following matrix in reduced row echelon form.
$$\begin{bmatrix}
 3 & -1 & 0 \\
 -1 & 0 & -1 \\
 -1 & 1 & 2 \\
 0 & 2 & 6
\end{bmatrix}$$
\end{problem}
\begin{solution}
$$\begin{bmatrix}
 3 & -1 & 0 \\
 -1 & 0 & -1 \\
 -1 & 1 & 2 \\
 0 & 2 & 6
\end{bmatrix}
\sim
\begin{bmatrix}
 -1 & 0 & -1 \\
 3 & -1 & 0 \\
 -1 & 1 & 2 \\
 0 & 2 & 6
\end{bmatrix}
\sim
\begin{bmatrix}
 1 & 0 & 1 \\
 3 & -1 & 0 \\
 -1 & 1 & 2 \\
 0 & 2 & 6
\end{bmatrix}
$$
$$
\sim
\begin{bmatrix}
 1 & 0 & 1 \\
 0 & -1 & -3 \\
 0 & 1 & 3 \\
 0 & 2 & 6
\end{bmatrix}
\sim
\begin{bmatrix}
 1 & 0 & 1 \\
 0 & 1 & 3 \\
 0 & -1 & -3 \\
 0 & 2 & 6
\end{bmatrix}
\sim
\begin{bmatrix}
 1 & 0 & 1 \\
 0 & 1 & 3 \\
 0 & 0 & 0 \\
0 & 0 & 0
\end{bmatrix}$$
\end{solution}

\begin{extract}\newpage\end{extract}
\begin{problem}{E3}
Solve the following linear system.
\begin{align*}
4x_1+4x_2+3x_3-6x_4 &= 5 \\
-2x_3-4x_4 &= 3 \\
2x_1+2x_2+x_3-4x_4 &= -1 \\
\end{align*}
\end{problem}
\begin{solution}
Let \(A =
  \begin{bmatrix}[cccc|c]
    4 & 4 & 3 & -6 & 5 \\
    0 & 0 & -2 & -4 & 3 \\
    2 & 2 & 1 & -4 & -1
  \end{bmatrix}
\), so \(\RREF A =
  \begin{bmatrix}[cccc|c]
    1 & 1 & 0 & -3 & 0 \\
    0 & 0 & 1 & 2 & 0 \\
    0 & 0 & 0 & 0 & 1
  \end{bmatrix}
\). It follows that the system is inconsistent with no solutions
(since the bottom row implies the contradiction \(0=1\)).
\end{solution}

\begin{problem}{E4}
Find a basis for the solution set to the homogeneous system of equations
given by
\begin{align*}
3x+2y+z &= 0 \\
x+y+z &= 0
\end{align*}
\end{problem}
\begin{solution}
Let \(A =
  \begin{bmatrix}[ccc|c]
    3 & 2 & 1 & 0 \\
    1 & 1 & 1 & 0
  \end{bmatrix}
\), so \(\RREF A =
  \begin{bmatrix}[ccc|c]
    1 & 0 & -1 & 0 \\
    0 & 1 & 2 & 0
  \end{bmatrix}
\).
It follows that the basis for the solution set is given by \(\left\{
  \begin{bmatrix}
    1 \\
    -2 \\
    1
  \end{bmatrix}
\right\}\).
\end{solution}

\begin{extract}\newpage\end{extract}
\begin{problem}{V1}
Let $V$ be the set of all polynomials with the operations, for any $f, g \in V$, $c \in \IR$,
\begin{align*}
f \oplus g &= f^\prime + g^\prime \\
c \odot f &= c f^\prime
\end{align*}
(here $f^\prime$ denotes the derivative of $f$).
\begin{enumerate}[(a)]
\item Show that scalar multiplication \textbf{distributes scalars} over
      vector addition:
      \(c\odot(f \oplus g)=
      c\odot f \oplus c\odot g\).
\item Determine if $V$ is a vector space or not.  Justify your answer.
\end{enumerate}
\end{problem}
\begin{solution}
Let $f,g \in \mathcal{P}$, and let $c \in \IR$.
$$c \odot (f \oplus g) = c \odot (f^\prime+g^\prime) =
c(f^\prime+g^\prime)^\prime = cf^{\prime\ \prime}+cg^{\prime\ \prime} =
cf^\prime\oplus cg^\prime= c \odot f \oplus c \odot g.$$
However, this is not a vector space, as there is no zero vector.  Additionally, $1 \odot f \neq f$ for any nonzero polynomial $f$.
\end{solution}


\begin{problem}{V2}
  Determine if
  \(\begin{bmatrix} 3 \\ -2 \\ 4 \end{bmatrix}\)
  belongs to the span of the set
  \(\left\{
    \begin{bmatrix} 1 \\ 2 \\ -3 \end{bmatrix},
    \begin{bmatrix} 2 \\ 4 \\ -6 \end{bmatrix},
    \begin{bmatrix} 0 \\ 0 \\ 0 \end{bmatrix}
    \right\}
  \).
\end{problem}
\begin{solution}
  Since
  \[
    \RREF\left(
      \begin{bmatrix}[ccc|c]
        1 & 2 & 0 & 3 \\
        2 & 4 & 0 & -2 \\
        -3 & -6 & 0 & 4
      \end{bmatrix}
    \right) =
    \begin{bmatrix}[ccc|c]
      1 & 2 & 0 & 0 \\
      0 & 0 & 0 & 1 \\
      0 & 0 & 0 & 0
    \end{bmatrix}
  \]
  contains the contradiction \(0=1\),
  \(\begin{bmatrix} 3 \\ -2 \\ 4 \end{bmatrix}\) is
  not a linear combination of the three vectors.
\end{solution}
\begin{extract}\newpage\end{extract}
\begin{problem}{V3}
Does
\(
  \operatorname{span}\left\{
    \begin{bmatrix} 2 \\ -1 \\ 4 \end{bmatrix},
    \begin{bmatrix} 3 \\ 12 \\ -9 \end{bmatrix},
    \begin{bmatrix} 1 \\ 4 \\ -3 \end{bmatrix},
    \begin{bmatrix} -4 \\ 2 \\ -8 \end{bmatrix}
  \right\} = \IR^3
\)?
\end{problem}
\begin{solution}
Since
\[
  \RREF\begin{bmatrix}
    2 & 3 & 1 & -4 \\
    -1 & 12 & 4 & 2 \\
    4 & -9 & -3 & -8
  \end{bmatrix} =
  \begin{bmatrix}
    1 & 0 & 0 & -2 \\
    0 & 1 & 1/3 & 0 \\
    0 & 0 & 0 & 0
  \end{bmatrix}
\]
has a zero row, the vectors fail to span \(\IR^3\).
\end{solution}

\begin{problem}{V4} Let $W$ be the set of all polynomials of even degree.  Determine if $W$ is a subspace of the vector space of all polynomials.
\end{problem}
\begin{solution}
$W$ is closed under scalar multiplication, but not under addition.  For example, $x-x^2$ and $x^2$ are both in $W$, but $(x-x^2)+(x^2)=x \notin W$.
\end{solution}


\begin{extract}\newpage\end{extract}
\begin{problem}{S1}
Determine if the set of polynomials $\left\{ -3x^3-8x^2, x^3+2x^2+2, -x^2+3\right\}$ is  linearly dependent or linearly independent
\end{problem}
\begin{solution}
$$\RREF\left( \begin{bmatrix}-3 & 1 & 0 \\ -8 & 2 & -1 \\ 0 & 0 & 0 \\ 0 & 2 & 3 \end{bmatrix}\right) = \begin{bmatrix} 1 & 0 & \frac{1}{2} \\ 0 & 1 & \frac{3}{2} \\ 0 & 0 & 0 \\ 0 & 0 & 0 \end{bmatrix}$$
This has a non pivot column, therefore the set is linearly dependent.
\end{solution}


\begin{problem}{S2}
  Determine if the set \(\left\{
    \begin{bmatrix} 3 \\ -1 \\ 2 \\3 \end{bmatrix},
    \begin{bmatrix} 2 \\ 0 \\ 2 \\ 4\end{bmatrix},
    \begin{bmatrix} 1 \\ -1 \\ 0 \\ -1\end{bmatrix},
    \begin{bmatrix} -1 \\ 3 \\ 0 \\ 5\end{bmatrix}
  \right\}\) is a basis of $\IR^4$.
\end{problem}
\begin{solution}
  \[\RREF\left(
    \begin{bmatrix}
      3 & 2 & 1 & -1\\
      -1 & 0 & -1 & 3\\
      2 & 2 & 0 & 0\\
      3 & 4 & -1 & 5\\
    \end{bmatrix} \right)= \begin{bmatrix}
      1 & 0 & 1 & 0 \\
      0 & 1 & -1 & 0 \\
      0 & 0 & 0 & 1 \\
      0 & 0 & 0 & 0
    \end{bmatrix}
  \]
Since the resulting matrix is not the identity matrix, it is not a basis.
\end{solution}


\begin{extract}\newpage\end{extract}
\begin{problem}{S3}
Let $W={\rm span}\left(\left\{\begin{bmatrix} 2 \\ 0 \\ -2 \\ 0 \end{bmatrix}, \begin{bmatrix} 3 \\ 1 \\ 3 \\ 6 \end{bmatrix}, \begin{bmatrix} 0 \\ 0 \\ 1 \\ 1 \end{bmatrix}, \begin{bmatrix}1 \\ 2 \\ 0 \\ 1 \end{bmatrix}\right\}\right)$.  Find a basis of $W$.
\end{problem}
\begin{solution}
$$\RREF\left( \begin{bmatrix} 2 & 3 & 0 & 1 \\ 0 & 1 & 0 & 2 \\ -2 & 3 & 1 & 0 \\ 0 & 6 & 1 & 1\end{bmatrix} \right) = \begin{bmatrix}1 & 0 & 0 & -\frac{5}{2} \\ 0 & 1 & 0 & 2 \\ 0 & 0 & 1 & -11\\ 0 & 0 & 0 & 0  \end{bmatrix} $$
Then $\left\{\begin{bmatrix} 2 \\ 0 \\ -2 \\ 0 \end{bmatrix}, \begin{bmatrix} 3 \\ 1 \\ 3 \\ 6 \end{bmatrix}, \begin{bmatrix} 0 \\ 0 \\ 1 \\ 1 \end{bmatrix}\right\}$ is a basis of $W$.
\end{solution}


\begin{problem}{S4}
Let $W = {\rm span} \left( \left\{  \begin{bmatrix} -3 \\ -8 \\ 0 \end{bmatrix}, \begin{bmatrix} 1 \\ 2 \\ 2 \end{bmatrix}, \begin{bmatrix} 0 \\ -1 \\ 3 \end{bmatrix} \right\} \right)$.  Compute the dimension of $W$.
\end{problem}
\begin{solution}
Let $A= \begin{bmatrix}-3 & 1 & 0 \\ -8 & 2 & -1 \\ 0 & 2 & 3\end{bmatrix}$, and compute $\RREF(A) = \begin{bmatrix} 1 & 0 & \frac{1}{2} \\ 0 & 1 & \frac{3}{2} \\ 0 & 0 & 0 \end{bmatrix}$.
Since there are two pivot columns, ${\rm dim}\ W = 2$.
\end{solution}


\end{document}