\documentclass{sbgLAsemi}

\begin{extract*}
\usepackage{amsmath,amssymb,amsthm,enumerate}
\coursetitle{Math 237}
\courselabel{Linear Algebra}
\calculatorpolicy{You may use a calculator, but you must show all relevant work to receive credit for a standard.}

\newcommand{\IR}{\mathbb{R}}

\makeatletter
\renewcommand*\env@matrix[1][*\c@MaxMatrixCols c]{%
  \hskip -\arraycolsep
  \let\@ifnextchar\new@ifnextchar
  \array{#1}}
\makeatother


\title{Semifinal}
\version{1}
\end{extract*}

\begin{document}

\begin{problem}{E1}
Write a system of linear equations corresponding to the following
augmented matrix.
\[
\begin{bmatrix}[cccc|c]
3 & -1 & 0 & 1 & 5 \\
-1 & 9 & 1 & -7 & 0 \\
1 & 0 & -1 & 0 & -3
\end{bmatrix}
\]
\end{problem}
\begin{solution}
\begin{align*}
3x_1-x_2+x_4 &= 5 \\
-x_1+9x_2+x_3-7x_4 &= 0 \\
x_1-x_3 &= -3
\end{align*}
\end{solution}

\begin{problem}{E2}
Find \(\RREF A\), where
\[
  A =
  \begin{bmatrix}[cccc|c]
    3 & -2 & 1 & 8 & -5 \\
    2 & 2 & 0 & 6 & -2 \\
    -1 & 1 & 1 & -4 & 6 \\
  \end{bmatrix}
\]
\end{problem}
\begin{solution}
\[
  \RREF A =
  \begin{bmatrix}[cccc|c]
    1 & 0 & 0 & 3 & -2 \\
    0 & 1 & 0 & 0 & 1 \\
    0 & 0 & 1 & -1 & 3
  \end{bmatrix}
\]
\end{solution}

\begin{problem}{E3}
Find the solution set for the following system of linear equations.
\begin{align*}
2x_1+3x_2-5x_3+14x_4 &= 8 \\
x_1+x_2-x_3+5x_4&= 3
\end{align*}
\end{problem}
\begin{solution}
Let \(A =
  \begin{bmatrix}[cccc|c]
    2 & 3 & -5 & 14 & 8 \\
    1 & 1 & -1 & 5 & 3
  \end{bmatrix}
\), so \(\RREF A =
  \begin{bmatrix}[cccc|c]
    1 & 0 & 2 & 1 & 1 \\
    0 & 1 & -3 & 4 & 2 \\
  \end{bmatrix}
\). It follows that the solution set is given by \(
  \begin{bmatrix}
    1 - 2a - b \\
    2 + 3a - 4b \\
    a \\
    b
  \end{bmatrix}
\) for all real numbers \(a,b\).
\end{solution}

\begin{problem}{E4}
Find a basis for the solution set to the system of equations
\begin{align*}
x+2y-3z &= 0 \\
2x+y-4z &= 0 \\
3y -2z & = 0 \\
x -y -z &= 0
\end{align*}
\end{problem}
\begin{solution}
$$\RREF \left(\begin{bmatrix} 1 & 2 & -3 \\ 2 & 1 & -4 \\ 0 & 3 & -2 \\ 1 & -1 & -1 \end{bmatrix} \right) = \begin{bmatrix} 1 & 0 & -\frac{5}{3} \\ 0 & 1 & -\frac{2}{3} \\ 0 & 0 & 0 \\ 0 & 0 & 0 \end{bmatrix}$$
Then the solution set is
$$\left\{ \begin{bmatrix} \frac{5}{3}a \\ \frac{2}{3}a \\ a \end{bmatrix} \bigg|\ a \in \IR \right\}$$
So a basis is $\left\{ \begin{bmatrix} \frac{5}{3} \\ \frac{2}{3} \\ 1 \end{bmatrix} \right\}$ or $\left\{ \begin{bmatrix} 5 \\  2 \\ 3 \end{bmatrix} \right\}$.
\end{solution}

\begin{problem}{V1}
Let $V$ be the  set of all real numbers together with the operations $\oplus$ and $\odot$ defined by, for any $x,y \in V$ and $c \in \IR$,
\begin{align*}
x\oplus y  &= x+y-3 \\
c \odot x &= cx-3(c-1)
\end{align*}
\begin{enumerate}[(a)]
\item Show that \textbf{scalar multiplication} is
      \textbf{associative}: \(a\odot(b\odot x)=(ab)\odot x\).
\item Determine if $V$ is a vector space or not.  Justify your answer
\end{enumerate}
\end{problem}

\begin{solution}
Let $x,y \in V$, $c,d \in \IR$.
To show associativity:
\begin{align*}
c\odot \left( d \odot x\right) &= c\odot \left( dx-3(d-1) \right) \\
&= c\left(dx-3(d-1)\right)-3(c-1) \\
&= cdx-3(cd-1) \\
&= (cd) \odot x
\end{align*}

We verify the remaining 7 properties to see that $V$ is a vector space.
\begin{enumerate}[1)]
\item Real addition is associative, so $\oplus$ is associative.
\item $x\oplus 3 = x+3-3=x$, so $3$ is the additive identity.
\item $x \oplus (6-x) = x+(6-x)-3=3$, so $6-x$ is the additive inverse of $x$.
\item Real addition is commutative, so $\oplus$ is commutative.
\item Associativity shown above
\item $1 \odot x = x-3(1-1)=x$
\item \begin{align*} c \odot (x \oplus y) &=
c \odot (x+y-3) \\
&= c(x+y-3)-3(c-1) \\
&= cx-3(c-1) + cy-3(c-1) -3 \\
&= (c\odot x ) \oplus (c\odot y)
\end{align*}
\item \begin{align*} (c+d) \odot x &= (c+d)x-3(c+d-1) \\
&= cx-3(c-1)+dx-3(c-1)-3 \\
&= (c\odot x ) \oplus (d \odot x)
\end{align*}
\end{enumerate}

Therefore $V$ is a vector space.
\end{solution}


\begin{problem}{V2}
Determine if $\begin{bmatrix}0 \\ -1 \\ 2 \\ 6 \end{bmatrix}$ can be written as a linear combination of the vectors $\begin{bmatrix} 3 \\ -1 \\ -1 \\ 0 \end{bmatrix}$ and $\begin{bmatrix} -1 \\ 0 \\ 1 \\ 2 \end{bmatrix}$.
\end{problem}
\begin{solution}
$$\RREF\left(\left[\begin{array}{cc|c} 3 & -1 & 0 \\ -1 & 0 & -1 \\ -1 & 1 & 2 \\ 0 & 2 & 6\end{array} \right] \right)= \left[ \begin{array}{cc|c} 1 & 0 & 1 \\ 0 & 1 & 3 \\ 0 & 0 & 0 \\ 0 & 0 & 0 \end{array} \right]$$

Since this system has a solution, $\begin{bmatrix}0 \\ -1 \\ 2 \\ 6 \end{bmatrix}$ can be written as a linear combination of the vectors $\begin{bmatrix} 3 \\ -1 \\ -1 \\ 0 \end{bmatrix}$ and $\begin{bmatrix} -1 \\ 0 \\ 1 \\ 2 \end{bmatrix}$, namely
$$\begin{bmatrix}0 \\ -1 \\ 2 \\ 6 \end{bmatrix}=\begin{bmatrix} 3 \\ -1 \\ -1 \\ 0 \end{bmatrix}+3\begin{bmatrix} -1 \\ 0 \\ 1 \\ 2 \end{bmatrix}.$$
\end{solution}

\begin{problem}{V3}
Does
\(
  \operatorname{span}\left\{
    \begin{bmatrix} 2 \\ -1 \\ 4 \\ 2 \\ 1 \end{bmatrix},
    \begin{bmatrix} -1 \\ 3 \\ 5 \\ 2 \\ 0 \end{bmatrix},
    \begin{bmatrix} 1 \\ 0 \\ 5 \\ 1 \\ -3 \end{bmatrix}
  \right\} = \IR^5
\)?
\end{problem}
\begin{solution}
Since there are only three vectors, they cannot span \(\IR^5\).
\end{solution}
\begin{problem}{V4} Let \(W\) be the set of all complex numbers \(a+bi\)
satisfying  \(a=2b\).
Determine if \(W\) is a subspace of \(\IC\).
\end{problem}
\begin{solution}
Yes, because \(c(2b_1+b_1i)+d(2b_2+b_2i)=2(cb_1+db_2)+(cb_1+db_2)i\) belongs to
\(W\). Alternately, yes because \(W\) is isomorphic to \(\IR\).
\end{solution}
\begin{problem}{S1}
Determine if the set of polynomials  $\left\{x^2+x, x^2+2x-1, x^2+3x-2\right\}$ is  linearly dependent or linearly independent
\end{problem}
\begin{solution}
$$\RREF\left( \begin{bmatrix} 1 &  1 & 1 \\ 1  & 2 & 3 \\ 0  & -1 & -2 \end{bmatrix} \right) = \begin{bmatrix} 1 &  0 & -1 \\ 0  & 1 & 2 \\ 0& 0 & 0  \end{bmatrix}$$
Since there is a nonpivot column, the set is linearly dependent.
\end{solution}


\begin{problem}{S2}
Determine if the set $\left\{
\begin{bmatrix} 1 & -3 \\ 2 & 2 \end{bmatrix},
\begin{bmatrix} -1 & 4 \\ -1 & 1 \end{bmatrix},
\begin{bmatrix} -1 & 0 \\ 2 & 1 \end{bmatrix},
\begin{bmatrix} 0 & 3 \\ 3 & 9 \end{bmatrix}
\right\}$ is a basis of $\IR^{2 \times 2}$ or not.
\end{problem}

\begin{solution}
$$\RREF \begin{bmatrix} 1 & -1 & -1 & 0 \\ -3 & 4 & 0 & 3 \\ 2 & -1 & 2 & 3 \\ 2 & 1 & 1 & 9 \end{bmatrix}=\begin{bmatrix} 1 &0 & 0 & 3 \\ 0 & 1 & 0 & 3 \\ 0 & 0 & 1 & 0 \\ 0 & 0 & 0 & 0 \end{bmatrix}$$
Since this is not the identity matrix, the set is not a basis.
\end{solution}

\begin{problem}{S3}
Let \(
  W={\rm span}\left\{
    \begin{bmatrix} 2 & 0 \\ -2 & 0 \end{bmatrix},
    \begin{bmatrix} 3 & 1 \\ 3 & 6 \end{bmatrix},
    \begin{bmatrix} 0 & 0 \\ 1 & 1 \end{bmatrix},
    \begin{bmatrix} 1 & 2 \\ 0 & 1 \end{bmatrix}
  \right\}
\). Find a basis for this vector space.
\end{problem}
\begin{solution}
\[
  \RREF\left(\begin{bmatrix}
    2 & 3 & 0 & 1 \\
    0 & 1 & 0 & 2 \\
    -2 & 3 & 1 & 0 \\
    0 & 6 & 1 & 1
  \end{bmatrix} \right) =
  \begin{bmatrix}
    1 & 0 & 0 & -\frac{5}{2} \\
    0 & 1 & 0 & 2 \\
    0 & 0 & 1 & -11\\
    0 & 0 & 0 & 0
  \end{bmatrix}
\]
Thus \(\left\{
  \begin{bmatrix} 2 & 0 \\ -2 & 0 \end{bmatrix},
  \begin{bmatrix} 3 & 1 \\ 3 & 6 \end{bmatrix},
  \begin{bmatrix} 0 & 0 \\ 1 & 1 \end{bmatrix}
\right\}\) is a basis of $W$.
\end{solution}


\begin{problem}{S4}
Let $W$ be the subspace of $\IR^{2\times2}$ given by $W={\rm span}\left(\left\{\begin{bmatrix} 2 & 0 \\ -2 & 0 \end{bmatrix}, \begin{bmatrix} 3 & 1 \\ 3 & 6 \end{bmatrix}, \begin{bmatrix} 0 & 0 \\ 1 & 1 \end{bmatrix}, \begin{bmatrix}1 & 2 \\ 0 & 1 \end{bmatrix}\right\}\right)$. Compute the dimension of $W$.
\end{problem}
\begin{solution}
$$\RREF\left( \begin{bmatrix} 2 & 3 & 0 & 1 \\ 0 & 1 & 0 & 2 \\ -2 & 3 & 1 & 0 \\ 0 & 6 & 1 & 1\end{bmatrix} \right) = \begin{bmatrix}1 & 0 & 0 & -\frac{5}{2} \\ 0 & 1 & 0 & 2 \\ 0 & 0 & 1 & -11\\ 0 & 0 & 0 & 0  \end{bmatrix} $$
This has 3 pivot columns so  $\dim(W) =3$.
\end{solution}


\begin{problem}{A1}
Let $T: \IR^4 \rightarrow \IR^2$ be the linear transformation given by $$T\left(\begin{bmatrix} x_1 \\ x_2 \\ x_3 \\ x_4 \end{bmatrix} \right) = \begin{bmatrix} x_1+3x_3 \\ 3x_2-5x_3 \end{bmatrix}.$$ Write the matrix for $T$ with respect to the standard bases of $\IR^4$ and $\IR^2$.
\end{problem}
\begin{solution}
$$\begin{bmatrix} 1 & 0 & 3 & 0 \\ 0 & 3 & -5 & 0 \end{bmatrix}$$
\end{solution}


\begin{problem}{A2}
Determine if the map $T: \P^3 \rightarrow \P^4$ given by $T(f(x))=xf(x)-f(x)$ is a linear transformation or not.
\end{problem}
\begin{problem}{A3}
Determine if each of the following linear transformations is injective (one-to-one) and/or surjective (onto).
\begin{enumerate}[(a)]
\item $S: \IR^2 \rightarrow \IR^4$ given by the standard matrix $\begin{bmatrix} 2 & 1 \\ 1 & 2 \\ 0 & 1 \\ 3 & -3 \end{bmatrix}$.
\item $T: \IR^4 \rightarrow \IR^3$ given by the standard matrix $\begin{bmatrix} 2 & 3 & -1 & 1 \\ -1 & 1 & 1 & 1 \\ 4 & 7 & -1 & 5 \end{bmatrix}$
\end{enumerate}
\end{problem}
\begin{solution}
\begin{enumerate}[(a)]
\item $ \begin{bmatrix} 2 & 1 \\ 1 & 2 \\ 0 & 1 \\ 3 & -3 \end{bmatrix}=\begin{bmatrix}1 & 0 \\ 0 & 1 \\ 0 & 0 \\ 0 & 0  \end{bmatrix}$.  Since each column is a pivot column, $S$ is injective.  Since there a no zero row, $S$ is not surjective.
\item Since $\dim \IR^4 > \dim \IR^3$, $T$ is not injective.
$$\RREF\left(\begin{bmatrix} 2 & 3 & -1 & 1 \\ -1 & 1 & 1 & 1 \\ 4 & 7 & -1 & 5 \end{bmatrix}\right) = \begin{bmatrix} 1 & 0  & 0 & 2 \\ 0 & 1 & 0 & 0  \\ 0 & 0 & 1 & 3 \end{bmatrix}$$
Since there is not a zero row, $T$ is surjective.
\end{enumerate}
\end{solution}

\begin{problem}{A4}
Let $T: \IR^4\rightarrow \IR^4$ be the linear transformation given by $$T\left(\begin{bmatrix} x \\ y \\ z \\ w \end{bmatrix} \right) = \begin{bmatrix} x+3y+3z+7w \\ x+3y-z-w \\ 2x+6y+3z+8w \\ x+3y-2z-3w \end{bmatrix}$$
Compute a basis for the kernel and a basis for the image of $T$.
\end{problem}
\begin{solution}

$$\RREF\left(\begin{bmatrix} 1 & 3 & 3 & 7 \\ 1 & 3 & -1 & -1 \\ 2 & 6 & 3 & 8 \\ 1 & 3 & -2 & -3 \end{bmatrix}\right) = \begin{bmatrix} 1 & 3 & 0 & 1 \\ 0 & 0 & 1 & 2 \\ 0 & 0 & 0 & 0 \end{bmatrix}$$

Then a basis for the kernel is
$$\left\{ \begin{bmatrix} -3 \\ 1 \\ 0 \\ 0 \end{bmatrix} , \begin{bmatrix} -1\\0 \\ -2 \\ 1 \end{bmatrix} \right\}$$
and a basis for the image is
$$\left\{ \begin{bmatrix} 1 \\ 1 \\ 2 \\ 1 \end{bmatrix} , \begin{bmatrix} 3 \\ -1 \\ 3 \\ -2 \end{bmatrix}\right\} $$
\end{solution}

\begin{problem}{M1}
Let 
\begin{align*}
A &= \begin{bmatrix} 3 \\ 5 \\ -1  \end{bmatrix} & B &= \begin{bmatrix} 2 & -1 \\ 0 & 4 \\ 3 & 1 \end{bmatrix} & C&=\begin{bmatrix} 1 & -1 & 3 & -3 \\ 2 & 1 & -1 & 2 \end{bmatrix}
\end{align*}
Exactly one of the six products $AB$, $AC$, $BA$, $BC$, $CA$, $CB$ can be computed.  Determine which one, and compute it.
\end{problem}
\begin{solution}
$BC$ is the only one that can be computed, and
$$BC=\begin{bmatrix} 0 & -3 & 7 & -8 \\ 8 & 4 & -4 & 8 \\ 5 & -2 & 8 & -7 \end{bmatrix}$$
\end{solution}

\begin{problem}{M2}
Determine if the matrix $\begin{bmatrix} 3 & -1 & 0 & 4 \\ 2 & 1 & 1 & 1 \\ 0 & 1 & 1 & -1 \\ 1 & -2 & 0 & 3 \end{bmatrix}$ is invertible.
\end{problem}
\begin{solution}
$$\RREF \begin{bmatrix} 3 & -1 & 0 & 4 \\ 2 & 1 & 1 & 1 \\ 0 & 1 & 1 & -1 \\ 1 & -2 & 0 & 3 \end{bmatrix} = \begin{bmatrix} 1 & 0 & 0 & 1 \\ 0 & 1 & 0 & -1 \\ 0 & 0 & 1 & 0 \\ 0 & 0 & 0 & 0 \end{bmatrix}$$
This matrix is not row equivalent to the identity matrix, so it is not invertible.
\end{solution}


\begin{problem}{M3}
Find the inverse of the matrix $\begin{bmatrix} 3 & -1 & 0  \\ 2 & 1 & 1  \\ 0 & 1 & 1   \end{bmatrix}$.
\end{problem}
\begin{solution}
$$\begin{bmatrix} 3 & -1 & 0  \\ 2 & 1 & 1  \\ 0 & 1 & 1   \end{bmatrix}^{-1} = \begin{bmatrix} 0 & \frac{1}{2} & -\frac{1}{2} \\ -1 & \frac{3}{2} & -\frac{3}{2} \\ 1 & -\frac{3}{2} & \frac{5}{2}\end{bmatrix}$$
\end{solution}



\begin{problem}{G1}
Compute the determinant of the matrix $\begin{bmatrix} 3 & -1 & 0  & 7 \\ 2 & 1 & 1 & -1  \\ 0 & 1 & 1 & 3 \\ 0 & 0 & 0 & 1   \end{bmatrix}$.
\end{problem}
\begin{solution}
$2$
\end{solution}

\begin{problem}{G2} 
Compute the eigenvalues, along with their algebraic multiplicities, of the matrix $ \begin{bmatrix}3 & -3 & 2 \\108 & -9 & 5 \\ 10 & -7 & 3 \end{bmatrix}$.
\end{problem}
\begin{solution}
The eigenvalues are $1$ with multiplicity 1 and $-2$, with algebraic multiplicity 2.
\end{solution}
\begin{problem}{G3}
Compute the eigenspace associated to the eigenvalue $2$ in the matrix $\begin{bmatrix} -1 & 1 & 0 \\ -9 & 5 & 0 \\ 15 & -5 & 2 \end{bmatrix}$.
\end{problem}

\begin{solution}
The eigenspace is the solution space of the system $(B-2I)X=0$.
$$\RREF(B-2I)=\RREF\left(\begin{bmatrix} -3 & 1 & 0 \\ -9 & 3 & 0 \\ 15 & - 5 & 0 \end{bmatrix} \right) = \begin{bmatrix} 1 & -\frac{1}{3} & 0 \\ 0 & 0 & 0 \\ 0 & 0 & 0 \end{bmatrix}$$
So the system simplifies to $x-\frac{y}{3}=0$, or $3x=y$.  Thus the eigenspace is $$E_2 = {\rm span}\left( \left\{ \begin{bmatrix} 1 \\ 3 \\ 0 \end{bmatrix}, \begin{bmatrix} 0 \\ 0 \\ 1\end{bmatrix} \right\} \right)$$
\end{solution}
\begin{problem}{G4}
Compute the geometric multiplicity of the eigenvalue $2$ in the matrix $\begin{bmatrix} -1 & 1 & 0 \\ -9 & 5 & 0 \\ 15 & -5 & 2 \end{bmatrix}$.
\end{problem}

\begin{solution}
The eigenspace is the solution space of the system $(B-2I)X=0$.
$$\RREF(B-2I)=\RREF\left(\begin{bmatrix} -3 & 1 & 0 \\ -9 & 3 & 0 \\ 15 & - 5 & 0 \end{bmatrix} \right) = \begin{bmatrix} 1 & -\frac{1}{3} & 0 \\ 0 & 0 & 0 \\ 0 & 0 & 0 \end{bmatrix}$$
Thus the geometric multiplicity is 2.
\end{solution}
\end{document}