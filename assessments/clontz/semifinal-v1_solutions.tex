\documentclass{sbgLAsemi}

\begin{extract*}
\usepackage{amsmath,amssymb,amsthm,enumerate}
\coursetitle{Math 237}
\courselabel{Linear Algebra}
\calculatorpolicy{You may use a calculator, but you must show all relevant work to receive credit for a standard.}

\newcommand{\IR}{\mathbb{R}}

\makeatletter
\renewcommand*\env@matrix[1][*\c@MaxMatrixCols c]{%
  \hskip -\arraycolsep
  \let\@ifnextchar\new@ifnextchar
  \array{#1}}
\makeatother


\title{Semifinal}
\version{1}
\end{extract*}

\begin{document}

\begin{problem}{E1}
Write a system of linear equations corresponding to the following
augmented matrix.
\[
\begin{bmatrix}[cccc|c]
3 & -1 & 0 & 1 & 5 \\
-1 & 9 & 1 & -7 & 0 \\
1 & 0 & -1 & 0 & -3
\end{bmatrix}
\]
\end{problem}
\begin{solution}
\begin{align*}
3x_1-x_2+x_4 &= 5 \\
-x_1+9x_2+x_3-7x_4 &= 0 \\
x_1-x_3 &= -3
\end{align*}
\end{solution}

\begin{problem}{E2}
Put the following matrix in reduced row echelon form.
$$\begin{bmatrix}
 3 & -1 & 0 \\
 -1 & 0 & -1 \\
 -1 & 1 & 2 \\
 0 & 2 & 6
\end{bmatrix}$$
\end{problem}
\begin{solution}
$$\begin{bmatrix}
 3 & -1 & 0 \\
 -1 & 0 & -1 \\
 -1 & 1 & 2 \\
 0 & 2 & 6
\end{bmatrix}
\sim
\begin{bmatrix}
 -1 & 0 & -1 \\
 3 & -1 & 0 \\
 -1 & 1 & 2 \\
 0 & 2 & 6
\end{bmatrix}
\sim
\begin{bmatrix}
 1 & 0 & 1 \\
 3 & -1 & 0 \\
 -1 & 1 & 2 \\
 0 & 2 & 6
\end{bmatrix}
$$
$$
\sim
\begin{bmatrix}
 1 & 0 & 1 \\
 0 & -1 & -3 \\
 0 & 1 & 3 \\
 0 & 2 & 6
\end{bmatrix}
\sim
\begin{bmatrix}
 1 & 0 & 1 \\
 0 & 1 & 3 \\
 0 & -1 & -3 \\
 0 & 2 & 6
\end{bmatrix}
\sim
\begin{bmatrix}
 1 & 0 & 1 \\
 0 & 1 & 3 \\
 0 & 0 & 0 \\
0 & 0 & 0
\end{bmatrix}$$
\end{solution}

\begin{problem}{E3}
Solve the following linear system.
\begin{align*}
3x+2y+z &= 7 \\
x+y+z &= 1 \\
-2x+3z &= -11
\end{align*}
\end{problem}
\begin{solution}
Let \(A =
  \begin{bmatrix}[ccc|c]
    3 & 2 & 1 & 7 \\
    1 & 1 & 1 & 1 \\
    -2 & 0 & 3 & 11
  \end{bmatrix}
\), so \(\RREF A =
  \begin{bmatrix}[ccc|c]
    1 & 0 & 0 & 4 \\
    0 & 1 & 0 & -2 \\
    0 & 0 & 1 & -1
  \end{bmatrix}
\). It follows that the system has exactly one solution:
\(\begin{bmatrix}
  4 & -2 & -1
\end{bmatrix}\)
\end{solution}
\begin{problem}{E4}
Find a basis for the solution set to the homogeneous system of equations
\begin{align*}
4x_1+4x_2+3x_3-6x_4 &= 0 \\
-2x_3-4x_4 &= 0 \\
2x_1+2x_2+x_3-4x_4 &= 0 \\
\end{align*}
\end{problem}
\begin{solution}
Let \(A =
  \begin{bmatrix}[cccc|c]
    4 & 4 & 3 & -6 & 0 \\
    0 & 0 & -2 & -4 & 0 \\
    2 & 2 & 1 & -4 & 0
  \end{bmatrix}
\), so \(\RREF A =
  \begin{bmatrix}[cccc|c]
    1 & 1 & 0 & -3 & 0 \\
    0 & 0 & 1 & 2 & 0 \\
    0 & 0 & 0 & 0 & 0
  \end{bmatrix}
\).
It follows that the basis for the solution set is given by \(\left\{
  \begin{bmatrix}
    -1 \\
    1 \\
    0 \\
    0
  \end{bmatrix},
  \begin{bmatrix}
    3 \\
    0 \\
    -2 \\
    1
  \end{bmatrix}
\right\}\).
\end{solution}

\begin{problem}{V1}
Let $V$ be the set of all points on the parabola $y=x^2$ with the operations, for any $(x_1,y_1), (x_2,y_2) \in V$, $c \in \IR$,
\begin{align*}
(x_1,y_1) \oplus (x_2,y_2) &= (x_1+x_2,y_1+y_2+2x_1x_2) \\
c \odot (x_1,y_1) &= (cx_1, c^2y_1)
\end{align*}
\begin{enumerate}[(a)]
\item Show that the vector \textbf{addition} $\oplus$ is \textbf{associative}:
      \((x_1,y_1) \oplus ((x_2,y_2) \oplus (x_3,y_3))=((x_1,y_1)\oplus (x_2,y_2))\oplus (x_3,y_3)\).
\item Determine if $V$ is a vector space or not.  Justify your answer.
\end{enumerate}
\end{problem}

\begin{problem}{V2}
  Determine if
  \(\begin{bmatrix} 4 \\ -1 \\ 6 \\ -7 \end{bmatrix}\)
  belongs to the span of the set
  \(\left\{
    \begin{bmatrix} 2 \\ 0 \\ -1 \\ 5 \end{bmatrix},
    \begin{bmatrix} 4 \\ -1 \\ 4 \\ 3 \end{bmatrix}
    \right\}
  \).
\end{problem}
\begin{solution}
  Since
  \[
    \RREF\left(
      \begin{bmatrix}[cc|c]
        2 & 4 & 4 \\
        0 & -1 & -1 \\
        -1 & 4 & 6 \\
        5 & 3 & -7
      \end{bmatrix}
    \right) =
    \begin{bmatrix}[cc|c]
      1 & 0 & 0 \\
      0 & 1 & 0 \\
      0 & 0 & 1 \\
      0 & 0 & 0
    \end{bmatrix}
  \]
  contains the contradiction \(0=1\),
  \(\begin{bmatrix} 4 \\ -1 \\ 6 \\ -7 \end{bmatrix}\) is
  not a linear combination of the three vectors.
\end{solution}


\begin{problem}{V3}
Determine if the vectors  $\begin{bmatrix} 8 \\ 21 \\ -7 \end{bmatrix}$, $\begin{bmatrix} -3 \\ -8 \\ 3 \end{bmatrix}$, $\begin{bmatrix} -1 \\ -3 \\ 2 \end{bmatrix}$, and $\begin{bmatrix} 4 \\ 11 \\ -5 \end{bmatrix}$ span $\IR^3$.
\end{problem}
\begin{solution}
$$\RREF\left(\begin{bmatrix} 8 & -3 & -1 & 4 \\ 21 & -8 & -3 & 11 \\ -7 & 3 & 2 & -5  \end{bmatrix} \right) = \begin{bmatrix} 1 & 0 & 1 & -1 \\ 0 & 1 & 3 & -4 \\ 0 & 0 & 0 & 0\end{bmatrix} $$
Since the rank is less than 3, they do not span $\IR^3$.
\end{solution}

\begin{problem}{V4} Let $W$ be the set of all polynomials of even degree.  Determine if $W$ is a subspace of the vector space of all polynomials.
\end{problem}
\begin{solution}
$W$ is closed under scalar multiplication, but not under addition.  For example, $x-x^2$ and $x^2$ are both in $W$, but $(x-x^2)+(x^2)=x \notin W$.
\end{solution}


\begin{problem}{S1}
Determine if the set of vectors  $\left\{\begin{bmatrix} 1 \\ 0 \\ 1 \end{bmatrix}, \begin{bmatrix} 1 \\ 2 \\ -1 \end{bmatrix}, \begin{bmatrix} 1 \\ 3 \\ -2 \end{bmatrix}\right\}$ is  linearly dependent or linearly independent
\end{problem}
\begin{solution}
$$\RREF\left( \begin{bmatrix} 1 &  1 & 1 \\ 0  & 2 & 3 \\ 1  & -1 & -2 \end{bmatrix} \right) = \begin{bmatrix} 1 &  0 & -\frac{1}{2} \\ 0  & 1 & \frac{3}{2} \\ 0& 0 & 0  \end{bmatrix}$$
Since there is a nonpivot column, the set is linearly dependent.
\end{solution}

\begin{problem}{S2}
Determine if the set $\left\{ x^3-x, x^2+x+1, x^3-x^2+2, 2x^2-1 \right\}$ is a basis of $\P^3$.
\end{problem}
\begin{solution}
$$\RREF\left(\begin{bmatrix} 1 & 0 & 1 & 0 \\ 0 & 1 & -1 & 2 \\ -1 & 1 & 0 & 0 \\ 0 & 1 & 2 & -1 \end{bmatrix} \right)= \begin{bmatrix} 1 & 0 &0 & 1 \\ 0 & 1 & 0 & 1 \\ 0 & 0 & 1 & -1 \\ 0 & 0 & 0 & 0 \end{bmatrix}$$
Since the resulting matrix is not the identity matrix, it is not a basis.
\end{solution}


\begin{problem}{S3}
Let $W={\rm span}\left(\left\{\begin{bmatrix} 2 \\ 0 \\ -2 \\ 0 \end{bmatrix}, \begin{bmatrix} 3 \\ 1 \\ 3 \\ 6 \end{bmatrix}, \begin{bmatrix} 0 \\ 0 \\ 1 \\ 1 \end{bmatrix}, \begin{bmatrix}1 \\ 2 \\ 0 \\ 1 \end{bmatrix}\right\}\right)$.  Find a basis of $W$.
\end{problem}
\begin{solution}
$$\RREF\left( \begin{bmatrix} 2 & 3 & 0 & 1 \\ 0 & 1 & 0 & 2 \\ -2 & 3 & 1 & 0 \\ 0 & 6 & 1 & 1\end{bmatrix} \right) = \begin{bmatrix}1 & 0 & 0 & -\frac{5}{2} \\ 0 & 1 & 0 & 2 \\ 0 & 0 & 1 & -11\\ 0 & 0 & 0 & 0  \end{bmatrix} $$
Then $\left\{\begin{bmatrix} 2 \\ 0 \\ -2 \\ 0 \end{bmatrix}, \begin{bmatrix} 3 \\ 1 \\ 3 \\ 6 \end{bmatrix}, \begin{bmatrix} 0 \\ 0 \\ 1 \\ 1 \end{bmatrix}\right\}$ is a basis of $W$.
\end{solution}


\begin{problem}{S4}
Let $W$ be the subspace of $\P_3$ given by $W={\rm span}\left( \left\{ x^3-x^2+3x-3, 2x^3+x+1, 3x^3-x^2+4x-2, x^3+x^2+x-7\right\}\right)$.  Compute the dimension of $W$.
\end{problem}
\begin{solution}
$$ \RREF \left( \begin{bmatrix} 1 & 2 & 3 & 1 \\ -1 & 0 & -1 & 1 \\ 3 & 1 & 4 & 1 \\ -3 & 1 & -2 & -7 \end{bmatrix} \right) =  \begin{bmatrix} 1 & 0 & 1 & 0 \\ 0 & 1 & 1 & 0 \\ 0 & 0 & 0 & 1 \\ 0 & 0 & 0 & 0\end{bmatrix}$$
This has 3 pivot columns so $\dim(W)=3$.
\end{solution}


\begin{problem}{A1}
Let $T: \IR^3\rightarrow \IR^4$ be the linear transformation given by $$T\left(\begin{bmatrix} x \\ y \\ z \\  \end{bmatrix} \right) = \begin{bmatrix} -3x+y \\ -8x+2y-z \\ 7x+2y+3z \\ 0 \end{bmatrix}.$$  Write the matrix for $T$ with respect to the standard bases of $\IR^3$ and $\IR^4$.
\end{problem}
\begin{solution}
$$\begin{bmatrix} 3 & 1 & 0 \\ -8 & 2 & -1 \\ 7 & 2 & 3 \\ 0 & 0 & 0 \end{bmatrix}$$
\end{solution}

\begin{problem}{A2}
Determine if the map $T: \P^3 \rightarrow \P^4$ given by $T(f(x))=xf(x)-f(x)$ is a linear transformation or not.
\end{problem}
\begin{problem}{A3}
Determine if each of the following linear transformations is injective (one-to-one) and/or surjective (onto).
\begin{enumerate}[(a)]
\item $S: \IR^2 \rightarrow \IR^2$ given by the standard matrix $\begin{bmatrix} 0 & 1 \\ -1 & 0 \end{bmatrix}$.
\item $T: \IR^4 \rightarrow \IR^3$ given by the standard matrix $\begin{bmatrix} 2 & 3 & -1 & -2 \\ 0 & 1 & 3 & 1 \\ 2 & 1 & -7 & -4 \end{bmatrix}$
\end{enumerate}
\end{problem}
\begin{solution}
\begin{enumerate}[(a)]
\item $ \RREF\begin{bmatrix} 0 & 1 \\ -1 & 0 \end{bmatrix}=\begin{bmatrix}1 & 0 \\ 0 & 1 \end{bmatrix}$.  Since each column is a pivot column, $S$ is injective.  Since there is no zero row, $S$ is surjective.
\item Since $\dim \IR^4 > \dim \IR^3$, $T$ is not injective.
$$\RREF\left(\begin{bmatrix} 2 & 3 & -1 & -2 \\ 0 & 1 & 3 & 1 \\ 2 & 1 & -7 & -4 \end{bmatrix}\right) = \begin{bmatrix} 1 & 0 & -5 & -\frac{5}{2} \\ 0 & 1 & 3 & 1 \\ 0 & 0 & 0 & 0\end{bmatrix}$$
Since there are only two pivot columns, $T$ is not surjective.
\end{enumerate}
\end{solution}


\begin{problem}{A4}
Let $T: \IR^{2\times 2} \rightarrow \IR^3$ be the linear map given by \(
  T\left(\begin{bmatrix} a & b \\ x & y \end{bmatrix} \right) =
  \begin{bmatrix}
    a+x \\ 0 \\ b+y
  \end{bmatrix}
\). Compute a basis for the kernel and a basis for the image of $T$.
\end{problem}
\begin{solution}
Rewrite as \(
  T'\left(\begin{bmatrix} a \\ b \\ x \\ y \end{bmatrix} \right) =
  \begin{bmatrix}
    a+x \\ 0 \\ b+y
  \end{bmatrix}
\).
\[
  \RREF \left( \begin{bmatrix}
    1 & 0 & 1 & 0 \\
    0 & 0 & 0 & 0 \\
    0 & 1 & 0 & 1
  \end{bmatrix} \right) = \begin{bmatrix}
    1 & 0 & 1 & 0 \\
    0 & 0 & 0 & 0 \\
    0 & 1 & 0 & 1
  \end{bmatrix}
\]

Thus \(\left\{
  \begin{bmatrix} 1 \\ 0 \\ 0 \end{bmatrix},
  \begin{bmatrix} 0 \\ 0 \\ 1 \end{bmatrix}
\right\} \) is a basis for the image, and \(\left\{
  \begin{bmatrix} -1 & 0 \\ 1 & 0 \end{bmatrix},
  \begin{bmatrix} 0 & -1 \\ 0 & 1 \end{bmatrix}
\right\} \) is a basis for the kernel.
\end{solution}
\begin{problem}{M1}
Let 
\begin{align*}
A &= \begin{bmatrix} 1 & 3 & -1 & -1 \\ 0 & 0 & 7 & 2 \end{bmatrix} & B &= \begin{bmatrix} 0 & 1 & 7 & 7 \\ -1 & -2 & 0 & 4 \\ 0 & 0 & 1 & 5 \end{bmatrix} & C&=\begin{bmatrix} 3 & 2 \\ 0 & 1 \\ -2 & -1 \end{bmatrix}
\end{align*}
Exactly one of the six products $AB$, $AC$, $BA$, $BC$, $CA$, $CB$ can be computed.  Determine which one, and compute it.
\end{problem}
\begin{solution}
$CA$ is the only one that can be computed, and 
$$CA = \begin{bmatrix} 3 & 9 & 11 & 1 \\ 0 & 0 & 7 & 2 \\ -2 & - 6 & -5 & 0 \end{bmatrix}$$
\end{solution}

\begin{problem}{M2}
Determine if the matrix $\begin{bmatrix} 2 & 1 & 0 & 3 \\ 1 & -1 & 0 & 1 \\ 3 & 2 & -1 & 7 \\ 4 & 1 & 2 & 0 \end{bmatrix}$ is invertible.
\end{problem}
\begin{solution}
$$\RREF \begin{bmatrix} 2 & 1 & 0 & 3 \\ 1 & -1 & 0 & 1 \\ 3 & 2 & -1 & 7 \\ 4 & 1 & 2 & 0 \end{bmatrix}=\begin{bmatrix} 1 & 0 & 0 & 0 \\ 0 & 1 & 0 & 0 \\ 0 & 0 & 1 & 0 \\ 0 & 0 & 0 & 1 \end{bmatrix}$$
Since it is row equivalent to the identity matrix, it is  invertible.
\end{solution}
\begin{problem}{M3}
Compute the inverse of the matrix $\begin{bmatrix} 1 & 2 & 3 & 0 \\ 0 & -1 & 4 & -2 \\ 0 & 0 & 1 & 3 \\ 0 & 0 & 0 & 1 \end{bmatrix}$.
\end{problem}
\begin{solution}
$$\RREF(A|I) = \begin{bmatrix}[cccc|cccc] 1 & 0 & 0 &0 & 1 & 2 & -11 & 37 \\ 0 & 1 & 0 & 0 & 0 & -1 & 4 & -14 \\ 0 & 0 & 1 & 0 & 0 & 0 & 1 & -3 \\ 0 & 0 & 0 & 1 & 0 & 0 & 0 & 1 \end{bmatrix}$$
So the inverse is $\begin{bmatrix}  1 & 2 & -11 & 37 \\ 0 & -1 & 4 & -14 \\  0 & 0 & 1 & -3 \\ 0 & 0 & 0 & 1 \end{bmatrix}$
\end{solution}


\begin{problem}{G1}
Compute the determinant of the matrix $\begin{bmatrix} 2 & 3 & 0 & 1 \\ -1 & 3 & 1 & 4 \\ 0 & 2 & 0 & 3 \\ 1 & -1 & 3 & 5 \end{bmatrix}$.
\end{problem}
\begin{solution}
$-60$.
\end{solution}

\begin{problem}{G2} 
Compute the eigenvalues, along with their algebraic multiplicities, of the matrix $ \begin{bmatrix} 8 & -3 & -1 \\ 21 & -8 & -3 \\ -7 & 3 & 2\end{bmatrix}$.
\end{problem}
\begin{solution}
\begin{align*}
\det(A-\lambda I) &= (8-\lambda) \det \begin{bmatrix} -8-\lambda & -3 \\ 3 & 2-\lambda \end{bmatrix}-(-3) \det \begin{bmatrix} 21 & -3 \\ -7 & 2-\lambda \end{bmatrix} +(-1) \det \begin{bmatrix} 21 & -8-\lambda \\ -7 & 3 \end{bmatrix} \\
&=(8-\lambda)\left(\lambda ^2+6\lambda-7\right)+3\left(-21\lambda + 21 \right)-\left(-7\lambda +7 \right) \\
&=(\lambda -1) \left( (8-\lambda)(\lambda+7)-63+7 \right) \\
&= (\lambda-1)(\lambda -\lambda ^2) \\
&= -\lambda (\lambda-1)^2
\end{align*}
So the eigenvalues are $0$ (with algebraic multiplicity 1) and $1$ (with algebraic multiplicity 2).
\end{solution}


\begin{problem}{G3}
Find the eigenspace associated to the eigenvalue $-1$ in the matrix $A=\begin{bmatrix}9 & -3 & 2 \\ 19 & -6 & 5 \\ -11 & 4 & -2 \end{bmatrix}$
\end{problem}
\begin{solution}
The eigenspace is spanned by $\begin{bmatrix} -\frac{5}{7} \\ - \frac{12}{7} \\ 1 \end{bmatrix}$.
\end{solution}

\begin{problem}{G4}
Compute the geometric multiplicity of the eigenvalue $2$ in the matrix $A=\begin{bmatrix}0 & -2 & -1 & 0 \\ -4 & -2 & -2 & 0 \\ 14 & 12 & 10 & 2 \\ -13 & -10 & -8 & -1 \end{bmatrix}$.
\end{problem}
\begin{solution}
The eigenspace is spanned by $\begin{bmatrix} -1 \\ \frac{1}{2} \\ 1 \\ 0 \end{bmatrix}$ and  $\begin{bmatrix} -1 \\ 1 \\ 0 \\ 1 \end{bmatrix}$, so the geometric multiplicity is $2$.
\end{solution}

\end{document}