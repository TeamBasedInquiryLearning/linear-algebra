\documentclass{sbgLAsemi}

\begin{extract*}
\usepackage{amsmath,amssymb,amsthm,enumerate}
\coursetitle{Math 237}
\courselabel{Linear Algebra}
\calculatorpolicy{You may use a calculator, but you must show all relevant work to receive credit for a standard.}

\newcommand{\IR}{\mathbb{R}}

\makeatletter
\renewcommand*\env@matrix[1][*\c@MaxMatrixCols c]{%
  \hskip -\arraycolsep
  \let\@ifnextchar\new@ifnextchar
  \array{#1}}
\makeatother


\title{Semifinal}
\version{1}
\end{extract*}

\begin{document}

\begin{problem}{E1}
Write a system of linear equations corresponding to the following
augmented matrix.
\[
\begin{bmatrix}[ccc|c]
2 & -1 & 0 & 1  \\
-1 & 4 & 1 & -7  \\
1 & 2 & -1 & 0
\end{bmatrix}
\]
\end{problem}
\begin{solution}
\begin{align*}
2x_1-x_2&=1 \\
-x_1+4x_2+x_3&=-7 \\
x_1+2x_2-x_3 &= 0
\end{align*}
\end{solution}

\begin{problem}{E2}
Find \(\RREF A\), where
\[
  A =
  \begin{bmatrix}[ccc|c]
    2 & -1 & 5 & 4 \\
    -1 & 0 & -2 & -1 \\
    1 & 3 & -1 & -5
  \end{bmatrix}
\]
\end{problem}
\begin{solution}
\[
  \RREF A =
  \begin{bmatrix}[ccc|c]
    1 & 0 & 2 & 1 \\
    0 & 1 & -1 & -2 \\
    0 & 0 & 0 & 0
  \end{bmatrix}
\]
\end{solution}

\begin{problem}{E3}
Solve the following linear system.
\begin{align*}
4x_1+4x_2+3x_3-6x_4 &= 5 \\
-2x_3-4x_4 &= 3 \\
2x_1+2x_2+x_3-4x_4 &= -1 \\
\end{align*}
\end{problem}
\begin{solution}
Let \(A =
  \begin{bmatrix}[cccc|c]
    4 & 4 & 3 & -6 & 5 \\
    0 & 0 & -2 & -4 & 3 \\
    2 & 2 & 1 & -4 & -1
  \end{bmatrix}
\), so \(\RREF A =
  \begin{bmatrix}[cccc|c]
    1 & 1 & 0 & -3 & 0 \\
    0 & 0 & 1 & 2 & 0 \\
    0 & 0 & 0 & 0 & 1
  \end{bmatrix}
\). It follows that the system is inconsistent with no solutions
(since the bottom row implies the contradiction \(0=1\)).
\end{solution}

\begin{problem}{E4}
Find a basis for the solution set of the system of equations
\begin{align*}
x+3y+3z+7w &= 0 \\
 x+3y-z-w &= 0 \\
  2x+6y+3z+8w &= 0 \\
   x+3y-2z-3w &= 0
\end{align*}
\end{problem}
\begin{solution}
$$\RREF\left(\begin{bmatrix} 1 & 3 & 3 & 7 \\ 1 & 3 & -1 & -1 \\ 2 & 6 & 3 & 8 \\ 1 & 3 & -2 & -3 \end{bmatrix}\right) = \begin{bmatrix} 1 & 3 & 0 & 1 \\ 0 & 0 & 1 & 2 \\ 0 & 0 & 0 & 0 \\ 0 & 0 & 0 & 0 \end{bmatrix}$$
Then the solution set is
$$\left\{ \begin{bmatrix} -3a-b \\ a \\ -2b \\ b \end{bmatrix}\ \big|\ a,b \in \IR\right\}$$
So a basis for the solution set is $$ \left\{ \begin{bmatrix} 3 \\ -1 \\ 0 \\ 0 \end{bmatrix} , \begin{bmatrix} 1\\0 \\ 2 \\ -1 \end{bmatrix} \right\} $$
\end{solution}


\begin{problem}{V1}
Let $V$ be the set of all points on the parabola $y=x^2$ with the operations, for any $(x_1,y_1), (x_2,y_2) \in V$, $c \in \IR$,
\begin{align*}
(x_1,y_1) \oplus (x_2,y_2) &= (x_1+x_2,y_1+y_2+2x_1x_2) \\
c \odot (x_1,y_1) &= (cx_1, c^2y_1)
\end{align*}
\begin{enumerate}[(a)]
\item Show that the vector \textbf{addition} $\oplus$ is \textbf{associative}:
      \((x_1,y_1) \oplus ((x_2,y_2) \oplus (x_3,y_3))=((x_1,y_1)\oplus (x_2,y_2))\oplus (x_3,y_3)\).
\item Determine if $V$ is a vector space or not.  Justify your answer.
\end{enumerate}
\end{problem}

\begin{problem}{V2}
  Determine if
  \(\begin{bmatrix} 3 \\ -2 \\ 4 \end{bmatrix}\)
  belongs to the span of the set
  \(\left\{
    \begin{bmatrix} 1 \\ 2 \\ -3 \end{bmatrix},
    \begin{bmatrix} 2 \\ 4 \\ -6 \end{bmatrix},
    \begin{bmatrix} 0 \\ 0 \\ 0 \end{bmatrix}
    \right\}
  \).
\end{problem}
\begin{solution}
  Since
  \[
    \RREF\left(
      \begin{bmatrix}[ccc|c]
        1 & 2 & 0 & 3 \\
        2 & 4 & 0 & -2 \\
        -3 & -6 & 0 & 4
      \end{bmatrix}
    \right) =
    \begin{bmatrix}[ccc|c]
      1 & 2 & 0 & 0 \\
      0 & 0 & 0 & 1 \\
      0 & 0 & 0 & 0
    \end{bmatrix}
  \]
  contains the contradiction \(0=1\),
  \(\begin{bmatrix} 3 \\ -2 \\ 4 \end{bmatrix}\) is
  not a linear combination of the three vectors.
\end{solution}
\begin{problem}{V3}
Determine if the vectors $\begin{bmatrix} 1 \\ 0 \\ 2 \\1 \end{bmatrix}$, $\begin{bmatrix} 3 \\ 1 \\ 0 \\ -3 \end{bmatrix}$,$\begin{bmatrix} 0 \\ 3 \\ 0 \\ -2 \end{bmatrix}$, and $\begin{bmatrix}-1 \\ 1 \\ -1 \\ -1 \end{bmatrix}$ span $\IR^4$.
\end{problem}
\begin{solution}
$$\RREF\left(\begin{bmatrix}1 & 3 & 0 & -1 \\ 0 & 1 & 3 & 1 \\ 2 & 0 & 0 & -1 \\ 1 & -3 & -2 & -1 \end{bmatrix} \right) =\begin{bmatrix} 1 & 0 & 0 & 0 \\ 0 & 1 & 0 & 0 \\ 0 & 0 & 1 & 0 \\ 0 & 0 & 0 & 1 \end{bmatrix}$$
Since every row contains a pivot, the vectors span $\IR^4$.
\end{solution}

\begin{problem}{V4} Let \(W\) be the set of all complex numbers
that are purely real (i.e of the form $a+0i$)  or purely imaginary (i.e. of the form $0+bi$).
Determine if \(W\) is a subspace of \(\IC\).
\end{problem}
\begin{solution}
No, because \(1\) is purely real and \(i\) is purely imaginary, but
the linear combination \(1+i\) is neither.
\end{solution}


\begin{problem}{S1}
Determine if the set of polynomials  $\left\{x^2+x, x^2+2x-1, x^2+3x-2\right\}$ is  linearly dependent or linearly independent
\end{problem}
\begin{solution}
$$\RREF\left( \begin{bmatrix} 1 &  1 & 1 \\ 1  & 2 & 3 \\ 0  & -1 & -2 \end{bmatrix} \right) = \begin{bmatrix} 1 &  0 & -1 \\ 0  & 1 & 2 \\ 0& 0 & 0  \end{bmatrix}$$
Since there is a nonpivot column, the set is linearly dependent.
\end{solution}


\begin{problem}{S2}
Determine if the set $\left\{ x^3-x, x^2+x+1, x^3-x^2+2, 2x^2-1 \right\}$ is a basis of $\P^3$.
\end{problem}
\begin{solution}
$$\RREF\left(\begin{bmatrix} 1 & 0 & 1 & 0 \\ 0 & 1 & -1 & 2 \\ -1 & 1 & 0 & 0 \\ 0 & 1 & 2 & -1 \end{bmatrix} \right)= \begin{bmatrix} 1 & 0 &0 & 1 \\ 0 & 1 & 0 & 1 \\ 0 & 0 & 1 & -1 \\ 0 & 0 & 0 & 0 \end{bmatrix}$$
Since the resulting matrix is not the identity matrix, it is not a basis.
\end{solution}


\begin{problem}{S3}
Let \(
  W={\rm span}\left\{
    \begin{bmatrix} 2 & 0 \\ -2 & 0 \end{bmatrix},
    \begin{bmatrix} 3 & 1 \\ 3 & 6 \end{bmatrix},
    \begin{bmatrix} 0 & 0 \\ 1 & 1 \end{bmatrix},
    \begin{bmatrix} 1 & 2 \\ 0 & 1 \end{bmatrix}
  \right\}
\). Find a basis for this vector space.
\end{problem}
\begin{solution}
\[
  \RREF\left(\begin{bmatrix}
    2 & 3 & 0 & 1 \\
    0 & 1 & 0 & 2 \\
    -2 & 3 & 1 & 0 \\
    0 & 6 & 1 & 1
  \end{bmatrix} \right) =
  \begin{bmatrix}
    1 & 0 & 0 & -\frac{5}{2} \\
    0 & 1 & 0 & 2 \\
    0 & 0 & 1 & -11\\
    0 & 0 & 0 & 0
  \end{bmatrix}
\]
Thus \(\left\{
  \begin{bmatrix} 2 & 0 \\ -2 & 0 \end{bmatrix},
  \begin{bmatrix} 3 & 1 \\ 3 & 6 \end{bmatrix},
  \begin{bmatrix} 0 & 0 \\ 1 & 1 \end{bmatrix}
\right\}\) is a basis of $W$.
\end{solution}


\begin{problem}{S4}
  Let \(
    W={\rm span}\left\{ 2x^2-x+3, 2x^2+2, -x^2+4x+1 \right\}\).
  Find the dimension of \(W\).
\end{problem}
\begin{solution}
  \[\RREF\left(
    \begin{bmatrix}
      2 & 2 & -1 \\
      -1 & 0 & 4 \\
      3 & 2 & 1
    \end{bmatrix} \right)= \begin{bmatrix}
      1 & 0 &0 \\
      0 & 1 & 0 \\
      0 & 0 & 1
    \end{bmatrix}
  \]
  Since it has three pivot columns, its dimension is \(3\).
\end{solution}
\begin{problem}{A1}
Let $T: \IR^3\rightarrow \IR^4$ be the linear transformation given by $$T\left(\begin{bmatrix} x \\ y \\ z \\  \end{bmatrix} \right) = \begin{bmatrix} -3x+y \\ -8x+2y-z \\ 7x+2y+3z \\ 0 \end{bmatrix}.$$  Write the matrix for $T$ with respect to the standard bases of $\IR^3$ and $\IR^4$.
\end{problem}
\begin{solution}
$$\begin{bmatrix} 3 & 1 & 0 \\ -8 & 2 & -1 \\ 7 & 2 & 3 \\ 0 & 0 & 0 \end{bmatrix}$$
\end{solution}

\begin{problem}{A2}
Determine if the map $T: \P^6  \rightarrow \P^7$ given by $T(f) = xf(x)-f(1)$ is a linear transformation or not.
\end{problem}

\begin{problem}{A3}
Determine if each of the following linear transformations is injective (one-to-one) and/or surjective (onto).
\begin{enumerate}[(a)]
\item $S: \IR^2 \rightarrow \IR^2$ given by the standard matrix $\begin{bmatrix} 0 & 1 \\ -1 & 0 \end{bmatrix}$.
\item $T: \IR^4 \rightarrow \IR^3$ given by the standard matrix $\begin{bmatrix} 2 & 3 & -1 & -2 \\ 0 & 1 & 3 & 1 \\ 2 & 1 & -7 & -4 \end{bmatrix}$
\end{enumerate}
\end{problem}
\begin{solution}
\begin{enumerate}[(a)]
\item $ \RREF\begin{bmatrix} 0 & 1 \\ -1 & 0 \end{bmatrix}=\begin{bmatrix}1 & 0 \\ 0 & 1 \end{bmatrix}$.  Since each column is a pivot column, $S$ is injective.  Since there is no zero row, $S$ is surjective.
\item Since $\dim \IR^4 > \dim \IR^3$, $T$ is not injective.
$$\RREF\left(\begin{bmatrix} 2 & 3 & -1 & -2 \\ 0 & 1 & 3 & 1 \\ 2 & 1 & -7 & -4 \end{bmatrix}\right) = \begin{bmatrix} 1 & 0 & -5 & -\frac{5}{2} \\ 0 & 1 & 3 & 1 \\ 0 & 0 & 0 & 0\end{bmatrix}$$
Since there are only two pivot columns, $T$ is not surjective.
\end{enumerate}
\end{solution}


\begin{problem}{A4}
Let $T: \IR^{2\times 2} \rightarrow \IR^3$ be the linear map given by \(
  T\left(\begin{bmatrix} a & b \\ x & y \end{bmatrix} \right) =
  \begin{bmatrix}
    a+x \\ 0 \\ b+y
  \end{bmatrix}
\). Compute a basis for the kernel and a basis for the image of $T$.
\end{problem}
\begin{solution}
Rewrite as \(
  T'\left(\begin{bmatrix} a \\ b \\ x \\ y \end{bmatrix} \right) =
  \begin{bmatrix}
    a+x \\ 0 \\ b+y
  \end{bmatrix}
\).
\[
  \RREF \left( \begin{bmatrix}
    1 & 0 & 1 & 0 \\
    0 & 0 & 0 & 0 \\
    0 & 1 & 0 & 1
  \end{bmatrix} \right) = \begin{bmatrix}
    1 & 0 & 1 & 0 \\
    0 & 0 & 0 & 0 \\
    0 & 1 & 0 & 1
  \end{bmatrix}
\]

Thus \(\left\{
  \begin{bmatrix} 1 \\ 0 \\ 0 \end{bmatrix},
  \begin{bmatrix} 0 \\ 0 \\ 1 \end{bmatrix}
\right\} \) is a basis for the image, and \(\left\{
  \begin{bmatrix} -1 & 0 \\ 1 & 0 \end{bmatrix},
  \begin{bmatrix} 0 & -1 \\ 0 & 1 \end{bmatrix}
\right\} \) is a basis for the kernel.
\end{solution}
\begin{problem}{M1}
Let 
\begin{align*}
A &= \begin{bmatrix} 3 \\ 5 \\ -1  \end{bmatrix} & B&=\begin{bmatrix} 1 & -1 & 3 & -3 \\ 2 & 1 & -1 & 2 \end{bmatrix} & C &= \begin{bmatrix} 2 & -1 \\ 0 & 4 \\ 3 & 1 \end{bmatrix} \end{align*}
Exactly one of the six products $AB$, $AC$, $BA$, $BC$, $CA$, $CB$ can be computed.  Determine which one, and compute it.
\end{problem}
\begin{solution}
$CB$ is the only one that can be computed, and
$$CB=\begin{bmatrix} 0 & -3 & 7 & -8 \\ 8 & 4 & -4 & 8 \\ 5 & -2 & 8 & -7 \end{bmatrix}$$
\end{solution}

\begin{problem}{M2}
Determine if the matrix $\begin{bmatrix} 1 & 3 & 3 & 7 \\ 1 & 3 & -1 & -1 \\ 2 & 6 & 3 & 8 \\ 1 & 3 & -2 & -3 \end{bmatrix}$ is invertible.
\end{problem}
\begin{solution}
The second column is a multiple of the first, so it is not invertible.
\end{solution}



\begin{problem}{M3}
Find the inverse of the matrix $\begin{bmatrix} 8 & 5 & 3 & 0 \\ 3 & 2 & 1 & 1 \\ 5 & -3 & 1 & -2 \\ -1 & 2 & 0 & 1\end{bmatrix} $.
\end{problem}
\begin{solution}
$$\RREF \left(\begin{bmatrix}[cccc|cccc] 8 & 5 & 3 & 0 & 1 & 0 & 0 & 0\\ 3 & 2 & 1 & 1 & 0 & 1 & 0 & 0 \\ 5 & -3 & 1 & -2 & 0 & 0 & 1 & 0 \\ -1 & 2 & 0 & 1 & 0 & 0 & 0 & 1 \end{bmatrix} \right) = \begin{bmatrix}[cccc|cccc] 1 & 0 & 0 & 0 & 1 & 2 & -5 & 12 \\ 0 &  1 & 0 & 0 & 1 & 1 & -4 & -9 \\ 0 & 0 & 1 & 0 & -4 & -7 & 20 & 47 \\ 0 & 0 & 0 & 1 & -1 & 0 & 3 & 7 \end{bmatrix}$$

So the inverse is $\begin{bmatrix} 1 & 2 & -5 & 12 \\  1 & 1 & -4 & -9 \\  -4 & -7 & 20 & 47 \\-1 & 0 & 3 & 7 \end{bmatrix}$.
\end{solution}


\begin{problem}{G1}
Compute the determinant of the matrix
\[
  \begin{bmatrix}
    0 & -4 & 1 & 1 \\
    0 & 1 & 0 & 1 \\
    -2 & 3 & -1 & 1 \\
    5 & 0 & -4 & 0 \\
  \end{bmatrix}
.\]
\end{problem}
\begin{solution}
\(55\).
\end{solution}

\begin{problem}{G2} 
Compute the eigenvalues, along with their algebraic multiplicities, of the matrix $ \begin{bmatrix}3 & -3 & 2 \\108 & -9 & 5 \\ 10 & -7 & 3 \end{bmatrix}$.
\end{problem}
\begin{solution}
The eigenvalues are $1$ with multiplicity 1 and $-2$, with algebraic multiplicity 2.
\end{solution}
\begin{problem}{G3}
Compute the eigenspace associated to the eigenvalue $2$ in the matrix $\begin{bmatrix} -1 & 1 & 0 \\ -9 & 5 & 0 \\ 15 & -5 & 2 \end{bmatrix}$.
\end{problem}

\begin{solution}
The eigenspace is the solution space of the system $(B-2I)X=0$.
$$\RREF(B-2I)=\RREF\left(\begin{bmatrix} -3 & 1 & 0 \\ -9 & 3 & 0 \\ 15 & - 5 & 0 \end{bmatrix} \right) = \begin{bmatrix} 1 & -\frac{1}{3} & 0 \\ 0 & 0 & 0 \\ 0 & 0 & 0 \end{bmatrix}$$
So the system simplifies to $x-\frac{y}{3}=0$, or $3x=y$.  Thus the eigenspace is $$E_2 = {\rm span}\left( \left\{ \begin{bmatrix} 1 \\ 3 \\ 0 \end{bmatrix}, \begin{bmatrix} 0 \\ 0 \\ 1\end{bmatrix} \right\} \right)$$
\end{solution}
\begin{problem}{G4}
Compute the geometric multiplicity of the eigenvalue $2$ in the matrix $\begin{bmatrix} -1 & 1 & 0 \\ -9 & 5 & 0 \\ 15 & -5 & 2 \end{bmatrix}$.
\end{problem}

\begin{solution}
The eigenspace is the solution space of the system $(B-2I)X=0$.
$$\RREF(B-2I)=\RREF\left(\begin{bmatrix} -3 & 1 & 0 \\ -9 & 3 & 0 \\ 15 & - 5 & 0 \end{bmatrix} \right) = \begin{bmatrix} 1 & -\frac{1}{3} & 0 \\ 0 & 0 & 0 \\ 0 & 0 & 0 \end{bmatrix}$$
Thus the geometric multiplicity is 2.
\end{solution}
\end{document}