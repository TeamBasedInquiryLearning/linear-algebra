\documentclass{sbgLAexam}

\begin{extract*}
\usepackage{amsmath,amssymb,amsthm,enumerate}
\coursetitle{Math 237}
\courselabel{Linear Algebra}
\calculatorpolicy{You may use a calculator, but you must show all relevant work to receive credit for a standard.}

\newcommand{\IR}{\mathbb{R}}

\makeatletter
\renewcommand*\env@matrix[1][*\c@MaxMatrixCols c]{%
  \hskip -\arraycolsep
  \let\@ifnextchar\new@ifnextchar
  \array{#1}}
\makeatother


\title{Final Exam}
\standard{E1,E2,E3,E4,V1,V2,V3,V4,S1,S2,S3,S4,A1,A2,A3,A4,M1,M2,M3,G1,G2,G3,G4}
\version{5}
\end{extract*}

\begin{document}

\begin{problem}{E1}
Write a system of linear equations corresponding to the following
augmented matrix.
\[
\begin{bmatrix}[cccc|c]
3 & -1 & 0 & 1 & 5 \\
-1 & 9 & 1 & -7 & 0 \\
1 & 0 & -1 & 0 & -3
\end{bmatrix}
\]
\end{problem}
\begin{solution}
\begin{align*}
3x_1-x_2+x_4 &= 5 \\
-x_1+9x_2+x_3-7x_4 &= 0 \\
x_1-x_3 &= -3
\end{align*}
\end{solution}

\begin{problem}{E2}
Find the reduced row echelon form of the matrix below.
$$\begin{bmatrix}[cccc|c] 2 & 1 & -1 & 0 & 5 \\ 3 & -1 & 0 & -2 & 0 \\ -1 & 0 & 5 & 0 & -1 \end{bmatrix}$$
\end{problem}
\begin{solution}
$$\begin{bmatrix}[cccc|c]
 2 & 1 & -1 & 0 & 5 \\
 3 & -1 & 0 & -2 & 0 \\
 -1 & 0 & 5 & 0 & -1
\end{bmatrix} \sim
\begin{bmatrix}[cccc|c]
 -1 & 0 & 5 & 0 & -1  \\
 2 & 1 & -1 & 0 & 5 \\
 3 & -1 & 0 & -2 & 0
\end{bmatrix} \sim
\begin{bmatrix}[cccc|c]
 1 & 0 & -5 & 0 & 1 \\
 2 & 1 & -1 & 0 & 5 \\
 3 & -1 & 0 & -2 & 0
\end{bmatrix} $$
$$\sim
\begin{bmatrix}[cccc|c]
 1 & 0 & -5 & 0 & 1  \\
 0 & 1 & 9 & 0 & 3 \\
 0 & -1 & 15 & -2 & -3 \\
\end{bmatrix} \sim
\begin{bmatrix}[cccc|c]
 1 & 0 & -5 & 0 & 1  \\
 0 & 1 & 9 & 0 & 3 \\
 0 & 0 & 24 & -2 & 0 \\
\end{bmatrix} \sim
\begin{bmatrix}[cccc|c]
 1 & 0 & -5 & 0 & 1 \\
 0 & 1 & 9 & 0 & 3 \\
 0 & 0 & 1 & -\frac{1}{12} & 0 \\
\end{bmatrix} \sim
\begin{bmatrix}[cccc|c]
1 & 0 & 0 & -\frac{5}{12} & 1 \\
 0 & 1 & 0 & \frac{3}{4} & 3 \\
 0 & 0 & 1 & -\frac{1}{12} & 0
\end{bmatrix}$$
\end{solution}

\begin{extract}\newpage\end{extract}
\begin{problem}{E3}
Find the solution set for the following system of linear equations.
\begin{align*}
2x_1-2x_2+6x_3-x_4 &=-1 \\
3x_1+6x_3+x_4 &= 5 \\
-4x_1+x_2-9x_3+2x_4&=-7
\end{align*}
\end{problem}
\begin{solution}
Let \(A =
  \begin{bmatrix}[cccc|c]
    2 & -2 & 6 & -1 & -1 \\
    3 & 0 & 6 & 1 & 5 \\
    -4 & 1 & -9 & 2 & -7
  \end{bmatrix}
\), so \(\RREF A =
  \begin{bmatrix}[cccc|c]
    1 & 0 & 2 & 0 & 2 \\
    0 & 1 & -1 & 0 & 3 \\
    0 & 0 & 0 & 1 & -1
  \end{bmatrix}
\). It follows that the solution set is given by \(
  \begin{bmatrix}
    2 - 2a \\
    3 + a \\
    a \\
    -1
  \end{bmatrix}
\) for all real numbers \(a\).
\end{solution}

\begin{problem}{E4}
Find a basis for the solution set of the system of equations
\begin{align*}
x+2y+3z+w &= 0 \\
3x-y+z+w &= 0 \\
2x-3y-2z &= 0
\end{align*}
\end{problem}
\begin{solution}
$$\RREF \left(\begin{bmatrix} 1 & -2 & 3 & 1 \\ 3 & -1 & 1 & 1 \\ 2 & -3 & -2 & 0  \end{bmatrix} \right) = \begin{bmatrix} 1 & 0 & \frac{5}{7} & \frac{3}{7} \\ 0 & 1 & \frac{8}{7} & \frac{2}{7} \\ 0 & 0 & 0 & 0\end{bmatrix}$$
Then the solution set is
$$\left\{ \begin{bmatrix} -\frac{5}{7}a-\frac{3}{7}b \\ -\frac{8}{7}a-\frac{2}{7}b \\ a \\ b \end{bmatrix} \bigg |\ a,b \in \IR \right\}$$
So a basis for the solution set is $\left\{\begin{bmatrix} -\frac{5}{7} \\ -\frac{8}{7} \\ 1 \\ 0\end{bmatrix}, \begin{bmatrix} - \frac{3}{7} \\ -\frac{2}{7} \\ 0 \\ 1 \end{bmatrix} \right\}$, or $\left\{\begin{bmatrix} 5 \\ 8 \\ -7 \\ 0 \end{bmatrix}, \begin{bmatrix} 3 \\ 2 \\ 0 \\ -7 \end{bmatrix}\right\}$.
\end{solution}

\begin{extract}\newpage\end{extract}
\begin{problem}{V1}
Let $V$ be the set of all polynomials with the operations, for any $f, g \in V$, $c \in \IR$,
\begin{align*}
f \oplus g &= f^\prime + g^\prime \\
c \odot f &= c f^\prime
\end{align*}
(here $f^\prime$ denotes the derivative of $f$).
\begin{enumerate}[(a)]
\item Show that scalar multiplication \textbf{distributes scalars} over
      vector addition:
      \(c\odot(f \oplus g)=
      c\odot f \oplus c\odot g\).
\item Determine if $V$ is a vector space or not.  Justify your answer.
\end{enumerate}
\end{problem}
\begin{solution}
Let $f,g \in \mathcal{P}$, and let $c \in \IR$.
$$c \odot (f \oplus g) = c \odot (f^\prime+g^\prime) =
c(f^\prime+g^\prime)^\prime = cf^{\prime\ \prime}+cg^{\prime\ \prime} =
cf^\prime\oplus cg^\prime= c \odot f \oplus c \odot g.$$
However, this is not a vector space, as there is no zero vector.  Additionally, $1 \odot f \neq f$ for any nonzero polynomial $f$.
\end{solution}


\begin{problem}{V2}
  Determine if
  \(\begin{bmatrix} 4 \\ -1 \\ 6 \\ -7 \end{bmatrix}\)
  belongs to the span of the set
  \(\left\{
    \begin{bmatrix} 2 \\ 0 \\ -1 \\ 5 \end{bmatrix},
    \begin{bmatrix} 4 \\ -1 \\ 4 \\ 3 \end{bmatrix}
    \right\}
  \).
\end{problem}
\begin{solution}
  Since
  \[
    \RREF\left(
      \begin{bmatrix}[cc|c]
        2 & 4 & 4 \\
        0 & -1 & -1 \\
        -1 & 4 & 6 \\
        5 & 3 & -7
      \end{bmatrix}
    \right) =
    \begin{bmatrix}[cc|c]
      1 & 0 & 0 \\
      0 & 1 & 0 \\
      0 & 0 & 1 \\
      0 & 0 & 0
    \end{bmatrix}
  \]
  contains the contradiction \(0=1\),
  \(\begin{bmatrix} 4 \\ -1 \\ 6 \\ -7 \end{bmatrix}\) is
  not a linear combination of the three vectors.
\end{solution}


\begin{extract}\newpage\end{extract}
\begin{problem}{V3}
Determine if the vectors $\begin{bmatrix} 2 \\ 0 \\ -2 \\ 0 \end{bmatrix}$, $\begin{bmatrix} 3 \\ 1 \\ 3 \\ 6 \end{bmatrix}$, $\begin{bmatrix} 0 \\ 0 \\ 1 \\ 1 \end{bmatrix}$, and $\begin{bmatrix}1 \\ 2 \\ 0 \\ 1 \end{bmatrix}$ span $\IR^4$.
\end{problem}
\begin{solution}
$$\RREF\left( \begin{bmatrix} 2 & 3 & 0 & 1 \\ 0 & 1 & 0 & 2 \\ -2 & 3 & 1 & 0 \\ 0 & 6 & 1 & 1\end{bmatrix} \right) = \begin{bmatrix}1 & 0 & 0 & -\frac{5}{2} \\ 0 & 1 & 0 & 2 \\ 0 & 0 & 1 & -11\\ 0 & 0 & 0 & 0  \end{bmatrix} $$
Since there is a zero row, the vectors do not span $\IR^4$.
\end{solution}

\begin{problem}{V4} Let \(W\) be the set of all complex numbers
that are purely real (i.e of the form $a+0i$)  or purely imaginary (i.e. of the form $0+bi$).
Determine if \(W\) is a subspace of \(\IC\).
\end{problem}
\begin{solution}
No, because \(1\) is purely real and \(i\) is purely imaginary, but
the linear combination \(1+i\) is neither.
\end{solution}


\begin{extract}\newpage\end{extract}
\begin{problem}{S1}
Determine if the set of matrices $\left\{\begin{bmatrix} 3 & -1 \\ 0 & 4 \end{bmatrix}, \begin{bmatrix} 1  & 2 \\ -2 & 1 \end{bmatrix}, \begin{bmatrix} 3 & -8 \\ 6 & 5 \end{bmatrix} \right\}$  is linearly dependent or linearly independent.
\end{problem}
\begin{solution}
$$\RREF\left(\begin{bmatrix} 3 & 1 & 3 \\ -1 & 2 & -8 \\ 0 & -2 & 6 \\ 4 & 1 & 5 \end{bmatrix} \right) = \begin{bmatrix} 1 & 0 & 2 \\ 0 & 1 & -3 \\ 0 & 0 & 0 \\ 0 & 0 & 0 \end{bmatrix}$$
Since the reduced row echelon form has a nonpivot column, the vectors are linearly dependent.
\end{solution}
\begin{problem}{S2}
Determine if the set $\left\{
\begin{bmatrix} 1 & -3 \\ 2 & 2 \end{bmatrix},
\begin{bmatrix} -1 & 4 \\ -1 & 1 \end{bmatrix},
\begin{bmatrix} -1 & 0 \\ 2 & 1 \end{bmatrix},
\begin{bmatrix} 0 & 3 \\ 3 & 9 \end{bmatrix}
\right\}$ is a basis of $\IR^{2 \times 2}$ or not.
\end{problem}

\begin{solution}
$$\RREF \begin{bmatrix} 1 & -1 & -1 & 0 \\ -3 & 4 & 0 & 3 \\ 2 & -1 & 2 & 3 \\ 2 & 1 & 1 & 9 \end{bmatrix}=\begin{bmatrix} 1 &0 & 0 & 3 \\ 0 & 1 & 0 & 3 \\ 0 & 0 & 1 & 0 \\ 0 & 0 & 0 & 0 \end{bmatrix}$$
Since this is not the identity matrix, the set is not a basis.
\end{solution}

\begin{extract}\newpage\end{extract}
\begin{problem}{S3}
Let $W = {\rm span} \left( \left\{ \begin{bmatrix} 1 \\ 1 \\ 2 \\ 1 \end{bmatrix}, \begin{bmatrix} 3 \\ 3 \\ 6 \\ 3 \end{bmatrix}, \begin{bmatrix} 3 \\ -1 \\ 3 \\ -2 \end{bmatrix}, \begin{bmatrix} 7 \\ -1 \\ 8 \\ -3 \end{bmatrix} \right\} \right)$.  Find a basis for $W$.
\end{problem}
\begin{solution}
$$\RREF\left(\begin{bmatrix} 1 & 3 & 3 & 7 \\ 1 & 3 & -1 & -1 \\ 2 & 6 & 3 & 8 \\ 1 & 3 & -2 & -3 \end{bmatrix}\right) = \begin{bmatrix} 1 & 3 & 0 & 1 \\ 0 & 0 & 1 & 2 \\ 0 & 0 & 0 & 0 \\  0 & 0 & 0 & 0 \end{bmatrix}$$

Then a basis is
$ \left\{ \begin{bmatrix} 1 \\ 1 \\ 2 \\ 1 \end{bmatrix} , \begin{bmatrix} 3 \\ -1 \\ 3 \\ -2 \end{bmatrix}\right\} $.
\end{solution}


\begin{problem}{S4}
Let $W={\rm span}\left( \left\{ \begin{bmatrix} 1 \\ -1 \\ 3 \\ -3 \end{bmatrix},\begin{bmatrix} 2 \\ 0 \\ 1 \\ 1 \end{bmatrix}, \begin{bmatrix} 3 \\ -1 \\ 4 \\ -2 \end{bmatrix},  \begin{bmatrix} 1 \\ 1 \\ 1 \\ -7 \end{bmatrix} \right\}\right)$.  Compute the dimension of $W$.
\end{problem}
\begin{solution}
$$ \RREF \left( \begin{bmatrix} 1 & 2 & 3 & 1 \\ -1 & 0 & -1 & 1 \\ 3 & 1 & 4 & 1 \\ -3 & 1 & -2 & -7 \end{bmatrix} \right) =  \begin{bmatrix} 1 & 0 & 1 & 0 \\ 0 & 1 & 1 & 0 \\ 0 & 0 & 0 & 1 \\ 0 & 0 & 0 & 0\end{bmatrix}$$
This has 3 pivot columns so $\dim(W)=3$.
\end{solution}


\begin{extract}\newpage\end{extract}
\begin{problem}{A1}
Let $T: \IR^3 \rightarrow \IR$ be the linear transformation given by $$T\left(\begin{bmatrix} x_1 \\ x_2 \\ x_3  \end{bmatrix} \right) = \begin{bmatrix} x_2+3x_3 \end{bmatrix}.$$ Write the matrix for $T$ with respect to the standard bases of $\IR^3$ and $\IR$.
\end{problem}
\begin{solution}
$$\begin{bmatrix} 0 & 1 & 3 \end{bmatrix}$$
\end{solution}


\begin{problem}{A2}
Determine if $D: \IR^{2\times 2} \rightarrow \IR$ given by $D\left(\begin{bmatrix} a & b \\ c & d \end{bmatrix} \right) = ad-bc$ is a linear transformation or not.
\end{problem}
\begin{solution}
$D(I)=1$ but $D(2I)=4 \neq 2D(I)$, so $D$ is not linear.
\end{solution}

\begin{extract}\newpage\end{extract}
\begin{problem}{A3}
Determine if each of the following linear transformations is injective (one-to-one) and/or surjective (onto).
\begin{enumerate}[(a)]
\item $S: \IR^2 \rightarrow \IR^4$ given by the standard matrix $\begin{bmatrix} 2 & 1 \\ 1 & 2 \\ 0 & 1 \\ 3 & -3 \end{bmatrix}$.
\item $T: \IR^4 \rightarrow \IR^3$ given by the standard matrix $\begin{bmatrix} 2 & 3 & -1 & 1 \\ -1 & 1 & 1 & 1 \\ 4 & 11 & -1 & 5 \end{bmatrix}$
\end{enumerate}
\end{problem}
\begin{solution}
\begin{enumerate}[(a)]
\item $ \begin{bmatrix} 2 & 1 \\ 1 & 2 \\ 0 & 1 \\ 3 & -3 \end{bmatrix}=\begin{bmatrix}1 & 0 \\ 0 & 1 \\ 0 & 0 \\ 0 & 0  \end{bmatrix}$.  Since each column is a pivot column, $S$ is injective.  Since there a no zero row, $S$ is not surjective.
\item Since $\dim \IR^4 > \dim \IR^3$, $T$ is not injective.
$$\RREF\left(\begin{bmatrix} 2 & 3 & -1 & 1 \\ -1 & 1 & 1 & 1 \\ 4 & 11 & -1 & 5 \end{bmatrix}\right) = \begin{bmatrix} 1 & 0  & -\frac{4}{5} & -\frac{2}{5} \\ 0 & 1 & \frac{1}{5} & \frac{3}{5} \\ 0 & 0 & 0 & 0 \end{bmatrix}$$
Since there is a zero row, $T$ is not surjective.
\end{enumerate}
\end{solution}

\begin{problem}{A4}
Let $T: \IR^3\rightarrow \IR^3$ be the linear transformation given by $$T\left(\begin{bmatrix} x \\ y \\ z \\  \end{bmatrix} \right) = \begin{bmatrix} -3x+y \\ -8x+2y-z \\ 2y+3z \end{bmatrix}$$
Compute a basis for the kernel and a basis for the image of $T$.
\end{problem}
\begin{solution}
Let $A= \begin{bmatrix}-3 & 1 & 0 \\ -8 & 2 & -1 \\ 0 & 2 & 3\end{bmatrix}$, and compute $\RREF(A) = \begin{bmatrix} 1 & 0 & \frac{1}{2} \\ 0 & 1 & \frac{3}{2} \\ 0 & 0 & 0 \end{bmatrix}$.  Then a basis for the image is its columns, $$\left\{ \begin{bmatrix} -3 \\ -8 \\ 0 \end{bmatrix}, \begin{bmatrix} 1 \\ 2 \\ 2 \end{bmatrix} \right\}$$
And the kernel is the solution set of $AX=0$, so a basis would be
$$\left\{\begin{bmatrix} 1 \\ 3 \\ -2 \end{bmatrix} \right\}$$
\end{solution}


\begin{extract}\newpage\end{extract}
\begin{problem}{M1}
Let 
\begin{align*}
A &= \begin{bmatrix} 3 \\ 5 \\ -1  \end{bmatrix} & B&=\begin{bmatrix}  2 & 1 & -1 & 2 \\ 1 & -1 & 3 & -3  \end{bmatrix} & C &= \begin{bmatrix} 2 & -1 \\ 0 & 4 \\ 3 & 1 \end{bmatrix} \end{align*}
Exactly one of the six products $AB$, $AC$, $BA$, $BC$, $CA$, $CB$ can be computed.  Determine which one, and compute it.
\end{problem}
\begin{solution}
$CB$ is the only one that can be computed, and
$$CB=\begin{bmatrix} 3 & 3 & -5 & 7 \\ 4 & -4 & 12 & -12 \\ 7 & 2 & 0 & 3 \end{bmatrix}$$
\end{solution}
\begin{problem}{M2}
Determine if the matrix $\begin{bmatrix} 3 & -1 & 0  \\ 2 & 1 & 1  \\ 0 & 1 & 1   \end{bmatrix}$ is invertible.
\end{problem}
\begin{solution}
It is row equivalent to the identity matrix, so it is invertible.
\end{solution}


\begin{extract}\newpage\end{extract}
\begin{problem}{M3}
Find the inverse of the matrix $\begin{bmatrix} 8 & 5 & 3 & 0 \\ 3 & 2 & 1 & 1 \\ 5 & -3 & 1 & -2 \\ -1 & 2 & 0 & 1\end{bmatrix} $.
\end{problem}
\begin{solution}
$$\RREF \left(\begin{bmatrix}[cccc|cccc] 8 & 5 & 3 & 0 & 1 & 0 & 0 & 0\\ 3 & 2 & 1 & 1 & 0 & 1 & 0 & 0 \\ 5 & -3 & 1 & -2 & 0 & 0 & 1 & 0 \\ -1 & 2 & 0 & 1 & 0 & 0 & 0 & 1 \end{bmatrix} \right) = \begin{bmatrix}[cccc|cccc] 1 & 0 & 0 & 0 & 1 & 2 & -5 & 12 \\ 0 &  1 & 0 & 0 & 1 & 1 & -4 & -9 \\ 0 & 0 & 1 & 0 & -4 & -7 & 20 & 47 \\ 0 & 0 & 0 & 1 & -1 & 0 & 3 & 7 \end{bmatrix}$$

So the inverse is $\begin{bmatrix} 1 & 2 & -5 & 12 \\  1 & 1 & -4 & -9 \\  -4 & -7 & 20 & 47 \\-1 & 0 & 3 & 7 \end{bmatrix}$.
\end{solution}


\begin{problem}{G1}
Compute the determinant of the matrix $\begin{bmatrix} 2 & 3 & 0 & 1 \\ -1 & 3 & 1 & 4 \\ 0 & 2 & 0 & 3 \\ 1 & -1 & 3 & 5 \end{bmatrix}$.
\end{problem}
\begin{solution}
$-60$.
\end{solution}

\begin{extract}\newpage\end{extract}
\begin{problem}{G2}
Compute the eigenvalues, along with their algebraic multiplicities, of the matrix $ \begin{bmatrix} 8 & -3 & -1 \\ 21 & -8 & -3 \\ -7 & 3 & 2\end{bmatrix}$.
\end{problem}
\begin{solution}
\begin{align*}
\det(A-\lambda I) &= (8-\lambda) \det \begin{bmatrix} -8-\lambda & -3 \\ 3 & 2-\lambda \end{bmatrix}-(-3) \det \begin{bmatrix} 21 & -3 \\ -7 & 2-\lambda \end{bmatrix} +(-1) \det \begin{bmatrix} 21 & -8-\lambda \\ -7 & 3 \end{bmatrix} \\
&=(8-\lambda)\left(\lambda ^2+6\lambda-7\right)+3\left(-21\lambda + 21 \right)-\left(-7\lambda +7 \right) \\
&=(\lambda -1) \left( (8-\lambda)(\lambda+7)-63+7 \right) \\
&= (\lambda-1)(\lambda -\lambda ^2) \\
&= -\lambda (\lambda-1)^2
\end{align*}
So the eigenvalues are $0$ (with algebraic multiplicity 1) and $1$ (with algebraic multiplicity 2).
\end{solution}


\begin{problem}{G3}
Find the eigenspace associated to the eigenvalue $3$ in the matrix $A=\begin{bmatrix}1 & -2 & -1 & 0 \\ -4 & -1 & -2 & 0 \\ 14 & 12 & 11 & 2 \\ -14 & -10 & -9 & -1\end{bmatrix}$.
\end{problem}
\begin{solution}
The eigenspace is spanned by $\begin{bmatrix} -1 \\ \frac{1}{2} \\ 1 \\ 0 \end{bmatrix}$ and  $\begin{bmatrix} -1 \\ 1 \\ 0 \\ 1 \end{bmatrix}$.
\end{solution}


\begin{extract}\newpage\end{extract}
\begin{problem}{G4}
Compute the geometric multiplicity of the eigenvalue $2$ in the matrix $A=\begin{bmatrix} 0 & 1 & 0 & 0 \\ -4 & 4 & 0 & 0 \\ 11 & -6 & 1 & -1 \\ -9 & 5 & 1 & 3 \end{bmatrix}$.
\end{problem}
\begin{solution}
The eigenspace is spanned by $\begin{bmatrix} -1 \\ -2 \\ 1 \\ 0 \end{bmatrix}$ and $\begin{bmatrix} -1 \\ -2 \\ 0 \\ 1 \end{bmatrix}$, so the geometric multiplicity is $2$.
\end{solution}

\end{document}