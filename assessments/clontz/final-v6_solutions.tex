\documentclass{sbgLAexam}

\begin{extract*}
\usepackage{amsmath,amssymb,amsthm,enumerate}
\coursetitle{Math 237}
\courselabel{Linear Algebra}
\calculatorpolicy{You may use a calculator, but you must show all relevant work to receive credit for a standard.}

\newcommand{\IR}{\mathbb{R}}

\makeatletter
\renewcommand*\env@matrix[1][*\c@MaxMatrixCols c]{%
  \hskip -\arraycolsep
  \let\@ifnextchar\new@ifnextchar
  \array{#1}}
\makeatother


\title{Final Exam}
\standard{E1,E2,E3,E4,V1,V2,V3,V4,S1,S2,S3,S4,A1,A2,A3,A4,M1,M2,M3,G1,G2,G3,G4}
\version{6}
\end{extract*}

\begin{document}

\begin{problem}{E1}
Write a system of linear equations corresponding to the following
augmented matrix.
\[
\begin{bmatrix}[ccc|c]
1 & 0 & 4 & 1 \\
0 & 1 & -1 & 7 \\
1 & -1 & 3 & -1
\end{bmatrix}
\]
\end{problem}
\begin{solution}
\begin{align*}
x_1+4x_3 &= 1 \\
x_2-x_3 &= 7 \\
x_1-x_2+3x_3 &= -1
\end{align*}
\end{solution}
\begin{problem}{E2}
Find \(\RREF A\), where
\[
  A =
  \begin{bmatrix}[ccc|c]
    2 & -1 & 5 & 4 \\
    -1 & 0 & -2 & -1 \\
    1 & 3 & -1 & -5
  \end{bmatrix}
\]
\end{problem}
\begin{solution}
\[
  \RREF A =
  \begin{bmatrix}[ccc|c]
    1 & 0 & 2 & 1 \\
    0 & 1 & -1 & -2 \\
    0 & 0 & 0 & 0
  \end{bmatrix}
\]
\end{solution}

\begin{extract}\newpage\end{extract}
\begin{problem}{E3}
Find the solution set for the following system of linear equations.
\begin{align*}
2x_1+3x_2-5x_3+14x_4 &= 8 \\
x_1+x_2-x_3+5x_4&= 3
\end{align*}
\end{problem}
\begin{solution}
Let \(A =
  \begin{bmatrix}[cccc|c]
    2 & 3 & -5 & 14 & 8 \\
    1 & 1 & -1 & 5 & 3
  \end{bmatrix}
\), so \(\RREF A =
  \begin{bmatrix}[cccc|c]
    1 & 0 & 2 & 1 & 1 \\
    0 & 1 & -3 & 4 & 2 \\
  \end{bmatrix}
\). It follows that the solution set is given by \(
  \begin{bmatrix}
    1 - 2a - b \\
    2 + 3a - 4b \\
    a \\
    b
  \end{bmatrix}
\) for all real numbers \(a,b\).
\end{solution}

\begin{problem}{E4}
Find a basis for the solution set of the system of equations
\begin{align*}
x+3y+3z+7w &= 0 \\
 x+3y-z-w &= 0 \\
  2x+6y+3z+8w &= 0 \\
   x+3y-2z-3w &= 0
\end{align*}
\end{problem}
\begin{solution}
$$\RREF\left(\begin{bmatrix} 1 & 3 & 3 & 7 \\ 1 & 3 & -1 & -1 \\ 2 & 6 & 3 & 8 \\ 1 & 3 & -2 & -3 \end{bmatrix}\right) = \begin{bmatrix} 1 & 3 & 0 & 1 \\ 0 & 0 & 1 & 2 \\ 0 & 0 & 0 & 0 \\ 0 & 0 & 0 & 0 \end{bmatrix}$$
Then the solution set is
$$\left\{ \begin{bmatrix} -3a-b \\ a \\ -2b \\ b \end{bmatrix}\ \big|\ a,b \in \IR\right\}$$
So a basis for the solution set is $$ \left\{ \begin{bmatrix} 3 \\ -1 \\ 0 \\ 0 \end{bmatrix} , \begin{bmatrix} 1\\0 \\ 2 \\ -1 \end{bmatrix} \right\} $$
\end{solution}


\begin{extract}\newpage\end{extract}
\begin{problem}{V1}
Let $V$ be the set of all points on the line $x+y=2$ with the operations, for any $(x_1,y_1), (x_2,y_2) \in V$, $c \in \IR$,
\begin{align*}
(x_1,y_1) \oplus (x_2,y_2) &= (x_1+x_2-1,y_1+y_2-1) \\
c \odot (x_1,y_1) &= (cx_1-(c-1), cy_1-(c-1))
\end{align*}
\begin{enumerate}[(a)]
\item Show that this vector space has an \textbf{additive identity} element
      \(\mathbf{0}\) satisfying \((x,y)\oplus\mathbf{0}=(x,y)\).
\item Determine if $V$ is a vector space or not.  Justify your answer.
\end{enumerate}
\end{problem}
\begin{solution}
Let $(x_1,y_1) \in V$; then $(x_1,y_1) \oplus (1,1) = (x_1,y_1)$, so $(1,1)$ is an additive identity element.

Now we will show the other seven properties.  Let $(x_1,y_1), (x_2,y_2) \in V$, and let $c,d \in \IR$.
\begin{enumerate}[1)]
\item Since real addition is associative, $\oplus$ is associative.
\item Since real addition is commutative, $\oplus$ is commutative.
\item The additive identity is $(1,1)$.
\item $(x_1,y_1) \oplus (2-x_1,2-y_1) = (1,1)$, so $(2-x_1,2-y_1)$ is the additive inverse of $(x_1,y_1)$.
\item \begin{align*} c\odot \left(d \odot (x_1,y_1) \right) &=c\odot \left( dx_1-(d-1),dy_1-(d-1)\right) \\
&= \left( c\left(dx_1-(d-1) \right)-(c-1), c\left(dy_1-(d-1) \right) \right) \\
&= \left(cdx_1-cd+c-(c-1), cdy_1-cd+c-(c-1) \right) \\
&= \left(cdx_1-(cd-1), cdy_1-(cd-1) \right) \\
&= (cd) \odot (x_1,y_1)
\end{align*}
\item $1 \odot (x_1,y_1) = (x_1-(1-1),y_1-(1-1)=(x_1,y_1)$
\item \begin{align*} c \odot \left( (x_1,y_1)\oplus(x_2,y_2) \right) &=
c\odot \left( x_1+y_1-1,x_2+y_2-1 \right) \\
&= \left( c(x_1+y_1-1)-(c-1), c(x_2+y_2-1)-(c-1) \right) \\
&= (cx_1+cx_2-2c+1, cy_1+cy_2-2c+1) \\
&= \left(cx_1-(c-1),cy_1-(c-1) \right) \oplus (cx_2-(c-1),cy_2-(c-1)) \\
&=c \odot (x_1,y_1) \oplus c\odot (x_2,y_2)
\end{align*}
\item \begin{align*} (c+d) \odot (x_1,y_1) &=
\left( (c+d)x_1-(c+d-1), (c+d)y_1-(c+d-1) \right) \\
&= \left( cx_1-(c-1), cy_1-(c-1) \right) \oplus (dx_1-(d-1), dy_1-(d-1) ) \\
&= c\odot (x_1,y_1) \oplus c \odot (x_2,y_2)
\end{align*}
\end{enumerate}
Therefore $V$ is a vector space.
\end{solution}

\begin{problem}{V2}
Determine if $\begin{bmatrix} 1 \\ 4 \\ 3 \end{bmatrix}$ is a linear combination of the vectors $\begin{bmatrix} 2 \\ 3 \\ -1 \end{bmatrix}$, $\begin{bmatrix} 1 \\ -1 \\ 0 \end{bmatrix}$, and $\begin{bmatrix} -3 \\ -2 \\ 5 \end{bmatrix}$.
\end{problem}
\begin{solution}
$$\RREF\left( \begin{bmatrix}[ccc|c] 2 & 1 & -3  & 1 \\ 3 & -1 & -2 & 4 \\ -1 & 0 & 5 & 3 \end{bmatrix} \right) = \begin{bmatrix}[ccc|c] 1 & 0 & 0 & 2 \\ 0 & 1 & 0 & 0 \\ 0 & 0 &  1 & 1 \end{bmatrix}$$
Since this system has a solution,  $\begin{bmatrix} 1 \\ 4 \\ 3 \end{bmatrix}$ is a linear combination of the three vectors.
\end{solution}

\begin{extract}\newpage\end{extract}
\begin{problem}{V3}
Does
\(
  \operatorname{span}\left\{
    \begin{bmatrix} 2 \\ -1 \\ 4 \end{bmatrix},
    \begin{bmatrix} 3 \\ 12 \\ -9 \end{bmatrix},
    \begin{bmatrix} 1 \\ 4 \\ -3 \end{bmatrix},
    \begin{bmatrix} -4 \\ 2 \\ -8 \end{bmatrix}
  \right\} = \IR^3
\)?
\end{problem}
\begin{solution}
Since
\[
  \RREF\begin{bmatrix}
    2 & 3 & 1 & -4 \\
    -1 & 12 & 4 & 2 \\
    4 & -9 & -3 & -8
  \end{bmatrix} =
  \begin{bmatrix}
    1 & 0 & 0 & -2 \\
    0 & 1 & 1/3 & 0 \\
    0 & 0 & 0 & 0
  \end{bmatrix}
\]
has a zero row, the vectors fail to span \(\IR^3\).
\end{solution}

\begin{problem}{V4} Let \(W\) be the set of all complex numbers \(a+bi\)
satisfying  \(a=2b\).
Determine if \(W\) is a subspace of \(\IC\).
\end{problem}
\begin{solution}
Yes, because \(c(2b_1+b_1i)+d(2b_2+b_2i)=2(cb_1+db_2)+(cb_1+db_2)i\) belongs to
\(W\). Alternately, yes because \(W\) is isomorphic to \(\IR\).
\end{solution}
\begin{extract}\newpage\end{extract}
\begin{problem}{S1}
Determine if the vectors $\begin{bmatrix} 1 \\ 1 \\ -1 \end{bmatrix}$, $\begin{bmatrix} 3 \\ -1 \\ 1 \end{bmatrix}$, and $\begin{bmatrix} 2 \\ 0 \\ -2 \end{bmatrix}$ are linearly dependent or linearly independent
\end{problem}
\begin{solution}
$$\RREF\left(\begin{bmatrix} 1 & 3 & 2 \\ 1 & -1 & 0 \\ -1 & 1 & -2 \end{bmatrix} \right)= \begin{bmatrix} 1 & 0 &0 \\ 0 & 1 & 0 \\ 0 & 0 & 1\end{bmatrix}$$
Since each column is a pivot column, the vectors are linearly independent.
\end{solution}


\begin{problem}{S2}
Determine if the set $\left\{\begin{bmatrix} 1 \\ 1 \\ -1 \end{bmatrix}, \begin{bmatrix} 3 \\ -1 \\ 1 \end{bmatrix},\begin{bmatrix} 2 \\ 0 \\ -2 \end{bmatrix}\right\}$ is a basis of $\IR^3$.
\end{problem}
\begin{solution}
$$\RREF\left(\begin{bmatrix} 1 & 3 & 2 \\ 1 & -1 & 0 \\ -1 & 1 & -2 \end{bmatrix} \right)= \begin{bmatrix} 1 & 0 &0 \\ 0 & 1 & 0 \\ 0 & 0 & 1\end{bmatrix}$$
Since the resulting matrix is the identity matrix, it is a basis.
\end{solution}


\begin{extract}\newpage\end{extract}
\begin{problem}{S3}
Let \(
  W={\rm span}\left\{
    \begin{bmatrix} 2 \\ 0 \\ 2 \\ 1 \end{bmatrix},
    \begin{bmatrix} 3 \\ 1 \\ -1 \\ 1 \end{bmatrix},
    \begin{bmatrix} 0 \\ 2 \\ -8 \\ -1 \end{bmatrix}
  \right\}
\). Find a basis for this vector space.
\end{problem}
\begin{solution}
\[
  \RREF\left(\begin{bmatrix}
    2 & 3 & 0 \\
    0 & 1 & 2 \\
    2 & -1 & -8 \\
    1 & 1 & -1
  \end{bmatrix} \right) =
  \begin{bmatrix}
    1 & 0 & -3 \\
    0 & 1 & 2 \\
    0 & 0 & 0 \\
    0 & 0 & 0
  \end{bmatrix}
\]
Thus \(\left\{
  \begin{bmatrix} 2 \\ 0 \\ 2 \\ 1 \end{bmatrix},
  \begin{bmatrix} 3 \\ 1 \\ -1 \\ 1 \end{bmatrix}
\right\}\) is a basis of $W$.
\end{solution}
\begin{problem}{S4}
Let $W = {\rm span} \left( \left\{  \begin{bmatrix} -3 \\ -8 \\ 0 \end{bmatrix}, \begin{bmatrix} 1 \\ 2 \\ 2 \end{bmatrix}, \begin{bmatrix} 0 \\ -1 \\ 3 \end{bmatrix} \right\} \right)$.  Compute the dimension of $W$.
\end{problem}
\begin{solution}
Let $A= \begin{bmatrix}-3 & 1 & 0 \\ -8 & 2 & -1 \\ 0 & 2 & 3\end{bmatrix}$, and compute $\RREF(A) = \begin{bmatrix} 1 & 0 & \frac{1}{2} \\ 0 & 1 & \frac{3}{2} \\ 0 & 0 & 0 \end{bmatrix}$.
Since there are two pivot columns, ${\rm dim}\ W = 2$.
\end{solution}


\begin{extract}\newpage\end{extract}
\begin{problem}{A1}
Let $T: \IR^3\rightarrow \IR^4$ be the linear transformation given by $$T\left(\begin{bmatrix} x \\ y \\ z \\  \end{bmatrix} \right) = \begin{bmatrix} -3x+y \\ -8x+2y-z \\ 7x+2y+3z \\ 0 \end{bmatrix}.$$  Write the matrix for $T$ with respect to the standard bases of $\IR^3$ and $\IR^4$.
\end{problem}
\begin{solution}
$$\begin{bmatrix} 3 & 1 & 0 \\ -8 & 2 & -1 \\ 7 & 2 & 3 \\ 0 & 0 & 0 \end{bmatrix}$$
\end{solution}

\begin{problem}{A2}
Determine if $D: \IR^{2\times 2} \rightarrow \IR$ given by $D\left(\begin{bmatrix} a & b \\ c & d \end{bmatrix} \right) = ad-bc$ is a linear transformation or not.
\end{problem}
\begin{solution}
$D(I)=1$ but $D(2I)=4 \neq 2D(I)$, so $D$ is not linear.
\end{solution}

\begin{extract}\newpage\end{extract}
\begin{problem}{A3}
Determine if each of the following linear transformations is injective (one-to-one) and/or surjective (onto).
\begin{enumerate}[(a)]
\item
  \(S: \IR^2 \rightarrow \IR^3\) where
  \(S(\vec e_1)=\begin{bmatrix}
    2 \\
    1 \\
    0
  \end{bmatrix}\) and
  \(S(\vec e_2)=\begin{bmatrix}
    1 \\
    2 \\
    1
  \end{bmatrix}\).
\item
  \(T: \IR^3 \rightarrow \IR^2\) where
  \(T(\vec e_1)=\begin{bmatrix}
    2 \\
    2
  \end{bmatrix}\),
  \(T(\vec e_2)=\begin{bmatrix}
   1  \\
   0
  \end{bmatrix}\), and
  \(T(\vec e_3)=\begin{bmatrix}
    1 \\
    1
  \end{bmatrix}\).
\end{enumerate}
\end{problem}
\begin{solution}
\begin{enumerate}[(a)]
\item
  \(\RREF\begin{bmatrix}
    2 & 1 \\
    1 & 2 \\
    0 & 1
  \end{bmatrix}=\begin{bmatrix}
    1 & 0 \\
    0 & 1 \\
    0 & 0
  \end{bmatrix}\).
  The map is injective since every column has a pivot, but is not surjective
  because there is a row without a pivot.
\item
  \(\RREF\begin{bmatrix}
    2 & 1 & 1 \\
    2 & 0 & 1
  \end{bmatrix}=\begin{bmatrix}
    1 & 0 & 1/2 \\
    0 & 1 & 1/2
  \end{bmatrix}\).
  The map is not injective since there is a column without a pivot,
  but it is surjective because every row has a pivot.
\end{enumerate}
\end{solution}

\begin{problem}{A4}
Let $T: \IR^4\rightarrow \IR^4$ be the linear transformation given by $$T\left(\begin{bmatrix} x \\ y \\ z \\ w \end{bmatrix} \right) = \begin{bmatrix} x+3y+3z+7w \\ x+3y-z-w \\ 2x+6y+3z+8w \\ x+3y-2z-3w \end{bmatrix}$$
Compute a basis for the kernel and a basis for the image of $T$.
\end{problem}
\begin{solution}

$$\RREF\left(\begin{bmatrix} 1 & 3 & 3 & 7 \\ 1 & 3 & -1 & -1 \\ 2 & 6 & 3 & 8 \\ 1 & 3 & -2 & -3 \end{bmatrix}\right) = \begin{bmatrix} 1 & 3 & 0 & 1 \\ 0 & 0 & 1 & 2 \\ 0 & 0 & 0 & 0 \end{bmatrix}$$

Then a basis for the kernel is
$$\left\{ \begin{bmatrix} -3 \\ 1 \\ 0 \\ 0 \end{bmatrix} , \begin{bmatrix} -1\\0 \\ -2 \\ 1 \end{bmatrix} \right\}$$
and a basis for the image is
$$\left\{ \begin{bmatrix} 1 \\ 1 \\ 2 \\ 1 \end{bmatrix} , \begin{bmatrix} 3 \\ -1 \\ 3 \\ -2 \end{bmatrix}\right\} $$
\end{solution}

\begin{extract}\newpage\end{extract}
\begin{problem}{M1}
Let 
\begin{align*}
A &= \begin{bmatrix} 1 & 3 & -1 & -1 \\ 0 & 0 & 7 & 2 \end{bmatrix} & B &= \begin{bmatrix} 0 & 1 & 7 & 7 \\ -1 & -2 & 0 & 4 \\ 0 & 0 & 1 & 5 \end{bmatrix} & C&=\begin{bmatrix} 3 & 2 \\ 0 & 1 \\ -2 & -1 \end{bmatrix}
\end{align*}
Exactly one of the six products $AB$, $AC$, $BA$, $BC$, $CA$, $CB$ can be computed.  Determine which one, and compute it.
\end{problem}
\begin{solution}
$CA$ is the only one that can be computed, and 
$$CA = \begin{bmatrix} 3 & 9 & 11 & 1 \\ 0 & 0 & 7 & 2 \\ -2 & - 6 & -5 & 0 \end{bmatrix}$$
\end{solution}

\begin{problem}{M2} Determine if the matrix $\begin{bmatrix} 1 & 3 & -1 \\ 2 & 7 & 0 \\ -1 & -1 & 5 \end{bmatrix}$ is invertible.
\end{problem}
\begin{solution}
$$\RREF\begin{bmatrix} 1 & 3 & -1 \\ 2 & 7 & 0 \\ -1 & -1 & 5 \end{bmatrix} =\begin{bmatrix} 1 & 0 & -7 \\ 0 & 1 & 2 \\ 0 & 0 & 0 \end{bmatrix}$$ 
Since it is not equivalent to the identity matrix, it is not invertible.
\end{solution}


\begin{extract}\newpage\end{extract}
\begin{problem}{M3}
  Find the inverse of the matrix
  \(\begin{bmatrix}
    4 & -1 & -8  \\
    2 & 1 & 3  \\
    1 & 1 & 4
  \end{bmatrix}\).
\end{problem}
\begin{solution}
\(\begin{bmatrix}[ccc|ccc]
  4 & -1 & -8 & 1 & 0 & 0  \\
  2 & 1 & 3   & 0 & 1 & 0 \\
  1 & 1 & 4   & 0 & 0 & 1
\end{bmatrix}\sim\begin{bmatrix}[ccc|ccc]
  1 & 0 & 0 & 1 & -4 & 5 \\
  0 & 1 & 0 & -5 & 24 & -28 \\
  0 & 0 & 1 & 1 & -5 & 6
\end{bmatrix}\). Thus the inverse is
\(\begin{bmatrix}
  1 & -4 & 5  \\
  -5 & 24 & -28  \\
  1 & -5 & 6
\end{bmatrix}\).
\end{solution}


\begin{problem}{G1}
Compute the determinant of the matrix $\begin{bmatrix} 8 & 5 & 3 & 0 \\ 3 & 2 & 1 & 1 \\ 5 & -3 & 1 & -2 \\ -1 & 2 & 0 & 1\end{bmatrix} $.
\end{problem}
\begin{solution}
$-1$.
\end{solution}

\begin{extract}\newpage\end{extract}
\begin{problem}{G2}
Compute the eigenvalues, along with their algebraic multiplicities, of the matrix $ \begin{bmatrix} 9 & -3 & 2 \\ 23 & -8 & 5 \\  -1 & 0 & 0 \end{bmatrix}$.
\end{problem}
\begin{solution}
1 (with algebraic multiplicy 2), and -1 (with algebraic multiplicity 1).
\end{solution}

\begin{problem}{G3}
Find the eigenspace associated to the eigenvalue $1$ in the matrix $A=\begin{bmatrix} -3 & 1 & 0 \\ -8 & 2 & -1 \\ 0 & 2 & 3 \end{bmatrix}$
\end{problem}
\begin{solution}
The eigenspace is spanned by $\begin{bmatrix} -\frac{1}{4} \\ -1 \\ 1 \end{bmatrix}$.
\end{solution}

\begin{extract}\newpage\end{extract}
\begin{problem}{G4}
Compute the geometric multiplicity of the eigenvalue $2$ in the matrix $A=\begin{bmatrix} 0 & 1 & 0 & 0 \\ -4 & 4 & 0 & 0 \\ 11 & -6 & 1 & -1 \\ -9 & 5 & 1 & 3 \end{bmatrix}$.
\end{problem}
\begin{solution}
The eigenspace is spanned by $\begin{bmatrix} -1 \\ -2 \\ 1 \\ 0 \end{bmatrix}$ and $\begin{bmatrix} -1 \\ -2 \\ 0 \\ 1 \end{bmatrix}$, so the geometric multiplicity is $2$.
\end{solution}

\end{document}