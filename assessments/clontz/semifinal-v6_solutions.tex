\documentclass{sbgLAsemi}

\begin{extract*}
\usepackage{amsmath,amssymb,amsthm,enumerate}
\coursetitle{Math 237}
\courselabel{Linear Algebra}
\calculatorpolicy{You may use a calculator, but you must show all relevant work to receive credit for a standard.}

\newcommand{\IR}{\mathbb{R}}

\makeatletter
\renewcommand*\env@matrix[1][*\c@MaxMatrixCols c]{%
  \hskip -\arraycolsep
  \let\@ifnextchar\new@ifnextchar
  \array{#1}}
\makeatother


\title{Semifinal}
\version{6}
\end{extract*}

\begin{document}

\begin{problem}{E1}
Write a system of linear equations corresponding to the following
augmented matrix.
\[
\begin{bmatrix}[ccc|c]
-4 & -1 & 3 & 2  \\
1 & 2 & -1 & 0 \\
-1 & 4 & 1 & 4
\end{bmatrix}
\]
\end{problem}
\begin{solution}
\begin{align*}
-4x_1-x_2+3x_3&=2 \\
x_1+2x_2-x_3 &= 0 \\
-x_1+4x_2+x_3&=4
\end{align*}
\end{solution}

\begin{problem}{E2}
Find the reduced row echelon form of the matrix below.
$$\begin{bmatrix}[cccc|c] 2 & 1 & -1 & 0 & 5 \\ 3 & -1 & 0 & -2 & 0 \\ -1 & 0 & 5 & 0 & -1 \end{bmatrix}$$
\end{problem}
\begin{solution}
$$\begin{bmatrix}[cccc|c]
 2 & 1 & -1 & 0 & 5 \\
 3 & -1 & 0 & -2 & 0 \\
 -1 & 0 & 5 & 0 & -1
\end{bmatrix} \sim
\begin{bmatrix}[cccc|c]
 -1 & 0 & 5 & 0 & -1  \\
 2 & 1 & -1 & 0 & 5 \\
 3 & -1 & 0 & -2 & 0
\end{bmatrix} \sim
\begin{bmatrix}[cccc|c]
 1 & 0 & -5 & 0 & 1 \\
 2 & 1 & -1 & 0 & 5 \\
 3 & -1 & 0 & -2 & 0
\end{bmatrix} $$
$$\sim
\begin{bmatrix}[cccc|c]
 1 & 0 & -5 & 0 & 1  \\
 0 & 1 & 9 & 0 & 3 \\
 0 & -1 & 15 & -2 & -3 \\
\end{bmatrix} \sim
\begin{bmatrix}[cccc|c]
 1 & 0 & -5 & 0 & 1  \\
 0 & 1 & 9 & 0 & 3 \\
 0 & 0 & 24 & -2 & 0 \\
\end{bmatrix} \sim
\begin{bmatrix}[cccc|c]
 1 & 0 & -5 & 0 & 1 \\
 0 & 1 & 9 & 0 & 3 \\
 0 & 0 & 1 & -\frac{1}{12} & 0 \\
\end{bmatrix} \sim
\begin{bmatrix}[cccc|c]
1 & 0 & 0 & -\frac{5}{12} & 1 \\
 0 & 1 & 0 & \frac{3}{4} & 3 \\
 0 & 0 & 1 & -\frac{1}{12} & 0
\end{bmatrix}$$
\end{solution}

\begin{problem}{E3}
Solve the system of equations
\begin{align*}
-3x +y &= 2\\
-8x+2y-z &= 6 \\
2y+3z &= -2
\end{align*}


\end{problem}

\begin{solution}
$$\RREF\left(\begin{bmatrix}[ccc|c] -3 & 1 & 0 & 2 \\ -8 & 2 & -1 & 6 \\ 0 & 2 & 3 & -2 \end{bmatrix} \right) = \begin{bmatrix}[ccc|c] 1 & 0 & \frac{1}{2} & -1 \\ 0 & 1 & \frac{3}{2} & -1 \\ 0 & 0 & 0 & 0 \end{bmatrix}$$
The solutions are $$\left\{ \begin{bmatrix} -1-\frac{c}{2} \\ -1-\frac{3c}{2} \\ c \end{bmatrix}\ \bigg|\ c\in \IR\right\} = \left\{ \begin{bmatrix} c-1 \\ 3c-1 \\ -2c  \end{bmatrix}\ \bigg|\ c\in \IR\right\}$$
\end{solution}

\begin{problem}{E4}
Find a basis for the solution set of the system of equations
\begin{align*}
x+2y+3z+w &= 0 \\
3x-y+z+w &= 0 \\
2x-3y-2z &= 0
\end{align*}
\end{problem}
\begin{solution}
$$\RREF \left(\begin{bmatrix} 1 & -2 & 3 & 1 \\ 3 & -1 & 1 & 1 \\ 2 & -3 & -2 & 0  \end{bmatrix} \right) = \begin{bmatrix} 1 & 0 & \frac{5}{7} & \frac{3}{7} \\ 0 & 1 & \frac{8}{7} & \frac{2}{7} \\ 0 & 0 & 0 & 0\end{bmatrix}$$
Then the solution set is
$$\left\{ \begin{bmatrix} -\frac{5}{7}a-\frac{3}{7}b \\ -\frac{8}{7}a-\frac{2}{7}b \\ a \\ b \end{bmatrix} \bigg |\ a,b \in \IR \right\}$$
So a basis for the solution set is $\left\{\begin{bmatrix} -\frac{5}{7} \\ -\frac{8}{7} \\ 1 \\ 0\end{bmatrix}, \begin{bmatrix} - \frac{3}{7} \\ -\frac{2}{7} \\ 0 \\ 1 \end{bmatrix} \right\}$, or $\left\{\begin{bmatrix} 5 \\ 8 \\ -7 \\ 0 \end{bmatrix}, \begin{bmatrix} 3 \\ 2 \\ 0 \\ -7 \end{bmatrix}\right\}$.
\end{solution}

\begin{problem}{V1}
Let $V$ be the set of all points on the parabola $y=x^2$ with the operations, for any $(x_1,y_1), (x_2,y_2) \in V$, $c \in \IR$,
\begin{align*}
(x_1,y_1) \oplus (x_2,y_2) &= (x_1+x_2,y_1+y_2+2x_1x_2) \\
c \odot (x_1,y_1) &= (cx_1, c^2y_1)
\end{align*}
\begin{enumerate}[(a)]
\item Show that the vector \textbf{addition} $\oplus$ is \textbf{associative}:
      \((x_1,y_1) \oplus ((x_2,y_2) \oplus (x_3,y_3))=((x_1,y_1)\oplus (x_2,y_2))\oplus (x_3,y_3)\).
\item Determine if $V$ is a vector space or not.  Justify your answer.
\end{enumerate}
\end{problem}

\begin{problem}{V2}
  Determine if
  \(\begin{bmatrix} 0 \\ -1 \\ 6 \\ -7 \end{bmatrix}\)
  belongs to the span of the set
  \(\left\{
    \begin{bmatrix} 2 \\ 0 \\ -1 \\ 5 \end{bmatrix},
    \begin{bmatrix} 4 \\ -1 \\ 4 \\ 3 \end{bmatrix}
    \right\}
  \).
\end{problem}
\begin{solution}
  Since
  \[
    \RREF\left(
      \begin{bmatrix}[cc|c]
        2 & 4 & 0 \\
        0 & -1 & -1 \\
        -1 & 4 & 6 \\
        5 & 3 & -7
      \end{bmatrix}
    \right) =
    \begin{bmatrix}[cc|c]
      1 & 0 & -2 \\
      0 & 1 & 1 \\
      0 & 0 & 0 \\
      0 & 0 & 0
    \end{bmatrix}
  \]
  does not contain a contradiction,
  \(\begin{bmatrix} 0 \\ -1 \\ 6 \\ -7 \end{bmatrix}\) is
  a linear combination of the three vectors.
\end{solution}


\begin{problem}{V3}
Determine if the vectors  $\begin{bmatrix} 8 \\ 21 \\ -7 \end{bmatrix}$, $\begin{bmatrix} -3 \\ -8 \\ 3 \end{bmatrix}$, $\begin{bmatrix} -1 \\ -3 \\ 2 \end{bmatrix}$, and $\begin{bmatrix} 4 \\ 11 \\ -5 \end{bmatrix}$ span $\IR^3$.
\end{problem}
\begin{solution}
$$\RREF\left(\begin{bmatrix} 8 & -3 & -1 & 4 \\ 21 & -8 & -3 & 11 \\ -7 & 3 & 2 & -5  \end{bmatrix} \right) = \begin{bmatrix} 1 & 0 & 1 & -1 \\ 0 & 1 & 3 & -4 \\ 0 & 0 & 0 & 0\end{bmatrix} $$
Since the rank is less than 3, they do not span $\IR^3$.
\end{solution}

\begin{problem}{V4} Let \(W\) be the set of all \(\IR^3\) vectors
\(\begin{bmatrix} x \\ y \\ z \end{bmatrix}\) satisfying \(x+y+z=0\) (this forms a plane).
Determine if \(W\) is a subspace of \(\IR^3\).
\end{problem}
\begin{solution}
Yes, because \(z=-x-y\) and
\(
  a\begin{bmatrix} x_1 \\ y_1 \\ -x_1-y_1 \end{bmatrix}+
  b\begin{bmatrix} x_2 \\ y_2 \\ -x_2-y_2 \end{bmatrix}=
  \begin{bmatrix}
    ax_1+bx_2 \\
    ay_1+by_2 \\
    -(ax_1+bx_2)-(ay_1+by_2)
  \end{bmatrix}
\).
Alternately, yes because \(W\) is isomorphic to \(\IR^2\).
\end{solution}


\begin{problem}{S1}
Determine if the set of vectors $\left\{\begin{bmatrix} 3 \\ -1 \\ 0 \\ 4 \end{bmatrix}, \begin{bmatrix} 1  \\ 2 \\ -2 \\ 1 \end{bmatrix}, \begin{bmatrix} 3 \\ -8 \\ 6 \\ 5 \end{bmatrix} \right\}$  is linearly dependent or linearly independent.
\end{problem}
\begin{solution}
$$\RREF\left(\begin{bmatrix} 3 & 1 & 3 \\ -1 & 2 & -8 \\ 0 & -2 & 6 \\ 4 & 1 & 5 \end{bmatrix} \right) = \begin{bmatrix} 1 & 0 & 2 \\ 0 & 1 & -3 \\ 0 & 0 & 0 \\ 0 & 0 & 0 \end{bmatrix}$$
Since the reduced row echelon form has a nonpivot column, the vectors are linearly dependent.
\end{solution}

\begin{problem}{S2}
  Determine if the set \(\left\{
    \begin{bmatrix} 3 \\ -1 \\ 2 \end{bmatrix},
    \begin{bmatrix} 2 \\ 0 \\ 2 \end{bmatrix},
    \begin{bmatrix} 1 \\ 4 \\ -1 \end{bmatrix}
  \right\}\) is a basis of $\IR^3$.
\end{problem}
\begin{solution}
  \[\RREF\left(
    \begin{bmatrix}
      3 & 2 & 1 \\
      -1 & 0 & 4 \\
      2 & 2 & -1
    \end{bmatrix} \right)= \begin{bmatrix}
      1 & 0 &0 \\
      0 & 1 & 0 \\
      0 & 0 & 1
    \end{bmatrix}
  \]
Since the resulting matrix is the identity matrix, it is a basis.
\end{solution}


\begin{problem}{S3}
Let \(
  W={\rm span}\left\{
    \begin{bmatrix} 2 \\ 0 \\ 2 \\ 1 \end{bmatrix},
    \begin{bmatrix} 3 \\ 1 \\ -1 \\ 1 \end{bmatrix},
    \begin{bmatrix} 0 \\ 2 \\ -8 \\ -1 \end{bmatrix}
  \right\}
\). Find a basis for this vector space.
\end{problem}
\begin{solution}
\[
  \RREF\left(\begin{bmatrix}
    2 & 3 & 0 \\
    0 & 1 & 2 \\
    2 & -1 & -8 \\
    1 & 1 & -1
  \end{bmatrix} \right) =
  \begin{bmatrix}
    1 & 0 & -3 \\
    0 & 1 & 2 \\
    0 & 0 & 0 \\
    0 & 0 & 0
  \end{bmatrix}
\]
Thus \(\left\{
  \begin{bmatrix} 2 \\ 0 \\ 2 \\ 1 \end{bmatrix},
  \begin{bmatrix} 3 \\ 1 \\ -1 \\ 1 \end{bmatrix}
\right\}\) is a basis of $W$.
\end{solution}
\begin{problem}{S4}
Let $W = {\rm span} \left( \left\{ \begin{bmatrix} 1 \\ 1 \\ 2 \\ 1 \end{bmatrix}, \begin{bmatrix} 3 \\ 3 \\ 6 \\ 3 \end{bmatrix}, \begin{bmatrix} 3 \\ -1 \\ 3 \\ -2 \end{bmatrix}, \begin{bmatrix} 7 \\ -1 \\ 8 \\ -3 \end{bmatrix} \right\} \right)$.  Find the dimension of $W$.
\end{problem}
\begin{solution}
$$\RREF\left(\begin{bmatrix} 1 & 3 & 3 & 7 \\ 1 & 3 & -1 & -1 \\ 2 & 6 & 3 & 8 \\ 1 & 3 & -2 & -3 \end{bmatrix}\right) = \begin{bmatrix} 1 & 3 & 0 & 1 \\ 0 & 0 & 1 & 2 \\ 0 & 0 & 0 & 0 \\  0 & 0 & 0 & 0 \end{bmatrix}$$

This has two pivot columns, so $W$ has dimension 2.
\end{solution}


\begin{problem}{A1}
Let $T: \IR^4 \rightarrow \IR^2$ be the linear transformation given by $$T\left(\begin{bmatrix} x_1 \\ x_2 \\ x_3 \\ x_4 \end{bmatrix} \right) = \begin{bmatrix} x_1+3x_3 \\ 3x_2-x_3 \end{bmatrix}.$$ Write the matrix for $T$ with respect to the standard bases of $\IR^4$ and $\IR^2$.
\end{problem}
\begin{solution}
$$\begin{bmatrix} 1 & 0 & 3 & 0 \\ 0 & 3 & -1 & 0 \end{bmatrix}$$
\end{solution}


\begin{problem}{A2}
Determine if $D: \IR^{2\times 2} \rightarrow \IR$ given by $D\left(\begin{bmatrix} a & b \\ c & d \end{bmatrix} \right) = ad-bc$ is a linear transformation or not.
\end{problem}
\begin{solution}
$D(I)=1$ but $D(2I)=4 \neq 2D(I)$, so $D$ is not linear.
\end{solution}

\begin{problem}{A3}
Determine if the following linear maps are injective (one-to-one) and/or surjective (onto).
\begin{enumerate}[(a)]
\item $S: \IR^2 \rightarrow \IR^3$ given by $S\left(\begin{bmatrix} x \\ y  \end{bmatrix} \right) = \begin{bmatrix} 3x+2y \\ x-y \\ x+4y \end{bmatrix} $
\item $T: \IR^3 \rightarrow \IR^3$ given by $T\left(\begin{bmatrix} x \\ y \\ z  \end{bmatrix} \right) = \begin{bmatrix} x+y+z \\ 2y+3z \\ x-y-2z \end{bmatrix} $
\end{enumerate}
\end{problem}

\begin{solution}
\begin{enumerate}[(a)]
\item $$\RREF\left( \begin{bmatrix} 1 &  1 & 1 \\ 0  & 2 & 3 \\ 1  & -1 & -2 \end{bmatrix} \right) = \begin{bmatrix} 1 &  0 & -\frac{1}{2} \\ 0  & 1 & \frac{3}{2} \\ 0& 0 & 0  \end{bmatrix}$$
Since there is a nonpivot column, $T$ is not injective.  Since there is a zero row, $T$ is not surjective.
\item $$\RREF \left( \begin{bmatrix} 3 & 2 \\ 1 & -1 \\ 1 & 4 \end{bmatrix} \right) = \begin{bmatrix} 1 & 0 \\ 0 & 1 \\ 0 & 0 \end{bmatrix}$$
Since all columns are pivot columns, $S$ is injective.  Since there is a zero row, $S$ is not surjective.
\end{enumerate}
\end{solution}



\begin{problem}{A4}
Let $T: \P^3 \rightarrow \P^3$ be the linear transformation given by $$T\left( ax^3+bx^2+cx+d \right)  = (a+3b+3c+7d)x^3 + (a+3b-c-d)x^2+  (2a+6b+3c+8d)x+  (a+3b-2c-3d)$$
Compute a basis for the kernel and a basis for the image of $T$.
\end{problem}
\begin{solution}

$$\RREF\left(\begin{bmatrix} 1 & 3 & 3 & 7 \\ 1 & 3 & -1 & -1 \\ 2 & 6 & 3 & 8 \\ 1 & 3 & -2 & -3 \end{bmatrix}\right) = \begin{bmatrix} 1 & 3 & 0 & 1 \\ 0 & 0 & 1 & 2 \\ 0 & 0 & 0 & 0 \end{bmatrix}$$

Then a basis for the kernel is
$$\left\{ -3x^3+x^2, -x^3-2x+1 \right\}$$
and a basis for the image is
$$\left\{ x^3+x^2+2x+1 , 3x^3-x^2+3x-2\right\} $$
\end{solution}



\begin{problem}{M1}
Let 
\begin{align*}
A &= \begin{bmatrix} 2 & 3 \\ 0 & 1 \end{bmatrix} & B&= \begin{bmatrix} 3 & 1 & 0 \end{bmatrix} & C&= \begin{bmatrix} 0 & -1 & 4 \\ 1 & -1 & 2 \end{bmatrix}
\end{align*}

Exactly one of the six products $AB$, $AC$, $BA$, $BC$, $CA$, $CB$ can be computed.  Determine which one, and compute it.
\end{problem}
\begin{solution}
$AC$ is the only one that can be computed, and 
$$AC = \begin{bmatrix} 3 & -5 & 11 \\ 1 & -1 & 2 \end{bmatrix}$$
\end{solution}

\begin{problem}{M2}
Determine if the matrix $\begin{bmatrix} 2 & 1 & 0 & 3 \\ 1 & -1 & 3 & 1 \\ 3 & 2 & -1 & 7 \\ 4 & 1 & 2 & 0 \end{bmatrix}$ is invertible.
\end{problem}
\begin{solution}
$$\RREF \begin{bmatrix} 2 & 1 & 0 & 3 \\ 1 & -1 & 3 & 1 \\ 3 & 2 & -1 & 7 \\ 4 & 1 & 2 & 0 \end{bmatrix}=\begin{bmatrix} 1 & 0 & 1 & 0 \\ 0 & 1 & -2 & 0 \\ 0 & 0 & 0 & 1 \\ 0 & 0 & 0 & 0 \end{bmatrix}$$
Since it is not row equivalent to the identity matrix, it is not invertible.
\end{solution}

\begin{problem}{M3}
  Find the inverse of the matrix
  \(\begin{bmatrix}
    4 & -1 & -8  \\
    2 & 1 & 3  \\
    1 & 1 & 4
  \end{bmatrix}\).
\end{problem}
\begin{solution}
\(\begin{bmatrix}[ccc|ccc]
  4 & -1 & -8 & 1 & 0 & 0  \\
  2 & 1 & 3   & 0 & 1 & 0 \\
  1 & 1 & 4   & 0 & 0 & 1
\end{bmatrix}\sim\begin{bmatrix}[ccc|ccc]
  1 & 0 & 0 & 1 & -4 & 5 \\
  0 & 1 & 0 & -5 & 24 & -28 \\
  0 & 0 & 1 & 1 & -5 & 6
\end{bmatrix}\). Thus the inverse is
\(\begin{bmatrix}
  1 & -4 & 5  \\
  -5 & 24 & -28  \\
  1 & -5 & 6
\end{bmatrix}\).
\end{solution}


\begin{problem}{G1}
Compute the determinant of the matrix
\[
  \begin{bmatrix}
    1 & 3 & 2 & 4 \\
    5 & 0 & -4 & 0 \\
    -2 & 3 & -1 & 1 \\
    0 & 1 & 0 & 1
  \end{bmatrix}
.\]
\end{problem}
\begin{solution}
\(15\).
\end{solution}

\begin{problem}{G2}
Let $A= \begin{bmatrix}-3 & 1 & 0 \\ -8 & 2 & -1 \\ 0 & 2 & 3\end{bmatrix}$.
List the eigenvalues of $A$ along with their algebraic multiplicities.
\end{problem}
\begin{solution}

\begin{align*}
\det(A-\lambda I) &= \det \begin{bmatrix} -3-\lambda & 1 & 0 \\ -8 & 2-\lambda & -1 \\ 0 & 2 & 3-\lambda \end{bmatrix} \\
&=(-3-\lambda) \det \begin{bmatrix} 2-\lambda & -1 \\ 2 & 3-\lambda \end{bmatrix} -(1) \det \begin{bmatrix} -8 & -1 \\ 0 & 3-\lambda \end{bmatrix} \\
&=(-3-\lambda)\left( (2-\lambda)(3-\lambda)+2 \right)-\left(-8(3-\lambda) \right) \\
&=(-3-\lambda)(8-5\lambda+\lambda ^2) +24-8\lambda \\
&=-\lambda ^3 +2\lambda ^2+7\lambda -24 +24-8\lambda \\
&= -\lambda ^3+2\lambda ^2 - \lambda \\
&= -\lambda (\lambda ^2-2\lambda +1 ) \\
&= -\lambda(\lambda-1)^2
\end{align*}
So $A$ has eigenvalues $0$ (with multiplicity 1) and $1$ (with algebraic multiplicity 2).
\end{solution}


\begin{problem}{G3}
Find the eigenspace associated to the eigenvalue $2$ in the matrix $A=\begin{bmatrix}0 & -2 & -1 & 0 \\ -4 & -2 & -2 & 0 \\ 14 & 12 & 10 & 2 \\ -13 & -10 & -8 & -1 \end{bmatrix}$.
\end{problem}
\begin{solution}
The eigenspace is spanned by $\begin{bmatrix} -1 \\ \frac{1}{2} \\ 1 \\ 0 \end{bmatrix}$ and  $\begin{bmatrix} -1 \\ 1 \\ 0 \\ 1 \end{bmatrix}$.
\end{solution}

\begin{problem}{G4}
Compute the geometric multiplicity of the eigenvalue $1$ in the matrix $A=\begin{bmatrix} 8 & -3 & -1 \\ 21 & -8 & -3 \\ -7 & 3  & 2 \end{bmatrix}$
\end{problem}
\begin{solution}
The eigenspace is spanned by $\begin{bmatrix} \frac{3}{7} \\ 1 \\ 0 \end{bmatrix}$ and $\begin{bmatrix}\frac{1}{7} \\ 0 \\ 1 \end{bmatrix}$, so the geometric multiplicity is $2$.
\end{solution}

\end{document}