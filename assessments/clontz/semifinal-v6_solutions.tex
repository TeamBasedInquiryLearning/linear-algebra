\documentclass{sbgLAsemi}

\begin{extract*}
\usepackage{amsmath,amssymb,amsthm,enumerate}
\coursetitle{Math 237}
\courselabel{Linear Algebra}
\calculatorpolicy{You may use a calculator, but you must show all relevant work to receive credit for a standard.}

\newcommand{\IR}{\mathbb{R}}

\makeatletter
\renewcommand*\env@matrix[1][*\c@MaxMatrixCols c]{%
  \hskip -\arraycolsep
  \let\@ifnextchar\new@ifnextchar
  \array{#1}}
\makeatother


\title{Semifinal}
\version{6}
\end{extract*}

\begin{document}

\begin{problem}{E1}
Write a system of linear equations corresponding to the following
augmented matrix.
\[
\begin{bmatrix}[cccc|c]
3 & -1 & 0 & 1 & 5 \\
-1 & 9 & 1 & -7 & 0 \\
1 & 0 & -1 & 0 & -3
\end{bmatrix}
\]
\end{problem}
\begin{solution}
\begin{align*}
3x_1-x_2+x_4 &= 5 \\
-x_1+9x_2+x_3-7x_4 &= 0 \\
x_1-x_3 &= -3
\end{align*}
\end{solution}

\begin{problem}{E2}
Find \(\RREF A\), where
\[
  A =
  \begin{bmatrix}[ccc|c]
    2 & -1 & 5 & 4 \\
    -1 & 0 & -2 & -1 \\
    1 & 3 & -1 & -5
  \end{bmatrix}
\]
\end{problem}
\begin{solution}
\[
  \RREF A =
  \begin{bmatrix}[ccc|c]
    1 & 0 & 2 & 1 \\
    0 & 1 & -1 & -2 \\
    0 & 0 & 0 & 0
  \end{bmatrix}
\]
\end{solution}

\begin{problem}{E3}
Solve the following linear system.
\begin{align*}
3x+2y+z &= 7 \\
x+y+z &= 1 \\
-2x+3z &= -11
\end{align*}
\end{problem}
\begin{solution}
Let \(A =
  \begin{bmatrix}[ccc|c]
    3 & 2 & 1 & 7 \\
    1 & 1 & 1 & 1 \\
    -2 & 0 & 3 & 11
  \end{bmatrix}
\), so \(\RREF A =
  \begin{bmatrix}[ccc|c]
    1 & 0 & 0 & 4 \\
    0 & 1 & 0 & -2 \\
    0 & 0 & 1 & -1
  \end{bmatrix}
\). It follows that the system has exactly one solution:
\(\begin{bmatrix}
  4 & -2 & -1
\end{bmatrix}\)
\end{solution}
\begin{problem}{E4}
Find a basis for the solution set of the system of equations
\begin{align*}
x+2y+3z+w &= 0 \\
3x-y+z+w &= 0 \\
2x-3y-2z &= 0 \\
-x+2z+5w &=0
\end{align*}
\end{problem}
\begin{solution}
$$\RREF \left(\begin{bmatrix} 1 & -2 & 3 & 1 \\ 3 & -1 & 1 & 1 \\ 2 & -3 & -2 & 0 \\ -1 & 0 & 2 & 5 \end{bmatrix} \right) = \begin{bmatrix} 1 & 0 & 0 & -1 \\ 0 & 1 & 0 & -2 \\ 0 & 0 & 1 & 2 \\ 0 & 0 & 0 & 0\end{bmatrix}$$
Then the solution set is
$$\left\{ \begin{bmatrix} a \\ 2a \\ -2a \\ a \end{bmatrix} \bigg |\ a \in \IR \right\}$$
So a basis for the solution set is $\left\{\begin{bmatrix} 1 \\ 2 \\ -2 \\ 1\end{bmatrix} \right\}$.
\end{solution}


\begin{problem}{V1}
Let $V$ be the set of all pairs of real numbers with the operations, for any $(x_1,y_1), (x_2,y_2) \in V$, $c \in \IR$,
\begin{align*}
(x_1,y_1) \oplus (x_2,y_2) &= (x_1+x_2,y_1+y_2) \\
c \odot (x_1,y_1) &= (c^2x_1, c^3y_1)
\end{align*}
\begin{enumerate}[(a)]
\item Show that scalar multiplication \textbf{distributes scalars} over
      vector addition:
      \(c\odot((x_1,y_1) \oplus (x_2,y_2))=
      c\odot(x_1,y_1) \oplus c\odot(x_2,y_2)\).
\item Determine if $V$ is a vector space or not.  Justify your answer.
\end{enumerate}
\end{problem}
\begin{solution}
Let $(x_1,y_1), (x_2,y_2) \in V$ and let $c \in \IR$.
\begin{align*}
c \odot \left( (x_1,y_1) \oplus (x_2,y_2) \right) &= c \odot (x_1+x_2,y_1+y_2) \\
&= (c^2(x_1+x_2), c^3(y_1+y_2) ) \\
&= (c^2x_1,c^3y_1) \oplus (c^2x_2,c^3y_2) \\
&= c \odot (x_1,y_1) \oplus c \odot (x_2,y_2)
\end{align*}
However, $V$ is not a vector space, as the other distributive law fails:
$$(c+d) \odot (x_1,y_1) = ( (c+d)^2 x_1, (c+d)^3 y_1) \neq ((c^2+d^2)x_1, (c^3+d^3)y_1) = c \odot (x_1,y_1) \oplus d \odot (x_1,y_1).$$
\end{solution}


\begin{problem}{V2}
  Determine if
  \(\begin{bmatrix} 0 \\ -1 \\ 6 \\ -7 \end{bmatrix}\)
  belongs to the span of the set
  \(\left\{
    \begin{bmatrix} 2 \\ 0 \\ -1 \\ 5 \end{bmatrix},
    \begin{bmatrix} 4 \\ -1 \\ 4 \\ 3 \end{bmatrix}
    \right\}
  \).
\end{problem}
\begin{solution}
  Since
  \[
    \RREF\left(
      \begin{bmatrix}[cc|c]
        2 & 4 & 0 \\
        0 & -1 & -1 \\
        -1 & 4 & 6 \\
        5 & 3 & -7
      \end{bmatrix}
    \right) =
    \begin{bmatrix}[cc|c]
      1 & 0 & -2 \\
      0 & 1 & 1 \\
      0 & 0 & 0 \\
      0 & 0 & 0
    \end{bmatrix}
  \]
  does not contain a contradiction,
  \(\begin{bmatrix} 0 \\ -1 \\ 6 \\ -7 \end{bmatrix}\) is
  a linear combination of the three vectors.
\end{solution}


\begin{problem}{V3}
Determine if the vectors  $\begin{bmatrix} 8 \\ 21 \\ -7 \end{bmatrix}$, $\begin{bmatrix} -3 \\ -8 \\ 3 \end{bmatrix}$, $\begin{bmatrix} -1 \\ -3 \\ 2 \end{bmatrix}$, and $\begin{bmatrix} 4 \\ 11 \\ -5 \end{bmatrix}$ span $\IR^3$.
\end{problem}
\begin{solution}
$$\RREF\left(\begin{bmatrix} 8 & -3 & -1 & 4 \\ 21 & -8 & -3 & 11 \\ -7 & 3 & 2 & -5  \end{bmatrix} \right) = \begin{bmatrix} 1 & 0 & 1 & -1 \\ 0 & 1 & 3 & -4 \\ 0 & 0 & 0 & 0\end{bmatrix} $$
Since the rank is less than 3, they do not span $\IR^3$.
\end{solution}

\begin{problem}{V4} Let $W$ be the set of all polynomials of the form
\(ax^3+bx\).  Determine if $W$ is a subspace of \(\P^3\).
\end{problem}
\begin{solution}
Yes because \(s(a_1x^3+b_1x)+t(a_2x^3+b_2x)=
(sa_1+ta_2)x^3+(sb_1+tb_2)x\) also belongs to \(W\).
Alternately, yes because \(W\) is isomorphic to \(\IR^2\).
\end{solution}


\begin{problem}{S1}
Determine if the set of polynomials $\left\{ -3x^3-8x^2, x^3+2x^2+2, -x^2+3\right\}$ is  linearly dependent or linearly independent
\end{problem}
\begin{solution}
$$\RREF\left( \begin{bmatrix}-3 & 1 & 0 \\ -8 & 2 & -1 \\ 0 & 0 & 0 \\ 0 & 2 & 3 \end{bmatrix}\right) = \begin{bmatrix} 1 & 0 & \frac{1}{2} \\ 0 & 1 & \frac{3}{2} \\ 0 & 0 & 0 \\ 0 & 0 & 0 \end{bmatrix}$$
This has a non pivot column, therefore the set is linearly dependent.
\end{solution}


\begin{problem}{S2}
  Determine if the set \(\left\{
    \begin{bmatrix} 3 \\ -1 \\ 2 \end{bmatrix},
    \begin{bmatrix} 2 \\ 0 \\ 2 \end{bmatrix},
    \begin{bmatrix} 1 \\ 4 \\ -1 \end{bmatrix}
  \right\}\) is a basis of $\IR^3$.
\end{problem}
\begin{solution}
  \[\RREF\left(
    \begin{bmatrix}
      3 & 2 & 1 \\
      -1 & 0 & 4 \\
      2 & 2 & -1
    \end{bmatrix} \right)= \begin{bmatrix}
      1 & 0 &0 \\
      0 & 1 & 0 \\
      0 & 0 & 1
    \end{bmatrix}
  \]
Since the resulting matrix is the identity matrix, it is a basis.
\end{solution}


\begin{problem}{S3}
Let \(
  W={\rm span}\left\{
    \begin{bmatrix} 2 & 0 \\ -2 & 0 \end{bmatrix},
    \begin{bmatrix} 3 & 1 \\ 3 & 6 \end{bmatrix},
    \begin{bmatrix} 0 & 0 \\ 1 & 1 \end{bmatrix},
    \begin{bmatrix} 1 & 2 \\ 0 & 1 \end{bmatrix}
  \right\}
\). Find a basis for this vector space.
\end{problem}
\begin{solution}
\[
  \RREF\left(\begin{bmatrix}
    2 & 3 & 0 & 1 \\
    0 & 1 & 0 & 2 \\
    -2 & 3 & 1 & 0 \\
    0 & 6 & 1 & 1
  \end{bmatrix} \right) =
  \begin{bmatrix}
    1 & 0 & 0 & -\frac{5}{2} \\
    0 & 1 & 0 & 2 \\
    0 & 0 & 1 & -11\\
    0 & 0 & 0 & 0
  \end{bmatrix}
\]
Thus \(\left\{
  \begin{bmatrix} 2 & 0 \\ -2 & 0 \end{bmatrix},
  \begin{bmatrix} 3 & 1 \\ 3 & 6 \end{bmatrix},
  \begin{bmatrix} 0 & 0 \\ 1 & 1 \end{bmatrix}
\right\}\) is a basis of $W$.
\end{solution}


\begin{problem}{S4}
Let $W$ be the subspace of $\P_3$ given by $W={\rm span}\left( \left\{ x^3-x^2+3x-3, 2x^3+x+1, 3x^3-x^2+4x-2, x^3+x^2+x-7\right\}\right)$.  Compute the dimension of $W$.
\end{problem}
\begin{solution}
$$ \RREF \left( \begin{bmatrix} 1 & 2 & 3 & 1 \\ -1 & 0 & -1 & 1 \\ 3 & 1 & 4 & 1 \\ -3 & 1 & -2 & -7 \end{bmatrix} \right) =  \begin{bmatrix} 1 & 0 & 1 & 0 \\ 0 & 1 & 1 & 0 \\ 0 & 0 & 0 & 1 \\ 0 & 0 & 0 & 0\end{bmatrix}$$
This has 3 pivot columns so $\dim(W)=3$.
\end{solution}


\begin{problem}{A1}
Let $T: \IR^3\rightarrow \IR^4$ be the linear transformation given by $$T\left(\begin{bmatrix} x \\ y \\ z \\  \end{bmatrix} \right) = \begin{bmatrix} -3x+y \\ -8x+2y-z \\ 7x+2y+3z \\ 0 \end{bmatrix}.$$  Write the matrix for $T$ with respect to the standard bases of $\IR^3$ and $\IR^4$.
\end{problem}
\begin{solution}
$$\begin{bmatrix} 3 & 1 & 0 \\ -8 & 2 & -1 \\ 7 & 2 & 3 \\ 0 & 0 & 0 \end{bmatrix}$$
\end{solution}

\begin{problem}{A2}
Determine if the map $T: \P^3 \rightarrow \P^4$ given by $T(f(x))=xf(x)-f(x)$ is a linear transformation or not.
\end{problem}
\begin{problem}{A3}
Determine if each of the following linear transformations is injective (one-to-one) and/or surjective (onto).
\begin{enumerate}[(a)]
\item $S: \IR^2 \rightarrow \IR^4$ given by the standard matrix $\begin{bmatrix} 2 & 1 \\ 1 & 2 \\ 0 & 1 \\ 3 & -3 \end{bmatrix}$.
\item $T: \IR^4 \rightarrow \IR^3$ given by the standard matrix $\begin{bmatrix} 2 & 3 & -1 & 1 \\ -1 & 1 & 1 & 1 \\ 4 & 11 & -1 & 5 \end{bmatrix}$
\end{enumerate}
\end{problem}
\begin{solution}
\begin{enumerate}[(a)]
\item $ \begin{bmatrix} 2 & 1 \\ 1 & 2 \\ 0 & 1 \\ 3 & -3 \end{bmatrix}=\begin{bmatrix}1 & 0 \\ 0 & 1 \\ 0 & 0 \\ 0 & 0  \end{bmatrix}$.  Since each column is a pivot column, $S$ is injective.  Since there a no zero row, $S$ is not surjective.
\item Since $\dim \IR^4 > \dim \IR^3$, $T$ is not injective.
$$\RREF\left(\begin{bmatrix} 2 & 3 & -1 & 1 \\ -1 & 1 & 1 & 1 \\ 4 & 11 & -1 & 5 \end{bmatrix}\right) = \begin{bmatrix} 1 & 0  & -\frac{4}{5} & -\frac{2}{5} \\ 0 & 1 & \frac{1}{5} & \frac{3}{5} \\ 0 & 0 & 0 & 0 \end{bmatrix}$$
Since there is a zero row, $T$ is not surjective.
\end{enumerate}
\end{solution}

\begin{problem}{A4}
Let $T: \IR^{2\times 2} \rightarrow \IR^3$ be the linear map given by $T\left(\begin{bmatrix} x & y \\ z & w \end{bmatrix} \right) = \begin{bmatrix}  8x-3y-z+4w \\ y+3z-4w \\ -7x+3y+2z-5w\end{bmatrix} $.
Compute a basis for the kernel and a basis for the image of $T$.
\end{problem}
\begin{solution}
$$\RREF \left( \begin{bmatrix} 8 & -3 & -1 & 4 \\ 0 & 1 & 3 & -4 \\ -7 & 3 & 2 & -5 \end{bmatrix} \right) = \begin{bmatrix} 1 & 0 & 1 & -1 \\ 0 & 1 & 3 & -4 \\ 0 & 0 & 0 & 0 \end{bmatrix}$$

Thus \(\left\{ \begin{bmatrix} 8 \\ 0 \\ -7 \end{bmatrix}, \begin{bmatrix} -3 \\ 1 \\ 3 \end{bmatrix} \right\}\) is a basis for the image, and \( \left\{ \begin{bmatrix} -1 & -3 \\ 1 & 0 \end{bmatrix}, \begin{bmatrix} 1 & 4 \\ 0 & 1 \end{bmatrix} \right\} \) is a basis for the kernel.
\end{solution}


\begin{problem}{M1}
Let 
\begin{align*}
A &= \begin{bmatrix} 1 & 3 & -1  \\ 0 & 0 & 7  \end{bmatrix} & B &= \begin{bmatrix} 0 & 1 & 7 & 7 \\ -1 & -2 & 0 & 4 \\ 0 & 0 & 1 & 5 \end{bmatrix} & C&=\begin{bmatrix} 3  \\  1 \end{bmatrix}
\end{align*}
Exactly one of the six products $AB$, $AC$, $BA$, $BC$, $CA$, $CB$ can be computed.  Determine which one, and compute it.
\end{problem}
\begin{solution}
$AB$ is the only ones that can be computed, and 
$$AB = \begin{bmatrix} -3 & -5 & 6 & 14 \\ 0 & 0 & 7 & 35 \end{bmatrix}$$
\end{solution}


\begin{problem}{M2}
Determine if the matrix $\begin{bmatrix}-3 & 1 & 0 \\ -8 & 2 & -1 \\ 0 & 2 & 3\end{bmatrix}$ is invertible.
\end{problem}
\begin{solution}
$$\RREF \begin{bmatrix}-3 & 1 & 0 \\ -8 & 2 & -1 \\ 0 & 2 & 3\end{bmatrix} = \begin{bmatrix} 1 & 0 & \frac{1}{2} \\ 0 & 1 & \frac{3}{2} \\ 0 & 0 & 0 \end{bmatrix}$$
Since it is not equivalent to the identity matrix, it is not invertible.
\end{solution}


\begin{problem}{M3}
  Find the inverse of the matrix
  \(\begin{bmatrix}
    4 & -1 & -8  \\
    2 & 1 & 3  \\
    1 & 1 & 4
  \end{bmatrix}\).
\end{problem}
\begin{solution}
\(\begin{bmatrix}[ccc|ccc]
  4 & -1 & -8 & 1 & 0 & 0  \\
  2 & 1 & 3   & 0 & 1 & 0 \\
  1 & 1 & 4   & 0 & 0 & 1
\end{bmatrix}\sim\begin{bmatrix}[ccc|ccc]
  1 & 0 & 0 & 1 & -4 & 5 \\
  0 & 1 & 0 & -5 & 24 & -28 \\
  0 & 0 & 1 & 1 & -5 & 6
\end{bmatrix}\). Thus the inverse is
\(\begin{bmatrix}
  1 & -4 & 5  \\
  -5 & 24 & -28  \\
  1 & -5 & 6
\end{bmatrix}\).
\end{solution}


\begin{problem}{G1}
Compute the determinant of the matrix $\begin{bmatrix} 1 & 3 & 0 & -1 \\ 0 & 1 & 3 & 1 \\ 2 & 0 & 0 & -1 \\ 1 & -3 & -2 & -1 \end{bmatrix}$.
\end{problem}
\begin{solution}
\begin{align*}
\det \begin{bmatrix} 1 & 3 & 0 & -1 \\ 0 & 1 & 3 & 1 \\ 2 & 0 & 0 & -1 \\ 1 & -3 & -2 & -1 \end{bmatrix} &= 2 \det \begin{bmatrix} 3 & 0 & -1 \\ 1 & 3 & 1 \\ -3 & -2 & -1 \end{bmatrix} -(-1) \det \begin{bmatrix} 1 & 3 & 0 \\ 0 & 1 & 3 \\ 1 & -3 & -2 \end{bmatrix} \\
&=2\left( 3 \det \begin{bmatrix} 3 & 1 \\ -2 & -1 \end{bmatrix} + (-1) \det \begin{bmatrix} 1 & 3 \\ -3 & -2 \end{bmatrix}\right) + \left( 1 \det \begin{bmatrix} 1 & 3 \\ -3 & -2 \end{bmatrix} -3 \begin{bmatrix} 0 & 3 \\ 1 & -2 \end{bmatrix} \right) \\
&=2 \left( 3(-1)+(-1)(7) \right) + \left( (1)(7)-3(-3) \right) \\
&= 2(-10)+16 \\
&=-4
\end{align*}
\end{solution}


\begin{problem}{G2} 
Compute the eigenvalues, along with their algebraic multiplicities, of the matrix $ \begin{bmatrix} 9 & -3 & 2 \\ 19 & -6 & 5 \\ -5 & 2 & 0 \end{bmatrix}$.
\end{problem}
\begin{solution}
1 with algebraic multiplicity 3
\end{solution}

\begin{problem}{G3}
Find the eigenspace associated to the eigenvalue $1$ in the matrix $A=\begin{bmatrix} 8 & -3 & -1 \\ 21 & -8 & -3 \\ -7 & 3  & 2 \end{bmatrix}$
\end{problem}
\begin{solution}
The eigenspace is spanned by $\begin{bmatrix} \frac{3}{7} \\ 1 \\ 0 \end{bmatrix}$ and $\begin{bmatrix}\frac{1}{7} \\ 0 \\ 1 \end{bmatrix}$.
\end{solution}

\begin{problem}{G4}
Compute the geometric multiplicity of the eigenvalue $2$ in the matrix $A=\begin{bmatrix}0 & -2 & -1 & 0 \\ -4 & -2 & -2 & 0 \\ 14 & 12 & 10 & 2 \\ -13 & -10 & -8 & -1 \end{bmatrix}$.
\end{problem}
\begin{solution}
The eigenspace is spanned by $\begin{bmatrix} -1 \\ \frac{1}{2} \\ 1 \\ 0 \end{bmatrix}$ and  $\begin{bmatrix} -1 \\ 1 \\ 0 \\ 1 \end{bmatrix}$, so the geometric multiplicity is $2$.
\end{solution}

\end{document}