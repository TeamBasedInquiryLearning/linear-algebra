\documentclass{sbgLAsemi}

\begin{extract*}
\usepackage{amsmath,amssymb,amsthm,enumerate}
\coursetitle{Math 237}
\courselabel{Linear Algebra}
\calculatorpolicy{You may use a calculator, but you must show all relevant work to receive credit for a standard.}

\newcommand{\IR}{\mathbb{R}}

\makeatletter
\renewcommand*\env@matrix[1][*\c@MaxMatrixCols c]{%
  \hskip -\arraycolsep
  \let\@ifnextchar\new@ifnextchar
  \array{#1}}
\makeatother


\title{Semifinal}
\version{2}
\end{extract*}

\begin{document}

\begin{problem}{E1}
Write a system of linear equations corresponding to the following
augmented matrix.
\[
\begin{bmatrix}[cccc|c]
3 & -1 & 0 & 1 & 5 \\
-1 & 9 & 1 & -7 & 0 \\
1 & 0 & -1 & 0 & -3
\end{bmatrix}
\]
\end{problem}
\begin{solution}
\begin{align*}
3x_1-x_2+x_4 &= 5 \\
-x_1+9x_2+x_3-7x_4 &= 0 \\
x_1-x_3 &= -3
\end{align*}
\end{solution}

\begin{problem}{E2}
Find \(\RREF A\), where
\[
  A =
  \begin{bmatrix}[cc|c]
    2 & -7 & 4 \\
    1 & -3 & 2 \\
    3 & 0 & 3
  \end{bmatrix}
\]
\end{problem}
\begin{solution}
\[
  \RREF A =
  \begin{bmatrix}[cc|c]
    1 & 0 & 0 \\
    0 & 1 & 0 \\
    0 & 0 & 1
  \end{bmatrix}
\]
\end{solution}

\begin{problem}{E3}
Find the solution set for the following system of linear equations.
\begin{align*}
2x_1-2x_2+6x_3-x_4 &=-1 \\
3x_1+6x_3+x_4 &= 5 \\
-4x_1+x_2-9x_3+2x_4&=-7
\end{align*}
\end{problem}
\begin{solution}
Let \(A =
  \begin{bmatrix}[cccc|c]
    2 & -2 & 6 & -1 & -1 \\
    3 & 0 & 6 & 1 & 5 \\
    -4 & 1 & -9 & 2 & -7
  \end{bmatrix}
\), so \(\RREF A =
  \begin{bmatrix}[cccc|c]
    1 & 0 & 2 & 0 & 2 \\
    0 & 1 & -1 & 0 & 3 \\
    0 & 0 & 0 & 1 & -1
  \end{bmatrix}
\). It follows that the solution set is given by \(
  \begin{bmatrix}
    2 - 2a \\
    3 + a \\
    a \\
    -1
  \end{bmatrix}
\) for all real numbers \(a\).
\end{solution}

\begin{problem}{E4}
Find a basis for the solution set of the system of equations
\begin{align*}
x+2y+3z+w &= 0 \\
3x-y+z+w &= 0 \\
2x-3y-2z &= 0 \\
-x+2z+5w &=0
\end{align*}
\end{problem}
\begin{solution}
$$\RREF \left(\begin{bmatrix} 1 & -2 & 3 & 1 \\ 3 & -1 & 1 & 1 \\ 2 & -3 & -2 & 0 \\ -1 & 0 & 2 & 5 \end{bmatrix} \right) = \begin{bmatrix} 1 & 0 & 0 & -1 \\ 0 & 1 & 0 & -2 \\ 0 & 0 & 1 & 2 \\ 0 & 0 & 0 & 0\end{bmatrix}$$
Then the solution set is
$$\left\{ \begin{bmatrix} a \\ 2a \\ -2a \\ a \end{bmatrix} \bigg |\ a \in \IR \right\}$$
So a basis for the solution set is $\left\{\begin{bmatrix} 1 \\ 2 \\ -2 \\ 1\end{bmatrix} \right\}$.
\end{solution}


\begin{problem}{V1}
Let $V$ be the set of all points on the line $x+y=2$ with the operations, for any $(x_1,y_1), (x_2,y_2) \in V$, $c \in \IR$,
\begin{align*}
(x_1,y_1) \oplus (x_2,y_2) &= (x_1+x_2-1,y_1+y_2-1) \\
c \odot (x_1,y_1) &= (cx_1-(c-1), cy_1-(c-1))
\end{align*}
\begin{enumerate}[(a)]
\item Show that this vector space has an \textbf{additive identity} element
      \(\mathbf{0}\) satisfying \((x,y)\oplus\mathbf{0}=(x,y)\).
\item Determine if $V$ is a vector space or not.  Justify your answer.
\end{enumerate}
\end{problem}
\begin{solution}
Let $(x_1,y_1) \in V$; then $(x_1,y_1) \oplus (1,1) = (x_1,y_1)$, so $(1,1)$ is an additive identity element.

Now we will show the other seven properties.  Let $(x_1,y_1), (x_2,y_2) \in V$, and let $c,d \in \IR$.
\begin{enumerate}[1)]
\item Since real addition is associative, $\oplus$ is associative.
\item Since real addition is commutative, $\oplus$ is commutative.
\item The additive identity is $(1,1)$.
\item $(x_1,y_1) \oplus (2-x_1,2-y_1) = (1,1)$, so $(2-x_1,2-y_1)$ is the additive inverse of $(x_1,y_1)$.
\item \begin{align*} c\odot \left(d \odot (x_1,y_1) \right) &=c\odot \left( dx_1-(d-1),dy_1-(d-1)\right) \\
&= \left( c\left(dx_1-(d-1) \right)-(c-1), c\left(dy_1-(d-1) \right) \right) \\
&= \left(cdx_1-cd+c-(c-1), cdy_1-cd+c-(c-1) \right) \\
&= \left(cdx_1-(cd-1), cdy_1-(cd-1) \right) \\
&= (cd) \odot (x_1,y_1)
\end{align*}
\item $1 \odot (x_1,y_1) = (x_1-(1-1),y_1-(1-1)=(x_1,y_1)$
\item \begin{align*} c \odot \left( (x_1,y_1)\oplus(x_2,y_2) \right) &=
c\odot \left( x_1+y_1-1,x_2+y_2-1 \right) \\
&= \left( c(x_1+y_1-1)-(c-1), c(x_2+y_2-1)-(c-1) \right) \\
&= (cx_1+cx_2-2c+1, cy_1+cy_2-2c+1) \\
&= \left(cx_1-(c-1),cy_1-(c-1) \right) \oplus (cx_2-(c-1),cy_2-(c-1)) \\
&=c \odot (x_1,y_1) \oplus c\odot (x_2,y_2)
\end{align*}
\item \begin{align*} (c+d) \odot (x_1,y_1) &=
\left( (c+d)x_1-(c+d-1), (c+d)y_1-(c+d-1) \right) \\
&= \left( cx_1-(c-1), cy_1-(c-1) \right) \oplus (dx_1-(d-1), dy_1-(d-1) ) \\
&= c\odot (x_1,y_1) \oplus c \odot (x_2,y_2)
\end{align*}
\end{enumerate}
Therefore $V$ is a vector space.
\end{solution}

\begin{problem}{V2}
  Determine if
  \(\begin{bmatrix} 4 \\ -1 \\ 6 \\ -7 \end{bmatrix}\)
  belongs to the span of the set
  \(\left\{
    \begin{bmatrix} 2 \\ 0 \\ -1 \\ 5 \end{bmatrix},
    \begin{bmatrix} 4 \\ -1 \\ 4 \\ 3 \end{bmatrix}
    \right\}
  \).
\end{problem}
\begin{solution}
  Since
  \[
    \RREF\left(
      \begin{bmatrix}[cc|c]
        2 & 4 & 4 \\
        0 & -1 & -1 \\
        -1 & 4 & 6 \\
        5 & 3 & -7
      \end{bmatrix}
    \right) =
    \begin{bmatrix}[cc|c]
      1 & 0 & 0 \\
      0 & 1 & 0 \\
      0 & 0 & 1 \\
      0 & 0 & 0
    \end{bmatrix}
  \]
  contains the contradiction \(0=1\),
  \(\begin{bmatrix} 4 \\ -1 \\ 6 \\ -7 \end{bmatrix}\) is
  not a linear combination of the three vectors.
\end{solution}


\begin{problem}{V3}
Determine if the vectors $\begin{bmatrix} 2 \\ 0 \\ -2 \\ 0 \end{bmatrix}$, $\begin{bmatrix} 3 \\ 1 \\ 3 \\ 6 \end{bmatrix}$, $\begin{bmatrix} 0 \\ 0 \\ 1 \\ 1 \end{bmatrix}$, and $\begin{bmatrix}1 \\ 2 \\ 0 \\ 1 \end{bmatrix}$ span $\IR^4$.
\end{problem}
\begin{solution}
$$\RREF\left( \begin{bmatrix} 2 & 3 & 0 & 1 \\ 0 & 1 & 0 & 2 \\ -2 & 3 & 1 & 0 \\ 0 & 6 & 1 & 1\end{bmatrix} \right) = \begin{bmatrix}1 & 0 & 0 & -\frac{5}{2} \\ 0 & 1 & 0 & 2 \\ 0 & 0 & 1 & -11\\ 0 & 0 & 0 & 0  \end{bmatrix} $$
Since there is a zero row, the vectors do not span $\IR^4$.
\end{solution}

\begin{problem}{V4} Let $W$ be the set of all polynomials of even degree.  Determine if $W$ is a subspace of the vector space of all polynomials.
\end{problem}
\begin{solution}
$W$ is closed under scalar multiplication, but not under addition.  For example, $x-x^2$ and $x^2$ are both in $W$, but $(x-x^2)+(x^2)=x \notin W$.
\end{solution}


\begin{problem}{S1}
Determine if the set of matrices $\left\{\begin{bmatrix} 3 & -1 \\ 0 & 4 \end{bmatrix}, \begin{bmatrix} 1  & 2 \\ -2 & 1 \end{bmatrix}, \begin{bmatrix} 3 & -8 \\ 6 & 5 \end{bmatrix} \right\}$  is linearly dependent or linearly independent.
\end{problem}
\begin{solution}
$$\RREF\left(\begin{bmatrix} 3 & 1 & 3 \\ -1 & 2 & -8 \\ 0 & -2 & 6 \\ 4 & 1 & 5 \end{bmatrix} \right) = \begin{bmatrix} 1 & 0 & 2 \\ 0 & 1 & -3 \\ 0 & 0 & 0 \\ 0 & 0 & 0 \end{bmatrix}$$
Since the reduced row echelon form has a nonpivot column, the vectors are linearly dependent.
\end{solution}
\begin{problem}{S2}
  Determine if the set \(\left\{
    \begin{bmatrix} 3 \\ -1 \\ 2 \\3 \end{bmatrix},
    \begin{bmatrix} 2 \\ 0 \\ 2 \\ 4\end{bmatrix},
    \begin{bmatrix} 1 \\ -1 \\ 0 \\ -1\end{bmatrix},
    \begin{bmatrix} -1 \\ 3 \\ 0 \\ 5\end{bmatrix}
  \right\}\) is a basis of $\IR^4$.
\end{problem}
\begin{solution}
  \[\RREF\left(
    \begin{bmatrix}
      3 & 2 & 1 & -1\\
      -1 & 0 & -1 & 3\\
      2 & 2 & 0 & 0\\
      3 & 4 & -1 & 5\\
    \end{bmatrix} \right)= \begin{bmatrix}
      1 & 0 & 1 & 0 \\
      0 & 1 & -1 & 0 \\
      0 & 0 & 0 & 1 \\
      0 & 0 & 0 & 0
    \end{bmatrix}
  \]
Since the resulting matrix is not the identity matrix, it is not a basis.
\end{solution}


\begin{problem}{S3}
Let $W$ be the subspace of $\P^3$ given by $W = {\rm span} \left( \left\{ x^3+x^2+2x+1, 3x^3+3x^2+6x+3, 3x^3-x^2+3x-2, 7x^3-x^2+8x-3 \right\} \right)$.  Find a basis for $W$.
\end{problem}
\begin{solution}
$$\RREF\left(\begin{bmatrix} 1 & 3 & 3 & 7 \\ 1 & 3 & -1 & -1 \\ 2 & 6 & 3 & 8 \\ 1 & 3 & -2 & -3 \end{bmatrix}\right) = \begin{bmatrix} 1 & 3 & 0 & 1 \\ 0 & 0 & 1 & 2 \\ 0 & 0 & 0 & 0 \\  0 & 0 & 0 & 0 \end{bmatrix}$$

Then a basis is
$ \left\{ x^3+x^2+2x+1, 3x^3-x^2+3x-2 \right\} $.
\end{solution}


\begin{problem}{S4}
Let \(
  W={\rm span}\left\{
    \begin{bmatrix} 2 \\ 0 \\ 2 \\ 1 \end{bmatrix},
    \begin{bmatrix} 3 \\ 1 \\ -1 \\ 1 \end{bmatrix},
    \begin{bmatrix} 0 \\ 2 \\ -8 \\ -1 \end{bmatrix}
  \right\}
\). Find the dimension of \(W\).
\end{problem}
\begin{solution}
\[
  \RREF\left(\begin{bmatrix}
    2 & 3 & 0 \\
    0 & 1 & 2 \\
    2 & -1 & -8 \\
    1 & 1 & -1
  \end{bmatrix} \right) =
  \begin{bmatrix}
    1 & 0 & -3 \\
    0 & 1 & 2 \\
    0 & 0 & 0 \\
    0 & 0 & 0
  \end{bmatrix}
\]
Since it has two pivot columns, its dimension is \(2\).
\end{solution}


\begin{problem}{A1}
Let $T: \IR^3 \rightarrow \IR$ be the linear transformation given by $$T\left(\begin{bmatrix} x_1 \\ x_2 \\ x_3  \end{bmatrix} \right) = \begin{bmatrix} x_2+3x_3 \end{bmatrix}.$$ Write the matrix for $T$ with respect to the standard bases of $\IR^3$ and $\IR$.
\end{problem}
\begin{solution}
$$\begin{bmatrix} 0 & 1 & 3 \end{bmatrix}$$
\end{solution}


\begin{problem}{A2} 
 Determine if $T: \IR^2 \rightarrow \IR^2$ given by $T\left(\begin{bmatrix} x \\ y \end{bmatrix}\right) = \begin{bmatrix} e^{x} \\ e^y \end{bmatrix}$ is a linear transformation.
\end{problem}
\begin{solution}
It is not linear.  For example, $$\begin{bmatrix}e^2 \\ 1 \end{bmatrix} = T\left(\begin{bmatrix} 2 \\ 0 \end{bmatrix} \right) \neq 2 T\left(\begin{bmatrix} 1 \\ 0 \end{bmatrix}\right)=\begin{bmatrix} 2e \\ 1 \end{bmatrix}$$
\end{solution}

\begin{problem}{A3}
Determine if each of the following linear transformations is injective (one-to-one) and/or surjective (onto).
\begin{enumerate}[(a)]
\item
  \(S: \IR^2 \rightarrow \IR^3\) where
  \(S(\vec e_1)=\begin{bmatrix}
    2 \\
    1 \\
    0
  \end{bmatrix}\) and
  \(S(\vec e_2)=\begin{bmatrix}
    1 \\
    2 \\
    1
  \end{bmatrix}\).
\item
  \(T: \IR^3 \rightarrow \IR^2\) where
  \(T(\vec e_1)=\begin{bmatrix}
    2 \\
    2
  \end{bmatrix}\),
  \(T(\vec e_2)=\begin{bmatrix}
   1  \\
   0
  \end{bmatrix}\), and
  \(T(\vec e_3)=\begin{bmatrix}
    1 \\
    1
  \end{bmatrix}\).
\end{enumerate}
\end{problem}
\begin{solution}
\begin{enumerate}[(a)]
\item
  \(\RREF\begin{bmatrix}
    2 & 1 \\
    1 & 2 \\
    0 & 1
  \end{bmatrix}=\begin{bmatrix}
    1 & 0 \\
    0 & 1 \\
    0 & 0
  \end{bmatrix}\).
  The map is injective since every column has a pivot, but is not surjective
  because there is a row without a pivot.
\item
  \(\RREF\begin{bmatrix}
    2 & 1 & 1 \\
    2 & 0 & 1
  \end{bmatrix}=\begin{bmatrix}
    1 & 0 & 1/2 \\
    0 & 1 & 1/2
  \end{bmatrix}\).
  The map is not injective since there is a column without a pivot,
  but it is surjective because every row has a pivot.
\end{enumerate}
\end{solution}

\begin{problem}{A4}
Let $T: \IR^3 \rightarrow \IR^3$ be the linear map given by \(
  T\left(\begin{bmatrix} x \\ y \\ z \end{bmatrix} \right) =
  \begin{bmatrix}
    8x-3y-z \\
    y+3z \\
    -7x+3y+2z
  \end{bmatrix}
\). Compute a basis for the kernel and a basis for the image of $T$.
\end{problem}
\begin{solution}
\[
  \RREF \left( \begin{bmatrix}
    8 & -3 & -1 \\
    0 & 1 & 3 \\
    -7 & 3 & 2
  \end{bmatrix} \right) = \begin{bmatrix}
    1 & 0 & 1 \\
    0 & 1 & 3 \\
    0 & 0 & 0
  \end{bmatrix}
\]

Thus \(\left\{
  \begin{bmatrix} 8 \\ 0 \\ -7 \end{bmatrix},
  \begin{bmatrix} -3 \\ 1 \\ 3 \end{bmatrix}
\right\} \) is a basis for the image, and \(\left\{
  \begin{bmatrix} -1 \\ -3 \\ 1 \end{bmatrix}
\right\} \) is a basis for the kernel.
\end{solution}


\begin{problem}{M1}
Let 
\begin{align*}
A &= \begin{bmatrix} 2 & 3 \\ 0 & 1 \end{bmatrix} & B&= \begin{bmatrix} 3 & 1 & 0 \end{bmatrix} & C&= \begin{bmatrix} 0 & -1 & 4 \\ 1 & -1 & 2 \end{bmatrix}
\end{align*}

Exactly one of the six products $AB$, $AC$, $BA$, $BC$, $CA$, $CB$ can be computed.  Determine which one, and compute it.
\end{problem}
\begin{solution}
$AC$ is the only one that can be computed, and 
$$AC = \begin{bmatrix} 3 & -5 & 11 \\ 1 & -1 & 2 \end{bmatrix}$$
\end{solution}

\begin{problem}{M2}
Determine if the matrix $\begin{bmatrix} 2 & 1 & 0 & 3 \\ 1 & -1 & 3 & 1 \\ 3 & 2 & -1 & 7 \\ 4 & 1 & 2 & 0 \end{bmatrix}$ is invertible.
\end{problem}
\begin{solution}
$$\RREF \begin{bmatrix} 2 & 1 & 0 & 3 \\ 1 & -1 & 3 & 1 \\ 3 & 2 & -1 & 7 \\ 4 & 1 & 2 & 0 \end{bmatrix}=\begin{bmatrix} 1 & 0 & 1 & 0 \\ 0 & 1 & -2 & 0 \\ 0 & 0 & 0 & 1 \\ 0 & 0 & 0 & 0 \end{bmatrix}$$
Since it is not row equivalent to the identity matrix, it is not invertible.
\end{solution}

\begin{problem}{M3}
Compute the inverse of the matrix $\begin{bmatrix} 1 & 2 & 3 & 0 \\ 0 & -1 & 4 & -2 \\ 0 & 0 & 1 & 3 \\ 0 & 0 & 0 & 1 \end{bmatrix}$.
\end{problem}
\begin{solution}
$$\RREF(A|I) = \begin{bmatrix}[cccc|cccc] 1 & 0 & 0 &0 & 1 & 2 & -11 & 37 \\ 0 & 1 & 0 & 0 & 0 & -1 & 4 & -14 \\ 0 & 0 & 1 & 0 & 0 & 0 & 1 & -3 \\ 0 & 0 & 0 & 1 & 0 & 0 & 0 & 1 \end{bmatrix}$$
So the inverse is $\begin{bmatrix}  1 & 2 & -11 & 37 \\ 0 & -1 & 4 & -14 \\  0 & 0 & 1 & -3 \\ 0 & 0 & 0 & 1 \end{bmatrix}$
\end{solution}


\begin{problem}{G1}
Compute the determinant of the matrix $\begin{bmatrix} 8 & 5 & 3 & 0 \\ 3 & 2 & 1 & 1 \\ 5 & -3 & 1 & -2 \\ -1 & 2 & 0 & 1\end{bmatrix} $.
\end{problem}
\begin{solution}
$-1$.
\end{solution}

\begin{problem}{G2} 
Compute the eigenvalues, along with their algebraic multiplicities, of the matrix $ \begin{bmatrix}2 & -3 & 2 \\ 8 & -9 & 5 \\ 8 & -7 & 3 \end{bmatrix}$.
\end{problem}
\begin{solution}
The eigenvalues are $0$ with multiplicity 1 and $-2$, with algebraic multiplicity 2.
\end{solution}

\begin{problem}{G3}
Find the eigenspace associated to the eigenvalue $-1$ in the matrix $A=\begin{bmatrix}9 & -3 & 2 \\ 19 & -6 & 5 \\ -11 & 4 & -2 \end{bmatrix}$
\end{problem}
\begin{solution}
The eigenspace is spanned by $\begin{bmatrix} -\frac{5}{7} \\ - \frac{12}{7} \\ 1 \end{bmatrix}$.
\end{solution}

\begin{problem}{G4}
Compute the geometric multiplicity of the eigenvalue $2$ in the matrix $A=\begin{bmatrix}8 & -3 & 2 \\ 15 & -5 & 5 \\ -3 & 2 & 1\end{bmatrix}$
\end{problem}
\begin{solution}
The eigenspace is spanned by $\begin{bmatrix} -\frac{1}{3} \\ 0 \\ 1 \end{bmatrix}$, so the geometric multiplicity is $1$.
\end{solution}


\end{document}