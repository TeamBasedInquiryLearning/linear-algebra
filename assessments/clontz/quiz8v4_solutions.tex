\documentclass{sbgLAquiz}

\begin{extract*}
\usepackage{amsmath,amssymb,amsthm,enumerate}
\coursetitle{Math 237}
\courselabel{Linear Algebra}
\calculatorpolicy{You may use a calculator, but you must show all relevant work to receive credit for a standard.}

\newcommand{\IR}{\mathbb{R}}

\makeatletter
\renewcommand*\env@matrix[1][*\c@MaxMatrixCols c]{%
  \hskip -\arraycolsep
  \let\@ifnextchar\new@ifnextchar
  \array{#1}}
\makeatother


\title{Mastery Quiz Day 8 }
\standard{E1,E3,E4,V1,E2}
\version{4}
\end{extract*}

\begin{document}

\begin{problem}{E1}
Write a system of linear equations corresponding to the following
augmented matrix.
\[
\begin{bmatrix}[ccc|c]
2 & -1 & 0 & 1  \\
-1 & 4 & 1 & -7  \\
1 & 2 & -1 & 0
\end{bmatrix}
\]
\end{problem}
\begin{solution}
\begin{align*}
2x_1-x_2&=1 \\
-x_1+4x_2+x_3&=-7 \\
x_1+2x_2-x_3 &= 0
\end{align*}
\end{solution}

\begin{problem}{E3}
Solve the system of equations
\begin{align*}
x+3y-4z &= 5 \\
3x+9y+z &= 2
\end{align*}
\end{problem}
\begin{solution}
$$\RREF \left(\begin{bmatrix}[ccc|c] 1 & 3 & -4 & 5 \\ 3 & 9 & 1 & 2 \end{bmatrix} \right) = \begin{bmatrix}[ccc|c] 1 & 3 & 0 & 1 \\ 0 & 0 & 1 & -1\end{bmatrix}$$
So the solution set is
$$\left\{ \begin{bmatrix} 1-3c \\ c \\ -1 \end{bmatrix} \bigg|\ c \in \IR \right\}$$
\end{solution}



\begin{extract}\newpage\end{extract}
\begin{problem}{E4}
Find a basis for the solution set to the homogeneous system of equations
given by
\begin{align*}
3x+2y+z &= 0 \\
x+y+z &= 0
\end{align*}
\end{problem}
\begin{solution}
Let \(A =
  \begin{bmatrix}[ccc|c]
    3 & 2 & 1 & 0 \\
    1 & 1 & 1 & 0
  \end{bmatrix}
\), so \(\RREF A =
  \begin{bmatrix}[ccc|c]
    1 & 0 & -1 & 0 \\
    0 & 1 & 2 & 0
  \end{bmatrix}
\).
It follows that the basis for the solution set is given by \(\left\{
  \begin{bmatrix}
    1 \\
    -2 \\
    1
  \end{bmatrix}
\right\}\).
\end{solution}

\begin{problem}{V1}
Let $V$ be the set of all pairs of real numbers with the operations, for any $(x_1,y_1), (x_2,y_2) \in V$, $c \in \IR$,
\begin{align*}
(x_1,y_1) \oplus (x_2,y_2) &= (x_1+x_2,y_1+y_2) \\
c \odot (x_1,y_1) &= (c^2x_1, c^3y_1)
\end{align*}
Determine if $V$ is a vector space or not.
\end{problem}
\begin{solution}
$V$ is not a vector space, as one of the distributive laws fails, namely
$$(c+d) \odot (x_1,y_1) = ( (c+d)^2 x_1, (c+d)^3 y_1) \neq ((c^2+d^2)x_1, (c^3+d^3)y_1) = c \odot (x_1,y_1) \oplus d \odot (x_1,y_1).$$
\end{solution}


\end{document}