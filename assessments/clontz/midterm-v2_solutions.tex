\documentclass{sbgLAexam}

\begin{extract*}
\usepackage{amsmath,amssymb,amsthm,enumerate}
\coursetitle{Math 237}
\courselabel{Linear Algebra}
\calculatorpolicy{You may use a calculator, but you must show all relevant work to receive credit for a standard.}

\newcommand{\IR}{\mathbb{R}}

\makeatletter
\renewcommand*\env@matrix[1][*\c@MaxMatrixCols c]{%
  \hskip -\arraycolsep
  \let\@ifnextchar\new@ifnextchar
  \array{#1}}
\makeatother


\title{Midterm Exam}
\standard{E1,E2,E3,E4,V1,V2,V3,V4,S1,S2,S3,S4}
\version{2}
\end{extract*}

\begin{document}

\begin{problem}{E1}
Write an augmented matrix corresponding to the following system of linear equations.
\begin{align*}
x_1+4x_3 &= 1 \\
x_2-x_3 &= 7 \\
x_1-x_2+3x_3 &= -1
\end{align*}
\end{problem}
\begin{solution}
\[
\begin{bmatrix}[ccc|c]
1 & 0 & 4 & 1 \\
0 & 1 & -1 & 7 \\
1 & -1 & 3 & -1
\end{bmatrix}
\]
\end{solution}

\begin{problem}{E2}
Put the following matrix in reduced row echelon form.
$$\begin{bmatrix}[ccc|c] -3 & 1 & 0 & 2 \\ -8 & 2 & -1 & 6 \\ 0 & 2 & 3 & -2 \end{bmatrix}$$
\end{problem}
\begin{solution}
$$\begin{bmatrix}[ccc|c]
-3 & 1 & 0 & 2 \\
 -8 & 2 & -1 & 6 \\
 0 & 2 & 3 & -2
\end{bmatrix} \sim
\begin{bmatrix}[ccc|c]
1 & -\frac{1}{3} & 0 & -\frac{2}{3} \\
 -8 & 2 & -1 & 6 \\
 0 & 2 & 3 & -2
\end{bmatrix} \sim
\begin{bmatrix}[ccc|c]
1 & -\frac{1}{3} & 0 & -\frac{2}{3} \\
 0 & -\frac{2}{3} & -1 & \frac{2}{3} \\
 0 & 2 & 3 & -2
\end{bmatrix} $$
$$\sim
\begin{bmatrix}[ccc|c]
1 & -\frac{1}{3} & 0 & -\frac{2}{3} \\
 0 & 1 & \frac{3}{2} & -1 \\
 0 & 2 & 3 & -2
\end{bmatrix} \sim
\begin{bmatrix}[ccc|c]
1 & 0 & \frac{1}{2} & -1 \\
 0 & 1 & \frac{3}{2} & -1 \\
 0 & 0 & 0 & 0
\end{bmatrix}$$
\end{solution}

\begin{extract}\newpage\end{extract}
\begin{problem}{E3}
Solve the system of linear equations.
\begin{align*}
2x+y-z+w &=5 \\
3x-y-2w &= 0 \\
-x+5z+3w&=-1
\end{align*}
\end{problem}
\begin{solution}
$$\RREF\left( \begin{bmatrix}[cccc|c] 2 & 1 & -1 & 0 & 5 \\ 3 & -1 & 0 & -2 & 0 \\ -1 & 0 & 5 & 0 & -1 \end{bmatrix} \right) = \begin{bmatrix}[cccc|c] 1 & 0 & 0 & -\frac{1}{12} & 1 \\ 0 & 1 & 0 & \frac{7}{4} & 3 \\ 0 & 0 & 1 & \frac{7}{12} & 0 \end{bmatrix}$$
So the solutions are $$\left\{ \begin{bmatrix} 1+a \\ 3-21a \\ -7a \\ 12a \end{bmatrix}\ \big|\ a \in \IR\right\}$$
\end{solution}

\begin{problem}{E4}
Find a basis for the solution set to the homogeneous system of equations
\begin{align*}
2x_1+3x_2-5x_3+14x_4 &= 0\\
x_1+x_2-x_3+5x_4 &= 0
\end{align*}
\end{problem}
\begin{solution}
Let \(A =
  \begin{bmatrix}[cccc|c]
    2 & 3 & -5 & 14 & 0 \\
    1 & 1 & -1 & 5 & 0
  \end{bmatrix}
\), so \(\RREF A =
  \begin{bmatrix}[cccc|c]
    1 & 0 & 2 & 1 & 1 \\
    0 & 1 & -3 & 4 & 2 \\
  \end{bmatrix}
\).
It follows that the basis for the solution set is given by \(\left\{
  \begin{bmatrix}
    - 2 \\
    3 \\
    1 \\
    0
  \end{bmatrix},
  \begin{bmatrix}
    -1 \\
    - 4 \\
    0 \\
    1
  \end{bmatrix}
\right\}\).
\end{solution}

\begin{extract}\newpage\end{extract}
\begin{problem}{V1}
Let $V$ be the set of all polynomials with the operations, for any $f, g \in V$, $c \in \IR$,
\begin{align*}
f \oplus g &= f^\prime + g^\prime \\
c \odot f &= c f^\prime
\end{align*}
(here $f^\prime$ denotes the derivative of $f$).
\begin{enumerate}[(a)]
\item Show that scalar multiplication \textbf{distributes scalars} over
      vector addition:
      \(c\odot(f \oplus g)=
      c\odot f \oplus c\odot g\).
\item Determine if $V$ is a vector space or not.  Justify your answer.
\end{enumerate}
\end{problem}
\begin{solution}
Let $f,g \in \mathcal{P}$, and let $c \in \IR$.
$$c \odot (f \oplus g) = c \odot (f^\prime+g^\prime) =
c(f^\prime+g^\prime)^\prime = cf^{\prime\ \prime}+cg^{\prime\ \prime} =
cf^\prime\oplus cg^\prime= c \odot f \oplus c \odot g.$$
However, this is not a vector space, as there is no zero vector.  Additionally, $1 \odot f \neq f$ for any nonzero polynomial $f$.
\end{solution}


\begin{problem}{V2}
  Determine if
  \(\begin{bmatrix} 0 \\ -1 \\ 6 \\ -7 \end{bmatrix}\)
  belongs to the span of the set
  \(\left\{
    \begin{bmatrix} 2 \\ 0 \\ -1 \\ 5 \end{bmatrix},
    \begin{bmatrix} 4 \\ -1 \\ 4 \\ 3 \end{bmatrix}
    \right\}
  \).
\end{problem}
\begin{solution}
  Since
  \[
    \RREF\left(
      \begin{bmatrix}[cc|c]
        2 & 4 & 0 \\
        0 & -1 & -1 \\
        -1 & 4 & 6 \\
        5 & 3 & -7
      \end{bmatrix}
    \right) =
    \begin{bmatrix}[cc|c]
      1 & 0 & -2 \\
      0 & 1 & 1 \\
      0 & 0 & 0 \\
      0 & 0 & 0
    \end{bmatrix}
  \]
  does not contain a contradiction,
  \(\begin{bmatrix} 0 \\ -1 \\ 6 \\ -7 \end{bmatrix}\) is
  a linear combination of the three vectors.
\end{solution}


\begin{extract}\newpage\end{extract}
\begin{problem}{V3}
Determine if the vectors  $\begin{bmatrix} -3 \\ 1 \\ 1 \end{bmatrix}$,$\begin{bmatrix} 5 \\ -1 \\ -2 \end{bmatrix}$,$\begin{bmatrix}2 \\ 0 \\ -1 \end{bmatrix}$, and $\begin{bmatrix} 0 \\ 2 \\ -1\end{bmatrix}$ span $\IR^3$
\end{problem}
\begin{solution}
$$\RREF\left(\begin{bmatrix}
-3 & 5 & 2 & 0 \\ 1 & -1 & 0 & 2 \\ 1 & -2 & -1 & -1 \end{bmatrix}\right)=\begin{bmatrix} 1 & 0 & 1 & 5 \\ 0 & 1 & 1 & 3 \\ 0 & 0 & 0 & 0\end{bmatrix}$$
Since the resulting matrix has only two pivot columns, the vectors do not span $\IR^3$.
\end{solution}


\begin{problem}{V4} Let \(W\) be the set of all \(\IR^3\) vectors
\(\begin{bmatrix} x \\ y \\ z \end{bmatrix}\) satisfying \(x+y+z=0\) (this forms a plane).
Determine if \(W\) is a subspace of \(\IR^3\).
\end{problem}
\begin{solution}
Yes, because \(z=-x-y\) and
\(
  a\begin{bmatrix} x_1 \\ y_1 \\ -x_1-y_1 \end{bmatrix}+
  b\begin{bmatrix} x_2 \\ y_2 \\ -x_2-y_2 \end{bmatrix}=
  \begin{bmatrix}
    ax_1+bx_2 \\
    ay_1+by_2 \\
    -(ax_1+bx_2)-(ay_1+by_2)
  \end{bmatrix}
\).
Alternately, yes because \(W\) is isomorphic to \(\IR^2\).
\end{solution}


\begin{extract}\newpage\end{extract}
\begin{problem}{S1}
Determine if the vectors $\begin{bmatrix} 1 \\ 1 \\ -1 \end{bmatrix}$, $\begin{bmatrix} 3 \\ -1 \\ 1 \end{bmatrix}$, and $\begin{bmatrix} 2 \\ 0 \\ -2 \end{bmatrix}$ are linearly dependent or linearly independent
\end{problem}
\begin{solution}
$$\RREF\left(\begin{bmatrix} 1 & 3 & 2 \\ 1 & -1 & 0 \\ -1 & 1 & -2 \end{bmatrix} \right)= \begin{bmatrix} 1 & 0 &0 \\ 0 & 1 & 0 \\ 0 & 0 & 1\end{bmatrix}$$
Since each column is a pivot column, the vectors are linearly independent.
\end{solution}


\begin{problem}{S2}
Determine if the set $\left\{ x^3-x, x^2+x+1, x^3-x^2+2, 2x^2-1 \right\}$ is a basis of $\P_3$
\end{problem}
\begin{solution}
$$\RREF\left(\begin{bmatrix} 1 & 0 & 1 & 0 \\ 0 & 1 & -1 & 2 \\ -1 & 1 & 0 & 0 \\ 0 & 1 & 2 & -1 \end{bmatrix} \right)= \begin{bmatrix} 1 & 0 &0 & 1 \\ 0 & 1 & 0 & 1 \\ 0 & 0 & 1 & -1 \\ 0 & 0 & 0 & 0 \end{bmatrix}$$
Since the resulting matrix is not the identity matrix, it is not a basis.
\end{solution}


\begin{extract}\newpage\end{extract}
\begin{problem}{S3}
Let $W$ be the subspace of $\P_3$ given by $W = {\rm span} \left( \left\{ x^3+x^2+2x+1, 3x^3+3x^2+6x+3, 3x^3-x^2+3x-2, 7x^3-x^2+8x-3 \right\} \right)$.  Find a basis for $W$.
\end{problem}
\begin{solution}
$$\RREF\left(\begin{bmatrix} 1 & 3 & 3 & 7 \\ 1 & 3 & -1 & -1 \\ 2 & 6 & 3 & 8 \\ 1 & 3 & -2 & -3 \end{bmatrix}\right) = \begin{bmatrix} 1 & 3 & 0 & 1 \\ 0 & 0 & 1 & 2 \\ 0 & 0 & 0 & 0 \\  0 & 0 & 0 & 0 \end{bmatrix}$$

Then a basis is 
$ \left\{ x^3+x^2+2x+1, 3x^3-x^2+3x-2 \right\} $.
\end{solution}


\begin{problem}{S4}  
Let $W$ be the subspace of $\P_3$ given by $W={\rm span}\left( \left\{ x^3-x^2+3x-3, 2x^3+x+1, 3x^3-x^2+4x-2, x^3+x^2+x-7\right\}\right)$.  Compute the dimension of $W$.
\end{problem}
\begin{solution}
$$ \RREF \left( \begin{bmatrix} 1 & 2 & 3 & 1 \\ -1 & 0 & -1 & 1 \\ 3 & 1 & 4 & 1 \\ -3 & 1 & -2 & -7 \end{bmatrix} \right) =  \begin{bmatrix} 1 & 0 & 1 & 0 \\ 0 & 1 & 1 & 0 \\ 0 & 0 & 0 & 1 \\ 0 & 0 & 0 & 0\end{bmatrix}$$
This has 3 pivot columns so $\dim(W)=3$.
\end{solution}


\end{document}