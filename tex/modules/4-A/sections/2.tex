\begin{applicationActivities}

\begin{remark}
Recall that a linear map \(T:V\rightarrow W\)
satisfies
\begin{enumerate}
\item \(T(\vec{v}+\vec{w}) = T(\vec{v})+T(\vec{w})\)
      for any \(\vec{v},\vec{w} \in V\).
\item \(T(c\vec{v}) = cT(\vec{v})\)
      for any \(c \in \IR,\vec{v} \in V\).
\end{enumerate}
In other words, a map is linear when vecor space operations
can be applied before or after the transformation without affecting the result.
\end{remark}

\begin{activity}{5}
Suppose \(T: \IR^3 \rightarrow \IR^2\) is a linear map, and you know
\(
  T\left(\begin{bmatrix} 1 \\ 0 \\ 0 \end{bmatrix} \right)
=
  \begin{bmatrix} 2 \\ 1 \end{bmatrix}
\)
and
\(
  T\left(\begin{bmatrix} 0 \\ 0 \\ 1 \end{bmatrix} \right)
=
  \begin{bmatrix} -3 \\ 2 \end{bmatrix}
\).
Compute \(T\left(\begin{bmatrix} 3 \\ 0 \\ 0 \end{bmatrix}\right)\).
\begin{multicols}{2}
\begin{enumerate}[(a)]
\item \(\begin{bmatrix} 6 \\ 3\end{bmatrix}\)
\item \(\begin{bmatrix} -9 \\ 6 \end{bmatrix}\)
\item \(\begin{bmatrix} -4 \\ -2 \end{bmatrix}\)
\item \(\begin{bmatrix} 6 \\ -4 \end{bmatrix}\)
\end{enumerate}
\end{multicols}
\end{activity}

\begin{activity}{3}
Suppose \(T: \IR^3 \rightarrow \IR^2\) is a linear map, and you know
\(
  T\left(\begin{bmatrix} 1 \\ 0 \\ 0 \end{bmatrix} \right)
=
  \begin{bmatrix} 2 \\ 1 \end{bmatrix}
\)
and
\(
  T\left(\begin{bmatrix} 0 \\ 0 \\ 1 \end{bmatrix} \right)
=
  \begin{bmatrix} -3 \\ 2 \end{bmatrix}
\).
Compute \(T\left(\begin{bmatrix} 1 \\ 0 \\ 1 \end{bmatrix}\right)\).
\begin{multicols}{2}
\begin{enumerate}[(a)]
\item \(\begin{bmatrix} 2 \\ 1\end{bmatrix}\)
\item \(\begin{bmatrix} 3 \\ -1 \end{bmatrix}\)
\item \(\begin{bmatrix} -1 \\ 3 \end{bmatrix}\)
\item \(\begin{bmatrix} 5 \\ -8 \end{bmatrix}\)
\end{enumerate}
\end{multicols}
\end{activity}

\begin{activity}{2}
Suppose \(T: \IR^3 \rightarrow \IR^2\) is a linear map, and you know
\(
  T\left(\begin{bmatrix} 1 \\ 0 \\ 0 \end{bmatrix} \right)
=
  \begin{bmatrix} 2 \\ 1 \end{bmatrix}
\)
and
\(
  T\left(\begin{bmatrix} 0 \\ 0 \\ 1 \end{bmatrix} \right)
=
  \begin{bmatrix} -3 \\ 2 \end{bmatrix}
\).
Compute \(T\left(\begin{bmatrix} -2 \\ 0 \\ -3 \end{bmatrix}\right)\).
\begin{multicols}{2}
\begin{enumerate}[(a)]
\item \(\begin{bmatrix} 2 \\ 1\end{bmatrix}\)
\item \(\begin{bmatrix} 3 \\ -1 \end{bmatrix}\)
\item \(\begin{bmatrix} -1 \\ 3 \end{bmatrix}\)
\item \(\begin{bmatrix} 5 \\ -8 \end{bmatrix}\)
\end{enumerate}
\end{multicols}
\end{activity}

\begin{activity}{5}
Suppose \(T: \IR^3 \rightarrow \IR^2\) is a linear map, and you know
\(
  T\left(\begin{bmatrix} 1 \\ 0 \\ 0 \end{bmatrix} \right)
=
  \begin{bmatrix} 2 \\ 1 \end{bmatrix}
\)
and
\(
  T\left(\begin{bmatrix} 0 \\ 0 \\ 1 \end{bmatrix} \right)
=
  \begin{bmatrix} -3 \\ 2 \end{bmatrix}
\).
Do you have enough information to compute
\(T(\vec{v})\) for \textit{any} \(\vec v\in\IR^3\)?
\begin{enumerate}[(a)]
\item Yes.
\item No, exactly one more piece of information is needed.
\item No, an infinite amount of information would be necessary to compute
      the transformation of infinitely-many vectors.
\end{enumerate}
\end{activity}

\begin{fact}
Consider any basis \(\{\vec b_1,\dots,\vec b_n\}\) for $V$.  Since every
vector \(\vec v\) can be written \textit{uniquely} as a linear combination of
basis vectors, \(x_1\vec b_1+\dots+ x_n\vec b_n\), we conclude that

\[
  T(\vec v)=T(x_1\vec b_1+\dots+ x_n\vec b_n)=
  x_1T(\vec b_1)+\dots+x_nT(\vec b_n)
.\]

Therefore any linear transformation \(T:V \rightarrow W\) can be defined
by just describing the values of \(T(\vec b_i)\).

Put another way, the basis vectors \term{determine} the transformation \(T\).
\end{fact}

\begin{definition}
Since linear transformation \(T:\IR^n\to\IR^m\) is determined by
the standard basis \(\{\vec e_1,\dots,\vec e_n\}\), it's convenient to
store this information in the \(m\times n\) \term{standard matrix}
\([T(\vec e_1) \,\cdots\, T(\vec e_n)]\).
\end{definition}


\begin{example}
Let \(T: \IR^3 \rightarrow \IR^2\) be the linear map determined by
the following values for \(T\) applied to the standard basis of \(\IR^3\).

\(
  T\left(\vec e_1 \right)
=
  T\left(\begin{bmatrix} 1 \\ 0 \\ 0 \end{bmatrix} \right)
=
  \begin{bmatrix} 3 \\ 2\end{bmatrix}
\)
\hfill
\(
  T\left(\vec e_2 \right)
=
  T\left(\begin{bmatrix} 0 \\ 1 \\ 0 \end{bmatrix} \right)
=
  \begin{bmatrix} -1 \\ 4\end{bmatrix}
\)
\hfill
\(
  T\left(\vec e_3 \right)
=
  T\left(\begin{bmatrix} 0 \\ 0 \\ 1 \end{bmatrix} \right)
=
  \begin{bmatrix} 5 \\ 0\end{bmatrix}
\)

Then the standard matrix corresponding to \(T\) is
\[
  \begin{bmatrix}T(\vec e_1) & T(\vec e_2) & T(\vec e_3)\end{bmatrix}
=
  \begin{bmatrix}3 & -1 & 5 \\ 2 & 4 & 0 \end{bmatrix}
.\]
\end{example}

\begin{activity}{5}TODO
  Let $T: \IR^3 \rightarrow \IR^2$ be the linear transformation given by
$$T\left(\begin{bmatrix} x\\ y \\ z \end{bmatrix} \right) = \begin{bmatrix} x+3z \\ 2x-y-4z \end{bmatrix}$$
Write the matrix corresponding to this linear transformation with respect to the standard basis.
\end{activity}

\begin{activity}{5}
  Let $T: \IR^3 \rightarrow \IR^2$ be the linear transformation given by the standard matrix $$\begin{bmatrix} 3  & -2 & -1  \\ 4 & 5 & 2 \end{bmatrix}.$$

Compute $T\left(\begin{bmatrix} x\\ y \\ z \end{bmatrix} \right) $.
\end{activity}

% \begin{fact}
%   $T: \IR^n \rightarrow \IR^m$ is given by the matrix
%   \([\vec v_1 \,\cdots\, \vec v_n]\) exactly when
%   \(T\left(
%     \begin{bmatrix}
%       x_1 \\ \vdots \\ x_n
%     \end{bmatrix}
%   \right)
%     =
%   x_1\vec v_1 + \dots + x_n\vec v_n\).
% \end{fact}

\begin{activity}{10}
Let $D: \P^3 \rightarrow \P^2$ be the derivative map \(D(f(x))=f'(x)\).
(Earlier we showed this is a linear transformation.)
\begin{subactivity}
Write down an equivalent linear transformation $T: \IR^4 \rightarrow \IR^3$
by converting \(\{1,x,x^2,x^3\}\) and \(\{D(1),D(x),D(x^2),D(x^3)\}\) into
appropriate vectors in \(\IR^4\) and \(\IR^3\).
\end{subactivity}
\begin{subactivity}
Write the standard matrix corresponding to $T$.
\end{subactivity}
\end{activity}

\end{applicationActivities}
