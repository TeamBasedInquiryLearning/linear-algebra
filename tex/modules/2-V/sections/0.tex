%!TEX root =../../../course-notes.tex
% ^ leave for LaTeXTools build functionality

\begin{applicationActivities}

% have students capture 8 properties of R2 from list of 11
% include ca=b, unique equidistant, nonzero orthogonal

\begin{activity}{20}\smallSlideText
Consider each of the following vector properties. Label each property
with \(\IR^1\), \(\IR^2\), and/or \(\IR^3\) if that property holds for
Euclidean vectors/scalars \(\vec u,\vec v,\vec w\) of that dimension.
\begin{multicols}{2}
\begin{enumerate}
  \item \textbf{Addition associativity.}

        \(\vec u+(\vec v+\vec w)=
        (\vec u+\vec v)+\vec w\).
  \item \textbf{Addition commutivity.}

        \(\vec u+\vec v=
        \vec v+\vec u\).
  \item \textbf{Addition identity.}

        There exists some \(\vec z\)
        where \(\vec v+\vec z=\vec v\).
  \item \textbf{Addition inverse.}

        There exists some \(-\vec v\)
        where \(\vec v+(-\vec v)=\vec z\).
  \item \textbf{Addition midpoint uniqueness.}

        There exists a unique \(\vec m\) where the distance from
        \(\vec u\) to \(\vec m\) equals the distance from \(\vec m\)
        to \(\vec v\).
  \item \textbf{Scalar multiplication associativity.}

        \(a(b\vec v)=(ab)\vec v\).
  \item \textbf{Scalar multiplication identity.}

        \(1\vec v=\vec v\).
  \item \textbf{Scalar multiplication relativity.}

        There exists some scalar \(c\) where either \(c\vec v=\vec w\)
        or \(c\vec w=\vec v\).
  \item \textbf{Scalar distribution.}

        \(a(\vec u+\vec v)=a\vec u+a\vec v\).
  \item \textbf{Vector distribution.}

        \((a+b)\vec v=a\vec v+b\vec v\).
  \item \textbf{Orthogonality.}

        There exists a non-zero vector \(\vec n\) such that
        \(\vec n\) is orthogonal to both \(\vec u\) and \(\vec v\).
  \item \textbf{Bidimensionality.}

        \(\vec v=a\vec i+b\vec j\) for some value of \(a,b\).
\end{enumerate}
\end{multicols}
\end{activity}

\begin{definition}
  A \term{vector space} \(V\) is any collection of mathematical objects with
  associated addition and scalar multiplication operations that satisfy
  the following properties. Let \(\vec u,\vec v,\vec w\) belong to \(V\),
  and let \(a,b\) be scalar numbers.

  \vectorSpaceProperties

  Any \term{Euclidean vector space} \(\IR^n\) satisfies all eight
  requirements regardless of the value of \(n\),
  but we will also study other types of vector spaces.
\end{definition}


\end{applicationActivities}
