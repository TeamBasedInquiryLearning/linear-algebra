


\begin{applicationActivities}

\begin{observation}
  Recall from last class: to compute a basis for the subspace \(\vspan\{\vect v_1,\dots,\vect v_m\}\),
  simply remove the vectors corresponding to the non-pivot columns of
  \(\RREF[\vect v_1\,\dots\,\vect v_m]\).
\end{observation}

\begin{activity}{10}
  Let 
  \begin{align*}
  S=\left\{
  \begin{bmatrix}2\\3\\0\\1\end{bmatrix},
  \begin{bmatrix}2\\0\\1\\-1\end{bmatrix},
  \begin{bmatrix}2\\-3\\2\\-3\end{bmatrix},
  \begin{bmatrix}1\\5\\-1\\0\end{bmatrix}
  \right\} & & \text{and}  & &
  T=\left\{
  \begin{bmatrix}2\\0\\1\\-1\end{bmatrix},
  \begin{bmatrix}2\\-3\\2\\-3\end{bmatrix},
  \begin{bmatrix}1\\5\\-1\\0\end{bmatrix},
  \begin{bmatrix}2\\3\\0\\1\end{bmatrix}
  \right\}
  \end{align*}
  \begin{subactivity}
  Find a basis for \(\vspan S\)
  \end{subactivity}
  \begin{subactivity}
  Find a basis for \(\vspan T\)
  \end{subactivity}
\end{activity}


\begin{fact}
  A vector space has a lot of bases, but all bases for a given vector space must be the same size.
\end{fact}

\begin{definition}
  The \term{dimension} of a vector space is given by the cardinality/size
  of any basis for the vector space.
\end{definition}

\begin{activity}{15}
  Find the dimension of each subspace of \(\IR^4\). \\
	\begin{tabular}{ll}
     \(\vspan\left\{
    \begin{bmatrix}1\\0\\0\\0\end{bmatrix},
    \begin{bmatrix}0\\1\\0\\0\end{bmatrix},
    \begin{bmatrix}0\\0\\1\\0\end{bmatrix},
    \begin{bmatrix}0\\0\\0\\1\end{bmatrix}
    \right\}
    \) &

     \(\vspan\left\{
    \begin{bmatrix}2\\3\\0\\-1\end{bmatrix},
    \begin{bmatrix}2\\0\\0\\3\end{bmatrix},
    \begin{bmatrix}4\\3\\0\\2\end{bmatrix},
    \begin{bmatrix}-3\\0\\1\\3\end{bmatrix}
    \right\}
    \) \\

 & \\
     \(\vspan\left\{
    \begin{bmatrix}2\\3\\0\\-1\end{bmatrix},
    \begin{bmatrix}2\\0\\0\\3\end{bmatrix},
    \begin{bmatrix}3\\13\\7\\16\end{bmatrix},
    \begin{bmatrix}-1\\10\\7\\14\end{bmatrix},
    \begin{bmatrix}4\\3\\0\\2\end{bmatrix}
    \right\}
    \)
 &
     \(\vspan\left\{
    \begin{bmatrix}2\\3\\0\\-1\end{bmatrix},
    \begin{bmatrix}4\\3\\0\\2\end{bmatrix},
    \begin{bmatrix}-3\\0\\1\\3\end{bmatrix},
    \begin{bmatrix}3\\6\\1\\5\end{bmatrix}
    \right\}
    \) \\

 & \\
     \(\vspan\left\{
    \begin{bmatrix}5\\3\\0\\-1\end{bmatrix},
    \begin{bmatrix}-2\\1\\0\\3\end{bmatrix},
    \begin{bmatrix}4\\5\\1\\3\end{bmatrix}
    \right\}
	\) \\
\end{tabular}
\end{activity}
\begin{fact}
  Every vector space with finite dimension, that is, every
  vector space with a basis of the form
  \(\{\vect v_1,\vect v_2,\dots,\vect v_n\}\) is isomorphic to a
  Euclidean space \(\IR^n\):

  \[
    c_1\vect v_1+c_2\vect v_2+\dots+c_n\vect v_n
    \leftrightarrow
    \begin{bmatrix}
      c_1\\c_2\\\vdots\\c_n
    \end{bmatrix}
  \]
\end{fact}

\begin{observation}
  Several interesting vector spaces are infinite-dimensional:
  \begin{itemize}
    \item The space of polynomials \(\P\) (consider the set
          \(\{1,x,x^2,x^3,\dots\}\)).
    \item The space of continuous functions \(C(\IR)\) (which contains
          all polynomials, in addition to other functions like
          \(e^x\)).
    \item The space of real number sequences \(\IR^\infty\) (consider
          the set \(\{(1,0,0,\dots),(0,1,0,\dots),(0,0,1,\dots),\dots\}\)).
  \end{itemize}
\end{observation}

\begin{definition}
A \textbf{homogeneous} system of linear equations is one of the form
\[x_1 \vec{v}_1 + \cdots+x_n \vec{v}_n = \vec{0} .\]

\vspace{1em}
Note that if \(\begin{bmatrix} a_1 \\ \vdots \\ a_n \end{bmatrix} \) and \(\begin{bmatrix} b_1 \\ \vdots \\ b_n \end{bmatrix} \) are solutions, so is  \(\begin{bmatrix} a_1 +b_1\\ \vdots \\ a_n+b_n \end{bmatrix} \)  i.e. if
\[a_1 \vec{v}_1+\cdots+a_n \vec{v}_n = \vec{0} \]
and
\[b_1 \vec{v}_1+\cdots+b_n \vec{v}_n = \vec{0} \]
then
\[(a_1 + b_1) \vec{v}_1+\cdots+(a_n+b_n) \vec{v}_n = \vec{0} .\]

Similarly, if \(c \in \IR\), \(\begin{bmatrix} ca_1 \\ \vdots \\ ca_n \end{bmatrix} \) is a solution.
\vspace{1em}
Thus the solution set of a homogeneous system is a subspace.
\end{definition}

\begin{activity}{10}
Consider the homogeneous system of equations 
\begin{alignat*}{5}
x_1&\,+\,&2x_2&\,\,& &\,+\,& x_4 &=& 0 \\
2x_1&\,+\,&4x_2&\,-\,&x_3 &\,-\,&2 x_4 &=& 0 \\
3x_1&\,+\,&6x_2&\,-\,&x_3 &\,-\,& x_4 &=& 0 \\
\end{alignat*}
\begin{subactivity}
Find the solution set.
\end{subactivity}
\begin{subactivity}
Rewrite the solution set in the form \[\setBuilder{ a \begin{bmatrix} ? \\ ? \\ ? \\ ?\end{bmatrix} + b \begin{bmatrix} ? \\ ? \\ ? \\ ? \end{bmatrix} }{a,b \in \IR}\].
\end{subactivity}
\begin{subactivity}
Find a basis for the solution set.
\end{subactivity}
\end{activity}

\begin{activity}{10}
Consider the homogeneous system of equations 
\begin{alignat*}{5}
x_1&\,-\,&3x_2&\,+\,& 2x_3&\,\,&  &=& 0 \\
2x_1&\,-\,&6x_2&\,+\,&4x_3 &\,+\,&3 x_4 &=& 0 \\
-2x_1&\,+\,&6x_2&\,-\,&4x_3 &\,-\,&4 x_4 &=& 0 \\
\end{alignat*}

Find a basis for the solution set.
\end{activity}

\begin{activity}{5}
Suppose \(W\) is a subspace of \(\P^8\), and you know that the set \(\{ x^3+x, x^2+1, x^4-x \}\) is a linearly independent subset of \(W\).  What can you conclude about \(W\)?
\begin{enumerate}[(a)]
\item The dimension of \(W\) is no more than 3
\item The dimension of \(W\) is  3
\item The dimension of \(W\) is  at least 3
\end{enumerate}
\end{activity}

\begin{activity}{5}
Suppose \(W\) is a subspace of \(\P^8\), and you know that \(W\) is spanned by the six vectors \[\{ x^4-x,x^3+x,x^3+x+1,x^4+2x,x^3,2x+1\}\]
Without doing any calculation, what can you conclude about \(W\)?
\begin{enumerate}[(a)]
\item The dimension of \(W\) is no more than 6
\item The dimension of \(W\) is  6
\item The dimension of \(W\) is  at least 6
\end{enumerate}
\end{activity}



\end{applicationActivities}
