\begin{applicationActivities}

\begin{remark}
Recall that the \term{product} \(AB\) of a 
\(m \times n\) matrix \(A\) and an \(n \times k\)
matrix \(B\) is the \(m \times k\) 
standard matrix of the composition map of the
two corresponding linear functions.

\vspace{1em}

For example, if \(S\) has a 
\(4\times \circledNumber{2}\) matrix \(A\) and
\(T\) has a \(\circledNumber{2}\times 3\) matrix 
\(B\), then \(S\circ T\) has a
\(4\times 3\) standard matrix:
\[
  AB
    =
  \begin{bmatrix} 1 & 2 \\ 0 & 1 \\ 3 & 5 \\ -1 & -2 \end{bmatrix}
  \begin{bmatrix} 2 & 1 & -3 \\ 5 & -3 & 4 \end{bmatrix}
\]
\[
    =
  \left[
  (S \circ T)(\vec{e}_1) \hspace{1em}
  (S\circ T)(\vec{e}_2) \hspace{1em}
  (S \circ T)(\vec{e}_3)
  \right]
    =
  \begin{bmatrix}
    12 & -5 & 5 \\
    5 & -3 & 4 \\
    31 & -12 & 11 \\
    -12 & 5 & -5
  \end{bmatrix}
.\]
\end{remark}






\begin{activity}{15}
Let \(B=\begin{bmatrix} 3 & -4 & 0 \\ 2 & 0 & -1 \\ 0 & -3 & 3 \end{bmatrix}\), 
and let \(A=\begin{bmatrix} 2 & 7 & -1 \\ 0 & 3 & 2 \\ 1 & 1 & -1 \end{bmatrix}\).  
\begin{subactivity}
  Compute the product \(BA\) by hand.
\end{subactivity}
\begin{subactivity}
  Check your work using technology. Using Octave:
  \begin{itemize}
    \item \texttt{ B = sym([3 -4 0 ; 2 0 -1 ; 0 -3 3]) }
    \item \texttt{ A = sym([2 7 -1 ; 0 3 2  ; 1 1 -1]) }
    \item \texttt{ B*A }
  \end{itemize}
\end{subactivity}
\end{activity}

\begin{activity}{5}
Let \(A=\begin{bmatrix} 2 & 7 & -1 \\ 0 & 3 & 2 \\ 1 & 1 & -1 \end{bmatrix}\).  
Find a \(3 \times 3\) matrix \(B\) such that \(BA=A\), that is,
\[
  \begin{bmatrix} \unknown & \unknown & \unknown \\ 
  \unknown & \unknown & \unknown 
  \\ \unknown & \unknown & \unknown \end{bmatrix}
  \begin{bmatrix} 2 & 7 & -1 \\ 0 & 3 & 2 \\ 1 & 1 & -1 \end{bmatrix}
=
  \begin{bmatrix} 2 & 7 & -1 \\ 0 & 3 & 2 \\ 1 & 1 & -1 \end{bmatrix}
\]
Check your guess using technology.
\end{activity}

\begin{definition}
The identity matrix $I_n$ (or just $I$ when $n$ is obvious from context) is  the $n \times n$ matrix $$I_n = \begin{bmatrix} 1 & 0  & \hdots & 0 \\ 0 & 1 & \ddots & \vdots  \\ \vdots & \ddots & \ddots & 0 \\ 0 & \hdots & 0 & 1 \end{bmatrix}.$$
It has a $1$ on each diagonal element and a $0$ in every other position.
\end{definition}

\begin{fact}
  For any square matrix \(A\), \(IA=AI=A\):

  \[
    \begin{bmatrix} 1 & 0 & 0 \\ 0 & 1 & 0 \\ 0 & 0 & 1 \end{bmatrix}
    \begin{bmatrix} 2 & 7 & -1 \\ 0 & 3 & 2 \\ 1 & 1 & -1 \end{bmatrix}
  =
    \begin{bmatrix} 2 & 7 & -1 \\ 0 & 3 & 2 \\ 1 & 1 & -1 \end{bmatrix}
      \begin{bmatrix} 1 & 0 & 0 \\ 0 & 1 & 0 \\ 0 & 0 & 1 \end{bmatrix}
  =
    \begin{bmatrix} 2 & 7 & -1 \\ 0 & 3 & 2 \\ 1 & 1 & -1 \end{bmatrix}
  \]
\end{fact}

\begin{activity}{20}
Tweaking the identity matrix slightly allows us to write row operations
in terms of matrix multiplication.
\begin{subactivity}
Create a matrix that doubles the third row of \(A\):
\[
 \begin{bmatrix} \unknown & \unknown & \unknown \\ \unknown & \unknown & \unknown \\ \unknown & \unknown & \unknown \end{bmatrix}
 \begin{bmatrix} 2 & 7 & -1 \\ 0 & 3 & 2 \\ 1 & 1 & -1 \end{bmatrix}
=
 \begin{bmatrix} 2 & 7 & -1 \\ 0 & 3 & 2 \\ 2 & 2 & -2 \end{bmatrix}
\]
\end{subactivity}
\begin{subactivity}
  Create a matrix that swaps the second and third rows of \(A\):
  \[
   \begin{bmatrix} \unknown & \unknown & \unknown \\ \unknown & \unknown & \unknown \\ \unknown & \unknown & \unknown \end{bmatrix}
   \begin{bmatrix} 2 & 7 & -1 \\ 0 & 3 & 2 \\ 1 & 1 & -1 \end{bmatrix}
  =
  \begin{bmatrix} 2 & 7 & -1 \\ 1 & 1 & -1 \\ 0 & 3 & 2 \end{bmatrix}
  \]
\end{subactivity}
\begin{subactivity}
Create a matrix that adds \(5\) times the third row of \(A\) to the first row:
\[
 \begin{bmatrix} \unknown & \unknown & \unknown \\ \unknown & \unknown & \unknown \\ \unknown & \unknown & \unknown \end{bmatrix}
 \begin{bmatrix} 2 & 7 & -1 \\ 0 & 3 & 2 \\ 1 & 1 & -1 \end{bmatrix}
=
 \begin{bmatrix} 2+5(1) & 7+5(1) & -1+5(-1) \\ 0 & 3 & 2 \\ 1 & 1 & -1 \end{bmatrix}
\]
\end{subactivity}
\end{activity}

\begin{fact}
If \(R\) is the result of applying a row operation to \(I\), then
\(RA\) is the result of applying the same row operation to \(A\).
\begin{itemize}
\item Scaling a row: \(R=
  \begin{bmatrix}
  c & 0 & 0 \\
  0 & 1 & 0 \\
  0 & 0 & 1
  \end{bmatrix}
\)
\item Swapping rows: \(R=
  \begin{bmatrix}
  0 & 1 & 0 \\
  1 & 0 & 0 \\
  0 & 0 & 1
  \end{bmatrix}
\)
\item Adding a row multiple to another row: \(R=
  \begin{bmatrix}
  1 & 0 & c \\
  0 & 1 & 0 \\
  0 & 0 & 1
  \end{bmatrix}
\)
\end{itemize}

Such matrices can be chained together to emulate multiple row operations.
In particular,
\[\RREF(A)=R_k\dots R_2R_1A\]
for some sequence of matrices \(R_1,R_2,\dots,R_k\).
\end{fact}

\begin{activity}{10}
Consider the two row operations 
\(R_2\leftrightarrow R_3\) and \(R_1+R_3\to R_1\)
applied as follows to show \(A\sim B\):
\begin{align*}
A
  =
\begin{bmatrix}
-1&4&5\\
0&3&-1\\
1&2&3\\
\end{bmatrix}
  &\sim
\begin{bmatrix}
-1&4&5\\
1&2&3\\
0&3&-1\\
\end{bmatrix}
  \\&\sim
\begin{bmatrix}
-1+1&4+2&5+3\\
1&2&3\\
0&3&-1\\
\end{bmatrix}
  =
\begin{bmatrix}
0&6&8\\
1&2&3\\
0&3&-1\\
\end{bmatrix}
  = 
B
\end{align*}
Express these row operations as matrix multiplication
by expressing \(B\) as the product of two matrices and \(A\):
\[
B =
\begin{bmatrix}
\unknown&\unknown&\unknown\\
\unknown&\unknown&\unknown\\
\unknown&\unknown&\unknown
\end{bmatrix}
\begin{bmatrix}
\unknown&\unknown&\unknown\\
\unknown&\unknown&\unknown\\
\unknown&\unknown&\unknown
\end{bmatrix}
A
\]
Check your work using technology.
\end{activity}






\end{applicationActivities}
