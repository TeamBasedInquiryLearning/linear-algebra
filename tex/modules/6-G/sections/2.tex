\begin{applicationActivities}

\begin{remark}
  Recall that the column versions of the three row-reducing operations
  a matrix may be used to simplify a determinant:
  \begin{enumerate}[(a)]
  \item Multiplying columns by scalars:
        \[\det([\cdots\hspace{0.5em}c\vec{v}\hspace{0.5em} \cdots])=
        c\det([\cdots\hspace{0.5em}\vec{v}\hspace{0.5em} \cdots])\]
  \item Swapping two columns:
        \[\det([\cdots\hspace{0.5em}\vec{v}\hspace{0.5em}
        \cdots\hspace{1em}\vec{w}\hspace{0.5em} \cdots])=
        -\det([\cdots\hspace{0.5em}\vec{w}\hspace{0.5em}
        \cdots\hspace{1em}\vec{v}\hspace{0.5em} \cdots])\]
  \item Adding a multiple of a column to another column:
        \[\det([\cdots\hspace{0.5em}\vec{v}\hspace{0.5em}
        \cdots\hspace{1em}\vec{w}\hspace{0.5em} \cdots])=
        \det([\cdots\hspace{0.5em}\vec{v}+c\vec{w}\hspace{0.5em}
        \cdots\hspace{1em}\vec{w}\hspace{0.5em} \cdots])\]
  \end{enumerate}
\end{remark}
\begin{remark}
The determinants of row operation matrices may be computed
by manipulating columns to reduce each matrix to the identity:
\begin{itemize}
\item Scaling a row: \(\det
  \begin{bmatrix}
  c & 0 & 0 \\
  0 & 1 & 0 \\
  0 & 0 & 1
  \end{bmatrix}
    =
  c\det
  \begin{bmatrix}
  1 & 0 & 0 \\
  0 & 1 & 0 \\
  0 & 0 & 1
  \end{bmatrix}
    =
  c
\)
\item Swapping rows: \(\det
  \begin{bmatrix}
  0 & 1 & 0 \\
  1 & 0 & 0 \\
  0 & 0 & 1
  \end{bmatrix}
    =
  -1\det
  \begin{bmatrix}
  1 & 0 & 0 \\
  0 & 1 & 0 \\
  0 & 0 & 1
  \end{bmatrix}
    =
  -1
\)
\item Adding a row multiple to another row: \(\det
  \begin{bmatrix}
  1 & 0 & c \\
  0 & 1 & 0 \\
  0 & 0 & 1
  \end{bmatrix}
    =
  \det
  \begin{bmatrix}
  1 & 0 & c-1c \\
  0 & 1 & 0-0c \\
  0 & 0 & 1-0c
  \end{bmatrix}
    =
  \det(I)=1
\)
\end{itemize}
\end{remark}
\begin{fact}
Thus we can also use row operations to simplify determinants:
\begin{enumerate}
\item Multiplying rows by scalars:
  \(\det\begin{bmatrix}\vdots\\cR\\\vdots\end{bmatrix}=
  c\det\begin{bmatrix}\vdots\\R\\\vdots\end{bmatrix}\)
\item Swapping two rows:
  \(\det\begin{bmatrix}\vdots\\R\\\vdots\\S\\\vdots\end{bmatrix}=
  -\det\begin{bmatrix}\vdots\\S\\\vdots\\R\\\vdots\end{bmatrix}\)
\item Adding multiples of rows to other rows:
  \(\det\begin{bmatrix}\vdots\\R\\\vdots\\S\\\vdots\end{bmatrix}=
  \det\begin{bmatrix}\vdots\\R+cS\\\vdots\\S\\\vdots\end{bmatrix}\)
\end{enumerate}
\end{fact}



\begin{observation}
  So we may compute the determinant of \(\begin{bmatrix} 2 & 4 \\ 2 & 3 \end{bmatrix}\) 
  by manipulating its rows/columns to reduce the matrix to \(I\):

  \begin{align*}
    \det\begin{bmatrix} 2 & 4 \\ 2 & 3 \end{bmatrix}
      &=
    2 \det \begin{bmatrix} 1 & 2 \\ 2 & 3 \end{bmatrix}\\
      &=
    %2 \det \begin{bmatrix} 1 & 2 \\ 2-2(1) & 3-2(2)\end{bmatrix}=
    2 \det \begin{bmatrix} 1 & 2 \\ 0 & -1 \end{bmatrix}\\
      &=
    %2(-1) \det \begin{bmatrix} 1 & -2 \\ 0 & +1 \end{bmatrix}=
    -2 \det \begin{bmatrix} 1 & -2 \\ 0 & 1 \end{bmatrix}\\
      &=
    %-2 \det \begin{bmatrix} 1+2(0) & -2+2(1) \\ 0 & 1\end{bmatrix} =
    -2 \det \begin{bmatrix} 1 & 0 \\ 0 & 1 \end{bmatrix}\\
      &=
    %-2\det I = 
	%-2(1) = 
	-2
  \end{align*}
\end{observation}

\begin{remark}
So we see that row reducing all the way into RREF gives us a method of computing determinants!

\vspace{1em}

However, we learned in module E that this can be tedious for large matrices.  Thus, we will try
to figure out how to turn the determinant of a larger matrix
into the determinant of a smaller matrix.
\end{remark}



\begin{activity}{5}
  The following image illustrates the transformation of the unit cube
  by the matrix
  $\begin{bmatrix} 3 & 1 & 0 \\  1 & 1 & 1 \\  0 & 0 & 1\end{bmatrix}$.

  \begin{center}
  \begin{tikzpicture}
  \fill[purple!50!white] (0,0,0) -- (1,0,1) -- (4,0,2) -- (3,0,1) -- (0,0,0);
  \draw[thin,gray,->] (0,0,0) -- (3,0,0);
  \draw[thin,gray,->] (0,0,0) -- (0,2,0);
  \draw[thin,gray,->] (0,0,0) -- (0,0,2);
  %(y,z,x)
  \draw[blue] (1,0,1) -- (4,0,2) -- (3,0,1);
  \draw[blue] (1,1,0) -- (2,1,1) -- (5,1,2) -- (4,1,1) -- (1,1,0);
  \draw[blue] (1,0,1) -- +(1,1,0);
  \draw[blue] (4,0,2) -- +(1,1,0);
  \draw[blue] (3,0,1) -- +(1,1,0);

  \draw[purple,thick,->] (0,0,0) -- (1,1,0)
    node[above left]{\tiny$\begin{bmatrix} 0 \\ 1 \\ 1\end{bmatrix}$};
  \draw[purple,thick,->] (0,0,0) -- (1,0,1)
    node[below]{\tiny$\begin{bmatrix} 1 \\ 1 \\ 0\end{bmatrix}$};
  \draw[purple,thick,->] (0,0,0) -- (3,0,1)
    node[above right]{\tiny$\begin{bmatrix} 3 \\ 1 \\ 0\end{bmatrix}$};
  \draw[purple,dashed,very thick] (0,0,0) -- node[left] {\tiny\(h=1\)} (0,1,0);
  \end{tikzpicture}
  \end{center}
  Recall that for this solid \(V=Bh\), where \(h\) is the height of the solid 
  and \(B\) is the area of its parallelogram base.
  So what must its volume be?
\begin{multicols}{4}
\begin{enumerate}[(a)]
\item $\det \begin{bmatrix} 3 & 1 \\ 1 & 1 \end{bmatrix}$
\item $\det \begin{bmatrix} 3 & 1 \\ 1 & 0 \end{bmatrix}$
\item $\det \begin{bmatrix} 3 & 1 \\ 0 & 1 \end{bmatrix}$
\item $\det \begin{bmatrix} 1 & 1 \\ 0 & 1 \end{bmatrix}$
\end{enumerate}
\end{multicols}
\end{activity}

\begin{fact}
If row \(i\) contains all zeros except for a \(1\) on the 
main (upper-left to lower-right) diagonal, 
then both column and row \(i\)
may be removed without changing the value of the determinant.
\[
  \det \begin{bmatrix}
    3 & \textcolor{red}{2} & -1 & 3 \\
    \textcolor{red}{0} & \textcolor{red}{1} 
      & \textcolor{red}{0} & \textcolor{red}{0} \\
    -1 & \textcolor{red}{4} & 1 & 0 \\
    5 & \textcolor{red}{0} & 11 & 1
  \end{bmatrix} =
  \det \begin{bmatrix}
    3 & -1 & 3 \\
    -1 & 1 & 0 \\
    5 & 11 & 1
  \end{bmatrix}
\]
Since row and column operations affect the determinants in the same
way, the same technique works for a column of all zeros except for
a \(1\) on the main diagonal.
\[
  \det \begin{bmatrix}
    3 & \textcolor{red}{0} & -1 & 5 \\
    \textcolor{red}{2} & \textcolor{red}{1} & \textcolor{red}{4} & 
       \textcolor{red}{0} \\
    -1 & \textcolor{red}{0} & 1 & 11 \\
    3 & \textcolor{red}{0} & 0 & 1
  \end{bmatrix} =
  \det \begin{bmatrix}
    3 & -1 & 5 \\
    -1 & 1 & 11 \\
    3 & 0 & 1
  \end{bmatrix}
\] 
\end{fact}


\begin{activity}{5}
  Remove an appropriate row and column of  
  $\det \begin{bmatrix} 1 & 0 & 0 \\ 1 & 5 & 12 \\ 3 & 2 & -1 \end{bmatrix}$
  to simplify the determinant to a \(2\times 2\) determinant.
\end{activity}

\begin{activity}{5}
  Simplify
  \(\det \begin{bmatrix} 0 & 3 & -2 \\ 2 & 5 & 12 \\ 0 & 2 & -1 \end{bmatrix}\)
  to a multiple of a \(2\times 2\) determinant by first doing the following:
  \begin{itemize}
    \item Factor out a \(2\) from a column.
    \item Swap rows or columns to put a \(1\) on the main diagonal.
  \end{itemize}
\end{activity}

\begin{activity}{5}
  Simplify
  \(\det \begin{bmatrix} 4 & -2 & 2 \\ 3 & 1 & 4 \\ 1 & -1 & 3\end{bmatrix}\)
  to a multiple of a \(2\times 2\) determinant by first doing the following:
  \begin{itemize}
    \item Use row/column operations to create two zeroes in the same row or column.
    \item Factor/swap as needed to get a row/column of all zeroes except 
      a \(1\) on the main diagonal.
  \end{itemize}
\end{activity}


\begin{observation}
Using row/column operations, you can introduce zeros
and reduce dimension to whittle down the determinant of a large
matrix to a determinant of a smaller matrix.

\begin{align*}
    \det\begin{bmatrix} 
      4 & 3 & 0 & 1 \\ 
      2 & -2 & 4 & 0 \\ 
      -1 & 4 & 1 & 5 \\ 
      2 & 8 & 0 & 3 
    \end{bmatrix}
  &=
    \det\begin{bmatrix} 
      4 & 3 & \textcolor{red}{0} & 1 \\ 
      6 & -18 & \textcolor{red}{0} & -20 \\ 
      \textcolor{red}{-1} & \textcolor{red}{4} & 
        \textcolor{red}{1} & \textcolor{red}{5} \\ 
      2 & 8 & \textcolor{red}{0} & 3 
    \end{bmatrix}
  =
    \det\begin{bmatrix} 
      4 & 3 & 1 \\ 
      6 & -18 & -20 \\ 
      2 & 8 & 3 
    \end{bmatrix}
  \\&=\dots=
    -2\det\begin{bmatrix}
      \textcolor{red}{1} & \textcolor{red}{3} & \textcolor{red}{4} \\ 
      \textcolor{red}{0} & 21 & 43 \\ 
      \textcolor{red}{0} & -1 & -10 
    \end{bmatrix}
  =
    -2\det\begin{bmatrix} 21 & 43 \\ -1 & -10 \end{bmatrix}
  \\&= \dots=
    -2\det\begin{bmatrix}
      -167 & \textcolor{red}{21} \\
      \textcolor{red}{0} & \textcolor{red}{1}
    \end{bmatrix}
   = -2\det[-167]
  \\&=-2(-167)\det(I)=
    334
\end{align*} 
\end{observation}

\begin{activity}{10}
  Compute 
  \(
    \det\begin{bmatrix} 
      2 & 3 & 5 & 0 \\ 
      0 & 3 & 2 & 0 \\ 
      1 & 2 & 0 & 3 \\ 
      -1 & -1 & 2 & 2 
    \end{bmatrix}
  \) by using any combination of row/column operations.
\end{activity}

\begin{observation}
Another option is to take advantage of the fact that the determinant is linear 
in each row or column.  This approach is called
\term{Laplace expansion} or \term{cofactor expansion}. 

For example, since 
\textcolor{blue}{\(
  \begin{bmatrix} 1 & 2 & 4 \end{bmatrix} 
= 
  1\begin{bmatrix} 1 & 0 & 0 \end{bmatrix}
+
  2\begin{bmatrix} 0 & 1 &  0 \end{bmatrix}
+
  4\begin{bmatrix} 0  & 0 & 1 \end{bmatrix}
\)},

  \begin{align*}
\det \begin{bmatrix} 2 & 3 & 5  \\ -1 & 3 & 5 \\ \textcolor{blue}{1} & \textcolor{blue}{2} & \textcolor{blue}{4} \end{bmatrix} &=
\textcolor{blue}{1}\det \begin{bmatrix} 2 & 3 & 5  \\ -1 & 3 & 5 \\ \textcolor{blue}{1} & \textcolor{blue}{0} & \textcolor{blue}{0} \end{bmatrix} +
\textcolor{blue}{2}\det \begin{bmatrix} 2 & 3 & 5  \\ -1 & 3 & 5 \\ \textcolor{blue}{0} & \textcolor{blue}{1} & \textcolor{blue}{0} \end{bmatrix} +
\textcolor{blue}{4}\det \begin{bmatrix} 2 & 3 & 5  \\ -1 & 3 & 5 \\ \textcolor{blue}{0} & \textcolor{blue}{0} & \textcolor{blue}{1} \end{bmatrix} \\
&= -1\det \begin{bmatrix}  5 & 3 & 2 \\ 5 & 3 & -1 \\ 0 & 0 & 1 \end{bmatrix} 
-2\det \begin{bmatrix} 2 & 5 & 3  \\ -1 & 5 & 3 \\ 0 & 0 & 1 \end{bmatrix} +
4\det \begin{bmatrix} 2 & 3 & 5  \\ -1 & 3 & 5 \\ 0 & 0 & 1 \end{bmatrix} \\
&= -\det \begin{bmatrix} 5 & 3 \\ 5 & 3 \end{bmatrix} 
-2 \det \begin{bmatrix} 2 & 5 \\ -1 & 5 \end{bmatrix}
+4 \det \begin{bmatrix} 2 & 3 \\ -1 & 3 \end{bmatrix}
\end{align*}

\end{observation}

\begin{observation}
Applying Laplace expansion to a \(2 \times 2\) matrix yields a short formula you may have seen:
\[
  \det \begin{bmatrix} \textcolor{blue}{a} & \textcolor{blue}{b} \\ c & d \end{bmatrix}
=
  \textcolor{blue}{a}\det \begin{bmatrix} \textcolor{blue}{1} & \textcolor{blue}{0} \\ 
    c & d \end{bmatrix} 
+ 
  \textcolor{blue}{b} \det \begin{bmatrix} \textcolor{blue}{0} & \textcolor{blue}{1} \\ 
    c & d \end{bmatrix} 
=
  a\det \begin{bmatrix} \textcolor{red}{1} & \textcolor{red}{0} \\ 
    \textcolor{red}{c} & d \end{bmatrix} 
-
  b \det \begin{bmatrix} \textcolor{red}{1} & \textcolor{red}{0} \\ 
    \textcolor{red}{d} & c \end{bmatrix} 
= 
  ad-bc
.\]

\vspace{1em}

There are formulas for the determinants of larger matrices,
but they can be pretty tedious to use. For example, writing out a
formula for a \(4\times 4\) determinant would require 24 different terms!

\[
   \det\begin{bmatrix}
     a_{11} & a_{12} & a_{13} & a_{14} \\
     a_{21} & a_{22} & a_{23} & a_{24} \\
     a_{31} & a_{32} & a_{33} & a_{34} \\
     a_{41} & a_{42} & a_{43} & a_{44}
   \end{bmatrix}
     =
   a_{11}(a_{22}(a_{33}a_{44}-a_{43}a_{34})-a_{23}(a_{32}a_{44}-a_{42}a_{34})+\dots)+\dots
\]

So this is why we either use Laplace expansion or row/column operations directly.
\end{observation}

\begin{activity}{10}
  Use Laplace expansion to compute 
  \(
    \det\begin{bmatrix} 
      2 & 2 & 1 & 0 \\ 
      0 & 3 & 2 & -1 \\ 
      3 & 2 & 0 & 3 \\ 
      0 & -3 & 2 & -2 
    \end{bmatrix}
  \).
\end{activity}

\begin{activity}{5}
Based on what we've done today, which technique is easier for computing determinants?
\begin{enumerate}[(a)]
\item Memorizing formulas.
\item Using row/column operations.
\item Laplace expansion.
\item Some other technique (be prepared to describe it).
\end{enumerate}
\end{activity}

\begin{activity}{10}
  Use your preferred technique to compute 
  \(
    \det\begin{bmatrix} 
      4 & -3 & 0 & 0 \\ 
      1 & -3 & 2 & -1 \\ 
      3 & 2 & 0 & 3 \\ 
      0 & -3 & 2 & -2 
    \end{bmatrix}
  \).
\end{activity}

\end{applicationActivities}
