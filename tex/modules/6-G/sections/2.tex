\begin{applicationActivities}

\begin{fact}
Since the row operation matrices may be obtained with column operations,
we can also use row operations to simplify determinants.
\begin{enumerate}
\item Multiplying rows by scalars:
  \(\det\begin{bmatrix}\vdots\\cR\\\vdots\end{bmatrix}=
  c\det\begin{bmatrix}\vdots\\R\\\vdots\end{bmatrix}\)
\item Swapping two rows:
  \(\det\begin{bmatrix}\vdots\\R\\\vdots\\S\\\vdots\end{bmatrix}=
  -\det\begin{bmatrix}\vdots\\S\\\vdots\\R\\\vdots\end{bmatrix}\)
\item Adding multiples of rows to other rows:
  \(\det\begin{bmatrix}\vdots\\R\\\vdots\\S\\\vdots\end{bmatrix}=
  \det\begin{bmatrix}\vdots\\R+cS\\\vdots\\S\\\vdots\end{bmatrix}\)
\end{enumerate}
\end{fact}

\begin{definition}
  The \term{transpose} of a matrix is given by rewriting its columns as
  rows and vice versa:
  \[
    \begin{bmatrix}
      1 & 2 & 3 \\
      4 & 5 & 6
    \end{bmatrix}^T
  =
    \begin{bmatrix}
      1 & 4  \\
      2 & 5  \\
      3 & 6
    \end{bmatrix}
  \]
\end{definition}

\begin{fact}
Since row and column operations both affect determinants the same way,
the determinant of a matrix and its transpose are the same.
For example, let \(A=\begin{bmatrix}3 & 3 \\ 0 & 2\end{bmatrix}\),
so \(A^T=\begin{bmatrix}3 & 0 \\ 3 & 2\end{bmatrix}\); both matrices
scale the unit square by \(6\), even though the parallelograms are not
congruent.

\begin{center}
  \begin{tikzpicture}[scale=0.5]
  \draw[thin,gray,<->] (-1,0)-- (4,0);
  \draw[thin,gray,<->] (0,-1)-- (0,4);
  \draw[blue,dashed] (0,0) -- (3,0) -- (6,2) -- (3,2) -- (0,0);
  \draw[thick,blue,->] (0,0) -- node[below] {$A \vec{e}_1= \begin{bmatrix}3 \\ 0 \end{bmatrix}$}++ (3,0);
  \draw[thick,blue,->] (0,0) -- ++(3,2) node[above] {$A \vec{e}_2 = \begin{bmatrix} 3 \\ 2 \end{bmatrix}$};
  \end{tikzpicture}
  \begin{tikzpicture}[scale=0.5]
  \draw[thin,gray,<->] (-1,0)-- (4,0);
  \draw[thin,gray,<->] (0,-1)-- (0,4);
  \draw[blue,dashed] (0,0) -- (3,3) -- (3,5) -- (0,2) -- (0,0);
  \draw[thick,blue,->] (0,0) -- node[right] {$A^T \vec{e}_1= \begin{bmatrix}3 \\ 3 \end{bmatrix}$}++ (3,3);
  \draw[thick,blue,->] (0,0) -- ++(0,2) node[left] {$A^T \vec{e}_2 = \begin{bmatrix} 0 \\ 2 \end{bmatrix}$};
  \end{tikzpicture}
\end{center}
\end{fact}

\begin{activity}{10}
  Compute the determinant of $\begin{bmatrix} 2 & 4 \\ 2 & 3 \end{bmatrix}$ by row reducing it to a nicer matrix. \\

\ \\

  For example, $\det \begin{bmatrix} 2 & 4 \\ 2 & 3 \end{bmatrix} = 2 \det \begin{bmatrix} 1 & 2 \\ 2 & 3 \end{bmatrix}$.

\end{activity}

\begin{fact}
  This same process allows us to prove a more convenient formula:
    \begin{align*}
    \det\begin{bmatrix} a & b \\ c & d \end{bmatrix} = ad-bc
    \end{align*}

In higher dimensions, the formulas become unreasonable.  For example, the formula for $4\times4$ matrices has 24 terms!
\end{fact}


\begin{activity}{5}
  The following image illustrates the transformation of the unit cube
  by the matrix
  $\begin{bmatrix} 3 & 1 & 0 \\  1 & 1 & 0 \\  0 & 0 & 1\end{bmatrix}$.

  \begin{center}
  \begin{tikzpicture}
  \draw[thin,gray,->] (0,0,0) -- (3,0,0);
  \draw[thin,gray,->] (0,0,0) -- (0,2,0);
  \draw[thin,gray,->] (0,0,0) -- (0,0,2);
  %(y,z,x)
  \draw[blue] (0,1,0) -- (1,1,1);
  \draw[blue] (0,1,0) -- (3,1,1);
  \draw[blue] (3,0,1) -- (4,0,2);
  \draw[blue] (3,0,1) -- (3,1,1);
  \draw[blue] (1,0,1) -- (4,0,2);
  \draw[blue] (1,0,1) -- (1,1,1);
  \draw[blue] (1,1,1) -- (4,1,2);
  \draw[blue] (4,0,2) -- (4,1,2);
  \draw[blue] (3,1,1) -- (4,1,2);

  \draw[purple,thick,->] (0,0,0) -- (0,1,0)
    node[left]{\tiny$\begin{bmatrix} 0 \\ 0 \\ 1\end{bmatrix}$};
  \draw[purple,thick,->] (0,0,0) -- (1,0,1)
    node[below]{\tiny$\begin{bmatrix} 1 \\ 1 \\ 0\end{bmatrix}$};
  \draw[purple,thick,->] (0,0,0) -- (3,0,1)
    node[right]{\tiny$\begin{bmatrix} 3 \\ 1 \\ 0\end{bmatrix}$};
  \end{tikzpicture}
  \end{center}
This volume is equal to which of the following areas?
\begin{multicols}{4}
\begin{enumerate}[(a)]
\item $\det \begin{bmatrix} 3 & 1 \\ 1 & 1 \end{bmatrix}$
\item $\det \begin{bmatrix} 3 & 1 \\ 1 & 0 \end{bmatrix}$
\item $\det \begin{bmatrix} 3 & 1 \\ 0 & 1 \end{bmatrix}$
\item $\det \begin{bmatrix} 1 & 1 \\ 0 & 1 \end{bmatrix}$
\end{enumerate}
\end{multicols}
\end{activity}

\begin{fact}
If column $i$ of a matrix is $\vec{e}_i$, then both column and row $i$
may be removed without changing the value of the determinant.
For example, the second column of the following matrix is \(\vec e_2\), so:
\[
  \det \begin{bmatrix}
    3 & 0 & -1 & 5 \\
    2 & 1 & 4 & 0 \\
    -1 & 0 & 1 & 11 \\
    3 & 0 & 0 & 1
  \end{bmatrix} =
  \det \begin{bmatrix}
    3 & -1 & 5 \\
    -1 & 1 & 11 \\
    3 & 0 & 1
  \end{bmatrix}
\]
Therefore the same holds for the transpose:
\[
  \det \begin{bmatrix}
    3 & 2 & -1 & 3 \\
    0 & 1 & 0 & 0 \\
    -1 & 4 & 1 & 0 \\
    5 & 0 & 11 & 1
  \end{bmatrix} =
  \det \begin{bmatrix}
    3 & -1 & 3 \\
    -1 & 1 & 0 \\
    5 & 11 & 1
  \end{bmatrix}
\]

Geometrically, this is the fact that if the height is 1, the base $\times$ height formula reduces to the area/volume/etc. of the $n-1$ dimensional base.
\end{fact}

% \begin{activity}{5}
%   The following image illustrates the transformation of the unit cube
%   by the matrix
%   $\begin{bmatrix} 3 & 1 & 0 \\  1 & 1 & 1 \\  0 & 0 & 1\end{bmatrix}$.
%
%
%   \begin{center}
%   \begin{tikzpicture}
%   \draw[thin,gray,->] (0,0,0) -- (4,0,0);
%   \draw[thin,gray,->] (0,0,0) -- (0,2,0);
%   \draw[thin,gray,->] (0,0,0) -- (0,0,2);
%   %(y,z,x)
%
%   \draw[blue] (0,0,0) -- (1,1,0);
%   \draw[blue] (0,0,0) -- (1,0,1);
%   \draw[blue] (0,0,0) -- (3,0,1);
%   \draw[blue] (1,1,0) -- (2,1,1);
%   \draw[blue] (1,1,0) -- (4,1,1);
%   \draw[blue] (3,0,1) -- (4,0,2);
%   \draw[blue] (3,0,1) -- (4,1,1);
%   \draw[blue] (1,0,1) -- (4,0,2);
%   \draw[blue] (1,0,1) -- (2,1,1);
%   \draw[blue] (2,1,1) -- (5,1,2);
%   \draw[blue] (4,0,2) -- (5,1,2);
%   \draw[blue] (4,1,1) -- (5,1,2);
%   \end{tikzpicture}
%   \end{center}
%
%   This volume is equal to which of the following?
%
% \begin{multicols}{2}
% \begin{enumerate}[(a)]
% \item $\det \begin{bmatrix} 3 & 1 & 0 \\  1 & 1 & 0 \\  0 & 0 & 1\end{bmatrix}
%       = \det \begin{bmatrix} 3 & 1 \\ 1 & 1 \end{bmatrix}$
% \item $\det \begin{bmatrix} 3 & 1 & 0 \\  1 & 0 & 0 \\  0 & 0 & 1\end{bmatrix}
%       = \det \begin{bmatrix} 3 & 1 \\ 1 & 0 \end{bmatrix}$
% \item $\det \begin{bmatrix} 3 & 1 & 0 \\  0 & 1 & 0 \\  0 & 0 & 1\end{bmatrix}
%       = \det \begin{bmatrix} 3 & 1 \\ 0 & 1 \end{bmatrix}$
% \item $\det \begin{bmatrix} 1 & 1 & 0 \\  0 & 1 & 0 \\  0 & 0 & 1\end{bmatrix}
%       = \det \begin{bmatrix} 1 & 1 \\ 0 & 1 \end{bmatrix}$
% \end{enumerate}
%
% \end{multicols}
% \end{activity}
%
% \begin{activity}{5}
%   Which of the following is the same as
%   $\det \begin{bmatrix}
%     2 & 0 & 1 \\
%     1 & 3 & 1 \\
%     0 & 0 & 2
%   \end{bmatrix}$?
%
% \begin{multicols}{2}
% \begin{enumerate}[(a)]
% \item $3\begin{bmatrix} 2 & 1 \\ 0 & 2 \end{bmatrix}$
% \item $3\begin{bmatrix} 2 & 0 \\ 1 & 3 \end{bmatrix}$
% \item $3\begin{bmatrix} 2 & 1 \\ 1 & 1 \end{bmatrix}$
% \item $3\begin{bmatrix} 3 & 1 \\ 0 & 2 \end{bmatrix}$
% \end{enumerate}
%
%
% \begin{center}
% \begin{tikzpicture}
% \draw[thin,gray,->] (0,0,0) -- (3,0,0);
% \draw[thin,gray,->] (0,0,0) -- (0,3,0);
% \draw[thin,gray,->] (0,0,0) -- (0,0,2);
% %(y,z,x)
%
% \draw[blue] (0,0,0) -- (1,0,2);
% \draw[blue] (0,0,0) -- (1,2,1);
% \draw[blue] (0,0,0) -- (3,0,0);
% \draw[blue] (1,0,2) -- (2,2,3);
% \draw[blue] (1,0,2) -- (4,0,2);
% \draw[blue] (3,0,0) -- (4,0,2);
% \draw[blue] (3,0,0) -- (4,2,1);
% \draw[blue] (1,2,1) -- (2,2,3);
% \draw[blue] (1,2,1) -- (4,2,1);
% \draw[blue] (2,2,3) -- (5,2,3);
% \draw[blue] (4,0,2) -- (5,2,3);
% \draw[blue] (4,2,1) -- (5,2,3);
% \end{tikzpicture}
% \end{center}
% \end{multicols}
%
% \end{activity}


% \begin{activity}{5}
%   Compute $\det \begin{bmatrix} 0 & 3 & -2 \\ 1 & 5 & 12 \\ 0 & 2 & -1 \end{bmatrix}$.
%
%   {\em Hint: Swap rows or columns to reduce to an easier problem}.
% \end{activity}
%
% \begin{activity}{10}
%    Using the fact that $\begin{bmatrix} 2 \\ 1 \\ 0 \end{bmatrix} = \begin{bmatrix} 2 \\ 0 \\ 0 \end{bmatrix} + \begin{bmatrix} 0 \\ 1 \\ 0 \end{bmatrix}$, compute $\det \begin{bmatrix} 2 & 2 & 3 \\ 1 & -2 & -5 \\ 0 & 3 & 3 \end{bmatrix}$.
% \end{activity}
%
% \begin{activity}{10}
%    Compute $\det \begin{bmatrix} 2 & 3 & 5  \\ 1 & 1 & 0  \\ -1 & 2 & -1 \end{bmatrix}$.
% \end{activity}
%
% \begin{activity}{10}
%    Compute $\det \begin{bmatrix} 2 & 3 & 5 & 0 \\ 0 & 1 & -1 & 0 \\ 1 & 2 & 0 & 3 \\ -1 & -1 & 2 & 2 \end{bmatrix}$.
% \end{activity}


\begin{activity}{5}
  Compute  $\det \begin{bmatrix} 1 & 0 & 0 \\ 1 & 5 & 12 \\ 3 & 2 & -1 \end{bmatrix}$.
\end{activity}

\begin{activity}{5}
  Compute   $\det \begin{bmatrix} 0 & 3 & -2 \\ 1 & 5 & 12 \\ 0 & 2 & -1 \end{bmatrix}$.
  \ \\
  \begin{multicols}{3}
  \begin{enumerate}[(a)]
  \item $-1$
  \item $0$
  \item $1$
  \end{enumerate}
  \end{multicols}
\end{activity}

\begin{activity}{10}
  Compute $\det \begin{bmatrix} 1 & 2 & 3 \\ 1 & -2 & -5 \\ 0 & 3 & 3 \end{bmatrix}$

\ \\

  {\em Hint:} $\begin{bmatrix} 1 \\ 1 \\ 0 \end{bmatrix} = \begin{bmatrix} 1 \\ 0 \\ 0 \end{bmatrix} + \begin{bmatrix} 0 \\ 1 \\ 0 \end{bmatrix}$.

  \begin{multicols}{4}
  \begin{enumerate}[(a)]
  \item $3$
  \item $6$
  \item $9$
  \item $12$
  \end{enumerate}
  \end{multicols}
  \end{activity}

\begin{activity}{15}
  Compute  $\det \begin{bmatrix} 2 & 3 & 5 & 0 \\ 0 & 3 & 2 & 0 \\ 1 & 2 & 0 & 3 \\ -1 & -1 & 2 & 2 \end{bmatrix}$.

\end{activity}

\begin{observation}
\newcommand{\tikzmark}[2]{\tikz[overlay,remember picture,baseline] \node [anchor=base] (#1) {$#2$};}

\newcommand{\DrawVLine}[3][]{%
  \begin{tikzpicture}[overlay,remember picture]
    \draw[#1] (#2.north) -- (#3.south);
  \end{tikzpicture}
}
\newcommand{\DrawHLine}[3][]{%
  \begin{tikzpicture}[overlay,remember picture]
    \draw[#1] (#2.west) -- (#3.east);
  \end{tikzpicture}
}

  \begin{align*}
\det \begin{bmatrix} 2 & 3 & 5 & 0 \\ 0 & 3 & 2 & 0 \\ 1 & 2 & 0 & 3 \\ -1 & -1 & 2 & 2 \end{bmatrix} &=
(-1) (0) \det \begin{bmatrix} \tikzmark{topA}{2} & 3 & 5 & 0 \\ \tikzmark{leftA}{0} & 3 & 2 & \tikzmark{rightA}{0} \\ 1 & 2 & 0 & 3 \\ \tikzmark{bottomA}{-1} & -1 & 2 & 2 \end{bmatrix}
+(1)(3) \det \begin{bmatrix} 2 & \tikzmark{topB}{3} & 5 & 0 \\ \tikzmark{leftB}{0} & 3 & 2 & \tikzmark{rightB}{0} \\ 1 & 2 & 0 & 3 \\ -1 & \tikzmark{bottomB}{-1} & 2 & 2 \end{bmatrix} +  \\
&\phantom{xx} (-1)(2) \det \begin{bmatrix} 2 & 3 & \tikzmark{topC}{5} & 0 \\ \tikzmark{leftC}{0} & 3 & 2 & \tikzmark{rightC}{0} \\ 1 & 2 & 0 & 3 \\ -1 & -1 & \tikzmark{bottomC}{2} & 2 \end{bmatrix}
+ (1)(0) \det \begin{bmatrix} 2 & 3 & 5 & \tikzmark{topD}{0} \\ \tikzmark{leftD}{0} & 3 & 2 & \tikzmark{rightD}{0} \\ 1 & 2 & 0 & 3 \\ -1 & -1 & 2 & \tikzmark{bottomD}{2} \end{bmatrix}  \\
&= 3 \det \begin{bmatrix}2 & 5 & 0 \\ 1 & 0 & 3 \\ -1 & 2 & 2 \end{bmatrix} +(-1)(2) \det \begin{bmatrix} 2 & 3 & 0 \\ 1 & 2 & 3 \\ -1 & -1 & 2 \end{bmatrix}
\end{align*}
\DrawHLine[red,thick]{leftA}{rightA}
\DrawVLine[red,thick]{topA}{bottomA}
\DrawHLine[red,thick]{leftB}{rightB}
\DrawVLine[red,thick]{topB}{bottomB}
\DrawHLine[red,thick]{leftC}{rightC}
\DrawVLine[red,thick]{topC}{bottomC}
\DrawHLine[red,thick]{leftD}{rightD}
\DrawVLine[red,thick]{topD}{bottomD}

This technique is called \term{Laplace expansion} or \term{cofactor expansion}.
  \end{observation}

\begin{activity}{10}
  Compute  $\det \begin{bmatrix} 1 & 2 & 1 & 0 \\ 0 & 3 & 2 & -1 \\ 1 & 2 & 0 & 3 \\ -1 & -3 & 2 & -2 \end{bmatrix}$.

\end{activity}


\end{applicationActivities}
