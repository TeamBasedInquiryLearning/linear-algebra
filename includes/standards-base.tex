
\usepackage[left=1in,right=1in,top=1in,bottom=1in]{geometry}
\usepackage{enumerate,amssymb}

\begin{document}
\pagestyle{empty}
\noindent Math 237 \hfill \sem \hfill \prof
\vspace{0.3in}
\hrule
\begin{center}{\large \bf Linear Algebra Standards}\end{center}
\hrule

\begin{multicols}{2}
\noindent\textbf{How can we solve systems of linear equations?}
\begin{enumerate}[{$\Box\ \Box$ \bf {E}1.}]
\item {\bf Systems as matrices}. I can translate back and forth between a system of linear equations and the corresponding augmented matrix.
\item {\bf Row reduction}.  I can put a matrix in reduced row echelon form.
\item {\bf Systems of linear equations}. I can compute the solution set for a system of linear equations.
\end{enumerate}

\noindent\textbf{What is a vector space?}
\begin{enumerate}[{$\Box\ \Box$ \bf {V}1.}]
\item {\bf Vector property verification}. I can show why an example satisfies a given vector space property, but does not satisfy another given property.
\item {\bf Vector space identification}. I can list all eight properties of a vector space, infer which of these properties a given example satisfies, and thus determine if the example is a vector space.
\item {\bf Linear combinations}. I can determine if a Euclidean vector can be written as a linear combination of a given set of Euclidean vectors.
\item {\bf Spanning sets}. I can determine if a set of Euclidean vectors spans \(\IR^n\).
\item {\bf Subspaces}. I can determine if a subset of \(\IR^n\) is a subspace or not.

\end{enumerate}

\noindent\textbf{What structure do vector spaces have?}
\begin{enumerate}[{$\Box\ \Box$ \bf {S}1.}]

\item {\bf Linear independence}. I can determine if a set of Euclidean vectors is linearly dependent or independent.
\item {\bf Basis verification}. I can determine if a set of Euclidean vectors is a basis of \(\IR^n\).
\item {\bf Basis computation}.  I can compute a basis for the subspace spanned by a given set of Euclidean vectors.
\item {\bf Dimension}.  I can compute the dimension of a subspace of \(\IR^n\).
\item {\bf Abstract vector spaces}. I can solve exercises related to standards V3-S4 when posed in terms of polynomials or matrices.
\item {\bf Basis of solution space}. I can find a basis for the solution set of a homogeneous system of equations.
\end{enumerate}

\noindent\textbf{How can we understand linear maps algebraically?}
\begin{enumerate}[{$\Box\ \Box$ \bf {A}1.}]
\item {\bf Linear maps and matrices}. I can translate back and forth between a
linear transformation of Euclidean spaces and its standard matrix, and perform related computations.
\item {\bf Linear map verification}. I can determine if a map between vector spaces of polynomials is linear or not.
\item {\bf Injectivity and surjectivity}.  I can determine if a given linear map is injective and/or surjective.
\item {\bf Kernel and Image}. I can compute a basis for the kernel and a basis for the image of a linear map.
\end{enumerate}


\noindent\textbf{What algebraic structure do matrices have?}
\begin{enumerate}[{$\Box\ \Box$ \bf {M}1.}]
\item {\bf Matrix Multiplication}. I can multiply matrices.
\item {\bf Invertible Matrices}. I can determine if a square matrix is invertible or not.
\item {\bf Matrix inverses}.  I can compute the inverse matrix of an invertible matrix.
\end{enumerate}


\noindent\textbf{How can we understand linear maps geometrically?}
\begin{enumerate}[{$\Box\ \Box$ \bf {G}1.}]
\item {\bf Row operations}.  I can represent a row operation as matrix multiplication, and compute how the operation affects the determinant.
\item {\bf Determinants}. I can compute the determinant of a square matrix.
\item {\bf Eigenvalues}. I can find the eigenvalues of a $2\times 2$ matrix.
\item {\bf Eigenvectors}. I can find a basis for the eigenspace of a square matrix associated with a given eigenvalue.
\end{enumerate}

\end{multicols}


\end{document}
