%!TEX root =../../course-notes.tex
% ^ leave for LaTeXTools build functionality

\begin{applicationActivities}{Day 2}

\begin{activity}{0}
Each row operation can be interpreted as a matrix multiplication.  Let $A \in M_{4,4}$
\begin{enumerate}[1)]
\item Find a matrix $S_1$ such that $S_1A$ is the result of swapping the second and fourth rows of $A$.
\item Find a matrix $S_2$ such that $S_2A$ is the result of adding $5$ times the third row of $A$ to the first.
\item Find a matrix $S_3$ such that $S_3A$ is the result of doubling the fourth row of $A$.
\end{enumerate}
\end{activity}


\begin{activity}{0}
Let $T: \IR^n \rightarrow \IR^m$ be a linear map with matrix $A \in M_{m,n}$ (for the standard basis).  Consider the following statements about $T$
\begin{enumerate}[(a)]
\item $T$ is injective
\item $T$ is surjective
\item $T$ is bijective (i.e. both injective and surjective)
\item $AX=B$ has a solution for all $B \in M_{m,1}$
\item $AX=B$ has a unique solution for all $B \in M_{m,1}$
\item $AX=0$ has a non-trivial solution.
\item The columns of $A$ span $\IR^m$
\item The columns of $A$ are linearly independent
\item The columns of $A$ are a basis of $\IR^m$
\item $\RREF(A)$ has $n$ pivot columns
\item $\RREF(A)$ has $m$ pivot columns
\end{enumerate}

Sort these statements into groups of equivalent statements.

\end{activity}

\begin{activity}{0}
Let $T: \IR^n \rightarrow \IR^m$ be a linear map with matrix $A \in M_{m,n}$ (for the standard basis).

If $T$ is injective, what must be true about how $m$ and $n$ are related?
\begin{enumerate}[(a)]
\item $m<n$
\item $m \leq n$
\item $m=n$
\item $m \geq n$
\item $m>n$
\end{enumerate}
\end{activity}

\begin{activity}{0}
If $T$ is surjective, what must be true about how $m$ and $n$ are related?
\begin{enumerate}[(a)]
\item $m<n$
\item $m \leq n$
\item $m=n$
\item $m \geq n$
\item $m>n$
\end{enumerate}
\end{activity}

\begin{activity}{0}
  If $T$ is bijective, what must be true about how $m$ and $n$ are related?
\begin{enumerate}[(a)]
\item $m<n$
\item $m \leq n$
\item $m=n$
\item $m \geq n$
\item $m>n$
\end{enumerate}
\end{activity}

\end{applicationActivities}
