%!TEX root =../../course-notes.tex
% ^ leave for LaTeXTools build functionality

\begin{applicationActivities}{Day 2}

\begin{definition}
  The following \term{row operations} produce equivalent
  augmented matrices:
  \begin{enumerate}
    \item Swap two rows.
    \item Multiply a row by a conzero constant.
    \item Add a constant multiple of one row to another row.
  \end{enumerate}
  Whenever two matrices \(A,B\) are equivalent (so whenever we do any of
  these operations), we write \(A\sim B\).
\end{definition}

\begin{activity}{15} % TODO add subactivity for finding solution
  Show that the following two linear systems:
  \begin{multicols}{2}\noindent
    \begin{alignat*}{4}
      3x_1 &\,-\,& 2x_2 &\,+\,& 13x_3 &\,=\,& 6 \\
      2x_1 &\,-\,& 2x_2 &\,+\,& 10x_3 &\,=\,& 2 \\
     -1x_1 &\,+\,& 3x_2 &\,-\,&  6x_3 &\,=\,& 11
    \end{alignat*}
    \begin{alignat*}{4}
       x_1 &\,-\,&  x_2  &\,+\,&  5x_3 &\,=\,& 1 \\
           &\, \,&  x_2 &\,-\,&  2x_3 &\,=\,& 3 \\
           &\, \,&      &\, \,&   x_3 &\,=\,& 2
    \end{alignat*}
  \end{multicols}
  are equivalent by converting the first system to an augmented matrix,
  and then performing the following row operations to obtain
  an augmented matrix equivalent to the second system.
  \begin{multicols}{2}\noindent
  \begin{enumerate}
    \item Swap \(R_1\) (first row) and \(R_2\) (second row).
    \item Multiply \(R_2\) by \(\frac{1}{2}\).
    \item Add \(R_1\) to \(R_3\).
    \item Add \(-3R_1\) to \(R_2\).
    \item Add \(-2R_2\) to \(R_3\).
    \item Multiply \(R_3\) by \(\frac{1}{3}\).
  \end{enumerate}
  \end{multicols}
\end{activity}

\begin{definition}
  The \term{leading term} of a matrix row is its first nonzero term.
  A matrix is in \term{row echelon form} if all leading terms are \(1\),
  the leading term of every row
  is farther right than every leading term on a higher row, and all zero
  rows are at the bottom of the matrix.
\end{definition}

\begin{activity}{10}
  % TODO rewrite column-by-column
  Reproduce the steps that manipulated the matrix
  \[
    \begin{bmatrix}[ccc|c]
      3 & -2 & 13 & 6 \\
      2 & -2 & 10 & 2 \\
      -1 & 3 & -6 & 11
    \end{bmatrix}\sim
    \begin{bmatrix}[ccc|c]
      1 & -1 &  5 & 1 \\
      0 &  1 & -2 & 3 \\
      0 &  0 &  1 & 2
    \end{bmatrix}
  \]
  into row echelon form by using the following algorithm.
  % TODO rewrite algorithm
  \begin{multicols}{2}\noindent
  \begin{enumerate}
    \item Identify
    the top cell of the first non-zero column as your \textbf{pivot position};
    you will ignore anything in the matrix that is above or left of your
    current pivot position.
    \item If the pivot position contains a \(0\), swap its row with a lower
          row that does not contain a \(0\) in its column.
    \item Divide the pivot row by the term in pivot position to change the
          pivot term to \(1\). (If convenient, you can first swap the
          pivot row with a lower row to make this division easier.)
    \item Add multiples of the pivot row to all lower rows so that all terms
          below pivot position become \(0\).
    \item Move your pivot position down and right one step.
    \item If all terms in and below pivot position are zero, move your
          pivot position right. Repeat this step as needed.
    \item If the matrix is not yet in row echelon form, return to Step 2.
  \end{enumerate}
  \end{multicols}
\end{activity}

\begin{definition}%TODO move after next activity
  A matrix is in \term{reduced row echelon form} if it is in row echelon form
  and all terms above leading terms are \(0\). %TODO change leading terms
  % to pivot positions
\end{definition}

\begin{activity}{10}
  % TODO remove explicit steps
  Show that the following two linear systems:
  \begin{multicols}{2}\noindent
    \begin{alignat*}{4}
       x_1 &\,-\,&  x_2  &\,+\,&  5x_3 &\,=\,& 1 \\
           &\, \,&  x_2 &\,-\,&  2x_3 &\,=\,& 3 \\
           &\, \,&      &\, \,&   x_3 &\,=\,& 2
    \end{alignat*}
      \begin{alignat*}{4}
         x_1 &\, \,&      &\, \,&       &\,=\,& -2 \\
             &\, \,&  x_2 &\, \,&       &\,=\,& 7 \\
             &\, \,&      &\, \,&   x_3 &\,=\,& 2
      \end{alignat*}
  \end{multicols}
  are equivalent by converting the first system to an augmented matrix,
  and then performing the following row operations to obtain
  an augmented matrix equivalent to the second system.
  \begin{enumerate}
    \item Add \(2R_3\) to \(R_2\).
    \item Add \(-5R_3\) to \(R_1\).
    \item Add \(R_2\) to \(R_1\).
  \end{enumerate}
  Then write the solution to the linear system.
\end{activity}

\begin{remark}
  We may verify that \((x_1,x_2,x_3)=(-2,7,2)\) is a solution to the
  original linear system
    \begin{alignat*}{4}
      3x_1 &\,-\,& 2x_2 &\,+\,& 13x_3 &\,=\,& 6 \\
      2x_1 &\,-\,& 2x_2 &\,+\,& 10x_3 &\,=\,& 2 \\
     -1x_1 &\,+\,& 3x_2 &\,-\,&  6x_3 &\,=\,& 11
    \end{alignat*}
  by plugging the solution into each equation.
\end{remark}

%TODO 3x3 example that is consistent with inf solutions

% \begin{activity}{10}
%   Reproduce the steps that manipulated the matrix
%   \[
%     \begin{bmatrix}[ccc|c]
%       1 & -1 &  5 & 1 \\
%       0 &  1 & -2 & 3 \\
%       0 &  0 &  1 & 2
%     \end{bmatrix}\sim
%     \begin{bmatrix}[ccc|c]
%       1 &  0 &  0 & -2 \\
%       0 &  1 &  0 & 7 \\
%       0 &  0 &  1 & 2
%     \end{bmatrix}
%   \]
%   into reduced row echelon form by adding multiples of lower rows into
%   higher rows, zeroing out all terms above leading terms.
% \end{activity}




\end{applicationActivities}
