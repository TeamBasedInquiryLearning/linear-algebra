%!TEX root =../../course-notes.tex
% ^ leave for LaTeXTools build functionality

\begin{applicationActivities}{Day 3}

\begin{fact}
  Every augmented matrix \(A\) reduces to a unique reduced row echelon form
  matrix. This matrix is denoted as \(\RREF(A)\).
\end{fact}

\begin{definition}
  The following algorithm that reduces \(A\) to \(\RREF(A)\) is known as
  \term{Gauss-Jordan elimination}.

  \begin{multicols}{2}\noindent
  \begin{enumerate}
    \item Identify
    the top cell of the first non-zero column as your pivot position;
    you will ignore anything in the matrix that is above or left of your
    current pivot position.
    \item If the pivot position contains a \(0\), swap its row with a lower
          row that does not contain a \(0\) in its column.
    \item Divide the pivot row by the term in pivot position to change the
          pivot term to \(1\). (If convenient, you can first swap the
          pivot row with a lower row to make this division easier.)
    \item Add multiples of the pivot row to all lower rows so that all terms
          below pivot position become \(0\).
    \item Move your pivot position down and right one step.
    \item If all terms in and below pivot position are zero, move your
          pivot position right. Repeat this step as needed.
    \item If the matrix is not yet in row echelon form, return to Step 2.
    \item Finally, add multiples of lower rows into higher rows, zeroing out
          all terms above leading terms.
  \end{enumerate}
  \end{multicols}
\end{definition}

\begin{activity}{15}
  Find \(\RREF(A)\) where
  \[A=
    \begin{bmatrix}[cccc|c]
      -1 &  1 & -3 &  2 &  0 \\
       2 & -1 &  5 &  3 & -11 \\
       3 &  2 &  4 &  1 &  1 \\
       0 &  1 & -1 &  1 &  1 \\
    \end{bmatrix}
  .\]
\end{activity}

\begin{definition}
  The columns of \(\RREF(A)\) without a leading term represent
  \term{free variables} of the linear system modeled by \(A\)
  that may be set equal to arbitrary parameters.
  The other \term{bounded variables} can then be expressed in terms
  of those parameters to describe the solution set
  to the linear system modeled by \(A\).
\end{definition}

\begin{activity}{10}
  Given the linear system and its equivalent augmented matrices
  \begin{multicols}{2}\noindent
    \begin{alignat*}{5}
      -x_1 &\,+\,&  x_2 &\,-\,&  3x_3 &\,+\,&  2x_4 &\,=\,& 0 \\
      2x_1 &\,-\,&  x_2 &\,+\,&  5x_3 &\,+\,&  3x_4 &\,=\,& -11 \\
      3x_1 &\,+\,& 2x_2 &\,+\,&  4x_3 &\,+\,&   x_4 &\,=\,& 1 \\
           &\, \,&  x_2 &\,-\,&   x_3 &\,+\,&   x_4 &\,=\,& 1 \\
    \end{alignat*}
  \[
    \begin{bmatrix}[cccc|c]
      -1 &  1 & -3 &  2 &  0 \\
       2 & -1 &  5 &  3 & -11 \\
       3 &  2 &  4 &  1 &  1 \\
       0 &  1 & -1 &  1 &  1 \\
    \end{bmatrix}\sim
    \begin{bmatrix}[cccc|c]
       1 &  0 &  2 &  0 & -1 \\
       0 &  1 & -1 &  0 &  3 \\
       0 &  0 &  0 &  1 & -2 \\
       0 &  0 &  0 &  0 &  0 \\
    \end{bmatrix}
  \]
  \end{multicols}
  describe the solution set
  \(
    \begin{bmatrix}
      x_1 \\
      x_2 \\
      x_3 \\
      x_4
    \end{bmatrix}=
    a\begin{bmatrix}
      r_1 \\
      r_2 \\
      r_3 \\
      r_4
    \end{bmatrix}+
    \begin{bmatrix}
      s_1 \\
      s_2 \\
      s_3 \\
      s_4
    \end{bmatrix}
  \) to the linear system by setting the free variable
  \(x_3=a\), and then expressing each of the
  bounded variables \(x_1,x_2,x_4\) equal to an expression in terms
  of \(a\).
\end{activity}

\begin{remark}
  It's not necessary to completely find \(\RREF(A)\) to
  deduce that a linear system is inconsistent.
\end{remark}

\begin{activity}{10}
  Find the contradiction in the inconsistent linear system
    \begin{alignat*}{4}
      2x_1 &\,-\,& 3x_2 &\,=\,& 17 \\
       x_1 &\,+\,& 2x_2 &\,=\,& -2 \\
      -x_1 &\,-\,&  x_2 &\,=\,& 1
    \end{alignat*}
  by considering the following equivalent augmented matrices:
  \[
    \begin{bmatrix}[cc|c]
       2 & -3 & 17 \\
       1 &  2 & -2 \\
      -1 & -1 &  1 \\
    \end{bmatrix}\sim
    \begin{bmatrix}[cc|c]
       1 &  2 & -2 \\
       0 &  1 &  3 \\
       0 &  0 &  2 \\
    \end{bmatrix}
  .\]
\end{activity}

\begin{definition}
  A \term{homogeneous system} is a linear system satisfying \(b_i=0\), that is,
  it is a linear system of the form
  \begin{alignat*}{5}
    a_{11}x_1 &\,+\,& a_{12}x_2 &\,+\,& \dots  &\,+\,& a_{1n}x_n &\,=\,& 0 \\
    a_{21}x_1 &\,+\,& a_{22}x_2 &\,+\,& \dots  &\,+\,& a_{2n}x_n &\,=\,& 0 \\
     \vdots&  &\vdots&   &&  &\vdots&&\vdots  \\
    a_{m1}x_1 &\,+\,& a_{m2}x_2 &\,+\,& \dots  &\,+\,& a_{mn}x_n &\,=\,& 0
  \end{alignat*}
\end{definition}

\begin{activity}{5}
  Prove that all homogeneous systems are consistent.
  (Hint: Find an obvious solution
  \(
    \begin{bmatrix}
      x_1 \\
      x_2 \\
      \vdots \\
      x_n
    \end{bmatrix}=
    \begin{bmatrix}
      s_1 \\
      s_2 \\
      \vdots \\
      s_n
    \end{bmatrix}
  \) that satisfies every homogeneous
  system.)
\end{activity}

\begin{fact}
  The solution set to any homogeneous system with infinitely-many solutions
  is generated by multiplying the parameters representing the free variables
  by Euclidean vectors, and adding these up. For example:
  \[
    \begin{bmatrix}
      x_1 \\
      x_2 \\
      x_3 \\
      x_4
    \end{bmatrix}=
    a\begin{bmatrix}
      3 \\
      1 \\
      -1 \\
      0
    \end{bmatrix}+
    b\begin{bmatrix}
      0 \\
      0 \\
      0 \\
      1
    \end{bmatrix}
  \]
\end{fact}

\begin{definition}
  A set of Euclidean vectors generating the solution set to a homogeneous
  system is called a \textbf{basis} for the solution
  set of the homogeneous system. For example:
  \begin{multicols}{2}\noindent
  \[
    \begin{bmatrix}
      x_1 \\
      x_2 \\
      x_3 \\
      x_4
    \end{bmatrix}=
    a\begin{bmatrix}
      3 \\
      1 \\
      -1 \\
      0
    \end{bmatrix}+
    b\begin{bmatrix}
      0 \\
      0 \\
      0 \\
      1
    \end{bmatrix}
  \]
  \[
    \textrm{Basis}=\left\{
    \begin{bmatrix}
      3 \\
      1 \\
      -1 \\
      0
    \end{bmatrix},
    \begin{bmatrix}
      0 \\
      0 \\
      0 \\
      1
    \end{bmatrix}\right\}
  \]
  \end{multicols}
\end{definition}

\begin{activity}{10}
  Find a basis for the solution set of the following homogeneous linear
  system.
  \begin{alignat*}{5}
    x_1 &\,+\,& 2x_2 &\, \,&     &\,-\,&  x_4 &\,=\,& 0 \\
        &\, \,&      &\, \,& x_3 &\,+\,& 4x_4 &\,=\,& 0 \\
   2x_1 &\,+\,& 4x_2 &\,+\,& x_3 &\,+\,& 2x_4 &\,=\,& 0 \\
  \end{alignat*}
\end{activity}




\end{applicationActivities}
