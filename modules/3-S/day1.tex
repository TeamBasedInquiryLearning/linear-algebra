%!TEX root =../../course-notes.tex
% ^ leave for LaTeXTools build functionality

\begin{applicationActivities}{Day 1}

\begin{activity}{15}
  In the previous module, we considered
  \[S=\left\{
  \begin{bmatrix}2\\3\\0\\-1\end{bmatrix},
  \begin{bmatrix}2\\0\\0\\3\end{bmatrix},
  \begin{bmatrix}3\\13\\7\\16\end{bmatrix},
  \begin{bmatrix}-1\\10\\7\\14\end{bmatrix},
  \begin{bmatrix}4\\3\\0\\2\end{bmatrix}
  \right\}
  \]
  and showed that \(\vspan S\not=\IR^4\). Find two vectors that
  are in the span of the other three vectors.

  \begin{TBLnote}
    Actually, the activity involved the corresponding vectors in \(\P^3\).
  \end{TBLnote}
\end{activity}

\begin{definition}
  We say that a set of vectors is \term{linearly dependent} if one vector
  in the set belongs to the span of the others. Otherwise, we say the set
  is \term{linearly independent}.
\end{definition}

\begin{activity}{10}
  Suppose \(x_1\vect v_1+x_2\vect v_2=\vect v_3\), so the set
  \(\{\vect v_1,\vect v_2,\vect v_3\}\) is linearly dependent.
  Is the vector equation \(x_1\vect v_1+x_2\vect v_2+x_3\vect v_3=\vect 0\)
  consistent with one solution, consistent with infinitely many solutions,
  or inconsistent?
\end{activity}

\begin{fact}
  The set \(\{\vect v_1,\dots\vect v_n\}\) is linearly dependent if and only
  if \(x_1\vect v_1+\dots+x_n\vect v_n=\vect 0\) is consistent with
  infinitely many solutions.
\end{fact}

\begin{activity}{10}
  Find
  \[\RREF\begin{bmatrix}[ccccc|c]
  2&2&3&-1&4&0\\
  3&0&13&10&3&0\\
  0&0&7&7&0&0\\
  -1&3&16&14&2&0
  \end{bmatrix}
  \]
  and circle the part of the matrix that demonstrates that
  \[S=\left\{
  \begin{bmatrix}2\\3\\0\\-1\end{bmatrix},
  \begin{bmatrix}2\\0\\0\\3\end{bmatrix},
  \begin{bmatrix}3\\13\\7\\16\end{bmatrix},
  \begin{bmatrix}-1\\10\\7\\14\end{bmatrix},
  \begin{bmatrix}4\\3\\0\\2\end{bmatrix}
  \right\}
  \]
  is linearly dependent.
\end{activity}

\begin{fact}
  A set of Euclidean vectors
  \(\{\vect v_1,\dots\vect v_n\}\) is linearly dependent if and only
  if \(\RREF\begin{bmatrix}\vect v_1&\dots&\vect v_n\end{bmatrix}\)
  has a column without a pivot position.
\end{fact}

\begin{activity}{15} TODO
  (compute RREF and label each set of vectors as linearly independent/dependent)
\end{activity}

\end{applicationActivities}
