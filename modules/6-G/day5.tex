%!TEX root =../../course-notes.tex
% ^ leave for LaTeXTools build functionality

\begin{applicationActivities}{5}{29}

\begin{activity}{10}

\textbf{A \$700,000,000,000 Problem:}

In the picture below, each circle represents a webpage, and each arrow
represents a link from one page to another.

\begin{center}
\begin{tikzpicture}
\begin{scope}[every node/.style={circle,thick,draw}]
    \node (1) at (0,4) {1};
    \node (2) at (0,0) {2};
    \node (3) at (4,0) {3};
    \node (4) at (4,4) {4};
    \node (5) at (6,4) {5};
    \node (6) at (10,4) {6} ;
    \node (7) at (6,0) {7} ;
\end{scope}

\begin{scope}[>={Stealth[red]}, every edge/.style={draw=red,very thick}]
    \draw [->] (1) edge[bend right=30] (2);
    \draw [->] (1) edge (4);
	\path [->] (2) edge[bend right=30] (1);
    \draw [->] (2) edge (3);
    \draw [->] (3) edge (4);
    \draw [->] (3) edge (7);
    \draw [->] (4) edge (2);
    \draw [->] (5) edge[bend right=30] (6);
    \draw [->] (5) edge (7);
    \draw [->] (6) edge[bend right=30] (5);
    \draw [->] (6) edge (7);
    \draw [->] (7) edge (4);
    \draw [->] (7) edge[bend left=30] (2);
\end{scope}
\end{tikzpicture}
\end{center}

Based on how these pages link to each other,
write a list of the 7 webpages in order from
most imporant to least important.
\end{activity}

\begin{observation}
\textbf{Two \$700,000,000,000 Ideas:}

\begin{enumerate}
\item
If a webpage is important, there should be other important webpages that
link to it.

\item
If a webpage links out to fewer pages, then its links are more valuable.
\end{enumerate}
\end{observation}

\begin{example}
Consider this small network with only three pages.

\begin{multicols}{2}
% \begin{align*}
% x_1 &\Leftarrow x_2 \\
% x_2 &\Leftarrow \frac{1}{2}x_1+ x_3 \\
% x_3 &\Leftarrow \frac{1}{2} x_1
% \end{align*}
\begin{center}
\begin{tikzpicture}
\begin{scope}[every node/.style={circle,thick,draw}]
    \node (1) at (0,4) {1};
    \node (2) at (0,0) {2};
    \node (3) at (4,0) {3};
\end{scope}

\begin{scope}[>={Stealth[red]}, every edge/.style={draw=red,very thick}]
    \draw [->] (1) edge[bend right=30] (2);
    \draw [->] (1) edge (3);
	\path [->] (2) edge[bend right=30] (1);
    \draw [->] (3) edge (2);
\end{scope}
\end{tikzpicture}
\end{center}

Each column of the following \term{transition matrix}
describes the percentage of links out from the corresponding page
into the pages corresponding to each row.

\[
  A
    =
  \begin{bmatrix}
    0\% & 100\% & 0\% \\
    50\% & 0\% & 100\% \\
    50\% & 0\% & 0\%
  \end{bmatrix}
    =
  \begin{bmatrix}
    0 & 1 & 0 \\
    \frac{1}{2} & 0 & 1 \\
    \frac{1}{2} & 0 & 0
  \end{bmatrix}
\]
\end{multicols}
\end{example}

\begin{activity}{5}
  Suppose \(x_i\) measures the importance of webpage \(i\), and
  \(\vec{x}=\begin{bmatrix}x_1\\x_2\\x_3\end{bmatrix}\).
  Then we may compute the product \(A\vec x\) as follows:

  \begin{multicols}{2}
  \begin{center}
  \begin{tikzpicture}[scale=0.5]
  \begin{scope}[every node/.style={circle,thick,draw}]
      \node (1) at (0,4) {1};
      \node (2) at (0,0) {2};
      \node (3) at (4,0) {3};
  \end{scope}

  \begin{scope}[>={Stealth[red]}, every edge/.style={draw=red,very thick}]
      \draw [->] (1) edge[bend right=30] (2);
      \draw [->] (1) edge (3);
  	\path [->] (2) edge[bend right=30] (1);
      \draw [->] (3) edge (2);
  \end{scope}
  \end{tikzpicture}
  \end{center}

  \[
    A\vec{x}
      =
    \begin{bmatrix}
      0 & 1 & 0 \\
      \frac{1}{2} & 0 & 1 \\
      \frac{1}{2} & 0 & 0
    \end{bmatrix}
    \begin{bmatrix}
      x_1 \\ x_2 \\ x_3
    \end{bmatrix}
      =
    \begin{bmatrix}
      x_2 \\
      \frac{x_1}{2}+x_3 \\
      \frac{x_1}{2}
    \end{bmatrix}
  \]
  \end{multicols}

Since each row represents the total value of incoming links to a webpage,
the resulting vector \(A\vec{x}\) should describe
the importance of each webpage, which was defined as \(\vec{x}\).

Since \(A\vec{x}\) and \(\vec{x}\) should be the same vector,
\(\vec{x}\) must be...

\begin{enumerate}[a)]
  \item The downward diagonal of \(A\)
  \item The upward diagonal of \(A\)
  \item An eigenvector of \(A\) corresponding to \(\lambda=1\)
\end{enumerate}
\end{activity}

\begin{fact}
% This is called a \term{Markov chain model}.  We can think about iteratively moving importance from webpages to each other by repeatedly multiplying by $A$.
The \term{steady state vector} for a matrix \(A\) is an eigenvector
satisfying \(A\vec{x}=\vec{x}\).

As we've just demonstrated, a steady state vector for the transition matrix
describes the importance of each webpage in the network.

Thus, the \$700,000,000,000 problem is an eigenvector problem!
\end{fact}

\begin{activity}{10}
Find a steady state vector (an eigenvector associated to the eigenvalue $1$)
for the following network's transition matrix \(A\).

\begin{multicols}{2}
\begin{center}
\begin{tikzpicture}[scale=0.75]
\begin{scope}[every node/.style={circle,thick,draw}]
    \node (1) at (0,4) {1};
    \node (2) at (0,0) {2};
    \node (3) at (4,0) {3};
\end{scope}

\begin{scope}[>={Stealth[red]}, every edge/.style={draw=red,very thick}]
    \draw [->] (1) edge[bend right=30] (2);
    \draw [->] (1) edge (3);
  \path [->] (2) edge[bend right=30] (1);
    \draw [->] (3) edge (2);
\end{scope}
\end{tikzpicture}
\end{center}

\[
  A
    =
  \begin{bmatrix}
    0 & 1 & 0 \\
    \frac{1}{2} & 0 & 1 \\
    \frac{1}{2} & 0 & 0
  \end{bmatrix}
\]
\end{multicols}
\end{activity}

\begin{observation}
Row-reducing
\(
  A-I
    =
  \begin{bmatrix}
    -1 & 1 & 0 \\
    \frac{1}{2} & -1 & 1 \\
    \frac{1}{2} & 0 & -1
  \end{bmatrix}
    \sim
  \begin{bmatrix}
    1 & 0 & -2 \\
    0 & 1 & -2 \\
    0 & 0 & 0
  \end{bmatrix}
\)
yields the basic eigenvector \(\begin{bmatrix} 2 \\ 2 \\1 \end{bmatrix}\).

Therefore, we may conclude that pages \(1\) and \(2\) are equally important,
and both pages are twice as important as page \(3\).
\begin{center}
\begin{tikzpicture}[scale=0.75]
\begin{scope}[every node/.style={circle,thick,draw}]
    \node (1) at (0,4) {1};
    \node (2) at (0,0) {2};
    \node (3) at (4,0) {3};
\end{scope}

\begin{scope}[>={Stealth[red]}, every edge/.style={draw=red,very thick}]
    \draw [->] (1) edge[bend right=30] (2);
    \draw [->] (1) edge (3);
  \path [->] (2) edge[bend right=30] (1);
    \draw [->] (3) edge (2);
\end{scope}
\end{tikzpicture}
\end{center}
\end{observation}


\begin{activity}{10}
Complete the $7 \times 7$ transition matrix for the following network.
\begin{multicols}{2}
\begin{center}
\begin{tikzpicture}[scale=0.6]
\begin{scope}[every node/.style={circle,thick,draw}]
    \node (1) at (0,4) {1};
    \node (2) at (0,0) {2};
    \node (3) at (3,0) {3};
    \node (4) at (3,4) {4};
    \node (5) at (6,4) {5};
    \node (6) at (10,4) {6} ;
    \node (7) at (6,0) {7} ;
\end{scope}

\begin{scope}[>={Stealth[red]}, every edge/.style={draw=red,very thick}]
    \draw [->] (1) edge[bend right=30] (2);
    \draw [->] (1) edge (4);
	\path [->] (2) edge[bend right=30] (1);
    \draw [->] (2) edge (3);
    \draw [->] (3) edge (4);
    \draw [->] (3) edge (7);
    \draw [->] (4) edge (2);
    \draw [->] (5) edge[bend right=30] (6);
    \draw [->] (5) edge (7);
    \draw [->] (6) edge[bend right=30] (5);
    \draw [->] (6) edge (7);
    \draw [->] (7) edge (4);
    \draw [->] (7) edge[bend left=30] (2);
\end{scope}
\end{tikzpicture}
\end{center}

\[
  A
    =
  \begin{bmatrix}
    0 & \unknown & \unknown & \unknown & \unknown & \unknown & \unknown \\
    \frac{1}{2} & \unknown & \unknown & \unknown & \unknown & \unknown & \unknown \\
    0 & \unknown & \unknown & \unknown & \unknown & \unknown & \unknown \\
    \frac{1}{2} & \unknown & \unknown & \unknown & \unknown & \unknown & \unknown \\
    0 & \unknown & \unknown & \unknown & \unknown & \unknown & \unknown \\
    0 & \unknown & \unknown & \unknown & \unknown & \unknown & \unknown \\
    0 & \unknown & \unknown & \unknown & \unknown & \unknown & \unknown
  \end{bmatrix}
\]
\end{multicols}
\end{activity}

\begin{activity}{10}
Find a steady state vector for the transition matrix.

\begin{multicols}{2}
\begin{center}
\begin{tikzpicture}[scale=0.6]
\begin{scope}[every node/.style={circle,thick,draw}]
    \node (1) at (0,4) {1};
    \node (2) at (0,0) {2};
    \node (3) at (3,0) {3};
    \node (4) at (3,4) {4};
    \node (5) at (6,4) {5};
    \node (6) at (10,4) {6} ;
    \node (7) at (6,0) {7} ;
\end{scope}

\begin{scope}[>={Stealth[red]}, every edge/.style={draw=red,very thick}]
    \draw [->] (1) edge[bend right=30] (2);
    \draw [->] (1) edge (4);
	\path [->] (2) edge[bend right=30] (1);
    \draw [->] (2) edge (3);
    \draw [->] (3) edge (4);
    \draw [->] (3) edge (7);
    \draw [->] (4) edge (2);
    \draw [->] (5) edge[bend right=30] (6);
    \draw [->] (5) edge (7);
    \draw [->] (6) edge[bend right=30] (5);
    \draw [->] (6) edge (7);
    \draw [->] (7) edge (4);
    \draw [->] (7) edge[bend left=30] (2);
\end{scope}
\end{tikzpicture}
\end{center}

\[
A=\begin{bmatrix}
0 & \frac{1}{2} & 0 & 0 & 0 & 0 & 0 \\
\frac{1}{2} & 0 & 0 & 1 & 0 & 0 & \frac{1}{2} \\
0 & \frac{1}{2} & 0 & 0 & 0 & 0 & 0 \\
\frac{1}{2} & 0 & \frac{1}{2} & 0 & 0 & 0 & \frac{1}{2} \\
0 & 0 & 0 & 0 & 0 & \frac{1}{2} & 0 \\
0 & 0 & 0 & 0 & \frac{1}{2} & 0 & 0 \\
0 & 0 & \frac{1}{2} & 0 & \frac{1}{2} & \frac{1}{2} & 0
\end{bmatrix}
\]
\end{multicols}

Which webpage is most important?
\end{activity}

\begin{observation}
Since a steady state vector for the network is given by \(\vec x\),
it's reasonable to consider page \(2\) as the most important page.

\parbox{\textwidth}{
\begin{multicols}{2}
\begin{center}
\begin{tikzpicture}[scale=0.6]
\begin{scope}[every node/.style={circle,thick,draw}]
    \node (1) at (0,4) {1};
    \node (2) at (0,0) {2};
    \node (3) at (3,0) {3};
    \node (4) at (3,4) {4};
    \node (5) at (6,4) {5};
    \node (6) at (10,4) {6} ;
    \node (7) at (6,0) {7} ;
\end{scope}

\begin{scope}[>={Stealth[red]}, every edge/.style={draw=red,very thick}]
    \draw [->] (1) edge[bend right=30] (2);
    \draw [->] (1) edge (4);
	\path [->] (2) edge[bend right=30] (1);
    \draw [->] (2) edge (3);
    \draw [->] (3) edge (4);
    \draw [->] (3) edge (7);
    \draw [->] (4) edge (2);
    \draw [->] (5) edge[bend right=30] (6);
    \draw [->] (5) edge (7);
    \draw [->] (6) edge[bend right=30] (5);
    \draw [->] (6) edge (7);
    \draw [->] (7) edge (4);
    \draw [->] (7) edge[bend left=30] (2);
\end{scope}
\end{tikzpicture}
\end{center}
\[
  \vec{x}
    =
  \begin{bmatrix} 2 \\ 4 \\2 \\ 2.5 \\ 0 \\ 0 \\ 1\end{bmatrix}
\]
\end{multicols}}
\end{observation}

\end{applicationActivities}
