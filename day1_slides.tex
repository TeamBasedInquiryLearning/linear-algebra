
\documentclass{beamer}


\usetheme{Hannover}
\usecolortheme{rose}

%Center frame titles
\setbeamertemplate{frametitle}[default][center]

%This file is for defining specific commands that apply only to one or two
%of the four sections
\usepackage{etoolbox}
\newcommand{\sem}{Fall 2017}

\newbool{TR}
\newbool{TBL}

\newcommand{\course}{Math 237-%
  \ifbool{TR}{%
    \ifbool{TBL}{102}{103}%
  }{%
    \ifbool{TBL}{501}{101}%
  }%
}
\newcommand{\prof}{%
  \ifbool{TR}{%
    Dr. Drew Lewis%
  }{%
    Dr. Steven Clontz%
  }%
}
\newcommand{\profemail}{%
  \ifbool{TR}{%
    \texttt{drewlewis@southalabama.edu}%
  }{%
    \texttt{sclontz@southalabama.edu}%
  }%
}
\newcommand{\profoffice}{%
  \ifbool{TR}{%
    MSPB (ILB) 439%
  }{%
    MSPB (ILB) 314%
  }%
}
\newcommand{\profhours}{%
  \ifbool{TR}{%
    TBA%
  }{%
    W 10:30-3:30, 4:50-5:50%
  }%
}
\newcommand{\tbltime}{%
  \ifbool{TR}{%
    TR 9:30-10:45 COMM 0160%
  }{%
    MW 6:00-7:15 COMM 0160%
  }%
}
\newcommand{\lecturetime}{%
  \ifbool{TR}{%
    TR 12:30-1:45 MSPB (ILB) 350%
  }{%
    MW 3:35-4:50 MSPB (ILB) 360%
  }%
}
\newcommand{\classtime}{\ifbool{TBL}{\tbltime}{\lecturetime}}

\newcommand*{\TakeFourierOrnament}[1]{{%
\fontencoding{U}\fontfamily{futs}\selectfont\char#1}}
\newcommand*{\danger}{\TakeFourierOrnament{66}}

\newcommand{\masteryMark}{\ifbool{TR}{\textbf{M}}{\checkmark}}
\newcommand{\minorMark}{\textbf{*}}
\newcommand{\reattemptMark}{\ifbool{TR}{\textbf{R}}{\danger}}
\newcommand{\noMark}{\ifbool{TR}{\textbf{B}}{\(\times\)}}


% Include the following to configure course
% \booltrue{TR}   %Use for Drew's TR sections
% \boolfalse{TR}  %Use for Steven's MW sections
% \booltrue{TBL}  %Use for TBL sections
% \boolfalse{TBL} %Use for non-TBL sections
%This file is for defining specific commands that apply only to one or two
%of the four sections
\usepackage{etoolbox}
\newcommand{\sem}{Fall 2017}

\newbool{TR}
\newbool{TBL}

\newcommand{\course}{Math 237-%
  \ifbool{TR}{%
    \ifbool{TBL}{102}{103}%
  }{%
    \ifbool{TBL}{501}{101}%
  }%
}
\newcommand{\prof}{%
  \ifbool{TR}{%
    Dr. Drew Lewis%
  }{%
    Dr. Steven Clontz%
  }%
}
\newcommand{\profemail}{%
  \ifbool{TR}{%
    \texttt{drewlewis@southalabama.edu}%
  }{%
    \texttt{sclontz@southalabama.edu}%
  }%
}
\newcommand{\profoffice}{%
  \ifbool{TR}{%
    MSPB (ILB) 439%
  }{%
    MSPB (ILB) 314%
  }%
}
\newcommand{\profhours}{%
  \ifbool{TR}{%
    TBA%
  }{%
    W 10:30-3:30, 4:50-5:50%
  }%
}
\newcommand{\tbltime}{%
  \ifbool{TR}{%
    TR 9:30-10:45 COMM 0160%
  }{%
    MW 6:00-7:15 COMM 0160%
  }%
}
\newcommand{\lecturetime}{%
  \ifbool{TR}{%
    TR 12:30-1:45 MSPB (ILB) 350%
  }{%
    MW 3:35-4:50 MSPB (ILB) 360%
  }%
}
\newcommand{\classtime}{\ifbool{TBL}{\tbltime}{\lecturetime}}

\newcommand*{\TakeFourierOrnament}[1]{{%
\fontencoding{U}\fontfamily{futs}\selectfont\char#1}}
\newcommand*{\danger}{\TakeFourierOrnament{66}}

\newcommand{\masteryMark}{\ifbool{TR}{\textbf{M}}{\checkmark}}
\newcommand{\minorMark}{\textbf{*}}
\newcommand{\reattemptMark}{\ifbool{TR}{\textbf{R}}{\danger}}
\newcommand{\noMark}{\ifbool{TR}{\textbf{B}}{\(\times\)}}


% Include the following to configure course
% \booltrue{TR}   %Use for Drew's TR sections
% \boolfalse{TR}  %Use for Steven's MW sections
% \booltrue{TBL}  %Use for TBL sections
% \boolfalse{TBL} %Use for non-TBL sections
%This file is for defining specific commands that apply only to one or two
%of the four sections
\usepackage{etoolbox}
\newcommand{\sem}{Fall 2017}

\newbool{TR}
\newbool{TBL}

\newcommand{\course}{Math 237-%
  \ifbool{TR}{%
    \ifbool{TBL}{102}{103}%
  }{%
    \ifbool{TBL}{501}{101}%
  }%
}
\newcommand{\prof}{%
  \ifbool{TR}{%
    Dr. Drew Lewis%
  }{%
    Dr. Steven Clontz%
  }%
}
\newcommand{\profemail}{%
  \ifbool{TR}{%
    \texttt{drewlewis@southalabama.edu}%
  }{%
    \texttt{sclontz@southalabama.edu}%
  }%
}
\newcommand{\profoffice}{%
  \ifbool{TR}{%
    MSPB (ILB) 439%
  }{%
    MSPB (ILB) 314%
  }%
}
\newcommand{\profhours}{%
  \ifbool{TR}{%
    TBA%
  }{%
    W 10:30-3:30, 4:50-5:50%
  }%
}
\newcommand{\tbltime}{%
  \ifbool{TR}{%
    TR 9:30-10:45 COMM 0160%
  }{%
    MW 6:00-7:15 COMM 0160%
  }%
}
\newcommand{\lecturetime}{%
  \ifbool{TR}{%
    TR 12:30-1:45 MSPB (ILB) 350%
  }{%
    MW 3:35-4:50 MSPB (ILB) 360%
  }%
}
\newcommand{\classtime}{\ifbool{TBL}{\tbltime}{\lecturetime}}

\newcommand*{\TakeFourierOrnament}[1]{{%
\fontencoding{U}\fontfamily{futs}\selectfont\char#1}}
\newcommand*{\danger}{\TakeFourierOrnament{66}}

\newcommand{\masteryMark}{\ifbool{TR}{\textbf{M}}{\checkmark}}
\newcommand{\minorMark}{\textbf{*}}
\newcommand{\reattemptMark}{\ifbool{TR}{\textbf{R}}{\danger}}
\newcommand{\noMark}{\ifbool{TR}{\textbf{B}}{\(\times\)}}


% Include the following to configure course
% \booltrue{TR}   %Use for Drew's TR sections
% \boolfalse{TR}  %Use for Steven's MW sections
% \booltrue{TBL}  %Use for TBL sections
% \boolfalse{TBL} %Use for non-TBL sections
\input{class-macros.local}




\title{Welcome to Linear Algebra}
\author{\prof}

\date{\ifbool{TR}{August 17, 2017}{August 16,2017}}


\begin{document}

\begin{frame}
\titlepage
\end{frame}

\begin{frame} \frametitle{What is Linear Algebra? }
Linear algebra is the study of {\bf linear maps}.
\begin{itemize}
\item In Calculus, you learn how to approximate any function by a linear function.
\item In Linear Algebra, we learn about how linear maps behave.
\item Combining the two, we can approximate how any function behaves.
\end{itemize}
\end{frame}

\begin{frame} \frametitle{What is Linear Algebra good for?}
\begin{itemize}
\item In an abstract sense, linear algebra is arguably the most used tool in higher math.
\item In computer graphics, linear algebra is used to help represent 3-dimensional objects in a two dimensional grid of pixels.
\item Differential equations are often very difficult (or impossible) to solve exactly; we use linear algebra to understand approximate solutions in a vast number of engineering applications such as fluid flows, vibrations, heat transfer, etc.
\item Google's famed Page Rank algorithm is based on linear algebra
\end{itemize}
\end{frame}

\begin{frame} \frametitle{Learning Outcomes }
By the end of this class, you will be able to
\begin{itemize}
\item Solve systems of linear equations.
\pause \item Determine whether or not a set with given operations is a vector space or a subspace of another vector space.
\pause \item Determine properties of sets of vectors such as whether they are linearly independent, whether they span, and whether they are a basis.
\pause \item Perform fundamental operations in the algebra of matrices, including multiplying and inverting matrices.
\pause \item Use and apply algebraic properties of a linear tranformation.
\pause \item Determine geometric information about a linear transformation, including computing determinants, eigenvalues, and eigenvectors.
\end{itemize}
\end{frame}

\section{SBG}
\begin{frame}\frametitle{Standards Based Grading}
You have two jobs in this class
\begin{enumerate}
\item Master the material
\item Demonstrate to me that you have mastered the material
\end{enumerate}

\vspace{0.2in}

{\bf I do not care when* or how you demonstrate mastery, only that you demonstrate mastery.}
\end{frame}

\begin{frame}\frametitle{SBG}
The course material is broken down into 23 learning {\bf standards}
\begin{itemize}
\item On an assessment (e.g. quiz), you will not receive a numerical score; instead for each standard I will mark whether you've mastered it or not.
\item Standards will be assessed several times.  If you don't succeed the first time, practice more and try again!  
\item It doesn't matter how many attempts it takes, only that you eventually demonstrate mastery.
\item You can demonstrate mastery of each standard up to two times, making for 46 possible mastery checkmarks.  Your grade will be determined by how many checkmarks you earn.
\end{itemize}
\end{frame}

\begin{frame}\frametitle{Assessments}
There will be several different types of assessments
\begin{itemize}
\item {\bf Quizzes}: Each day at the end of class we will have a quiz.
\item {\bf Midterm}: There will be a single midterm exam the week of Fall Break.
\item {\bf Final Exam}: This will be your final opportunity to demonstrate mastery
\item {\bf Office Hours Reassessments}: See form in Sakai
\end{itemize}

\vspace{0.2in}

I do not care which assessment you demonstrate mastery on, only that you demonstrate mastery of a standard.
\end{frame}

\begin{frame}\frametitle{Interpreting Feedback}
On each assessment, for each standard you will receive one of the following marks
\begin{itemize}
\item {\bf M} means you demonstrated {\bf Mastery} of that standard.  Great job!  Check off another box on your progress sheet.
\item {\bf *} means you have a minor mistake. If you can determine your mistake on your own, and come explain it to me in my office hours in the next week, then I will modify the * into a M. 
\item {\bf R} means you are eligible to {\bf Reassess} in my office hours.  You will earn this mark if you made a good faith attempt and demonstrated partial understanding, but did not demonstrate full mastery of that standard on this assessment.  
\item {\bf N} means there was {\bf No Significant Evidence} of understanding. Your next attempt must come on an in-class assessment.  
\end{itemize}
\end{frame}

\begin{frame}\frametitle{Course Grades}

\begin{tabular}{l|l}
\hline \\
A & \begin{minipage}{0.7\textwidth}
\begin{itemize}
\item Obtain 40 mastery checkmarks;
\item Complete 10 homework reports;
\item \ifbool{TBL}{Have a 90\% Class Participation Score}{Have an 80\% attendence record.} \\
\end{itemize} 
\end{minipage} \\
\hline \\

B & \begin{minipage}{0.7\textwidth}
\begin{itemize}	
\item Obtain 35 mastery checkmarks;
\item Complete 8 homework reports;
\item \ifbool{TBL}{Have a 80\% Class Participation Score}{Have an 80\% attendence record.} \\
\end{itemize} 
\end{minipage} \\
\hline \\

C 	& \begin{minipage}{0.7\textwidth}
\begin{itemize}	
\item Obtain 30 mastery checkmarks;
\item Complete 6 homework reports;
\item \ifbool{TBL}{Have a 70\% Class Participation Score}{Have an 80\% attendence record.} \\
\end{itemize} 
\end{minipage} \\
\hline \\

D 	&\begin{minipage}{0.7\textwidth}
\begin{itemize}	
\item Obtain 20 mastery checkmarks;
\item Complete 4 homework reports;
\item \ifbool{TBL}{Have a 50\% Class Participation Score}{Have a 50\% attendence record.} \\
\end{itemize} 
\end{minipage} \\
\hline 

\end{tabular}

\end{frame}

\begin{frame}\frametitle{Homework}
Homework is practice.
\begin{itemize}
\item I will not collect or grade homework problems.  
\item A list of suggested exercise for practice is in Sakai, sorted by standard.  You should work as many or as few of these as you need to master the material.
\item {\bf Caveat discipulus}: Most students do not work as many homework exercises as they should.
\item If you need help or feedback, come to my office hours.
\item I will collect homework reports each week (blank form in Sakai).
\end{itemize}
\end{frame}

\section{TBL}

\begin{frame}\frametitle{Team-Based Learning}
In this class we will use {\bf Team-Based Learning}.
\begin{itemize}
\item The course is divided into six modules, each lasting about 2 weeks.
\item At the beginning of each module is the {\bf Readiness Assurance Process}.  The first day of the module will consist of individual and team Readiness Assurance Tests
\item The next 3-4 class days will consist of guided activities with you working in your team.
\item Research in other STEM disciplines show that TBL leads to improved student learning.
\end{itemize}
\end{frame}

\begin{frame}\frametitle{Readiness Assurance Process}
\begin{itemize}
\item In Sakai, you will find a list of the skills you should have {\bf before each module starts}, along with a list of resources to help you prepare.
\begin{itemize}
\item Sometimes these skills are from previous courses.
\item Sometimes these skills are standards from earlier in this course.
\end{itemize}
\pause \item On the first day of the module, the Readiness Assurance Tests will ensure you have these skills.
\begin{itemize}
\item First, you will individually take the RAT
\item After everyone is done, you will take the RAT again collaboratively as a team.
\end{itemize}
\item {\bf The first Readiness Assurance day is \ifbool{TR}{Tuesday}{Monday}!}
\end{itemize}
\end{frame}

\begin{frame}\frametitle{Teams}
Stand up.  Line up in alphabetical order by last name, with A at the front of the room.
\vspace{4in}
\end{frame}

\begin{frame}\frametitle{Office Hours}
Choose up 3 one-hour long periods your team would like me to have an office hour during.  Rank them in order of your preference.\\
\ \\

I have the following constraints:
\begin{itemize}
\item They must be during business hours, i.e. 8-5.
\item I teach another class from 12:30-1:45 on TR
\item I have departmental meetings/seminars 3:30-5 on Thursdays
\end{itemize}
\end{frame}

\begin{frame} \frametitle{What makes a good team member?}
Create a list of criteria that make an effective team member.
\vspace{4in}
\end{frame}

\begin{frame} \frametitle{Peer Evaluation Questions}
Create a list of questions your team thinks should be on the peer evaluation surveys.  Answers to the questions should be on a scale from 1 to 5.
\vspace{4in}
\end{frame}

\begin{frame} \frametitle{Class Participation Score}
There will be four components to your participation score

\begin{center}
\begin{tabular}{l|l}
\hline 
iRAT & \phantom{xxxx}\%   \\ \hline
tRAT & \phantom{xxxx}\%   \\ \hline
Peer Evaluation & \phantom{xxxx}\%   \\ \hline
Attendence & \phantom{xxxx}\%   \\ \hline
\end{tabular}
\end{center}


In your teams, decide what percentage each of the four components should have.  They should add to 100\%.
\end{frame}



\end{document}