
\begin{exerciseStatement}


Let \(A\) be a \(4 \times 4\) matrix with determinant \( -5 \).


\begin{enumerate}[(a)]
\item Let \(Q\) be the matrix obtained from \(A\) by applying the row operation \( R_3 \leftrightarrow R_1 \). What is \(\operatorname{det}\ Q\)?
\item Let \(N\) be the matrix obtained from \(A\) by applying the row operation \( R_1 \to R_1 + -3R_2 \). What is \(\operatorname{det}\ N\)?
\item Let \(M\) be the matrix obtained from \(A\) by applying the row operation \( R_1 \to -2R_1 \). What is \(\operatorname{det}\ M\)?
\end{enumerate}
    
\end{exerciseStatement}
    
\begin{exerciseAnswer} 

\begin{enumerate}[(a)]
\item \(\operatorname{det}\ Q= 5 \)
\item \(\operatorname{det}\ N= -5 \)
\item \(\operatorname{det}\ M= 10 \)
\end{enumerate}
    
\end{exerciseAnswer}
    
