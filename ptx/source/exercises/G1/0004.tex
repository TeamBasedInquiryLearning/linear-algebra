
\begin{exerciseStatement}


Let \(A\) be a \(4 \times 4\) matrix with determinant \( 6 \).


\begin{enumerate}[(a)]
\item Let \(N\) be the matrix obtained from \(A\) by applying the row operation \( R_3 \to -3R_3 \). What is \(\operatorname{det}\ N\)?
\item Let \(Q\) be the matrix obtained from \(A\) by applying the row operation \( R_1 \leftrightarrow R_2 \). What is \(\operatorname{det}\ Q\)?
\item Let \(C\) be the matrix obtained from \(A\) by applying the row operation \( R_1 \to R_1 + 5R_2 \). What is \(\operatorname{det}\ C\)?
\end{enumerate}
    
\end{exerciseStatement}
    
\begin{exerciseAnswer} 

\begin{enumerate}[(a)]
\item \(\operatorname{det}\ N= -18 \)
\item \(\operatorname{det}\ Q= -6 \)
\item \(\operatorname{det}\ C= 6 \)
\end{enumerate}
    
\end{exerciseAnswer}
    
