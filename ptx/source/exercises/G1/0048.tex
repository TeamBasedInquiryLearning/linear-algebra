
\begin{exerciseStatement}


Let \(A\) be a \(4 \times 4\) matrix with determinant \( -3 \).


\begin{enumerate}[(a)]
\item Let \(B\) be the matrix obtained from \(A\) by applying the row operation \( R_1 \leftrightarrow R_4 \). What is \(\operatorname{det}\ B\)?
\item Let \(Q\) be the matrix obtained from \(A\) by applying the row operation \( R_3 \to -2R_3 \). What is \(\operatorname{det}\ Q\)?
\item Let \(C\) be the matrix obtained from \(A\) by applying the row operation \( R_2 \to R_2 + 5R_4 \). What is \(\operatorname{det}\ C\)?
\end{enumerate}
    
\end{exerciseStatement}
    
\begin{exerciseAnswer} 

\begin{enumerate}[(a)]
\item \(\operatorname{det}\ B= 3 \)
\item \(\operatorname{det}\ Q= 6 \)
\item \(\operatorname{det}\ C= -3 \)
\end{enumerate}
    
\end{exerciseAnswer}
    
