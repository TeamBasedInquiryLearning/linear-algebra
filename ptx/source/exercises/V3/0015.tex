
\begin{exerciseStatement}


Consider the statement 
\begin{center}\begin{minipage}{0.8\textwidth}
 The set of vectors \( \left\{ \left[\begin{array}{c}
-2 \\
2 \\
-2 \\
-1
\end{array}\right] , \left[\begin{array}{c}
-3 \\
2 \\
-2 \\
3
\end{array}\right] , \left[\begin{array}{c}
2 \\
-2 \\
-3 \\
0
\end{array}\right] , \left[\begin{array}{c}
1 \\
-5 \\
-4 \\
-1
\end{array}\right] , \left[\begin{array}{c}
0 \\
3 \\
1 \\
1
\end{array}\right] , \left[\begin{array}{c}
1 \\
1 \\
4 \\
-5
\end{array}\right] \right\} \)spans \(\mathbb{R}^4\). 
\end{minipage}\end{center}
    


\begin{enumerate}[(a)]
\item  Write an equivalent statement using a vector equation.
\item  Explain why your statement is true or false.
\end{enumerate}
    
\end{exerciseStatement}
    
\begin{exerciseAnswer} 


\[\operatorname{RREF} \left[\begin{array}{cccccc}
-2 & -3 & 2 & 1 & 0 & 1 \\
2 & 2 & -2 & -5 & 3 & 1 \\
-2 & -2 & -3 & -4 & 1 & 4 \\
-1 & 3 & 0 & -1 & 1 & -5
\end{array}\right] = \left[\begin{array}{cccccc}
1 & 0 & 0 & 0 & \frac{11}{177} & \frac{148}{177} \\
0 & 1 & 0 & 0 & \frac{17}{177} & -\frac{254}{177} \\
0 & 0 & 1 & 0 & \frac{35}{59} & -\frac{44}{59} \\
0 & 0 & 0 & 1 & -\frac{137}{177} & -\frac{25}{177}
\end{array}\right] \]


\begin{enumerate}[(a)]
\item The statement 
\begin{center}\begin{minipage}{0.8\textwidth}
 The set of vectors \( \left\{ \left[\begin{array}{c}
-2 \\
2 \\
-2 \\
-1
\end{array}\right] , \left[\begin{array}{c}
-3 \\
2 \\
-2 \\
3
\end{array}\right] , \left[\begin{array}{c}
2 \\
-2 \\
-3 \\
0
\end{array}\right] , \left[\begin{array}{c}
1 \\
-5 \\
-4 \\
-1
\end{array}\right] , \left[\begin{array}{c}
0 \\
3 \\
1 \\
1
\end{array}\right] , \left[\begin{array}{c}
1 \\
1 \\
4 \\
-5
\end{array}\right] \right\} \) spans \(\mathbb{R}^4\). 
\end{minipage}\end{center}
     is equivalent to the statement 
\begin{center}\begin{minipage}{0.8\textwidth}
 The vector equation \( x_{1} \left[\begin{array}{c}
-2 \\
2 \\
-2 \\
-1
\end{array}\right] + x_{2} \left[\begin{array}{c}
-3 \\
2 \\
-2 \\
3
\end{array}\right] + x_{3} \left[\begin{array}{c}
2 \\
-2 \\
-3 \\
0
\end{array}\right] + x_{4} \left[\begin{array}{c}
1 \\
-5 \\
-4 \\
-1
\end{array}\right] + x_{5} \left[\begin{array}{c}
0 \\
3 \\
1 \\
1
\end{array}\right] + x_{6} \left[\begin{array}{c}
1 \\
1 \\
4 \\
-5
\end{array}\right] =\vec{v}\) has a solution for every vector \(\vec{v}\) in \(\mathbb{R}^4\). 
\end{minipage}\end{center}
    
\item  The set of vectors \( \left\{ \left[\begin{array}{c}
-2 \\
2 \\
-2 \\
-1
\end{array}\right] , \left[\begin{array}{c}
-3 \\
2 \\
-2 \\
3
\end{array}\right] , \left[\begin{array}{c}
2 \\
-2 \\
-3 \\
0
\end{array}\right] , \left[\begin{array}{c}
1 \\
-5 \\
-4 \\
-1
\end{array}\right] , \left[\begin{array}{c}
0 \\
3 \\
1 \\
1
\end{array}\right] , \left[\begin{array}{c}
1 \\
1 \\
4 \\
-5
\end{array}\right] \right\} \) spans \(\mathbb{R}^4\). 
\end{enumerate}
    
\end{exerciseAnswer}
    
