
\begin{exerciseStatement}


Consider the statement 
\begin{center}\begin{minipage}{0.8\textwidth}
 The set of polynomials \( \left\{ -4 \, x^{2} + 3 \, x + 4 , 2 \, x^{2} + 2 \, x - 4 , 4 \, x^{3} - 4 \, x^{2} - 5 \, x - 2 , 14 \, x^{2} - 7 \, x - 16 \right\} \) spans \(\mathcal{P}_3\). 
\end{minipage}\end{center}
    


\begin{enumerate}[(a)]
\item  Write an equivalent statement using a polynomial equation.
\item  Explain why your statement is true or false.
\end{enumerate}
    
\end{exerciseStatement}
    
\begin{exerciseAnswer} 


\[\operatorname{RREF} \left[\begin{array}{cccc}
4 & -4 & -2 & -16 \\
3 & 2 & -5 & -7 \\
-4 & 2 & -4 & 14 \\
0 & 0 & 4 & 0
\end{array}\right] = \left[\begin{array}{cccc}
1 & 0 & 0 & -3 \\
0 & 1 & 0 & 1 \\
0 & 0 & 1 & 0 \\
0 & 0 & 0 & 0
\end{array}\right] \]


\begin{enumerate}[(a)]
\item The statement 
\begin{center}\begin{minipage}{0.8\textwidth}
 The set of polynomials \( \left\{ -4 \, x^{2} + 3 \, x + 4 , 2 \, x^{2} + 2 \, x - 4 , 4 \, x^{3} - 4 \, x^{2} - 5 \, x - 2 , 14 \, x^{2} - 7 \, x - 16 \right\} \) spans \(\mathcal{P}_3\). 
\end{minipage}\end{center}
     is equivalent to the statement 
\begin{center}\begin{minipage}{0.8\textwidth}
 The polynomial equation \[ y_{1} \left( -4 \, x^{2} + 3 \, x + 4 \right) + y_{2} \left( 2 \, x^{2} + 2 \, x - 4 \right) + y_{3} \left( 4 \, x^{3} - 4 \, x^{2} - 5 \, x - 2 \right) + y_{4} \left( 14 \, x^{2} - 7 \, x - 16 \right) =f\] has a solution for every \(f \in \mathcal{P}_3\). 
\end{minipage}\end{center}
    
\item The set of polynomials \( \left\{ -4 \, x^{2} + 3 \, x + 4 , 2 \, x^{2} + 2 \, x - 4 , 4 \, x^{3} - 4 \, x^{2} - 5 \, x - 2 , 14 \, x^{2} - 7 \, x - 16 \right\} \) does not span \(\mathcal{P}_3\). 
\end{enumerate}
    
\end{exerciseAnswer}
    
